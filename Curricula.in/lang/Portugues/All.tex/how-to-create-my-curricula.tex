\section*{?`Cómo crear una malla para mi carrera?}
\addcontentsline{toc}{section}{?`Cómo crear una malla para mi carrera?}%

A pesar de que esta malla trata de abarcar la gran mayoría del cuerpo del conocimiento de esta área,  
es necesario personalizarla para cada institución. Para que esto sea posible, se ha dejado un 
margen de créditos en cada semestre de la malla presentada en la sección \ref{sec:tables-by-semester}.

Hay universidades, como la \acf{UCSP} que tienen un especial énfasis en la formación de la persona. 
Estos créditos libres han permitido completar el documento de manera rápida a la carrera. 
En el caso de la \acf{UNSA} los créditos libres fueron utilizados para incorporar cursos de inglés. 

Sabemos que cada universidad tiene un especial énfasis en alguna línea en especial pero nunca 
debemos perder el foco principal de la carrera. No debemos caer nuevamente el error de crear 
decenas mallas curriculares cada una de ellas tratando de abarcar el todo de los cinco perfiles 
internacionales de IEEE-CS, ACM. En la práctica lo que hemos provocado es que no sea posible 
encontrar dos carreras con el mismo nombre que sean realmente equivalentes. 

Si Ud. desea crear una malla curricular en base a este documento debe seguir los siguientes pasos:

\begin{itemize}
\item Leer el documento y entender su estructura,
\item Prestar especial atención a los cursos de la malla que están en sección \ref{sec:tables-by-semester} 
      así como a los créditos libres en cada semestre.
\item Pensar en los cursos necesarios para adaptar y/o completar esta propuesta a su institución,
\item Tener el logo de la institución en tamaño 7x7cm en formato eps (\textit{Encrypted Poscript}) y 
      el mismo logo con transparencia de 50\% para el fondo del documento,
\item Tener algunos datos generales: nombre de la facultad, nombre de la carrera, 
\item Entrar en contacto con los autores de esta propuesta para la generación del documento.
\end{itemize}
