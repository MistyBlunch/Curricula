\section{Objetivos de Aprendizaje ({\it Learning Outcomes})}
Cada KU dentro de un KA enumera tanto un conjunto de temas y los resultados de aprendizaje ({\it Learning Outcomes})
que los estudiantes deben alcanzar en lo que respecta a los temas especificados. 
Resultados de aprendizaje no son de igual tamaño y no tienen una asignación uniforme de horas curriculares; 
temas con el mismo número de horas pueden tener muy diferentes números de los resultados del aprendizaje asociados.

Cada resultado de aprendizaje tiene un nivel asociado de dominio. 
En la definición de los diferentes niveles que dibujamos de otros enfoques curriculares, 
especialmente la taxonomí­a de Bloom, que ha sido bien explorado dentro de la ciencia de la computación. 
En este documento no se aplicó directamente los niveles de Bloom en parte porque varios de 
ellos son impulsados por contexto pedagógico, que introducirí­a demasiada pluralidad en un documento de este tipo; 
en parte porque tenemos la intención de los niveles de dominio para ser indicativa y 
no imponer restricción teórica sobre los usuarios de este documento.

Nosotros usamos tres niveles de dominio esperados que son:
\begin{description}
 \item [Nivel 1 \Familiarity ({\it Familiarity})]: \LearningOutcomesTxtEsFamiliarity
 \item [Nivel 2 \Usage ({\it Usage})]: \LearningOutcomesTxtEsUsage
 \item [Nivel 3 \Assessment ({\it Assessment})]: \LearningOutcomesTxtEsAssessment
\end{description}

% As a concrete, although admittedly simplistic, example of these levels of mastery, we consider
% the notion of iteration in software development, for example for-loops, while-loops, and iterators.
% At the level of ``Familiarity,'' a student would be expected to have a definition of the concept of
% iteration in software development and know why it is a useful technique. In order to show
% mastery at the ``Usage'' level, a student should be able to write a program properly using a form
% of iteration. Understanding iteration at the ``Assessment'' level would require a student to
% understand multiple methods for iteration and be able to appropriately select among them for
% different applications.

Por ejemplo, para evaluar los niveles de dominio, consideremos la noción de iteración en el desarrollo de software (for, while e iteradores). 
En el plano de la ``familiaridad'', se espera que un estudiante tenga una definición del 
concepto de iteración en el desarrollo de software y saber por qué esta técnica es útil. 

Con el fin de mostrar el dominio del nivel ``Uso'', el estudiante debe ser capaz de 
escribir un programa adecuadamente usando una forma de iteración. 

En el nivel de ``Evaluación'', en la iteración se requerirí­a que un estudiante comprenda múltiples métodos de iteración y 
que sea capaz de seleccionar apropiadamente entre ellos para diferentes aplicaciones.

