 \chapter*{Resumen ejecutivo}
\AbstractIntro

Todo el contenido del documento está basado en la propuesta internacional denominada \textit{Computing Curricula}\footnote{http://www.sigcse.org/cc2001/} 
en el área especí­fica de Ciencia de Datos. Este documento es el resultado de un trabajo conjunto de la 
\textit{Association for Computing Machinery} (ACM) y la Sociedad de Computación de IEEE (IEEE-CS) y 
puede ser accesado a través de la dirección \href{http://www.acm.org/education}{http://www.acm.org/education} 
en sus versions CS2001, CS2008 y \href{cs2020.net}{CC2020}.

Considerando que existen peculiaridades menores al aplicar esta propuesta internacional a nuestros paises, el modelo de \textit{Computing Curricula} 
fue utilizado para proponer el documento base de la presente malla. 

\noindent La computación hoy en dí­a presenta 5 perfiles de formación profesional claramente definidos: 
\begin{itemize}
\item \textbf{Ciencia de la Computación} (\textit{Computer Science} -- CS),
\item Ingenierí­a de Computación (\textit{Computer Engineering} -- CE),
\item Sistemas de Información (\textit{Information Systems} -- IS),
\item Ingenierí­a de Software (\textit{Software Engineering} -- SE) y 
\item Tecnologí­a de la Información (\textit{Information Technology} -- IT).
\end{itemize}

Que a paretir de CC2020 serán 7 considerando las nuevas propuestas de {\it Data Science} y {\it Cyebrsecurity} 

Los pilares fundamentales que consideramos en esta propuesta curricular son:
\begin{itemize}
\item Una sólida formación profesional en el área de Ciencia de Datos,
\item Preparación para la generación de alto valor agregado en las empresas a partir del tratamiento de loa dstos,
\item Una sólida formación ética y proyección a la sociedad
\end{itemize}

Estos pilares redundarán en la formación de profesionales que se puedan desempeñar en 
cualquier parte del mundo y que ayuden de forma clara al desarrollo de la Industria 
de Software de nuestro paí­s. 

\OtherKeyStones

El resto de este documento está organizado de la siguiente forma: el Capí­tulo \ref{chap:intro}, 
define y explica el campo de acción de la Ciencia de la Computación, 
además se hace una muy breve explicación de las distintas carreras del área de 
computación propuestas por IEEE-CS y ACM.


\OnlySPC{
En el Capí­tulo \ref{chap:cs-market} se presenta el perfil profesional, un análisis de mercado 
que incluye un análisis de la oferta, demanda y tendencias del conocimiento.

Un análisis más detallado del mercado junto con una encuesta a empresarios es presentada en el 
Capí­tulo \ref{chap:cs-estudio-de-mercado}. Este estudio arrojá datos interesantes con 
relación a la percepción y la necesidad empresarial de este perfil profesional.
}

\OnlyUNSA{
En el Capí­tulo \ref{chap:cs-market} se presenta el perfil profesional, un análisis de mercado 
reconocidas que incluye un análisis de la oferta, demanda y tendencias del conocimiento.

Un análisis más detallado del mercado junto con una encuesta a empresarios es presentada en el 
Capí­tulo \ref{chap:cs-estudio-de-mercado}. Este estudio arrojá datos interesantes con relación 
a la percepción y la necesidad empresarial de este perfil profesional.

En el Capí­tulo \ref{chap:cs-resources} se presentan los recursos de plana docente, infraestructura de 
laboratorios y financieros necesarios para poder crear esta carrera profesional.
}

El Capí­tulo \ref{chap:BOK}, muestra las áreas de Conocimiento de esta área, 
indicando los tópicos y objetivos de aprendizaje de los temas, pertenecientes a estos grupos.

El Capí­tulo \ref{chap:GeneralInfo} contiene la distribución por semestres, por áreas, por niveles, 
visión gráfica de la malla curricular, comparación con las diversas propuestas internacionales, 
distribución de tópicos por curso así­ como la distribución de habilidades por materia.

El Capí­tulo \ref{chap:syllabi} contiene información detallada para cada uno de los cursos 
incluyendo las habilidades con las cuales contribuye, bibliografí­a por cada unidad así­ 
como el número de horas mí­nimas por cada unidad.

\OnlyUCSP{En el Capí­tulo \ref{chap:equivalence} se presentan las tablas de equivalencias con otros 
planes curriculares.}

Finalmente, en el Capí­tulo \ref{chap:laboratories} se presenta una sugerencia de los laboratorios 
requeridos para el dictado de clases las mismas que podrí­an variar de acuerdo al volumen de alumnos que se tenga.

\newpage