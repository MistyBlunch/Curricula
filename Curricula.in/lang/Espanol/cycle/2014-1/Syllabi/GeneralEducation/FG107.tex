\begin{syllabus}

\course{FG107. Fundamentos Antropológicos de la Ciencia}{Obligatorio}{FG107}

\begin{justification}
El estudio del hombre es importante para responder a las preguntas acerca de quién soy y para qué estoy en el mundo. Las respuestas que se den a dichas preguntas resultan fundamentales, pues de ellas dependerá mucho de la orientación que cada uno dé a su vida tanto en la dimensión personal como en la social. La aproximación al hombre desde la razón necesita la complementariedad de la fe, que le abre un horizonte de mayor comprensión de la verdad. La revelación necesita de la razón para ser aprehendida, comprendida y aplicada. Por lo tanto, el curso de antropología filosófica y teológica, desde la síntesis entre fe y razón, busca dar una respuesta integral acerca de quién es el hombre y para qué está en el mundo, considerando la unidad entre la filosofía y la teología -fundada en la armonía entre fe y razón- y respetando la distinción entre ambas disciplinas.
\end{justification}

\begin{goals}
\item Comprender las características o notas constitutivas de la persona humana y sus manifestaciones a partir de una aproximación existencial que tenga en cuenta la experiencia personal de contingencia y de anhelo de trascendencia, que son comunes a todos los seres humanos.
\end{goals}

\begin{outcomes}
\ExpandOutcome{FH}{2}
\ExpandOutcome{HU}{3}
\ExpandOutcome{TASDSH}{3}
\end{outcomes}

\begin{unit}{¿Quién es el ser humano?}{GarciaCuadrado,JuanPabloA,Edith,Amengual,ReydeCastro,Lepp63}{6}{2}
\begin{topics}
 		\item ¿Qué es una \"antropología filosófica y teológica\"?
 		\item Armonía entre fe y razón.
 		\item La riqueza del aporte cristiano a la comprensión del hombre.
 		\item Primeras evidencias de nuestro ser personal: una aproximación existencial.
 		\item El hecho humano.
 		\item Nuestra experiencia de ser.
 		\item Breve recorrido histórico sobre los diversos intentos de respuesta a la pregunta ¿Quién es el ser humano?. 
\end{topics}

\begin{unitgoals}
	\item Presentar la Antropología Filosófica y Teológica desde una aproximación existencial que considera la distinción entre filosofía y teología, así como el vínculo entre ambas que nace en la armonía de la fe con la razón.
	\item Respuestas a la pregunta por el ser humano.
	\item alorar la originalidad de la respuesta cristiana.
\end{unitgoals}
\end{unit}

\begin{unit}{El hombre en la actualidad}{Lepp63,Rojas,Frankl,Unamuno,Figari04,Buber}{6}{2}
\begin{topics}
 		\item La situación del hombre hoy.
 		\item Mi elección.
 		\item La dimisión de lo humano.
 		\item Agnosticismo funcional.
\end{topics}

\begin{unitgoals}
	\item Analizar la situación del hombre en nuestro tiempo. Considerar en particular el proceso de dimisión de lo humano. 
\end{unitgoals}
\end{unit}

\begin{unit}{El ser humano es persona}{GarciaQuesada,Marcel,GarciaCuadrado,Melendo,StoTomas,Wojtyla,Wojtyla2,Lepp63,MorandeCFP}{15}{2}
\begin{topics}
 		\item ¿Quién soy? Necesidad de conocerme.
 		\item Deseo de conocer la verdad.
 		\item La experiencia de ser un misterio.
 		\item La experiencia existencial del yo.
 		\item Soy persona. Origen de la noción de \"persona\".
 		\item La noción de persona en el cristianismo: Boecio y Santo Tomás de Aquino.
 		\item Los filósofos personalistas.
 		\item Personalidad, mismidad e identidad.
 		\item Misión y dignidad.
 		\item Alteridad de la persona. Experiencia de la relación con el otro y de la comunión.
 		\item Antropocentrismo teologal.
\end{topics}

\begin{unitgoals}
	\item Mostrar que frente al nihilismo el ser humano es alguien, es persona y tiene un sentido.
\end{unitgoals}
\end{unit}

\begin{unit}{La persona: unidad bio-psico-espiritual}{Edith,GarciaCuadrado,Lepp63,Wojtyla2}{3}{2}
\begin{topics}
 		\item Soy cuerpo.
 		\item Soy cuerpo y alma.
 		\item Soy cuerpo, alma y espíritu.  Dinamismos de la mente (el conocimiento), del corazón (las emociones) y de la acción (la libertad).
 		\item Apertura a la trascendencia.
 		\item La fe como respuesta al ansia de trascendencia.
\end{topics}

\begin{unitgoals}
	\item Presentar al hombre como unidad biológica, psíquica y espiritual
\end{unitgoals}
\end{unit}

\begin{unit}{Los dinamismos de la persona}{Lepp63,ReydeCastro2}{6}{2}
\begin{topics}
 	 		\item Dinamismos fundamentales.
 	 		\item Necesidades.
 	 		\item El conocimiento, las emociones y la acción como actos de la persona.
 	 		\item Fe de mente
 	 		\item Fe de corazón
 	 		\item Fe de acción
\end{topics}

\begin{unitgoals}
	\item Comprensión ontológica del ser personal desde sus dinamismos fundamentales y otros dinamismos.
	\item Comprensión de la fe, enraizada en el ser personal

\end{unitgoals}
\end{unit}

\begin{unit}{La persona: ser en relación}{GarciaQuesada,GarciaCuadrado,Congregacion,Edith}{6}{2}
\begin{topics}
 	 	 		\item Relacionalidad, encuentro y comunión en la persona.
 	 	 		\item La persona como ser sexuado y como ser social.
 	 	 		\item La familia.
 	 	 		\item La sociedad.
\end{topics}

\begin{unitgoals}
	\item Comprensión de la persona desde su dimensión ontológica relacional con Dios, consigo mismo, con lo demás y con la Creación.
\end{unitgoals}
\end{unit}



\begin{coursebibliography}
\bibfile{GeneralEducation/FG101}
\end{coursebibliography}

\end{syllabus}
