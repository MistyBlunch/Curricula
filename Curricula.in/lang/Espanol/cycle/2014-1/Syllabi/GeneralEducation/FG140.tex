\begin{syllabus}

\course{FG104. Pensamiento Crítico}{Obligatorio}{FG140}

\begin{justification}
El aprender a aprender en la legislación universitaria, es una oportunidad 
abierta a todas las corrientes y formas del pensamiento expuestas de manera 
científica. La presente facilitación dirige su actividad a la formación integral 
del estudiante universitario para forjar en él principio de justicia social, 
identidad institucional, pluralismo, paz, afirmar la democracia, los 
derechos humanos y la solidaridad entre todos los estamentos del convivir 
gregoriano. Se trabajará planteamientos, técnicas y estrategias, que 
propicien el diálogo que difundan y fortalezcan los valores en la 
sociedad estudiantil, la formación integral y el desarrollo humano, 
paralelo a la formación profesional, técnica y científica, puesto 
que resulta incompatible con los principios señalados, la violencia, 
intolerancia y discriminación, por lo que la normatividad establecida 
en la vida jurídica de la institución adopta la aplicación de políticas 
y mecanismos específicos que promueven y garantizan el pensamiento crítico 
y la conciencia social en los miembros de la comunidad universitaria; 
generando una respetuosa relación humana y una plena realización 
profesional y personal.
\end{justification}

\begin{goals}
\item Lograr que los estudiantes interioricen los procesos de la convivencia universitaria
\item Apropiar de estabilidad y conciencia humana a través de principos y valores que requiere el estudiante
\item Lograr una concienciación de los procesos de evaluación que asumen docentes, estudiantes e institución universitaria
\end{goals}

\begin{outcomes}
\ExpandOutcome{e}{2}
\ExpandOutcome{f}{2}
\ExpandOutcome{HU}{3}
\end{outcomes}

\begin{unit}{Introducción a la vida universitaria}{Asamblea08}{10}{2}
   \begin{topics}
	\item Naturaleza de la educación superior
	\item Visión, Misión
	\item Fines, Objetivos
   \end{topics}

   \begin{unitgoals}
      \item Dar a conocer las generalidades de la educación superior
   \end{unitgoals}
\end{unit}

\begin{unit}{Estructura orgánica funcional de la Universidad}{Gregorio08}{10}{2}
   \begin{topics}
        \item Nivel Directivo
	\item Nivel Consultivo
	\item Nivel Asesor
	\item Nivel de Apoyo
	\item Nivel Operativo
  \end{topics}

   \begin{unitgoals}
      \item Posesionar a los estudiantes del orden estructural y funcional que posee la universidad
   \end{unitgoals}
\end{unit}

\begin{unit}{Hermeneútica jurídica universitaria}{Congreso00}{10}{2}
   \begin{topics}
        \item Ley de Educación Superior 
	\item Reglamento a la Ley
	\item Estatuto
	\item Reglamento de Régimen Académico
   \end{topics}

   \begin{unitgoals}
      \item Faciitar conocimientos de organización, funcionalidad y responsabilidad que asumen las universidades a través de la Ley de Educación Superior
      \item Reconocer las funciones y derechos que le asisten a universidades, docentes y estudiantes
   \end{unitgoals}
\end{unit}

\begin{learning-strategies}
\FGLearningStrategies
\end{learning-strategies}

\begin{evaluation}
\FGEvaluation
\end{evaluation}

\begin{didactical-resources}
Aulas totalmente equipadas con la tecnología suficiente:

\begin{inparaenum}[ \bf I:]
\item Pizarra.
\item Borrador.
\item Pilots.
\item Proyector de vídeo (vídeo-beam).
\item Computadora.
\item Laboratorio de cómputo
\item Acceso a internet por cable modem.
\end{inparaenum}

\end{didactical-resources}

% \begin{schedule}
% \end{schedule}

\begin{coursebibliography}
\bibfile{GeneralEducation/FG122}
\end{coursebibliography}

\end{syllabus}
