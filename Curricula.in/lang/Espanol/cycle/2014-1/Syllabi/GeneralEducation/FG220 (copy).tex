\begin{syllabus}

\course{HU220. Análisis de la Realidad Peruana}{Obligatorio}{FG220}

\begin{justification}
La formación integral del alumno implica una objetiva valoración histórica de la 
realidad nacional de modo que su accionar profesional esté integrado y articulado 
con la identidad de la nación peruana. De esta manera es posible formar profesionales 
comprometidos con la construcción de una sociedad más justa y solidaria.
\end{justification}

\begin{goals}
\item Analizar y comprender la situación actual del Perú desde una perspectiva 
histórica y sociológica, que permita a los alumnos identificarse con una visión 
reconciliada e integral de la realidad nacional, y de ese modo, asumir la 
responsabilidad que les compete en la construcción de una sociedad más justa.
\end{goals}

\begin{outcomes}
\ExpandOutcome{e}{3}
\ExpandOutcome{g}{4}
\end{outcomes}

\begin{unit}{Los Orígenes de la Peruanidad}{VAB1965,Porras1935}{9}{2}
\begin{topics}
      \item 
\end{topics}
\begin{unitgoals}
      \item Comprender el origen y las características de la cultura andina, principalmente incaica.
      \item Analizar su aporte como un antecedente histórico en la construcción de la nación peruana.
\end{unitgoals}
\end{unit}

\begin{unit}{La formación de la conciencia nacional peruana}{Messori96,Klinge1999,Figari1992,Figari2000,delaPuente1970}{9}{2}
\begin{topics}
      \item Conquista española. ?`Encuentro o choque de las culturas? Hacia una comprensión integral del fenómeno. Debate conceptual. Elaboración de matriz: cultura española.
      \item Visión Crítica del encuentro entre España y América:Diferencias entre la conquista del Norte y Sur de América.
      \item Virreinato. Repaso de aspectos socio-culturales más importantes. Surgimiento de la identidad nacional peruana al calor de la Fe Católica. Elaboración de matriz: cultura virreinal.
      \item La búsqueda de la justicia en la evangelización constituyente (Francisco de Vitoria y la Escuela de Salamanca).
      \item El proceso de la emancipación peruana. Hacia una compresión integral del fenómeno. Debate conceptual. Elaboración de matriz.
      \item El Perú del Siglo XIX: Aspectos destacados de la primera etapa de la República.
\end{topics}
\begin{unitgoals}
      \item Estudiar el proceso de formación de la conciencia nacional peruana a partir del encuentro de la cultura indígena con la cultura occidental traída por España. Se analiza el proceso de mestizaje, y el aporte de la fe, como ejes del nacimiento de la nación peruana.
\end{unitgoals}
\end{unit}

\begin{unit}{El Perú en la actualidad: la educación, la vida y la familia}{Mujica1992,Cardo2005}{9}{2}
\begin{topics}
      \item La Educación Peruana: Crisis y posibilidades.
      \item Rol de la Educación en el desarrollo del Perú.
      \item La Vida y la Familia en la política y la legislación peruana.
      \item Estrategias del movimiento anti-vida.
\end{topics}
\begin{unitgoals}
      \item Conocer y analizar la situación de la educación en el Perú, como problema y alternativa de solución.
\end{unitgoals}
\end{unit}

\begin{unit}{El Perú en la actualidad: la informalidad}{Cardo2005,deSoto1987}{9}{2}
\begin{topics}
      \item Analizar la presencia y acción de los actores sociales actuales.
      \item Formas, causas y consecuencias de la Informalidad en el Perú.
\end{topics}

\begin{unitgoals}
      \item Conocer y evaluar un hecho social de especial importancia en el Perú actual: el fenómeno de la informalidad y sus implicancias en el desarrollo nacional.
\end{unitgoals}
\end{unit}

\begin{unit}{El Perú en la actualidad: Terrorismo y pacificación}{ADDCOT2000a,ADDCOT2000b}{9}{2}
\begin{topics}
      \item Características fundamentales de las décadas de los '80 y '90.
      \item El surgimiento de los grupos terroristas.
      \item Sendero Luminoso: su ideología, estrategia, y consecuencias de su acción violenta.
      \item La acción del estado frente a la guerra interna.
      \item La Comisión de la Verdad y la Reconciliación.
\end{topics}
\begin{unitgoals}
      \item Comprender las características de las décadas de los '80 y '90 que permitieron el surgimiento de los grupos terroristas.
      \item Profundizar en la ideología y el accionar de estos movimientos y la reacción del Estado frente a la violencia interna.
      \item Analizar la lectura que sobre este proceso hizo la CVR.
\end{unitgoals}
\end{unit}

\begin{coursebibliography}
\bibfile{GeneralEducation/FG220}
\end{coursebibliography}

\end{syllabus}

%\end{document}
