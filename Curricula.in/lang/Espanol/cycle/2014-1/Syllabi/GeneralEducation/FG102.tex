\begin{syllabus}

\course{FG102. Metodología del Estudio}{Obligatorio}{FG102}

\begin{justification}
El curso tiene su fundamentación en la necesidad de hacer que los estudiantes respondan a la exigencia académica de la Universidad para ser exitosos en el logro de sus objetivos. Ese  éxito debe ser consecuencia de un desempeño definidamente intencionado, de la asimilación de su responsabilidad y de la comprensión de los procesos intelectuales que realiza. 
Los alumnos en formación profesional necesitan mejorar su actitud frente al trabajo y exigencia académicos, entendida como el camino para ser mejor y alcanzar logros positivos. Además conviene que entiendan el proceso mental que se da en el ejercicio del estudio para lograr el aprendizaje; así  sabrán dónde y cómo hacer los ajustes más convenientes a sus necesidades. Asimismo, requieren dominar variadas formas de estudiar, para que puedan seleccionar las estrategias  más convenientes a su personal estilo de aprender y a la naturaleza de cada asignatura. De ese modo podrán  aplicarlos a su trabajo universitario, haciendo exitoso su esfuerzo.
Metodología del estudio es un curso de formación teórico-práctico cuyo propósito es  ayudar a los alumnos a  tomar consciencia de su rol como estudiantes, potenciar fortalezas que favorezcan la adaptación a la realidad universitaria, fortalecer la disposición y actitud para el trabajo académico, conocer los procesos mentales que comportan el aprendizaje  y ejercitarse en el dominio de recursos y técnicas de estudio que les permitan formular su propio método de trabajo académico para un exitoso desempeño en las demás asignaturas. El curso de metodología del estudio, por tanto, tiene un carácter instrumental que proporciona conceptos, promueve un cambio de actitud y favorece el dominio de técnicas para el  trabajo académico

\end{justification}

\begin{goals}
\item Demostrar una actitud frente al trabajo académico que lo lleve a interesarse por la comprensión del proceso de aprendizaje.
\item Comprender y potenciar las fortalezas que requiere un estudiante para  un mejor desempeño  y ejercitarse para dominar el uso de  técnicas de estudio que lo lleven a formular su propio método para el trabajo académico, optimizando su rendimiento con menor desgaste y mayor eficiencia.
\end{goals}

\begin{outcomes}
\ExpandOutcome{f}{3}
\ExpandOutcome{h}{3}
\ExpandOutcome{FH}{2}
\end{outcomes}

\begin{unit}{}{CHAVEZA,CARRAZCO,GARAYCOCHEA,Pauk,Pimienta,Bernedo,Tapia}{8}{3}
\begin{topics}
        \item La exigencia en el trabajo universitario. Objetivos del curso.
        \item La postura del estudiante ante el reto del trabajo universitario. Problemas  del universitario.     
        \item Las  habilidades intelectuales que requiere el universitario
        \item Organización personal y de los recursos. Plan de mejora personal.
        \item Técnicas de estudio. Primera parte: Ideas principales de textos y subrayado.       
\end{topics}
\begin{unitgoals}
        \item Comprender el  significado de estar en  la universidad, de la exigencia que representa, de los problemas del universitario que se inicia, de las habilidades que debe desarrollar como tal; para fortalecer una actitud comprometida con su ser estudiante.
        \item Conocer y aplicar los criterios para el aparato crítico, para el tratamiento de la información académica.
\end{unitgoals}
\end{unit}

\begin{unit}{}{CHAVEZA,CARRAZCO,GARAYCOCHEA,Pauk,Pimienta,Bernedo}{8}{2}
\begin{topics}
        \item Conocimiento de uno mismo: Auto concepto y autoestima.
        \item La voluntad: importancia y fortalecimiento.
        \item Las conductas personales: Conductas pasiva, agresiva y asertiva.
        \item Capacidad para superar el fracaso: La resiliencia.
        \item La inteligencia emocional.
        \item La mente, la memoria  y la atención.
        \item La Inteligencia y las inteligencias múltiples. 
        \item Técnicas de  estudio. El resumen, las notas al margen.       
\end{topics}
\begin{unitgoals}
        \item Comprender la necesidad de desarrollar  fortalezas y disposiciones que para el desempeño académico y relacional, en la universidad.
        \item Comprender los procesos mentales que se dan en el aprendizaje para evaluar y monitorear el propio proceso de estudiar para aprender.
\end{unitgoals}
\end{unit}

\begin{unit}{}{CHAVEZA,CARRAZCO,GARAYCOCHEA,Flores,Buzan,Pauk,Pimienta,Bernedo}{8}{2}
\begin{topics}
        \item El método de estudio, como conjunto de estrategias y uso de herramientas o técnicas que favorecen el aprendizaje.
        \item La lectura como herramienta principal para el aprendizaje.
        \item Tipos de lectura. La lectura académica.
        \item Etapas de la lectura: Sensorial, perceptiva, extrapolativa.
        \item El análisis en la lectura.
        \item Meza de Vernet y su Análisis de las partes, de la estructura, de las funciones de las relaciones.
        \item Técnicas de estudio. Los mapas conceptuales. Los esquemas.
\end{topics}
\begin{unitgoals}
        \item Comprender los procesos para el aprendizaje y la relación entre Conocimiento y Aprendizaje, la  lectura analítica como medio principal para aprender, las leyes del aprendizaje, los pasos o fases del aprendizaje. Optar por el aprendizaje  significativo.
        \item Explicar conceptos y ejemplificar situaciones  para el ejercicio de en la capacidad para expresar el conocimiento, con lenguaje adecuado, pertinente de forma oral y escrita.
\end{unitgoals}
\end{unit}

\begin{unit}{}{CHAVEZA,CARRAZCO,GARAYCOCHEA,Buzan,Pauk,Pimienta,Bernedo}{7}{3}
\begin{topics}
        \item Los estilos de aprendizaje. El estilo personal.
        \item Técnicas de estudio. Los mapas mentales para la toma de apuntes.
        \item Técnicas de estudio: Los mapas mentales para la exposición.
        \item Los grupos de estudio, como estrategia.
        \item Los exámenes y maneras de afrontarlos.
        \item Las técnicas de relajación.
\end{topics}
\begin{unitgoals}
        \item Elegir y usar un método de estudio como un conjunto de estrategias, organización  y técnicas de estudio libremente implementado, como consecuencia  de la comprensión del estilo personal de aprendizaje.
        \item Usar lenguaje inteligible y adecuado para expresar el conocimiento, con lenguaje adecuado, pertinente de forma oral y escrita.
\end{unitgoals}
\end{unit}



\begin{coursebibliography}
\bibfile{GeneralEducation/FG101}
\end{coursebibliography}

\end{syllabus}
