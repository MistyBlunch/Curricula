\begin{syllabus}

\course{FG120. Constitución y Realidad Nacional}{Obligatorio}{FG120}

\begin{justification}
La naturaleza de la asignatura radica en conocer los aspectos económicos políticos y socio cultural de nuestra realidad nacional y al mismo tiempo brindar información de los acontecimientos mas resaltantes a lo largo de la historia peruana. El contenido de la asignatura se estructura de la siguiente manera: Aspectos Generales. El Estado. La Población. El Perú y su Realidad Histórico-Política. La Problemática Social. La Problemática Educativa.
\end{justification}

\begin{goals}
\item Que el alumno entienda el contexto nacional sobre el cual tendra efecto su ejercicio profesional.
\item Que el alumno entienda el contexto legal existente sobre el cual ejercerá su profesión.
\end{goals}

\begin{outcomes}
\ExpandOutcome{HU}{2}
\end{outcomes}

\begin{unit}{Aspectos Generales}{Quijano92}{12}{2}
\begin{topics}
	\item Análisis Coyuntural
  	\item La Realidad Social
  	\item La Realidad Económica
  	\item La Realidad Política y Geográfica
\end{topics}

\begin{unitgoals}
      \item  Adquirir información básica acerca de nuestro pasado histórico para una reflexión analítica de nuestra realidad nacional.
   \end{unitgoals}
\end{unit}

\begin{unit}{El Estado}{Quijano92,Kapsoli93}{12}{2}
\begin{topics}
	\item El Estado
	\item Funciones del Estado Estado y Gobierno
	\item La Ciudadanía
	\item Deberes y Derechos del ciudadano
\end{topics}

\begin{unitgoals}
      \item Describir los diferentes aspectos de la problemática nacional.
      \item Desarrollan una serie de actividades dinámicas para una mejor comprensión de la realidad nacional.
   \end{unitgoals}
\end{unit}

\begin{unit}{La Población}{Marticona93,Mariategui91}{12}{2}
\begin{topics}
	\item La población en el Perú
	\item Distribución espacial de la población Migraciones
	\item Realidad indígena peruana
	\item La población en la actividad económica
\end{topics}

\begin{unitgoals}
      \item Conocer como esta ubicada la poblacion y cual es la actividad economica.
   \end{unitgoals}
\end{unit}

\begin{unit}{El Peru y su realidad historico política}{Kapsoli93}{6}{2}
\begin{topics}
	\item La República y sus coyunturas gobiernistas.
	\item El Primer Congreso Constituyente
	\item La Reconstrucción Nacional
	\item El Tercer Militarismo
\end{topics}

\begin{unitgoals}
      \item Tener conocimiento del Peru y su realidad historica
\end{unitgoals}
\end{unit}

\begin{unit}{Aspestoc Sociales}{Kapsoli93}{6}{2}
\begin{topics}
	\item Identidad
	\item La Nacionalidad
	\item La política social del Perú
	\item Entre la democracia y la dictadura
	\item Gobierno Revolucionario
	\item Gobierno Democrático
\end{topics}

\begin{unitgoals}
      \item Consolidar los conocimientos de la politica social del Perú
\end{unitgoals}
\end{unit}



\begin{coursebibliography}
\bibfile{GeneralEducation/FG120}
\end{coursebibliography}

\end{syllabus}
