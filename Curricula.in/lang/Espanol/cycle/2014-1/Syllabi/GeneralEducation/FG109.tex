\begin{syllabus}

\course{FG109. Realidad Nacional}{Obligatorio}{FG109}

\begin{justification}
Está encaminada a conocer profundamente los elementos que constituyen la realidad de 
nuestro entorno para concientizar a los estudiantes de la situación nacional, 
para dar el mayor número de elementos de juicio para que el estudiante aprenda a tomar 
posiciones en las mejores condiciones de racionalidad, libertad y sociabilidad.
\end{justification}

\begin{goals}
\item Aplicar técnicas dinámicas grupales que propician la integración de los participantes, rompiendo estructuras tradicionales en el aprendizaje.
\item Desarrollar en el estudiante la capacidad analítica y crítica sobre la realidad nacional.
\item Incentivar al estudiante sobre la importancia de conocer nuestra realidad.
\end{goals}

\begin{outcomes}
\ExpandOutcome{FH}{2}
\end{outcomes}

\begin{unit}{Un país pluriétnico y pluricultural}{Almeida94}{10}{2}
   \begin{topics}
	\item Los pueblos indígenas
	\item Poblaciones y poblamientos
	\item Los espacios etnoculturales
   \end{topics}

   \begin{unitgoals}
      \item Conocer la diversidad étnica y cultural de nuestro país
   \end{unitgoals}
\end{unit}

\begin{unit}{Los organismos del Estado}{Huerta99}{15}{2}
   \begin{topics}
      \item Funciones
	\item Atribuciones del presidente de la república
	\item Órganos de la función judicial
	\item Los organismos de control
  \end{topics}

   \begin{unitgoals}
      \item Analizar las funciones de cada una de las instancias gubernmentales
   \end{unitgoals}
\end{unit}

\begin{unit}{La realidad socioeconómica del Ecuador}{Acosta95}{10}{2}
   \begin{topics}
      \item Los espacios de bienestar y pobreza
	\item Desarrollo y subdesarrollo
	\item Las diferentes interpretaciones y propuestas de solución a los problemas socioeconómicos
	\item El contexto mundial de la economía
   \end{topics}

   \begin{unitgoals}
      \item Enfrentar los problemas socioeconómicos derivados de nuestro entorno con la situación mundial
   \end{unitgoals}
\end{unit}

\begin{unit}{Las diferentes perspectivas}{Vicuña07}{10}{2}
   \begin{topics}
      \item Los diferentes tipos de actores económicos
	\item Los sectores económicos
	\item La producción para la exportación y para el consumo interno
	\item La economía del petróleo
	\item Las economías modernas y tradicionales
	\item Hacia un proyecto nacional
	\item Regiones, autonomía, centralismo, bi-descentralización
	\item Cifras del Ecuador
	\item Emigraciones/Inmigraciones
	\item Objetivos del Milenio: inclusión social 
   \end{topics}

   \begin{unitgoals}
      \item Diseñar un plan de perspectivas encaminadas a conocer posibles soluciones a los problemas de nuestro país
   \end{unitgoals}
\end{unit}

\begin{coursebibliography}
\bibfile{GeneralEducation/FG109}
\end{coursebibliography}
\end{syllabus}
