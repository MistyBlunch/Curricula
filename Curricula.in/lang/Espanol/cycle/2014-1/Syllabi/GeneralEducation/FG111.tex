\begin{syllabus}

\course{FG111. Técnicas de investigación}{Obligatorio}{FG111}

\begin{justification}
La comprensión cabal del proceso que debe seguir las Técnicas de
investigación,  tratando de que los estudiantes, apliquen y lleven 
sus conocimientos teóricos a la  práctica, mediante la elaboración 
de un mini proyecto,  utilizando como temática  lo que acontece en 
el país,  a partir de la fijación de lineamientos  teóricos y 
prácticos desarrollados en el salón de clase.
\end{justification}

\begin{goals}
\item Contribuir al aprendizaje del estudiante, respecto al proceso que debe seguir la metodología investigativa a partir del estudio de las diferentes teorías, métodos, técnicas y procedimientos que puedan aplicarse a un determinado proyecto de Investigación 
\item Lograr, que el estudiante participe activamente  en cada una de las actividades a desarrollarse dentro del aula  y fuera de la misma, para el cumplimiento de trabajos de investigación o laboratorio
\item Incentivar al estudiante sobre la  importancia de vincular la teoría con la práctica, a fin de que se logre una mejor y mayor comprensión de los temas tratados
\item Utilizar el recurso bibliográfico suministrado, como auxiliar didáctico dinámico de discusiones conversacionales,  para el desarrollo del presente evento
\end{goals}

\begin{outcomes}
\ExpandOutcome{f}{4}
\ExpandOutcome{h}{3}
\ExpandOutcome{l}{3}
\end{outcomes}

\begin{unit}{Conocimiento e Investigación}{Aguilar92}{5}{2}
   \begin{topics}
      \item Definición origen, elementos y formas del conocimiento
	\item Conocimiento científico
	\item Definición de ciencia
	\item Clasificación de la ciencia.
	\item Investigación científica
	\item Definición
	\item Características
	\item Tipos de investigación
   \end{topics}

   \begin{unitgoals}
      \item Determinar la importancia de la investigación
   \end{unitgoals}
\end{unit}

\begin{unit}{Diferencia entre Método y Técnica}{Gutierrez88}{5}{2}
   \begin{topics}
      \item Diferencia entre técnica y Método.
	\item Definición del método de investigación
	\item Definición de técnica
	\item Características
	\item Proceso
	\item Cuadro comparativo entre técnica y método.
   \end{topics}

   \begin{unitgoals}
      \item Establecer claras diferencias entre método y técnica
   \end{unitgoals}
\end{unit}

\begin{unit}{Las técnicas de Investigación}{Arias99}{5}{3}
   \begin{topics}
      \item Técnicas de Investigación
	\item Definición e importancia de las técnicas de investigación
	\item Información primaria y secundaria
	\item Clasificación de las técnicas
	\item Observación
	\item Definición
	\item Importancia
	\item Tipos
	\item Proceso
	\item Instrumentos para la recolección 
	\item Ventajas y limitaciones de la observación
	\item Entrevista
	\item Características e importancia
	\item Tipos de entrevista
	\item Estructura de la entrevista
	\item Ventajas y limitaciones
	\item Encuesta
	\item Características e importancia
	\item Instrumentos
	\item Ventajas y limitaciones
	\item Muestreo
	\item Características e importancia
	\item Tipos de muestreo
   \end{topics}

   \begin{unitgoals}
      \item Conocer y manejar cada una de las técnicas necesarias en el proceso de investigación
   \end{unitgoals}
\end{unit}

\begin{unit}{El proceso de investigación}{Pardinas94}{10}{4}
   \begin{topics}
      \item Proceso de Investigación
	\item Concepto de investigación
	\item El anteproyecto de Investigación 
	\item Elementos básicos del proyecto de investigación
	\item Descripción y análisis de los elementos del proyecto.
	\item Marco referencial
	\item Marco Teórico
	\item Marco Metodológico
	\item Marco Administrativo
	\item Bibliografía 
	\item Anexos
   \end{topics}

   \begin{unitgoals}
      \item Diseñar un proyecto de investigación, tomando los elementos del proceso estudiado.
   \end{unitgoals}
\end{unit}

\begin{coursebibliography}
\bibfile{GeneralEducation/FG111}
\end{coursebibliography}
\end{syllabus}
