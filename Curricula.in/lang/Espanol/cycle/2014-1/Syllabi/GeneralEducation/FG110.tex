\begin{syllabus}

\course{FG110. Economía Digital}{Obligatorio}{FG110}

\begin{justification}
Es un espacio de estudios que facilita conocimientos técnicos, estratégicos y de proyección en el área de la computación. Los temas en estudio permitirán obtener conocimientos y al mismo tiempo facilitar habilidades y competencias que serán indispensables como medio del desarrollo de la información y comunicación. El mercado se fortalecerá con los avances del sistema computacional de tal manera que los conocimientos otorgados en el tópico de Economía Digital permitirá reconocer las nuevas tendencias de la economía, y logrará el perfeccionamiento directo o indirecto de los servicios que oferta a través de los medios digitales. Esta es probablemente uno de los mejores espacios del sistema económico moderno, podrá manipular la información a gusto y necesidad del ofertante como del demandante, ya que el área de conocimiento computacional agilitará los procesos de compra y venta, al mismo tiempo acelerará la información para tomar las decisiones oportunas y válidas. El sistema digital facilitará los beneficios que las empresas o el gobierno tiene como política de dearrollo económico a través del sistema computacional.
\end{justification}

\begin{goals}
\item Reconocer los avances tecnológicos y las nuevas formas de desarrollo empresarial con la participación del talento humano calificado.
\item Desarrollar en los estudiantes nuevos sistemas virtuales y digitales como medios de crecimiento productivo
\item Lograr suficiente comprensión de los aportes digitales, a los servicios que oferta los campos de producción
\item Desarrollar habilidades y capacidades en el sistema virtual que afiance la confianza de la economía electro-autónoma
\end{goals}

\begin{outcomes}
\ExpandOutcome{g}{3}
\ExpandOutcome{HU}{3}
\end{outcomes}

\begin{unit}{El accionar empresarial del siglo XXI}{Castells01}{8}{2}
   \begin{topics}
      \item Nuevas organizaciones del siglo XXI
	\item De las ideas a las acciones
	\item ecommerce, no es solamente tecnología: es una nueva forma de crear valor
	\item Como lanzar un negocio en la economía digital
	\item La evolución del presente al futuro
	\item Los recursos humanos, nuevos comportamientos y nuevos conocimientos
	\item Retener el saber en la economía digital
   \end{topics}

   \begin{unitgoals}
      \item Interiorizar conocimientos de organización y evolución que posee la economía digital en el siglo XXI
   \end{unitgoals}
\end{unit}

\begin{unit}{La nueva economía desde una perspectiva macroeconómica}{Barea00}{10}{2}
   \begin{topics}
      \item El crecimiento de la productividad
	\item Nuevas características para una nueva economía
	\item Una economía del conocimiento
	\item Una economía de la información
	\item Una economía virtual y digital
	\item Una economía que favorece la convergencia de tecnologías
	\item El mercado electrónico
	\item Los modelos de comercio en el mercado electrónico
	\item Cambios en las estructuras de las empresas o industrias y en las ventajas competitivas
  \end{topics}

   \begin{unitgoals}
      \item Posesionar la información de las nuevas tecnologías que aporten al crecimiento de la economía apoyados por los paradigmas macroeconómicos
   \end{unitgoals}
\end{unit}

\begin{unit}{Bienes y servicios de la nueva economía}{Tapscott01}{10}{3}
   \begin{topics}
      \item Servicios a la carta, incubados en la Internet
	\item Servicios educativos
	\item Servicios de comunicación y mercadeo
	\item Servicios de intermediación financiera
	\item Bienes ofrecidos e intercambios a través de la red
	\item Empleos, servicios básicos, servicios de desarrollo tecnológico y medicina
   \end{topics}

   \begin{unitgoals}
      \item Determinar elementos claros en las transacciones que transmite información como medio de principio y seguridad
   \end{unitgoals}
\end{unit}

\begin{unit}{El destino de las firmas}{Alvarez01}{10}{2}
   \begin{topics}
      \item El surgimiento de la economía electro-autónoma
	\item La verdadera fábrica virtual
	\item El desarrollo de productos en tiempo de Internet
	\item La confianza y la organización virtual
	\item Depredadores y presas: Una nueva ecología de la competencia
   \end{topics}

   \begin{unitgoals}
      \item Valorar la oportuna y eficiente utilización de las nuevas tecnologías como medios de competitividad en los negocios
   \end{unitgoals}
\end{unit}

\begin{unit}{Negocios y gobiernos en la red}{Velasco02}{10}{3}
   \begin{topics}
      \item Entorno legal, gobierno y control
	\item Negocios, servicios y riqueza, análisis
	\item Cuantificación de la riqueza generada por la red
	\item Ventajas y desventajas de la economía virtual
   \end{topics}

   \begin{unitgoals}
      \item Adquirir conocimientos de la estructura que desarrolla los sistemas de la economía virtual
   \end{unitgoals}
\end{unit}

\begin{coursebibliography}
\bibfile{GeneralEducation/FG110}
\end{coursebibliography}
\end{syllabus}
