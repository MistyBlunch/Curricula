\begin{syllabus}

\course{FG104. Introducción a la Filosofía}{Obligatorio}{FG104}

\begin{justification}
Todos los hombres por naturaleza quieren conocer  (Aristóteles, Metafísica). La riquísima experiencia del mundo propia del hombre no tiene instintos proporcionados que la dirijan. Es un ser radicalmente abierto: el ser de todo lo que existe es su  hábitat  natural (Platón). Necesita llegar al  en sí al ser de las cosas para sobrevivir y florecer. Por lo mismo, naturalmente se pregunta sobre la realidad a todo nivel y crece según su vastísimo potencial en la medida que enriquece su relación con la realidad, hasta llegar a su Fundamento. El curso versa sobre la dimensión de principios que, con el rigor propio de su nivel, dan razón de todas las particularidades.
\end{justification}

\begin{goals}
\item Lo propio de ingeniero es brindar soluciones de alta fiabilidad a problemas prácticos. Pero todas ellas suponen una buena comprensión de los principios que gobiernan esas soluciones. Este curso se extiende lógicamente a la exploración de los principios que gobiernan y dan razón de la realidad natural y humana.
\end{goals}

\begin{outcomes}
\ExpandOutcome{FH}{2}
\end{outcomes}

\begin{unit}{Introducción, Nociones, División de la Filosofía}{Melendo,Pieper,ReydeCastro2010}{15}{2}
\begin{topics}
	\item Presentación del curso: visión global y motivaciones para el curso.
	\item Introducción: Buscar y escuchar, La filosofía como respuesta, Exigencias para conocer la realidad, ?`Qué quiero?, El asombro como punto de partida de la filosofía.
	\item Noción de filosofía:  Filosofía , La filosofía abierta al ser, La tarea de la filosofía.
	\item División de la filosofía y distinción con otras ciencias: División de la filosofía, La filosofía y la fe, La filosofía y la ciencia,  La filosofía cristiana.
\end{topics}
\begin{unitgoals}
	\item Introducir al alumno a la naturaleza de la filosofía y a su importancia para la vida.
	\item Introducir las nociones esenciales de la filosofía.
	\item Que el alumno comprenda el alcance de la filosofía y su diferenciación con otras disciplinas.
\end{unitgoals}
\end{unit}

\begin{unit}{Antropología, Gnoseología, Ética}{ReydeCastro2010}{15}{2}
\begin{topics}
	\item Introducción a la Antropología: ?`Quién soy?, Persona, un ser para el encuentro, Persona libre, Persona que conoce, La persona y sus dinamismos fundamentales, La dignidad humana.
	\item Introducción a la Gnoseología: ?`Cómo conocemos la realidad?, Frente a la realidad cognoscible y misteriosa, Naturaleza y alcance del conocimiento.
	\item Introducción a la Ética: ?`Qué es bueno hacer?, Ética y moral, Fundamentos de la ética, La ley moral (La ley en general, La ley eterna o sobrenatural, La ley natural, Ley positiva), Acto del hombre y acto humano, El abandono de la metafísica y sus consecuencias para la ética.
\end{topics}

\begin{unitgoals}
	\item Introducir al alumno a las nociones y principios fundamentales de la Antropología Filosófica.
	\item Introducir al alumno a las nociones y principios fundamentales de una Teoría del Conocimiento general y científico que de hecho de razón de las acciones humanas de actuar en base a ese conocimiento, es decir, un teoría realista moderada
	\item Introducir al alumno a las nociones y principios fundamentales de la Ética como reflexión sobre la libertad y felicidad humana en relación con la realidad.
\end{unitgoals}
\end{unit}

\begin{unit}{Metafísica}{ReydeCastro2010,Pieper,JuanPabloA,Chavez2003HistoriaDoctrinas}{15}{2}
\begin{topics}
	\item Metafísica, parte I  Nociones fundamentales y la intuición del ser: El misterio como constitutivo de la realidad, El misterio del ser, Metafísica o la filosofía como pregunta por el ser, Ens y esse, Descripción de la noción de ente, Descripción de la noción de ser, La analogía.
	\item Metafísica, parte II  Ser y devenir, La clave de acto y potencia: Significado de la palabra  acto , Significado de la palabra  potencia , Clases de acto y potencia, Distinciones para una mejor comprensión, Potencia pasiva y acto primero, Potencia activa y acto segundo, Aporte de Santo Tomás en cuanto a la naturaleza del acto, Relación entre potencia y acto, La prioridad del acto, Aplicaciones del principio constitutivo de los entes.
	\item Metafísica, parte III  Sustancia y accidentes: La sustancia, Los accidentes, El ser como acto propio de la sustancia (El ser de la sustancia y de los accidentes, La sustancia como ente en sentido propio, Los accidentes como actualidad de la sustancia, Sustancia y accidentes; acto y potencia).
	\item Metafísica, parte IV  Ser y esencia: ?`Qué es la esencia? Polisemia de la palabra  ser , ?`Ser o existencia? Distinciones entre ser y esencia, La materia y la forma sustancial: constitución metafísica de las sustancias corpóreas, Las propiedades trascendentales (Unidad, Verdad, Bondad, Belleza)
	\item Metafísica, parte V: Causalidad: Causalidad (causa material, causa formal, causa eficiente, causa final), La causa eficiente, La causa final.
\end{topics}

\begin{unitgoals}
	\item Introducir al alumno a las nociones y principios fundamentales de la Metafísica como dimensión inherente a todo cuestionamiento humano sobre la realidad.
\end{unitgoals}
\end{unit}



\begin{coursebibliography}
\bibfile{GeneralEducation/FG101}
\end{coursebibliography}

\end{syllabus}
