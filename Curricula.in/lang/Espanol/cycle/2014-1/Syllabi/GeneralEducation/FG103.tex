\begin{syllabus}

\course{FG103. Introducción a la Vida Universitaria}{Obligatorio}{FG103}

\begin{justification}
El ingreso a la universidad es un momento de nuevos desafíos y decisiones en la vida de una persona. En ese sentido, la Universidad Católica San Pablo busca, mediante el presente espacio, escuchar y acoger al joven ingresante con sus inquietudes y anhelos personales, presentar la identidad y misión de la universidad como su "alma mater", señalando los principales desafíos que el futuro profesional enfrentará en el mundo actual  y orientando a nuestros jóvenes estudiantes, a través de diversos principios, medios y otros recursos, con el fin de que puedan formarse integralmente y desplegarse plenamente en la fascinante aventura de la vida universitaria.  Su realización como buen profesional depende de una buena formación personal y cultural que le brinde horizontes amplios, que sustenten y proyecten su conocimiento y quehacer técnicos e intelectuales y que le permitan contribuir siendo agentes de cambio cultural y social.
\end{justification}

\begin{goals}
\item Que el alumno canalice sus inquietudes y anhelos a través del encuentro y descubrimiento de sí mismo, que le brinden espacios de análisis y reflexión personales para asumir posturas bien fundamentadas hacia los valores e ideales de su entorno. Mediante su inserción en la vida universitaria, logrará una disposición de apertura a su propio mundo interior y a su misión en el mundo, cuestionando su cosmovisión y a sí mismo para obtener un conocimiento y crecimiento personales que permitan su despliegue integral y profesional. 
\end{goals}

\begin{outcomes}
\ExpandOutcome{FH}{3}
\end{outcomes}

\begin{unit}{El Universitario}{Sanz,Guardini,Rilke,Marias,Frankl,Fromm}{36}{3}
\begin{topics}
	\item Introducción al curso: presentación y dinámicas.
	\item Sentido de la Vida, búsqueda de propósito y vocación profesional.
	\item Obstáculos para el autoconocimiento: el ruido, la falta de comunicación, la mentira existencial, máscaras.
	\item Ofertas Intramundanas: Hedonismo, Relativismo, Consumismo, Individualismo, Inmanentismo
	\item Las consecuencias: la falta de interioridad, masificación y el desarraigo, soledad
	\item Encontrando el sentido:¿Quién soy? Conocimiento Personal y Ventana de Johari, Análisis del Amor y la Amistad, La libertad como elemento fundamental en las elecciones personales: la experiencia del mal, Aceptación y Reconciliación personales.
\end{topics}
\begin{unitgoals}
	\item Identificar y caracterizar la propia cosmovisión y los criterios personales predominantes en sí mismos acerca del propósito y sentido de la vida y la felicidad.
	\item Crear un vínculo de confianza con el docente del curso para lograr apertura a nuevas perspectivas.
	\item Descubrir el propio proyecto de vida que responda a sus anhelos y búsqueda de propósito.
\end{unitgoals}
\end{unit}

\begin{unit}{La Universidad}{JuanPablo,Guardini2,Maldonado,Identidad}{12}{2}
\begin{topics}
	\item Origen y propósito de la Universidad: breve reseña histórica.
	\item La identidad católica de la UCSP.
	\item La búsqueda de la Verdad: la formación intelectual.
	\item La formación integral: los hábitos y virtudes de un universitario
	\item Sentido de las normas y prácticas de la UCSP: vestimenta, actitudes, etc.
\end{topics}
\begin{unitgoals}
	\item Identificar y dominar los horizontes y la dinámica de la vida universitaria, particularmente,  los horizontes específicos de la Universidad Católica San Pablo.
	\item Descubrir que su tiempo de estancia en la institución constituye un proceso de verdadera formación integral de su persona, intelectual, académica y profesional, y, que, puede desde la Universidad, ofrecer un aporte consistente y responsable ante los desafíos del mundo actual.   
\end{unitgoals}
\end{unit}



\begin{coursebibliography}
\bibfile{GeneralEducation/FG101}
\end{coursebibliography}

\end{syllabus}
