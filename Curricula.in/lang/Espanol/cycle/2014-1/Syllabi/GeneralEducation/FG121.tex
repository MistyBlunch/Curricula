\begin{syllabus}

\curso{HU121. Econom�a}{Obligatorio}{HU121}

\begin{justification}
La sociedad y sus componentes interact�an en un sistema econ�mico que se caracteriza por ser de libre mercado, el mismo que ha demostrado ser el m�s eficiente en cuanto a la asignaci�n de los recursos e incentivos para la inversi�n por parte de las empresas privadas. En tal sentido se hace necesario que los profesionales estudien y conozcan los principios fundamentales inherentes al funcionamiento de los mercados de tal manera de desarrollar una capacidad anal�tica que permita entender los fen�menos que se presentan diariamente dentro de su entorno. Asimismo el estudio de la econom�a permitir�a entender la pol�tica econ�mica adoptada por los diferentes gobiernos a lo largo de nuestra historia econ�mica de tal manera de poder evaluarlas sobre la base de los resultados obtenidos.
\end{justification}

\begin{goals}
\item \OutcomeHU
\end{goals}

\begin{outcomes}
\ExpandOutcome{HU}
\end{outcomes}

\begin{unit}{Introduccion}{Case93}{16}
\begin{topics}
	\item Conceptos generales
	\item Metodolog�a de la ciencia econ�mica
	\item Descripci�n general de la micro y macroeconom�a
\end{topics}

\begin{unitgoals}
      \item Desarrollar la capacidad de analizar situaciones propias de la gesti�n con un enfoque econ�mico y de eficiencia
   \end{unitgoals}
\end{unit}

\begin{unit}{La teoria de la demanda}{Fernandez99}{16}
\begin{topics}
	\item  Utilidad, presupuesto y consumo �ptimo del consumidor.
	\item Bienes normales, inferiores, sustitutos y complementarios.
 	\item Caso: Indice de Precios del Consumidor
\end{topics}

\begin{unitgoals}
      \item Aprender a utilizar las herramientas b�sicas de la ciencia econ�mica en las diferentes situaciones profesionales que ser�n afrontadas por el alumno
   \end{unitgoals}
\end{unit}


\begin{unit}{Teor�a de la producci�n y de los costos econ�micos de corto plazo}{Blanchard00}{16}
\begin{topics}
	\item Funci�n de producci�n de corto plazo.
	\item Rendimientos marginales decrecientes.
	\item Costos de corto plazo
	\item Relaci�n entre la producci�n y los costosConceptos generales
\end{topics}

\begin{unitgoals}
      \item Desarrollar la capacidad de entendimiento de las pol�ticas econ�micas que normalmente adoptan los gobiernos de turno
   \end{unitgoals}
\end{unit}

\begin{coursebibliography}
\bibfile{GeneralEducation/FG121}
\end{coursebibliography}
\end{syllabus}





