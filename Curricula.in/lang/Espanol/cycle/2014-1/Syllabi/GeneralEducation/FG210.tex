\begin{syllabus}

\course{FG210. Ética}{Obligatorio}{FG210}

\begin{justification}
Brindar al alumno criterios de discernimiento general y particular, así como pautas morales para que con ellos oriente su conducta personal, de modo que se oriente a su realización integral mediante actos queridos, conscientes, libres y responsables. 
\end{justification}

\begin{goals}
\item Formar la conciencia del estudiante para que pueda conducirse moralmente en el ámbito personal y profesional.
\end{goals}

\begin{outcomes}
\ExpandOutcome{e}{2}
\ExpandOutcome{FH}{2}
\ExpandOutcome{TASDSH}{2}
\end{outcomes}

\begin{unit}{La Ética Filosófica}{Lewis, Bourmaud, RodriguezL, AristotelesE}{9}{2}
\begin{topics}
	\item	Presentación del curso. 
	\item	Lo ético y moral. La ética como rama de la filosofía.
	\item	La necesidad de la metafísica.
	\item	La experiencia moral.
	\item	El problema del relativismo y su solución.
	
\end{topics}
\begin{unitgoals}
	\item Presentar una primera noción de la ética y de los problemas relativos a esta rama de la filosofía.
\end{unitgoals}
\end{unit}

\begin{unit}{La acción moral}{SanchezM,Genta}{15}{4}
\begin{topics}
	\item	Caracterización del actuar humano. 
	\item	Libertad, conciencia y voluntariedad. Distintos niveles de libertad. Factores que afectan la voluntariedad.
	\item	El papel de la afectividad en la moralidad.
	\item	La felicidad como fin último del ser humano.

\end{topics}
\begin{unitgoals}
	\item Hacer un análisis del acto humano, presentando sus condiciones y especificando su moralidad.
\end{unitgoals}
\end{unit}

\begin{unit}{La vida virtuosa}{Piper,Droste,Lego,StoTomas}{12}{4}
\begin{topics}
	\item	Qué se entiende por virtud.
	\item	La virtud moral: caracterización y modo de adquisición; el carácter dinámico de la virtud.
	\item	Relación entre las distintas virtudes éticas. Las virtudes cardinales. Los vicios.

\end{topics}
\begin{unitgoals}
	\item Presentar el ideal filosófico de la vida virtuosa destacando algunas virtudes fundamentales.
\end{unitgoals}
\end{unit}

\begin{unit}{Lo éticamente correcto y su conocimiento}{ReydeCastro2010,SanchezM,Genta}{9}{4}
\begin{topics}
	\item 	La corrección en lo ético.
	\item 	El conocimiento de lo éticamente correcto.
	\item 	La llamada ``recta razón'' y la ``verdad práctica''. 
	\item 	Las leyes morales: ley natural y ley positiva.
	\item 	La conciencia moral: definición, tipos, deformaciones. 
	\item 	La valoración moral de las acciones concretas.

\end{topics}

\begin{unitgoals}
	\item Presentar las nociones de recta razón, conciencia moral, y moral natural destacando el conocimiento de la ley moral natural.
\end{unitgoals}
\end{unit}



\begin{coursebibliography}
\bibfile{GeneralEducation/FG101}
\end{coursebibliography}

\end{syllabus}
