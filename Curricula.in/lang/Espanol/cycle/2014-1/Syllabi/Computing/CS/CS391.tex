\begin{syllabus}

\course{CS391. Calidad de Software}{Obligatorio}{CS391}

\begin{justification}
Calidad: cómo asegurar y verificar la calidad, y la necesidad de una cultura de calidad. Como proveer patrones de
calidad por medio de los estándares y métricas como CMMI, PSP/TSP e ISO. Técnicas de prueba, verificación y validación.
Aseguramiento de proceso contra aseguramiento del producto. Estándares de proceso de calidad. Producto y
aseguramiento del proceso. Análisis y divulgación del problema. Acercamientos estadísticos al control de calidad.
\end{justification}

\begin{goals}
\item Los alumnos deben describir los conceptos fundamentales  y comprender la terminología del CMMI.
\item Los alumnos discutirán acerca de las 22 áreas de proceso CMMI así como reconocer el valor de este modelo en diferentes casos de estudio.
\item Los alumnos deben comprender los conceptos fundamentales  CMMI para que sean adoptados en los proyectos de software.
\item Describir y comprender los conceptos de calidad, las normas de la familia ISO en sus diferentes versiones.
\item El alumno debe comprender y aplicar el proceso de pruebas de en software desarrollado así como las estadísticas aplicadas a este proceso.
\item El alumno establecerá una metodología de pruebas para el software realizado.
\end{goals}

\begin{outcomes}
\ExpandOutcome{b}{4}
\ExpandOutcome{c}{4}
\ExpandOutcome{d}{3}
\ExpandOutcome{f}{3}
\ExpandOutcome{g}{4}
\ExpandOutcome{i}{3}
\ExpandOutcome{j}{3}
\ExpandOutcome{k}{3}
\end{outcomes}

\begin{unit}{CMMI v 1.2}{CMMI06,Mary07,Kulpa08}{18}{3}
   \begin{topics}
      \item Introducción.
      \item Conceptos de mejora de procesos y CMMI.
      \item Visión general a los componentes del modelo CMMI.
      \item Representaciones del modelo e institucionalización.
      \item Desarrollo del producto parte 1.
      \item Gestionando el proyecto.
      \item Soporte al proyecto y a la organización.
      \item Desarrollo del producto parte 2.
      \item Infraestructura de mejora.
      \item Gestionando cuantitativamente.
      \item Soportando ambientes complejos.
      \item Integrando los temas tratados.
      \item Siguientes pasos.
      \item Resumen.
   \end{topics}

   \begin{unitgoals}
      \item Describir los componentes y el contenido del modelo CMMI-DEV y sus relaciones.
      \item Discutir las 22 áreas de procesos que conforman el modelo.
      \item Ubicar información relevante en el modelo.   
   \end{unitgoals}
\end{unit}

\begin{unit}{People Software Process \& Team Software Process}{PSPBOK05,Humphrey95,Humphrey97,Humphrey00,Humphrey01,Humphrey05,Humphrey06,Humphrey06a}{12}{3}
\begin{topics}
      \item Fundamentos.
      \item Conceptos básicos de PSP.
      \item Medición de tamaño y estimación.
      \item Creación y seguimiento de planes de proyecto.
      \item Planificación y seguimiento de calidad de software.
      \item Diseño de software.
      \item Extensiones de proceso y personalizaciones.
      \item Conceptos básicos de TSP.
      \item Relaciones entre PSP/TSP y CMMI.
   \end{topics}

   \begin{unitgoals}
      \item En esta unidad se revisará el PSP como una herramienta de mejora del desempeño personal de los desarrolladores de software y cómo éstos pueden convertirse en un equipo de alto desempeño usando TSP.
      \item Se explicará la relación que existe entre PSP/TSP y CMMI.
   \end{unitgoals}
\end{unit}

\begin{unit}{Estándares ISO/IEC}{Peach02,Sue07,Gordon08,Khan06}{18}{3}
   \begin{topics}
      \item ISO 9001:2001.
      \item ISO 9000-3.
      \item ISO/IEC 9126.
      \item ISO/IEC 12207.
      \item ISO/IEC 15939.
      \item ISO/IEC 14598.
      \item ISO/IEC 15504-SPICE.
      \item IT Mark.
      \item SCRUM.
      \item SQuaRE.
      \item CISQ.
   \end{topics}

   \begin{unitgoals}
      \item Brindar a los participantes comprensión de los conceptos relacionados con la calidad, y con las normas de la familia ISO 9000, en sus diferentes versiones (la normas ISO 9001:2001, especificidades de la norma ISO 9000-3 para el caso del diseño, desarrollo, suministro, instalación y mantenimiento de software de computación y aplicación de estos conceptos y técnicas; las normas ISO/IEC 9126, ISO/IEC 12207, ISO/IEC 15939, ISO/IEC 14598, ISO/IEC 15504-SPICE, IT Mark, SCRUM, SQuaRE y CISQ, su utilización, etc.)
   \end{unitgoals}
\end{unit}

\begin{unit}{Técnicas de Prueba de Software}{Wang00,Peter08}{12}{4}
\begin{topics}
      \item Introducción
      \item Estadísticas relativas al proceso de pruebas.
      \item Estándares relativos a la prueba de software.
      \item El proceso de pruebas.
      \begin{inparaenum}
         \item Principios de prueba.
         \item El plan de calidad.
         \item El plan de pruebas.
         \item Técnicas de Verificación.
      \end{inparaenum}
      \item Software CAST (\textit{Computer Aided Software Testing}).
      \item Una metodología de pruebas.
   \end{topics}

   \begin{unitgoals}
      \item Elaborar planes de prueba y planes de calidad en sus proyectos de desarrollo.
      \item Aplicar técnicas de pruebas formales para la generación de casos de prueba.
      \item Definir las técnicas de prueba a aplicar, según los requerimientos de cada aplicación.
      \item Desarrollar un plan para implantar una metodología de pruebas en la organización.
   \end{unitgoals}
\end{unit}



\begin{coursebibliography}
\bibfile{Computing/CS/CS391}
\end{coursebibliography}

\end{syllabus}
