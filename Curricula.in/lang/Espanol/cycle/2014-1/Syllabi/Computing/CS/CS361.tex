\begin{syllabus}

\course{CS361. Tópicos en Inteligencia Artificial}{Electivo}{CS361}

\begin{justification}
Provee una serie de herramientas para resolver problemas que son difíciles de solucionar con los métodos algorítmicos tradicionales. Incluyendo heurísticas, planeamiento, formalísmos en la representación del conocimiento y del razonamiento, técnicas de aprendizaje en máquinas, técnicas aplicables a los problemas de acción y reacción: asi como el aprendizaje de lenguaje natural, visión artificial y robótica entre otros. 
\end{justification}

\begin{goals}
\item Realizar algún curso avanzado de Inteligencia Artificial sugerido por el curriculo de la ACM/IEEE.
\end{goals}

\begin{outcomes}
\ExpandOutcome{a}{4}
\ExpandOutcome{b}{4}
\ExpandOutcome{h}{4}
\ExpandOutcome{i}{3}
\ExpandOutcome{j}{5}
\ExpandOutcome{l}{4}
\end{outcomes}

\begin{itemize}
\item CS360. Sistemas Inteligentes
\item CS361. Razonamiento automatizado
\item CS362. Sistemas Basados en Conocimiento
\item CS363. Aprendizaje de Maquina \cite{Russell03},\cite{Haykin99}
\item CS364. Sistemas de Planeamiento
\item CS365. Procesamiento de Lenguaje Natural
\item CS366. Agentes
\item CS367. Robótica
\item CS368. Computación Simbólica
\item CS369. Algoritmos Genéticos \cite{Goldberg89}
\end{itemize}



\begin{coursebibliography}
\bibfile{Computing/CS/CS261T}
\end{coursebibliography}

\end{syllabus}
