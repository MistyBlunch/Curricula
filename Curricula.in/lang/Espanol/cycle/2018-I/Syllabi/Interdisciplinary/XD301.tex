\begin{syllabus}

\course{EG0009. Proyecto Interdisciplinario III}{Obligatorio}{EG0009} % Common.pm

\begin{justification}
Proyectos Interdisciplinarios III es un curso en el que los estudiantes trabajan en equipos en un proyecto de investigación y desarrollo o emprendimiento, con el fin de plantear una solución a un problema relevante. El desarrollo del proyecto se centra en el uso de herramientas de ingeniería, tecnologia y la ciencia de la computación para proponer soluciones a problemas técnicos, tecnológicos, científicos y/o sociales. La integración del
conocimiento y aspectos multidisciplinarios e interdisciplinarios es un elemento esencial para el éxito del proyecto. A lo largo del curso, el estudiante aprende sobre el proceso de
diseño, a aplicar los contenidos de su carrera a un contexto real; a identificar y adquirir nuevos conocimientos relevantes; y a colaborar interdisciplinariamente. En este tercer curso de Proyectos Interdisciplinarios, el estudiante está expuesto a problemas de complejidad moderada, con bajo nivel incertidumbre en la problemática y la solución, y cuenta con el
apoyo y supervisión cercana del asesor del proyecto. El curso enfatiza el desarrollo y reforzamiento de las habilidades de comunicación efectiva y colaboración, para propiciar la
formación de equipos de alto rendimiento. Se aprende a gestionar proyectos, aplicando buenas prácticas y estándares internacionales.
\end{justification}

\begin{goals}
   \item Identificar problemas
   \item Diseñar un componente o un proceso para satisfacer las necesidades deseadas dentro de restricciones realistas.  
\end{goals}

\begin{outcomes}
   \item \ShowOutcome{f}{2}
   \item \ShowOutcome{n}{2}
\end{outcomes}

\begin{competences}
    \item \ShowCompetence{C17}{f}
    \item \ShowCompetence{C19}{n}
\end{competences}

\begin{unit}{Proyecto Interdisciplinario III }{}{Zobel}{16}{C17}
\begin{topics}
      \item Desarrollar ideas relacionas a las multiples discipiplinas  que aproximen al alumno a una idea real de una empresa.
\end{topics}

\begin{learningoutcomes}
   \item Desarrollo del pensamiento crítico en la toma de decisiones en los procesos de diseño de productos o realización de las investigaciones.
\end{learningoutcomes}
\end{unit}



\begin{coursebibliography}
\bibfile{Interdisciplinary/XD101}
\end{coursebibliography}

\end{syllabus}
