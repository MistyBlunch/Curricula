\begin{syllabus}

\course{CS1D01. Estructuras Discretas I}{Obligatorio}{CS1D01} % Common.pm

\begin{justification}

Las estructuras discretas proporcionan los fundamentos teóricos necesarios para la computación. Estos fundamentos no sólo son útiles para desarrollar la computación desde un punto de vista teórico como sucede
En el curso de la teoría computacional, pero también es útil para la práctica de la informática; En particular en aplicaciones tales como verificación,
Criptografía, métodos formales, etc.

\end{justification}

\begin{goals}
\item Aplicar Correctamente conceptos de matemáticas finitas (conjuntos, relaciones, funciones) para representar datos de problemas reales.
\item Modelar situaciones reales descritas en lenguaje natural, usando lógica proposicional y lógica predicada.
\item Determine las propiedades abstractas de las relaciones binarias.
\item Elijir el método de demostración más apropiado para determinar la veracidad de una propuesta y construir argumentos matemáticos correctos.
\item Interpretar soluciones matemáticas a un problema y determinar su fiabilidad, ventajas y desventajas.
\item Expresar el funcionamiento de un circuito electrónico simple usando álgebra booleana.
\end{goals}

\begin{outcomes}
    \item \ShowOutcome{a}{2}
    \item \ShowOutcome{i}{3}
    \item \ShowOutcome{j}{2}
\end{outcomes}

\begin{competences}
    \item \ShowCompetence{C1}{a}
    \item \ShowCompetence{C20}{i,j}
\end{competences}

\begin{unit}{\DSSetsRelationsandFunctions}{}{Grimaldi03,Rosen2007}{13}{C1,C20}
   \begin{topics}
        \item \DSSetsRelationsandFunctionsTopicSets
        \item \DSSetsRelationsandFunctionsTopicRelations
        \item \DSSetsRelationsandFunctionsTopicFunctions
   \end{topics}
   \begin{learningoutcomes}
	\item \DSSetsRelationsandFunctionsLOExplainWith [\Assessment]
	\item \DSSetsRelationsandFunctionsLOPerformThe [\Assessment]
	\item \DSSetsRelationsandFunctionsLORelate [\Assessment]
   \end{learningoutcomes}
 \end{unit}

 \begin{unit}{\DSBasicLogic}{}{Rosen2007,Grimaldi03}{14}{C1,C20}
   \begin{topics}
        \item \DSBasicLogicTopicPropositional%
        \item \DSBasicLogicTopicLogical%
        \item \DSBasicLogicTopicTruth%
        \item \DSBasicLogicTopicNormal%
        \item \DSBasicLogicTopicValidity%
        \item \DSBasicLogicTopicPropositionalInference%
        \item \DSBasicLogicTopicPredicate%
        \item \DSBasicLogicTopicLimitations%
   \end{topics}
   \begin{learningoutcomes}
	\item \DSBasicLogicLOConvertLogical [\Usage ]
	\item \DSBasicLogicLOApplyFormal [\Usage ]
	\item \DSBasicLogicLOUseThe [\Usage]
	\item \DSBasicLogicLODescribeHowCan [\Familiarity]
	\item \DSBasicLogicLOApplyFormalAnd [\Usage ]
	\item \DSBasicLogicLODescribeTheLimitationsAnd [\Usage]
   \end{learningoutcomes}
 \end{unit}

\begin{unit}{\DSProofTechniques}{}{Rosen2007, Epp10, Scheinerman12}{14}{C1,C20}
\begin{topics}
        \item \DSProofTechniquesTopicNotions%
        \item \DSProofTechniquesTopicThe%
        \item \DSProofTechniquesTopicDirect%
        \item \DSProofTechniquesTopicDisproving%
        \item \DSProofTechniquesTopicProof%s
        \item \DSProofTechniquesTopicInduction%
        \item \DSProofTechniquesTopicStructural%
        \item \DSProofTechniquesTopicWeak%
        \item \DSProofTechniquesTopicRecursive%
        \item \DSProofTechniquesTopicWell%
\end{topics}

\begin{learningoutcomes}
    %% itemizar cada learning outcomes [nivel segun el curso]
	\item \DSProofTechniquesLOIdentifyTheUsed [\Assessment]
	\item \DSProofTechniquesLOOutline [\Usage ]
	\item \DSProofTechniquesLOApplyEach [\Usage ]
	\item \DSProofTechniquesLODetermineWhich [\Assessment]
	\item \DSProofTechniquesLOExplainTheIdeas [\Familiarity ]
	\item \DSProofTechniquesLOExplainTheWeak [\Assessment]
	\item \DSProofTechniquesLOStateThe [\Familiarity]
\end{learningoutcomes}
\end{unit}

\begin{unit}{Lógica Digital y Representación de Datos}{}{Rosen2007,Grimaldi03}{19}{C1,C20}
   \begin{topics}
	\item Órdenes parciales y Conjuntos parcialmente ordenados.   
 	\item Elementos extremos de un conjunto parcialmente ordenado.
	\item Reticulo: Tipos y propiedades.
	\item Álgebras booleanas.
	\item Funciones y expresiones booleanas.
	\item Representación de las funciones booleanas: Disjuntiva normal y forma conjunta.
	\item Puertas Lógicas.
	\item Minimización del Circuito.
   \end{topics}

   \begin{learningoutcomes}
	\item Explicar la importancia del álgebra booleana como una unificación de la teoría de conjuntos y la lógica proposicional [\Assessment].
	\item Conocer las estructuras algebraicas del retículo y sus tipos [\Assessment].
	\item Explicar la relación entre el retículo y el conjunto de ordenadas y el uso prudente para demostrar que un conjunto es un retículo [\Assessment].
	\item Conocer las propiedades que satisfacen un álgebra booleana [\Assessment].
	\item Demostrar si una terna formada por un conjunto y dos operaciones internas es o no Álgebra booleana [\Assessment].
	\item Encuentra las formas canónicas de una función booleana  [\Assessment].
	\item Representar una función booleana como un circuito booleano usando puertas lógica[\Assessment].
	\item Minimizar una función booleana [\Assessment].
    \end{learningoutcomes}
\end{unit}

\begin{coursebibliography}
\bibfile{Computing/CS/CS1D1}
\end{coursebibliography}

\end{syllabus}

%\end{document}
