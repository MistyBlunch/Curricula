\begin{syllabus}

\course{. }{}{} % Common.pm

\begin{justification}
Este curso tiene por objetivo que el alumno logre finalizar adecuadamente su borrador de tesis.
\end{justification}

\begin{goals}
\item Que el alumno complete este curso con su tesis elaborada en calidad suficiente como para una inmediata sustentación.
\item Que el alumno presente formalmente el borrador de tesis ante las autoridades de la facultad.
\item Los entregables de este curso son:
	\begin{description}
	\item [Parcial:] Avance del proyecto de tesis incluyendo en el documento: introducción, marco teorico, estado del arte, propuesta, análisis y/o experimentos y bibliografía sólida.
	\item [Final:] Documento de tesis completo y listo para sustentar en un plazo no mayor de quince días.
	\end{description}
\end{goals}

\begin{outcomes}
\item \ShowOutcome{a}{3}
\item \ShowOutcome{b}{3}
\item \ShowOutcome{c}{3}
\item \ShowOutcome{e}{3}
\item \ShowOutcome{f}{3}
\item \ShowOutcome{h}{3}
\item \ShowOutcome{i}{3}
\item \ShowOutcome{l}{3}
\end{outcomes}

\begin{competences}
\item \ShowCompetence{C1}{a,b,c} 
\item \ShowCompetence{C20}{e,f.g}
\item \ShowCompetence{CS2}{h,i,l}
\end{competences}

\begin{unit}{Escritura del Borrador del trabajo de final de carrera (tesis)}{}{ieee,acm,citeseer}{60}{C1,C20,CS2}
\begin{topics}
    \item Redacción y correccion del trabajo de final de carrera
\end{topics}

\begin{learningoutcomes}
    \item Parte experimental concluída (si fuese adecuado al proyecto) [\Assessment]
    \item Verificar que el documento cumpla con el formato de tesis de la carrera [\Assessment]
    \item Entrega del borrador de tesis finalizado y considerado listo para una sustentación pública del mismo (requisito de aprobación) [\Assessment]
\end{learningoutcomes}
\end{unit}



\begin{coursebibliography}
\bibfile{Computing/CS/CS401}
\end{coursebibliography}
\end{syllabus}
