\begin{syllabus}

\course{. }{}{} % Common.pm

\begin{justification}
Parte fundamental de la formación integral de un profesional es la habilidad de comunicarse en un idioma extranjero además del propio idioma nativo. No solamente amplTecnologíaa su horizonte cultural sino que permite una visión más humana y comprensiva de la vida. En el caso de los idiomas extranjeros, indudablemente el Inglés es el 
más práctico porque es hablado alrededor de todo el mundo. No hay paTecnologías alguno donde este no sea hablado. En las carreras relacionadas con los servicios al turista el inglés es tal vez la herramienta práctica más importante que el alumno debe dominar desde el primer momento como parte de su formación integral.
\end{justification}

\begin{goals}
\item Formar en el alumno de capacidad de comprender y retener una conversación.
\item Brindar técnicas de ilación de ideas.
\end{goals}

\begin{outcomes}
\item \ShowOutcome{f}{2}
\end{outcomes}

\begin{competences}
\item \ShowCompetence{C25}{f}
\end{competences}

\begin{unit}{Getting to know you!}{}{Soars022S, Cambridge06, MacGrew99}{0}{C25}
   \begin{topics}
      \item Tiempos Presente, Pasado y Futuro.
      \item Oraciones Interrogativas con Wh-.
      \item Palabras con más de un significado.
      \item Partes de la oración.
      \item Expresiones para tiempo libre.
   \end{topics}

   \begin{learningoutcomes}
      \item Al terminar la primera unidad, cada uno de los alumnos, comprendiendo la gramática de los tiempos presente, pasado y futuro es capaz de expresar una mayor cantidad de acciones en forma de oraciones.  Además es capaz de expresar ideas en forma de preguntas.  Asume la idea de palabras con más de un significado. Utiliza expresiones sociales en situaciones de entretenimiento. 
   \end{learningoutcomes}
\end{unit}

\begin{unit}{The way we live!}{}{Soars022S, Cambridge06, MacGrew99}{0}{C25}
   \begin{topics}
      \item Tiempo Presente Simple.
      \item Tiempo Presente Continuo.
      \item Colocaciones.
      \item Vocabulario de paí­ses del  mundo.
      \item Expresiones de enojo.
      \item Conectores.
   \end{topics}

   \begin{learningoutcomes}
      \item Al terminar la segunda unidad, los alumnos habiendo identificado la forma de expresar presente reconocen la diferencia entre las formas del mismo y las aplican adecuadamente. Describen paí­ses acuciosamente.  Asumen expresiones para demostrar interés. Utilizan conectores para unir ideas varias.  
   \end{learningoutcomes}
\end{unit}

\begin{unit}{It all went wrong!}{}{Soars022S, Cambridge06, MacGrew99}{0}{C25}
   \begin{topics}
      \item Tiempo Pasado Simple.
      \item Tiempo Pasado Continuo.
      \item Verbos Irregulares.
      \item Expresiones de Tiempo.
      \item Conectores de tiempo.
   \end{topics}

   \begin{learningoutcomes}
      \item Al terminar la tercera unidad, los alumnos habiendo reconocido las caracterTecnologíasticas de los tiempos en pasado los utilizan adecuadamente. Utilizan prefijos y sufijos para crear y reconocer nuevas palabras. Describen tiempo en forma amplia. Utilizarán conjunciones para unir ideas tipo. 
   \end{learningoutcomes}
\end{unit}

\begin{unit}{Lets go shopping!}{}{Soars022S, Cambridge06, MacGrew99}{0}{C25}
   \begin{topics}
      \item Expresiones de Cantidad Indefinida
      \item Oraciones Afirmativas, Negativas y Preguntas
      \item Uso de Artí­culos
      \item Precios de productos
      \item Llenado de formatos y encuestas
      \item Expresiones para ir de compras
   \end{topics}

   \begin{learningoutcomes}
      \item Al terminar la cuarta primera unidad, los alumnos habiendo identificado la idea de cantidad expresan diversas situaciones que la involucran. Reconocen y aplican artí­culos a sustantivos. Asumen la idea de ir de compras con la ayuda de expresiones. Expresan precios e ideas de dinero. Llenan formatos varios. Expresan actitudes.
   \end{learningoutcomes}
\end{unit}

\begin{unit}{What do you want to do?}{}{Soars022S, Cambridge06, MacGrew99}{0}{C25}
   \begin{topics}
      \item Patrones Verbales I.
      \item Intenciones Futuras.
      \item Verbos de Percepción.
      \item Vocabulario de sentimientos.
      \item Expresiones de Planes y Ambiciones.
   \end{topics}

   \begin{learningoutcomes}
      \item Al finalizar la quinta unidad, los alumnos, a partir de la comprensión de la idea de patrones verbales elaborarán oraciones utilizando los elementos necesarios. Asimilarán además la necesidad de expresar intenciones futuras. Adquirirán vocabulario para describir sentimientos. Se presentará expresiones para describir planes y ambiciones.
   \end{learningoutcomes}
\end{unit}

\begin{unit}{The best in the world!}{}{Soars022S, Cambridge06, MacGrew99}{0}{C25}
   \begin{topics}
      \item Whats it like?.
      \item Adjetivos.
      \item Comparativos y Superlativos.
      \item Sinónimos y Antónimos. 
      \item Indicaciones de Dirección.
      \item Lecturas.
   \end{topics}

   \begin{learningoutcomes}
      \item Al finalizar la sexta unidad, los alumnos habiendo conocido los fundamentos del uso de adjetivos, estructuran oraciones con diversas formas de los mismos en contextos adecuados. Enfatizan la diferencia entre tipos de ciudades y pueblos y estilos de vida. Utilizan expresiones indicación de direcciones.
   \end{learningoutcomes}
\end{unit}

\begin{unit}{Fame!}{}{Soars022S, Cambridge06, MacGrew99}{0}{C25}
   \begin{topics}
      \item Presente Perfecto y Pasado Simple
      \item Expresiones for, ever, since
      \item Adverbios
      \item Expresiones que vienen en pares 
      \item Respuestas cortas
      \item Celebridades
   \end{topics}

   \begin{learningoutcomes}
      \item Al finalizar la séptima unidad, los alumnos habiendo conocido los fundamentos de la estructuración del tiempo presente perfecto y lo diferencian del pasado simple. Enfatizan la diferencia entre formas de adjetivos. Describen ideas de la música. Utilizan expresiones para dar respuestas cortas. Asumen la idea de dar explicaciones extra de los elementos de una oración.
   \end{learningoutcomes}
\end{unit}



\begin{coursebibliography}
\bibfile{ForeignLanguages/ID101}
\end{coursebibliography}

\end{syllabus}
%\end{document}
