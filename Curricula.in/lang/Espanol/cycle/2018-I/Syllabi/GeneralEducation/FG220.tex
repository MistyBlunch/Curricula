\begin{syllabus}

\course{FG220. Realidad Económica Nacional y Global}{Obligatorio}{FG220}

\begin{justification}
La formación integral del alumno supone una adecuada valoración histórica de la realidad nacional de modo que su accionar profesional esté integrado y articulado con la identidad cultural peruana, que genera el compromiso de hacer de nuestra sociedad un ámbito más humano, solidario y justo.
\end{justification}

\begin{goals}
\item Analizar y comprender la situación actual del Perú desde una perspectiva histórica y sociológica, de modo que los alumnos puedan reconocerse y entenderse como parte de una Nación sellada en su núcleo más íntimo por los valores cristianos que impulsan la construcción de una sociedad más justa y reconciliada.  [\Familiarity]
\end{goals}

\begin{outcomes}
    \item \ShowOutcome{e}{2}
    \item \ShowOutcome{n}{2}
\end{outcomes}

\begin{competences}
    \item \ShowCompetence{C20}{e,n}
\end{competences}

\begin{unit}{}{Primera Unidad}{Belaunde1965,Messori1998,Morande1987,Vargas1996}{18}{C20}
\begin{topics}
	\item Aspectos conceptuales relevantes para elaboración de las matrices analíticas.
	    \begin{subtopics}
		\item Cultura.
		\item Identidad.
		\item Nación.
		\item Sociedad.
		\item Estado.
		\item Normas para elaboración de matrices.
	    \end{subtopics}
	\item El imperio de los Incas.
	     \begin{subtopics}
		\item Repaso de  aspectos socio-culturales más importantes.
		\item Elaboración de la matriz del imperio Inca.
	     \end{subtopics}
	\item Conquista española.
	    \begin{subtopics}
		\item ?`Encuentro o choque de las culturas?.
		\item Hacia una comprensión integral del fenómeno.
		\item Debate conceptual.
		\item Elaboración de matriz: cultura española.
	    \end{subtopics}
	\item Virreinato.
	    \begin{subtopics}
		\item Repaso de  aspectos socio-culturales más importantes.
		\item Surgimiento de la identidad nacional peruana al calor de la Fe Católica.
		\item Elaboración de matriz: cultura virreinal.	
	    \end{subtopics}
\end{topics}
\begin{learningoutcomes}
	\item Comprender adecuadamente el proceso histórico que determina el nacimiento de nuestra identidad nacional a partir de la síntesis cultural del virreinato.[\Familiarity]
\end{learningoutcomes}
\end{unit}

\begin{unit}{}{Segunda Unidad}{Pease1999,Basadre1994,Vargas1996}{6}{C20}
\begin{topics}
	\item El proceso de la emancipación peruana.
	\item Hacia una compresión integral del fenómeno.
	\item Debate conceptual.
	\item Elaboración de matriz.
\end{topics}
\begin{learningoutcomes}
	\item Comprender el proceso independentista peruano como expresión de la identidad nacional.[\Familiarity]
\end{learningoutcomes}
\end{unit}

\begin{unit}{}{Tercera Unidad}{Pease1999,Vargas1996}{9}{C20}
\begin{topics}
	\item Primeros cambios culturales.
	    \begin{subtopics}
		\item Inicio del proceso secularizador de la cultura.
		\item Primera República y Militarismo.
		\item Repaso de  aspectos socio-culturales más importantes.
		\item Elaboración de matriz.
	    \end{subtopics}
	\item Prosperidad Falaz.
	    \begin{subtopics}
		\item Repaso de  aspectos socio-culturales más importantes.
		\item Elaboración de matriz.
	     \end{subtopics}
	\item Guerra con Chile.
	      \begin{subtopics}
		\item Repaso de  aspectos socio-culturales más importantes.
		\item Elaboración de matriz.	
	    \end{subtopics}
\end{topics}
\begin{learningoutcomes}
	\item Identificar adecuadamente los procesos históricos de desintegración nacional en el siglo XIX.[\Familiarity]
\end{learningoutcomes}
\end{unit}

\begin{unit}{}{Cuarta Unidad}{Pease1999,Delapuente2006,Quiroz2006}{18}{C20}
\begin{topics}
    \item Principales ideologías políticas en el siglo XX en contrapunto con los principios de la Doctrina Social de la Iglesia.
    \item Ciclo liberal.
	\begin{subtopics}
	    \item Repaso de  aspectos socio-culturales más importantes.
	    \item Elaboración de matriz.
	\end{subtopics}
    \item Ciclo Nacional-Populista.
      \begin{subtopics}
	\item Primer subciclo (1930-1948).
	      \begin{subtopics}
		\item Repaso de  aspectos socio-culturales más importantes.
		\item Elaboración de matriz.
	      \end{subtopics}
	\item Segundo subciclo (1948-1968).
	      \begin{subtopics}
		\item Repaso de  aspectos socio-culturales más importantes.
		\item Elaboración de matriz.
	      \end{subtopics}
	\item Tercer subciclo (1968-1980).
	      \begin{subtopics}
		\item Repaso de  aspectos socio-culturales más importantes.
		\item Elaboración de matriz.
	      \end{subtopics}
	\item Cuarto subciclo (1980-1990).
	      \begin{subtopics}
		\item Repaso de  aspectos socio-culturales más importantes.
		\item Elaboración de matriz.
	    \end{subtopics}
      \end{subtopics}
    \item Ciclo Neoliberal (1990- ?`?).
	  \begin{subtopics}
	    \item Repaso de  aspectos socio-culturales más importantes.
	    \item Elaboración de matriz.
	  \end{subtopics}
    \item Situación de la Nación Peruana.
    \item Recuperación de la integración y la solidaridad socio-cultural.
\end{topics}
\begin{learningoutcomes}
	\item Identificar adecuadamente los procesos históricos de desintegración nacional en el siglo XX.[\Familiarity]
\end{learningoutcomes}
\end{unit}



\begin{coursebibliography}
\bibfile{GeneralEducation/FG220}
\end{coursebibliography}

\end{syllabus}
