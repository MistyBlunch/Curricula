\begin{syllabus}

\course{. }{}{} % Common.pm

\begin{justification}
La propuesta del Magisterio de la Iglesia para el correcto orden de la vida social -en los ámbitos políticos, social y económico- debería constituir la piedra angular de la organización social, logrando así una sociedad reconciliada para todos.
\end{justification}

\begin{goals}
\item Contribuir en la formación de agentes de cambio, quienes desde el rol que les toque desempeñar en la sociedad, sean partícipes en una sociedad orientada al desarrollo social solidario. [\Usage]
\end{goals}

\begin{outcomes}
    \item \ShowOutcome{e}{2}
    \item \ShowOutcome{n}{2}
    \item \ShowOutcome{o}{2}
\end{outcomes}

\begin{competences}
    \item \ShowCompetence{C10}{e,n}
    \item \ShowCompetence{C20}{n,o}
    \item \ShowCompetence{C21}{e,n,o}
\end{competences}

\begin{unit}{}{Primera Unidad: Misión de la Doctrina Social de la Iglesia}{Peruana2005,Centro2011,II2002,II2003}{6}{C21}
\begin{topics}
	\item Naturaleza de la Doctrina Social de la Iglesia.
	\item La Doctrina Social de la Iglesia en nuestro tiempo.
	\item La persona humana: múltiples dimensiones y su centralidad.
	\item Los derechos humanos.
\end{topics}
\begin{learningoutcomes}
	\item Comprender la naturaleza de la acción de la Iglesia en el mundo. [\Familiarity]
	\item Comprender la importancia de la centralidad del hombre en la sociedad. [\Familiarity]
\end{learningoutcomes}
\end{unit}

\begin{unit}{}{Segunda Unidad: Principios y Valores de la Doctrina Social de la Iglesia}{Peruana2005,II2002,Centro2011}{9}{C10,C20,C21}
\begin{topics}
	\item Bien común.
	\item Destino universal de los bienes.
	\item Subsidiaridad.
	\item Participación.
	\item Solidaridad.
	\item Valores Fundamentales.
	    \begin{subtopics}
		\item La verdad.
		\item La libertad.
		\item La justicia.
		\item El amor.
	    \end{subtopics}
\end{topics}
\begin{learningoutcomes}
	\item Conocer y comprender los principios permanentes y valores fundamentales que están presentes en la Enseñanza Magisterial, los cuales deben ser la base para la formación de las diversas instancias sociales.[\Familiarity]
\end{learningoutcomes}
\end{unit}

\begin{unit}{}{Tercera Unidad: La familia: Célula vital de la Sociedad}{Peruana2005,Centro2011,II2002}{6}{C10,C20}
\begin{topics}
	\item Importancia de la Familia para la persona y sociedad.
	\item Fundamento de la Familia: El Matrimonio.
	\item Familia necesaria para la vida social.
	\item Familia, centro de la civilización del amor.
\end{topics}
\begin{learningoutcomes}
	\item Comprender que de la naturaleza social del hombre derivan, algunos órdenes sociales necesarios, como la familia. [\Familiarity]
	\item Conocer, comprender y valorar la naturaleza de la familia y el matrimonio y su rol en la sociedad.[\Familiarity]
\end{learningoutcomes}
\end{unit}

\begin{unit}{}{Cuarta Unidad: Trabajo y vida económica}{Peruana2005,Benedicto2009,II2002,Centro2010,Benedicto2006,II2003}{9}{C10,C20,C21}
\begin{topics}
	\item La Dignidad del trabajo.
	\item Derecho al trabajo y derechos de los trabajadores.
	\item Solidaridad entre los trabajadores.
	\item El Trabajo en un mundo global.
	\item Relación entre la moral y la economía.
	\item Iniciativa Privada y empresa.
	\item Instituciones y nuevas organizaciones económicas al servicio del hombre.
	\item Frente a la economía global.
\end{topics}
\begin{learningoutcomes}
	\item Conocer y comprender los principios de la Doctrina Social de la Iglesia en el campo de la actividad económica. [\Familiarity]
	\item Formación de la conciencia cristiana para el posterior desenvolvimiento profesional.[\Usage]
	\item Comprender que los principios del Evangelio y de la ética natural pueden ser aplicados  a las concreciones del orden económico de la actividad humana.[\Familiarity]
\end{learningoutcomes}
\end{unit}

\begin{unit}{}{Quinta Unidad: Comunidad Política y Comunidad Internacional}{II2003, Peruana2005,Benedicto2006,II2002,Centro2010}{12}{C21}
\begin{topics}
	\item Elementos constitutivos de la comunidad política.
	\item El fundamento y fin de la comunidad política.
	\item Autoridad Política y Democracia.
	\item Relación entre Iglesia y Estado.
	\item Comunidad Política al servicio de la Sociedad Cívil.
	\item Las reglas fundamentales y organización de la comunidad internacional.
	\item Cooperación interna-cional para el desarrollo.
	\item La promoción de la paz.
	\item Salvaguarda del medio ambiente.

\end{topics}
\begin{learningoutcomes}
	\item Comprender que de la naturaleza social del hombre derivan, la nación y el Estado como órdenes sociales necesarios.[\Familiarity]
\end{learningoutcomes}
\end{unit}



\begin{coursebibliography}
\bibfile{GeneralEducation/FG301}
\end{coursebibliography}

\end{syllabus}
