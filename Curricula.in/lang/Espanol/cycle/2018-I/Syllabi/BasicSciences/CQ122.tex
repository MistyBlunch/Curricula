\begin{syllabus}

\curso{CQ122. Química General II}{Obligatorio}{CQ122}

\begin{justification}
Este curso es útil en esta carrera para que el alumno aprenda a mostrar un alto grado de dominio de las leyes de la Química General.
\end{justification}

\begin{goals}
\item Capacitar y presentar al estudiante los principios básicos de la química como ciencia natural abarcando sus tópicos más importantes y su relación con los problemas cotidianos.
\end{goals}

\begin{outcomes}
\ExpandOutcome{a}
\ExpandOutcome{i}
\ExpandOutcome{j}
\end{outcomes}

\begin{unit}{Termodinámica}{Chang99,Garritz94}{2}
\begin{topics}
	\item Sistemas termodinámicos y su clasificación. Variables termodinámicas y funciones de estado.
	\item Estados de un sistema. Estados de equilibrio. Variables extensivas e intensivas.
	\item Equilibrios térmicos. Principio cero de la termodinámica.
	\item Primer principio de la termodinámica. Capacidad calorífica. Procesos reversibles y trabajo máximo.
	\item Energía interna de los gases ideales. Transformaciones adiabáticas. Termoquímica. Ley de Lavoisier y La Place, Ley de Hess. Ley de Kirchhoff.
	\item Segunda Ley de la termodinámica. Entropía. Eficiencia de un ciclo reversible.
	\item Energía libre. Tercera ley de la termodinámica.
\end{topics}

\begin{unitgoals}
	\item Entender y trabajar con los principios de la Termodinámica.
	\item Abstraer de la naturaleza los conceptos de las transformaciones de los gases.
\end{unitgoals}
\end{unit}

\begin{unit}{Equilibrio químico}{Chang99,Garritz94}{4}
\begin{topics}
      \item Concepto. Constante de equilibrio.
      \item Ley de acción de las masas.
      \item Equilibrios homogéneos. Equilibrios heterogéneos. Equilibrios múltiples.
      \item Factores que afectan el equilibrio químico. Principio de Le Chatelier.
    \end{topics}
   \begin{unitgoals}
      \item Describir, conocer y aplicar los conceptos del equilibrio químico.
      \item Resolver problemas.
   \end{unitgoals}
\end{unit}

\begin{unit}{Ácidos y bases}{Chang99,Garritz94}{4}
\begin{topics}
	\item Ácidos y bases de Bronsted. Propiedades ácido-base del agua. El pH.
	\item Fuerza de los ácidos y bases. Ácidos débiles y su constante de ionización ácida. Bases débiles y su constante de ionización básica. 
	\item Relación entre la constante de acidez de los ácidos y sus bases conjugadas.
	\item Ácidos dipróticos y polipróticos. Propiedades ácido-base de las sales.
	\item Hidrólisis.  Ácidos y bases de Lewis
\end{topics}

\begin{unitgoals}
	\item Describir el comportamiento y características de los ácidos y las bases.
	\item Resolver problemas.
\end{unitgoals}
\end{unit}

\begin{unit}{Equilibrio ácido-base y equilibrio de solubilidad}{Whitten98,Brady98}{6}
\begin{topics}
	\item El efecto del ion común. Disoluciones reguladoras.
	\item Titulaciones ácido-base. Tipos.  Indicadores ácido-base.
	\item El producto de solubilidad. Separación de iones por precipitación. El efecto del ión común y la solubilidad. El pH y la solubilidad.
	\item Aplicación del principio del producto de solubilidad al análisis cualitativo.
\end{topics}

\begin{unitgoals}
	\item Entender los fundamentos del equilibrio ácido-base y de solubilidad.
	\item Resolver problemas.
\end{unitgoals}
\end{unit}

\begin{unit}{Equilibrios de iones complejos}{Whitten98,Brady98}{4}
\begin{topics}
	\item Ion complejo. Enlace covalente coordinado.
	\item La constante de formación. La formación del ion complejo y la solubilidad de una sustancia.
	\item Compuestos de coordinación. La contribución del químico Alfred Werner. Los aminocomplejos.
	\item Ligandos unidentados y polidentados. Complejos quelato.
	\item Índice de coordinación. La esfera de coordinación. Nomenclatura.
   \end{topics}

   \begin{unitgoals}
      \item Entender los fundamentos del equilibrio de iones complejos
      \item Resolver problemas.
   \end{unitgoals}
\end{unit}

\begin{unit}{Cinética química}{Whitten98,Brady98}{3}
\begin{topics}
      \item La velocidad de  reacción. Ecuaciones cinéticas. Orden de reacción, orden parcial, orden total.
      \item Mecanismos de reacción. Reacción elemental. Reacción simple. Reacción compleja.
	\item Seudo orden. Ecuaciones cinéticas y constantes de equilibrio de reacciones elementales.
	\item Molecularidad. Reacciones de primer orden. La vida media. Reacciones de segundo orden.
	\item Determinación de las ecuaciones cinéticas. Influencia de la temperatura en las constantes cinéticas.
	\item La ecuación de Arrhenius. Catálisis.
  \end{topics}

   \begin{unitgoals}
      \item Conocer y entender la cinética de las reacciones químicas.
      \item Resolver problemas.
   \end{unitgoals}
\end{unit}

\begin{unit}{Electroquímica}{Whitten92,Hill99}{4}
\begin{topics}
      \item Reacciones redox. Balanceo. Titulaciones redox.
      \item Sistemas electroquímicos. Celdas electroquímicas.
      \item Pila de Daniell. Diagramas de pilas y convenios IUPAC.
      \item La ecuación de Nernst.
      \item Electrodo: Definición. Potenciales normales de electrodo.
      \item Clasificación de las pilas galvánicas. Baterías. Corrosión.
      \item Células electrolíticas. Electrólisis. Electrólisis del agua.
   \end{topics}

   \begin{unitgoals}
      \item Conocer los conceptos básicos de la electroquímica.
      \item Resolver problemas.
   \end{unitgoals}
\end{unit}

\begin{unit}{La química de la atmósfera}{Whitten92,Hill99}{3}
\begin{topics}
      \item Conceptos de medio ambiente. Composición y capas atmosféricas.
      \item Disminución de ozono en la estratosfera. Agujeros en la capa de ozono.
      \item Química atmosférica. Principales oxidantes en la atmósfera. Oxidación del NO.
      \item Química Diurna. Química Nocturna.
      \item Gases invernadero. Efecto Invernadero. La lluvia ácida. El smog fotoquímico.
    \end{topics}

   \begin{unitgoals}
      \item Conocer conceptos básicos de la química de la atmósfera
   \end{unitgoals}
\end{unit}

\begin{coursebibliography}
\bibfile{BasicSciences/CQ122}
\end{coursebibliography}
\end{syllabus}
