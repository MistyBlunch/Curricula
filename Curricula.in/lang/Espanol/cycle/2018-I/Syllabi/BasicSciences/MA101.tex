\begin{syllabus}

\course{EG0005. Matemática II}{Obligatorio}{EG0005} % Common.pm

\begin{justification}

El curso desarrolla en los estudiantes las habilidades para manejar modelos de habilidades de ingeniería y ciencia. En la primera parte
Del curso un estudio de las funciones de varias variables, derivadas parciales, integrales múltiples y una
Introducción a campos vectoriales. Luego el estudiante utilizará los conceptos básicos de cálculo para modelar y resolver ecuaciones diferenciales ordinarias utilizando técnicas como las transformadas de Laplace y las series de Fourier.

\end{justification}

\begin{goals}
  \item Aplicar reglas de derivación y diferenciación parcial en funciones de varias variables.
  \item Aplicar técnicas para el cálculo de integrales múltiples.
  \item Comprender y utilizar los conceptos de cálculo vectorial.
  \item Comprender la importancia de las series.
  \item Identificar y resolver ecuaciones diferenciales de primer orden y sus aplicaciones en problemas químicos y físicos.
\end{goals}

\begin{outcomes}{V1}
    \item \ShowOutcome{a}{3}  
    \item \ShowOutcome{j}{3}
\end{outcomes}

\begin{competences}{V1}
    \item \ShowCompetence{C1}{a}
    \item \ShowCompetence{C20}{j}
\end{competences}

\begin{unit}{Multi-Variable Function Differential}{}{Stewart,DennisZ}{24}{C1,C20}
   \begin{topics}      
    \item Concepto de funciones multi-variables.
    \item Derivados Direccionales
    \item Línea tangente, plano normal a línea de curva y plano tangente, línea normal a un plano de curva. Conocer para calcular sus ecuaciones.
    \item Concepto de valor extremo y valor extremo condicional de funciones multi-variables.
    \item Problemas de aplicación tales como modelización de la producción total de un sistema económico, velocidad del sonido a través del océano, optimización del espesante, etc.
      \end{topics}

   \begin{learningoutcomes}
    \item Comprender el concepto de funciones multi-variables.
    \item Dominar el concepto y método de cálculo de la derivada direccional y gradiente de la guía.
    \item Dominar el método de cálculo de la derivada parcial de primer orden y de segundo orden de las funciones compuestas.
    \item DomEntender línea tangente, plano normal a línea de curva y plano tangente, línea normal a un plan de curva. Saber calcular sus ecuaciones.inar el método de cálculo de las derivadas parciales para funciones implícitas.
    \item Entender línea tangente, plano normal a línea de curva y plano tangente, línea normal a un plan de curva. Saber calcular sus ecuaciones.
    \item Aprenda el concepto de valor extremo y valor extremo condicional de funciones multi-variables; Saber para averiguar el valor extremo de la función binaria.
    \item Ser capaz de resolver problemas de aplicaciones simples.
    \end{learningoutcomes}
\end{unit}

\begin{unit}{Multi-Variable function Integral}{}{Stewart,DennisZ}{12}{C1,C20}
  \begin{topics}
    \item Integral doble, integral triple y naturaleza de la integral múltiple.
    \item Método de doble integral
    \item Línea integral
    \item La Divergencia, Rotación y Laplaciano
   \end{topics}
  
  \begin{learningoutcomes}
    \item Entender la integral doble, integral triple, y entender la naturaleza de la integral múltiple.
    \item Dominar el método de cálculo de la integral doble (coordenadas cartesianas, coordenadas polares), la integral triple (coordenadas cartesianas, coordenadas cilíndricas, coordenadas esféricas).
    \item Entender el concepto de línea Integral, sus propiedades y relaciones.
    \item Saber calcular la integral de línea.
    \item Dominar el cálculo de la rotación, la divergencia y Laplacian.
  
    \end{learningoutcomes}

\end{unit}

\begin{unit}{Series}{}{Stewart,DennisZ}{24}{C1,C20}
   \begin{topics}
    \item Serie convergente.
    \item Serie Taylor y MacLaurin.
    \item Funciones ortogonales.
 \end{topics}

   \begin{learningoutcomes}
    \item Dominio del cálculo si la serie es convergente, y si es convergente, encontrar la suma de la serie tratando de encontrar el radio de convergencia y el intervalo de convergencia de una serie de potencia.
    \item Representa una función como una serie de potencias y encuentra la serie de Taylor y MacLaurin para estimar los valores de las funciones con la precisión deseada.
    \item Entender los conceptos de funciones ortogonales y la expansión de una función dada f para encontrar su serie de Fourier.
     \end{learningoutcomes}
\end{unit}

\begin{unit}{Ordinary Differential Equations}{}{Stewart,DennisZ}{30}{C1,C20}
   \begin{topics}
    \item Concepto de ecuaciones diferenciales
    \item Métodos para resolver ecuaciones diferenciales
    \item Métodos para resolver las ecuaciones diferenciales lineales de segundo orden
    \item Ecuaciones diferenciales ordinarias lineales de orden superior
    \item Problemas de aplicaciones con las transformaciones de Laplace
      \end{topics}

   \begin{learningoutcomes}
    \item Comprender ecuaciones diferenciales, soluciones, orden, solución general, condiciones iniciales y soluciones especiales, etc.
    \item Dominar el método de cálculo para las variables ecuación separable y ecuaciones lineales de primer orden. Conocido para resolver la ecuación homogénea y las ecuaciones de Bernoulli (Bernoulli); Entender la sustitución de la variable para resolver la ecuación.
    \item Diminio  para resolver ecuaciones diferenciales totales.
    \item Ser capaz de utilizar el método de orden reducido para resolver ecuaciones.
    \item Comprender la estructura de la ecuación diferencial lineal de segundo orden.
    \item Dominio del cálculo para las ecuaciones diferenciales lineales homogéneas de coeficiente constante; Y comprender el método de cálculo para las ecuaciones diferenciales lineales homogéneas de orden superior.
    \item Saber aplicar el método de cálculo de ecuaciones diferenciales para resolver problemas simples de aplicación geométrica y física.
    \item Resolver correctamente ciertos tipos de ecuaciones diferenciales utilizando transformadas de Laplace.

   \end{learningoutcomes}
\end{unit}

\begin{coursebibliography}
\bibfile{BasicSciences/MA101}
\end{coursebibliography}

\end{syllabus}
