\begin{syllabus}

\course{MA307. Matemática aplicada a la computación}{Obligatorio}{MA307} % Common.pm

\begin{justification}
Este curso es importante porque desarrolla tópicos del Álgebra Lineal y de Ecuaciones Diferenciales Ordinarias útiles en todas aquellas áreas de la ciencia de la computación donde se trabaja con sistemas lineales y sistemas dinámicos.
\end{justification}

\begin{goals}
\item Que el alumno tenga la base matemática para el modelamiento de sistemas lineales y sistemas dinámicos necesarios en el àrea de Computaciòn Gràfica e Inteligencia Artificial.
\end{goals}

\begin{outcomes}
  \item \ShowOutcome{a}{1}
  \item \ShowOutcome{i}{1}
  \item \ShowOutcome{j}{2}
\end{outcomes}

\begin{competences}
    \item \ShowCompetence{C1}{a} 
    \item \ShowCompetence{C20}{i}
    \item \ShowCompetence{CS2}{j}
\end{competences}

\begin{unit}{}{Espacios Lineales}{Strang03, Apostol73}{0}{C1}
\begin{topics}
      \item Espacios vectoriales.
      \item Independencia, base y dimensión.
      \item Dimensiones y ortogonalidad de los cuatro subespacios.
      \item Aproximaciones por mínimos cuadrados.
      \item Proyecciones
      \item Bases ortogonales y Gram-Schmidt
   \end{topics}
   \begin{learningoutcomes}
      \item Identificar espacios generados por vectores linealmente independientes[\Usage]
      \item Construir conjuntos de vectores ortogonales[\Usage]
      \item Aproximar funciones por polinomios trigonométricos[\Usage]
   \end{learningoutcomes}
\end{unit}

\begin{unit}{}{Transformaciones lineales}{Strang03, Apostol73}{0}{C20}
\begin{topics}
      \item Concepto de transformación lineal.
      \item Matriz de una transformación lineal.
      \item Cambio de base.
      \item Diagonalización y pseudoinversa
   \end{topics}
   \begin{learningoutcomes}
      \item Determinar el núcleo y la imagen de una transformación[\Usage]
      \item Construir la matriz de una transformación[\Usage]
      \item Determinar la matriz de cambio de base[\Usage]
      \end{learningoutcomes}
\end{unit}

\begin{unit}{}{Autovalores y autovectores}{Strang03, Apostol73}{0}{C24}
\begin{topics}
      \item Diagonalización de una matriz
      \item Matrices simétricas
      \item Matrices definidas positivas
      \item Matrices similares
      \item La descomposición de valor singular
  \end{topics}
 \begin{learningoutcomes}
      \item Encontrar la representación diagonal de una matriz[\Usage]
      \item Determinar la similaridad entre matrices[\Usage]
      \item Reducir una forma cuadrática real a diagonal[\Usage]
   \end{learningoutcomes}
\end{unit}

\begin{unit}{}{Sistemas de ecuaciones diferenciales}{Zill02,Apostol73}{0}{C1}
\begin{topics}
      \item Exponencial de una matriz
      \item Teoremas de existencia y unicidad para sistemas lineales homogéneos con coeficientes constantes
      \item Sistemas lineales no homogéneas con coeficientes constantes.
   \end{topics}
\begin{learningoutcomes}
      \item Hallar la solución general de un sistema lineal no  homogéneo[\Usage]
      \item Resolver problemas donde intervengan sistemas de ecuaciones diferenciales[\Usage]
   \end{learningoutcomes}
\end{unit}

\begin{unit}{}{Teoría fundamental}{Hirsh74}{0}{C20}
\begin{topics}
      \item Sistemas dinámicos
      \item El teorema fundamental
      \item Existencia y unicidad
      \item El flujo de una ecuación diferencial
   \end{topics}
   \begin{learningoutcomes}
      \item Discutir la existencia y la unicidad de una ecuación diferencial[\Usage]
      \item Analizar la continuidad de las soluciones[\Usage]
      \item Estudiar la prolongación de una solución[\Usage]

   \end{learningoutcomes}
\end{unit}

\begin{unit}{}{Estabilidad de equilibrio}{Zill02, Hirsh74}{0}{C24}
\begin{topics}
      \item Estabilidad
      \item Funciones de Liapunov
      \item Sistemas gradientes
   \end{topics}
   \begin{learningoutcomes}
      \item Analizar la estabilidad de una solución[\Usage]
      \item Hallar la función de Liapunov para puntos de  equilibrio[\Usage]
      \item Trazar el retrato de fase un flujo gradiente[\Usage]
    \end{learningoutcomes}
\end{unit}



\begin{coursebibliography}
\bibfile{BasicSciences/MA307}
\end{coursebibliography}

\end{syllabus}
