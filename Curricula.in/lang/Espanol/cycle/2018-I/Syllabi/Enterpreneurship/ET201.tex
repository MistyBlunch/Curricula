\begin{syllabus}

\course{ET201. Formación de Empresas de Base Tecnológica I}{Obligatorio}{ET201}

\begin{justification}
Este es el primer curso dentro del área formación de empresas de
base tecnológica, tiene como objetivo dotar al futuro profesional 
de conocimientos, actitudes y aptitudes que le
permitan elaborar un plan de negocio para una empresa de base tecnológica.
El curso está dividido en las siguientes unidades:
Introducción, Creatividad, De la idea a la oportunidad, el modelo Canvas, Customer Development y Lean Startup, Aspectos Legales y Marketing, Finanzas de la empresa y Presentación.

Se busca aprovechar el potencial creativo e innovador y el esfuerzo de los alumnos en la creación de nuevas empresas.
\end{justification}

\begin{goals}
\item Que el alumno conozca como elaborar un plan de negocio para dar inicio a una empresa de base tecnológica.
\item Que el alumno sea capaz de realizar, usando modelos de negocio, la concepción y presentación de una propuesta de negocio.
\end{goals}

%% outcomes (a -> )
%% (1) familiar  (2)usar (3)evaluar
\begin{outcomes}
    \item \ShowOutcome{d}{2}
    \item \ShowOutcome{f}{3}
    \item \ShowOutcome{i}{2}
    \item \ShowOutcome{k}{3}
    \item \ShowOutcome{m}{3}
\end{outcomes}

%% competencias (C1 -> C25)
\begin{competences}
    \item \ShowCompetence{C2}{d} %%{2}
    \item \ShowCompetence{C10}{f} %%{1}
    \item \ShowCompetence{C17}{f}
    \item \ShowCompetence{C18}{i}
    \item \ShowCompetence{C19}{i}
    \item \ShowCompetence{C20}{k}
    \item \ShowCompetence{C23}{k}
    \item \ShowCompetence{CS5}{m}
    
\end{competences}

%%%%%%%%%%%%%%%%%%%%%%%%%
%%
%% \begin{unit}{TítuloUnidad}{Bibliografia1,Bibliografia2}{#horas}{Competencia}
%%   \begin{topics}
%%     \item Topico1
%%	   \item Topico2
%%   \end{topics}
%%   \begin{learningoutcomes}
%%     \item Outcome1 [Nivel]
%%	   \item Outcome2 [Nivel]
%%   \end{learningoutcomes}
%% \end{unit}
%%
%%%%%%%%%%%%%%%%%%%%%%%%%

%%%%%%%%%%%%

\begin{unit}{}{Introducción}{byers10,osterwalder10,garzozi14}{5}{C2}
\begin{topics}
      \item Emprendedor, emprendedurismo e innovación tecnológica
      \item Modelos de negocio
      \item Formación de equipos
   \end{topics}

    \begin{learningoutcomes} 
      \item Identificar características de los emprendedores  [\Familiarity]
      \item Introducir modelos de negocio  [\Familiarity]
    \end{learningoutcomes} 
\end{unit}

\begin{unit}{}{Creatividad}{byers10,blank12,garzozi14}{5}{C10}
\begin{topics}
      \item Visión
      \item Misión
      \item La Propuesta de valor
      \item Creatividad e invención
      \item Tipos y fuentes de innovación
      \item Estrategia y Tecnología
      \item Escala y ámbito
   \end{topics}

    \begin{learningoutcomes} 
      \item Plantear correctamente la vision y misión de empresa  [\Usage]
	  \item Caracterizar una propuesta de valor innovadora  [\Assessment]
      \item Identificar los diversos tipos y fuentes de innovación  [\Familiarity]
   \end{learningoutcomes} 
\end{unit}

\begin{unit}{}{De la Idea a la Oportunidad}{byers10,osterwalder10,ries11,garzozi14}{5}{C17}
\begin{topics}
      \item Estrategia de la Empresa
      \item Barreras 
      \item Ventaja competitiva sostenible
      \item Alianzas
      \item Aprendizaje organizacional
      \item Desarrollo y diseño de productos
   \end{topics}

  \begin{learningoutcomes} 
      \item Conocer estrategias empresariales  [\Familiarity]
      \item Caracterizar barreras y ventajas competitivas   [\Familiarity]
       
    \end{learningoutcomes} 
\end{unit}

\begin{unit}{}{El Modelo Canvas}{osterwalder10,blank12,garzozi14}{20}{C18}
	\begin{topics}
      \item Creación de un nuevo negocio
      \item El plan de negocio 
      \item Canvas
      \item Elementos del Canvas
   \end{topics}

  \begin{learningoutcomes} 
      \item Conocer los elementos del modelo Canvas  [\Usage]
      \item Elaborar un plan de negocio basado en el modelo Canvas  [\Usage]
    \end{learningoutcomes} 
\end{unit}

\begin{unit}{}{Customer Development y Lean Startup}{blank12,ries11,garzozi14}{20}{C19}
	\begin{topics}
      \item Aceleración versus incubación  
      \item Customer Development
      \item Lean Startup 
   \end{topics}

   \begin{learningoutcomes} 
      \item Conocer y aplicar el modelo Customer Development  [\Usage]
      \item Conocer y aplicar el modelo Lean Startup  [\Usage]
    \end{learningoutcomes} 
\end{unit}

\begin{unit}{}{Aspectos Legales y Marketing}{byers10,ries11,congreso96, congreso97,garzozi14}{5}{C20}
	\begin{topics}
	  \item Aspectos Legales y tributarios para la constitución de la empresa
      \item Propiedad intelectual
      \item Patentes
      \item Copyrights y marca registrada
      \item Objetivos de marketing  y segmentos de mercado
      \item Investigación de mercado y búsqueda de clientes
   \end{topics}

  \begin{learningoutcomes} 
      \item Conocer los aspectos legales necesarios para la formación de una empresa tecnológica  [\Familiarity]
      \item Identificar segmentos de mercado y objetivos de marketing     [\Familiarity]
   \end{learningoutcomes} 
\end{unit}

\begin{unit}{}{Finanzas de la Empresa}{byers10,blank12,garzozi14}{5}{C23}
	\begin{topics}
      \item Modelo de costos
      \item Modelo de utilidades
      \item Precio
      \item Plan financiero
      \item Formas de financiamiento
      \item Fuentes de capital
      \item Capital de riesgo
   \end{topics}

   \begin{learningoutcomes} 
      \item Definir um modelo de costos y utilidades  [\Assessment]
      \item Conocer las diversas fuentes de financiamento  [\Familiarity]
   \end{learningoutcomes} 
\end{unit}

\begin{unit}{}{Presentación}{byers10,blank12,garzozi14}{5}{CS5}
	\begin{topics}
      \item The Elevator Pitch
      \item Presentación
      \item Negociación
    \end{topics}

   \begin{learningoutcomes} 
      \item Conocer las diversas formas de presentar propuestas de negocio  [\Familiarity]
      \item Realizar la presentación de una propuesta de negocio  [\Usage]
   \end{learningoutcomes} 
\end{unit}



\begin{coursebibliography}
\bibfile{Enterpreneurship/ET201}
\end{coursebibliography}

\end{syllabus}

%\end{document}
