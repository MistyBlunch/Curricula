\begin{syllabus}

\course{CB101. �lgebra y Geometr�a}{Obligatorio}{CB101}

\begin{justification}
Este laboratorio est� orientado a consolidar las habilidades comunicativas del estudiante, tanto a nivel oral como escrito en el marco de la disciplina que se estudia. En particular, el estudiante fortalecer� sus capacidades expositivas al ejercitarse en toda la primera parte del curso en la escritura de un tipo de texto que
desarrollar� a lo largo de su carrera como ingeniero: los informes de laboratorio. Reflexionar� sobre la situaci�n ret�rica que enfrenta al escribir este tipo de texto: qui�n ser� su lector, cu�l es la intenci�n comunicativa de ese texto y el tema sobre el que est� escribiendo.
En una segunda parte, el curso se presenta como un espacio de discusi�n sobre el discurso argumentativo y de lectura cr�tica de textos argumentativos, para que el alumno reflexione, conozca y emplee las herramientas comunicativas para producir textos argumentativos formales. En este sentido, el curso se orienta hacia la producci�n
permanente de textos escritos y orales, por lo que el alumno participar� no solo en foros de discusi�n sino que se espera que sea capaz de debatir con sus compa�eros sobre un tema propuesto por el profesor. En suma, el curso busca consolidar las competencias de lectura, an�lisis y elaboraci�n de textos escritos y orales, tanto expositivos como argumentativos.
\end{justification}

\begin{goals}
\item Desarrollar habilidades que les permitan a los estudiantes mejorar sus capacidades comunicativas, tanto orales como escritas.
\item Comprender y producir textos expositivos en los que informen sobre la aplicaci�n del conocimiento te�rico en un experimento o contexto diferente.
\item Comprender y producir textos argumentativos orales y escritos.
\item Se capaz de debatir empleando argumentos s�lidos.
\item Emplear adecuadamente y reflexivamente la informaci�n obtenida en diferentes fuentes.
\item Mostrar apertura y respeto para escuchar la diversidad de opiniones o puntos de vista de los compa�eros de clase.

(informe de laboratorio)
\end{goals}

\begin{outcomes}
\ExpandOutcome{a}{3}
\ExpandOutcome{i}{2}
\ExpandOutcome{j}{4}
\end{outcomes}

\begin{unit}{Sistemas de coordenadas. La recta.}{Lehmann05}{12}{4}
   \begin{topics}
      \item 
      \item 
   \end{topics}
   \begin{unitgoals}
      \item 
   \end{unitgoals}
\end{unit}

\begin{unit}{C�nicas y Coordenadas polares}{Lehmann05}{24}{3}
   \begin{topics}
      \item 
      \item 
   \end{topics}

   \begin{unitgoals}
      \item 
      \item
      \item 
      \end{unitgoals}
\end{unit}

\begin{unit}{Sistemas de ecuaciones. Matrices y determinantes}{Strang03,Grossman96}{24}{3}
   \begin{topics}
      \item 
      \item 
      \item 
      \end{topics}

   \begin{unitgoals}
      \item 
      \item 
      \item 
     
   \end{unitgoals}
\end{unit}

\begin{unit}{Vectores en $R^2$ y vectores en $R^3$}{Grossman96}{30}{3}
   \begin{topics}
      \item 
      \item 
   \end{topics}

   \begin{unitgoals}
      \item 
      \item 
      \item 
   \end{unitgoals}
\end{unit}



\begin{coursebibliography}
\bibfile{GeneralFormation/FG101B}
\end{coursebibliography}

\end{syllabus}
