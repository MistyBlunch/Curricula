\begin{syllabus}

\course{CS355. Tópicos en Computación Gráfica}{Electivo}{CS355}

\begin{justification}
En este curso se puede profundizar en alguno de los tópicos
mencionados en el área de Computación Gráfica (\textit{Graphics and Visual
Computing} - GV).

Éste curso está destinado a realizar algun curso avanzado sugerido
por la curricula de la ACM/IEEE. \nocite{Foley90,HearnAndBaker94}
\end{justification}

\begin{goals}
\item Que el alumno utilice técnicas de computación gráfica más sofisticadas que involucren estructuras de datos y algoritmos complejos.
\item Que el alumno aplique los conceptos aprendidos para crear una aplicación sobre un problema real.
\item Que el alumno investigue la posibilidad de crear un nuevo algoritmo y/o técnica nueva para resolver un problema real.
\end{goals}

\begin{outcomes}
\ExpandOutcome{a}{4}
\ExpandOutcome{b}{4}
\ExpandOutcome{i}{4}
\ExpandOutcome{j}{4}
\end{outcomes}

\begin{itemize}
\item CS355. Computación Gráfica avanzada
\item CS356. Animación por computadora
\item CS313. Algoritmos Geométricos
\item CS357. Visualización
\item CS358. Realidad Virtual
\item CS359. Algoritmos Genéticos
\end{itemize}



\begin{coursebibliography}
\bibfile{Computing/CS/CS255}
\end{coursebibliography}

\end{syllabus}
