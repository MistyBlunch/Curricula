\begin{syllabus}

\course{CS3P3. Internet de las Cosas}{Obligatorio}{CS3P3}

\begin{justification}
La última década ha traído un crecimiento explosivo en computación con multiprocesadores, incluyendo 
los procesadores de varios núcleos y centros de datos distribuidos. Como resultado, la computación 
paralela y distribuida se ha convertido de ser un tema ampliamente electivo para ser uno de los principales componentes
en la malla estudios en ciencia de la computación de pregrado. Tanto la computación paralela como la distribuida implica 
la ejecución simultánea de múltiples procesos en diferentes dispositivos que cambian de posición.
\end{justification}

\begin{goals}
\item Que el alumno sea capaz de crear aplicaciones paralelas de mediana complejidad aprovechando eficientemente distintos dispositivos móviles.
% \item Que el alumno sea capaz de comparar aplicaciones secuenciales y paralelas.
% \item Que el alumno sea capaz de convertir, cuando la situación lo amerite, aplicaciones secuenciales a paralelas de forma eficiente.
\end{goals}

\begin{outcomes}
    \item \ShowOutcome{a}{2} 
    \item \ShowOutcome{b}{2} % Analizar problemas e identificar y definir los requerimientos computacionales apropiados para su solución
    \item \ShowOutcome{i}{2} % Utilizar técnicas y herramientas actuales necesarias para la práctica de la computación
    \item \ShowOutcome{j}{2} % Aplicar la base matemática, principios de algoritmos y la teoría de la Ciencia de la Computación en el modelamiento y diseño de sistemas computacionales de tal manera que demuestre comprensión de los puntos de equilibrio involucrados en la opción escogida.
\end{outcomes}

\begin{competences}
    \item \ShowCompetence{C2}{a} % Capacidad para tener una perspectiva crítica y creativa para identificar y resolver problemas utilizando el pensamiento computacional
    \item \ShowCompetence{C4}{b}
    \item \ShowCompetence{C16}{i} % Capacidad para identificar temas avanzados de computación y de la comprensión de las fronteras de la disciplina. 
    \item \ShowCompetence{CS2}{i} % Identificar y analizar los criterios y especificaciones apropiadas a los problemas específicos, y planificar estrategias para su solución.
    \item \ShowCompetence{CS3}{j} % Analizar el grado en que un sistema basado en el ordenador cumple con los criterios definidos para su uso actual y futuro desarrollo.
  \item \ShowCompetence{CS6}{j} % Evaluar los sistemas en términos de atributos de calidad en general y las posibles ventajas y desventajas que se presentan en el problema dado.
\end{competences}

\begin{unit}{\PDParallelismFundamentals}{}{peterpacheco,matloff,quinn}{18}{C2}
\begin{topics}%
    \item \PDParallelismFundamentalsTopicMultiple
    \item \PDParallelismFundamentalsTopicGoals
    \item \PDParallelismFundamentalsTopicParallelism
    \item \PDParallelismFundamentalsTopicProgramming
\end{topics}    
\begin{learningoutcomes}%
    \item \PDParallelismFundamentalsLODistinguishUsing~[\Familiarity] %
    \item \PDParallelismFundamentalsLODistinguishMultiple~[\Familiarity] %
    \item \PDParallelismFundamentalsLODistinguishData~[\Familiarity] %
\end{learningoutcomes}%
\end{unit}

\begin{unit}{\PDParallelArchitecture}{}{peterpacheco,wenmei,sanders}{12}{C4}
\begin{topics}%
    \item \PDParallelArchitectureTopicMulticore
    \item \PDParallelArchitectureTopicShared
    \item \PDParallelArchitectureTopicSymmetric
    \item \PDParallelArchitectureTopicSimd
    \item \PDParallelArchitectureTopicGpu
    \item \PDParallelArchitectureTopicFlynns
    \item \PDParallelArchitectureTopicInstruction
    \item \PDParallelArchitectureTopicMemory
    \item \PDParallelArchitectureTopicTopologies
\end{topics}
\begin{learningoutcomes}%
    \item \PDParallelArchitectureLOExplainTheShared~[\Assessment] %
    \item \PDParallelArchitectureLODescribeTheAndKey~[\Assessment] %
    \item \PDParallelArchitectureLOCharacterizeTheTasks~[\Usage] %
    \item \PDParallelArchitectureLODescribeTheLimitationsVs~[\Usage] %
    \item \PDParallelArchitectureLOExplainTheEach~[\Usage] %
    \item \PDParallelArchitectureLODescribeTheMaintaining~[\Familiarity] %
    \item \PDParallelArchitectureLODescribeTheChallenges~[\Familiarity] %
\end{learningoutcomes}%
\end{unit}

\begin{unit}{\PDParallelDecomposition}{}{peterpacheco,matloff,quinn}{18}{C16}
\begin{topics}%
    \item \PDParallelDecompositionTopicNeed
    \item \PDParallelDecompositionTopicIndependence
    \item \PDParallelDecompositionTopicBasic
    \item \PDParallelDecompositionTopicTask
    \item \PDParallelDecompositionTopicData
    \item \PDParallelDecompositionTopicActors
\end{topics}
\begin{learningoutcomes}%
    \item \PDParallelDecompositionLOExplainWhyNecessary~[\Usage] %
    \item \PDParallelDecompositionLOIdentifyOpportunities~[\Familiarity] %
    \item \PDParallelDecompositionLOWriteAScalable~[\Usage] %
    \item \PDParallelDecompositionLOParallelize~[\Usage] %
    \item \PDParallelDecompositionLOParallelizeAn~[\Usage] %
    \item \PDParallelDecompositionLOWriteAActors~[\Usage] %
\end{learningoutcomes}%
\end{unit}

\begin{unit}{\PDCommunicationandCoordination}{}{peterpacheco,matloff,quinn}{18}{C16}
\begin{topics}%
    \item \PDCommunicationandCoordinationTopicShared
    \item \PDCommunicationandCoordinationTopicConsistency
    \item \PDCommunicationandCoordinationTopicMessage
    \item \PDCommunicationandCoordinationTopicAtomicity
    \item \PDCommunicationandCoordinationTopicConsensus
    \item \PDCommunicationandCoordinationTopicConditional
\end{topics}
\begin{learningoutcomes}%
    \item \PDCommunicationandCoordinationLOUseMutual~[\Usage] %
    \item \PDCommunicationandCoordinationLOGiveAn~[\Familiarity] %
    \item \PDCommunicationandCoordinationLOGiveAnA~[\Usage] %
    \item \PDCommunicationandCoordinationLOExplainWhenMulticast~[\Familiarity] %
    \item \PDCommunicationandCoordinationLOWriteACorrectly~[\Usage] %
    \item \PDCommunicationandCoordinationLOGiveAnAWhich~[\Familiarity] %
    \item \PDCommunicationandCoordinationLOUseSemaphores~[\Usage] %
\end{learningoutcomes}%
\end{unit}

\begin{unit}{\PDParallelAlgorithmsAnalysisandProgramming}{}{matloff,quinn}{18}{CS2}
\begin{topics}%
    \item \PDParallelAlgorithmsAnalysisandProgrammingTopicCritical
    \item \PDParallelAlgorithmsAnalysisandProgrammingTopicSpeed
    \item \PDParallelAlgorithmsAnalysisandProgrammingTopicNaturally
    \item \PDParallelAlgorithmsAnalysisandProgrammingTopicParallel
    \item \PDParallelAlgorithmsAnalysisandProgrammingTopicParallelGraph
    \item \PDParallelAlgorithmsAnalysisandProgrammingTopicParallelMatrix
    \item \PDParallelAlgorithmsAnalysisandProgrammingTopicProducer
    \item \PDParallelAlgorithmsAnalysisandProgrammingTopicExamples
\end{topics}
\begin{learningoutcomes}%
        \item \PDParallelAlgorithmsAnalysisandProgrammingLODefineCritical~[\Familiarity] %
        \item \PDParallelAlgorithmsAnalysisandProgrammingLOComputeTheSpan~[\Usage] %
        \item \PDParallelAlgorithmsAnalysisandProgrammingLODefineSpeed~[\Familiarity] %
        \item \PDParallelAlgorithmsAnalysisandProgrammingLOIdentifyIndependent~[\Usage] %
        \item \PDParallelAlgorithmsAnalysisandProgrammingLOCharacterizeFeatures~[\Familiarity] %
        \item \PDParallelAlgorithmsAnalysisandProgrammingLOImplementAAnd~[\Usage] %
        \item \PDParallelAlgorithmsAnalysisandProgrammingLODecompose~[\Usage] %
        \item \PDParallelAlgorithmsAnalysisandProgrammingLOProvideAn~[\Usage] %
        \item \PDParallelAlgorithmsAnalysisandProgrammingLOGiveExamplesWhere~[\Usage] %
        \item \PDParallelAlgorithmsAnalysisandProgrammingLOImplementAAlgorithm~[\Usage] %
        \item \PDParallelAlgorithmsAnalysisandProgrammingLOIdentifyIssuesIn~[\Usage] %
\end{learningoutcomes}%
\end{unit}

\begin{unit}{\PDParallelPerformance}{}{peterpacheco,matloff,wenmei,sanders}{18}{CS3}
\begin{topics}%
    \item \PDParallelPerformanceTopicLoad
    \item \PDParallelPerformanceTopicPerformance
    \item \PDParallelPerformanceTopicScheduling
    \item \PDParallelPerformanceTopicEvaluating
    \item \PDParallelPerformanceTopicData
    \item \PDParallelPerformanceTopicPower
\end{topics}
\begin{learningoutcomes}%
    \item \PDParallelPerformanceLODetect~[\Usage] %
    \item \PDParallelPerformanceLOCalculateThe~[\Usage] %
    \item \PDParallelPerformanceLODescribeHowLayout~[\Familiarity] %
    \item \PDParallelPerformanceLODetectAnd~[\Usage] %
    \item \PDParallelPerformanceLOExplainTheScheduling~[\Familiarity] %
    \item \PDParallelPerformanceLOExplainPerformance~[\Familiarity] %
    \item \PDParallelPerformanceLOExplainTheTrade~[\Familiarity] %
\end{learningoutcomes}%
\end{unit}



\begin{coursebibliography}
\bibfile{Computing/CS/CS3P1}
\end{coursebibliography}

\end{syllabus}
