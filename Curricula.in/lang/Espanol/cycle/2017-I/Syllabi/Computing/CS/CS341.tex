\begin{syllabus}

\course{CS341. Lenguajes de Programaci�n}{Obligatorio}{CS341}

\begin{justification}
Los lenguajes de programaci�n son el medio a trav�s del cual los programadores describen con precisi�n los conceptos, 
formulan algoritmos y representan sus soluciones. Un cient�fico de la computaci�n trabajar� con diferentes lenguajes, 
por separado o en conjunto. Los cient�ficos de la computaci�n deben entender los modelos de programaci�n de los diferentes 
lenguajes, tomar decisiones de dise�o basados en el lenguaje de programaci�n y sus conceptos. El profesional a menudo 
necesitar� aprender nuevos lenguajes y construcciones de programaci�n y debe entender los fundamentos de como las 
caracter�sticas del lenguaje de programaci�n estan definidas, compuestas e implementadas. El uso eficaz de los lenguajes 
de programaci�n y la apreciaci�n de sus limitaciones, tambi�n requiere un conocimiento b�sico de traducci�n de lenguajes 
de programaci�n y su an�lisis de ambientes est�ticos y din�micos, as� como los componentes de tiempo de ejecuci�n tales 
como la gesti�n de memoria, entre otros detalles de relevancia.
\end{justification}

\begin{goals}
\item Capacitar a los estudiantes para entender los lenguajes de programaci�n desde diferentes tipos de vista, seg�n el 
modelo subyacente, los componentes fundamentales presentes en todo lenguaje de programaci�n y como objetos formales 
dotados de una estructura y un significado seg�n diversos enfoques.
\end{goals}

\begin{outcomes}
    \item \ShowOutcome{a}{2}
    \item \ShowOutcome{b}{2}
    \item \ShowOutcome{i}{2} 
    \item \ShowOutcome{j}{2} 
		
\end{outcomes}

\begin{competences}
    \item \ShowCompetence{C2}{a} 
    \item \ShowCompetence{CS1}{b}
    \item \ShowCompetence{CS2}{i}
    \item \ShowCompetence{CS3}{j}
    \item \ShowCompetence{CS6}{j}
\end{competences}

\begin{unit}{}{Evoluci�n de los lenguajes de programaci�n}{robsebesta, webber}{18}{C2, CS1, CS2}
\begin{topics}
	\item Historia de los Lenguajes de Programaci�n
	\item \PLProgramRepresentationTopicPrograms
	\item \PLProgramRepresentationTopicData
	\item Estructura de un programa: L�xico, Sint�ctico y Sem�ntico
	\item BNF
	\item \PLLanguageTranslationandExecutionTopicInterpretation [\Familiarity]
\end{topics}
\begin{learningoutcomes}
	\item Reconocer el desarrollo hist�rico de los lenguajes de programaci�n. [\Familiarity]
	\item Identificar los paradigmas que agrupan a la mayor�a de lenguajes de programaci�n existentes hoy en d�a. [\Familiarity]
	\item \PLProgramRepresentationLOExplainHowProcess [\Familiarity]
	\item \PLProgramRepresentationLODescribeAnTree [\Familiarity]
	\item \PLProgramRepresentationLOWriteAProcess [\Usage]	
	\item \PLLanguageTranslationandExecutionLODistinguishA [\Familiarity]
	\item Reconocer como funciona un programa a nivel de computador. [\Familiarity]
\end{learningoutcomes}
\end{unit}

\begin{unit}{\PLLanguagePragmatics}{}{robsebesta, webber, vanroy}{12}{C2, CS1, CS2}
\begin{topics}%
    \item \PLLanguagePragmaticsTopicPrinciples
    \item \PLLanguagePragmaticsTopicEvaluation
    \item \PLLanguagePragmaticsTopicEager
    \item \PLLanguagePragmaticsTopicDefining
    \item \PLLanguagePragmaticsTopicExternal
\end{topics}
\begin{learningoutcomes}%
    \item \PLLanguagePragmaticsLODiscussTheConcepts [\Usage]
    \item \PLLanguagePragmaticsLOUseCrisp [\Usage]
    \item \PLLanguagePragmaticsLOGiveAnWhose [\Usage]
    \item \PLLanguagePragmaticsLOShowUses [\Familiarity]
    \item \PLLanguagePragmaticsLODiscussTheAllowing [\Familiarity] %
\end{learningoutcomes}%
\end{unit}

\begin{unit}{\PLTypeSystems}{}{robsebesta, webber, vanroy}{18}{C2, CS1, CS2}
\begin{topics}%
    \item \PLTypeSystemsTopicCompositional
    \item \PLTypeSystemsTopicType
    \item \PLTypeSystemsTopicTypeSafety
    \item \PLTypeSystemsTopicTypeInference
    \item \PLTypeSystemsTopicStatic
\end{topics}
\begin{learningoutcomes}%
        \item \PLTypeSystemsLODefineAPrecisely [\Usage] %
        \item \PLTypeSystemsLOForVarious [\Familiarity] %
        \item \PLTypeSystemsLOPrecisely [\Familiarity] %
        \item \PLTypeSystemsLOProveType [\Usage] %
        \item \PLTypeSystemsLOImplementAType [\Usage] %
        \item \PLTypeSystemsLOExplainHowAndAlgorithms [\Familiarity] %
\end{learningoutcomes}%
\end{unit}

\begin{unit}{\PLObjectOrientedProgramming}{}{robsebesta, webber, vanroy}{12}{CS2, CS3, CS6}
\begin{topics}%
    \item \PLObjectOrientedProgrammingTopicObject
    \item \PLObjectOrientedProgrammingTopicDefinition
    \item \PLObjectOrientedProgrammingTopicSubclasses
    \item \PLObjectOrientedProgrammingTopicDynamic
    \item \PLObjectOrientedProgrammingTopicSubtyping
    \item \PLObjectOrientedProgrammingTopicObjectOriented
    \item \PLObjectOrientedProgrammingTopicUsing
\end{topics}
\begin{learningoutcomes}%
    \item \PLObjectOrientedProgrammingLODesignAndClass [\Usage]
    \item \PLObjectOrientedProgrammingLOUseSubclassing [\Usage]
    \item \PLObjectOrientedProgrammingLOCorrectly [\Usage]
    \item \PLObjectOrientedProgrammingLOCompareAndThe [\Assessment]
    \item \PLObjectOrientedProgrammingLOExplainTheObject [\Usage] 
    \item \PLObjectOrientedProgrammingLOUseObject [\Usage] 
    \item \PLObjectOrientedProgrammingLODefineAndAnd [\Usage] 
\end{learningoutcomes}%
\end{unit}

\begin{unit}{\PLFunctionalProgramming}{}{robsebesta, webber, vanroy}{18}{CS2, CS3, CS6}
\begin{topics}%
    \item \PLFunctionalProgrammingTopicEffect
    \item \PLFunctionalProgrammingTopicProcessing
    \item \PLFunctionalProgrammingTopicFirst
    \item \PLFunctionalProgrammingTopicFunction
    \item \PLFunctionalProgrammingTopicDefining
\end{topics}
\begin{learningoutcomes}%
        \item \PLFunctionalProgrammingLOWriteBasic [\Usage] %
        \item \PLFunctionalProgrammingLOWriteUseful [\Usage] %
        \item \PLFunctionalProgrammingLOCompareAndTheApproach [\Assessment] %
        \item \PLFunctionalProgrammingLOCorrectlyReason [\Usage] %
        \item \PLFunctionalProgrammingLOUseFunctional [\Usage] %
        \item \PLFunctionalProgrammingLODefineAndAndOn [\Usage] %
\end{learningoutcomes}%
\end{unit}

\begin{unit}{\PLEventDrivenandReactiveProgramming}{}{robsebesta}{12}{CS2, CS3, CS6}
\begin{topics}%
        \item \PLEventDrivenandReactiveProgrammingTopicEvents%
        \item \PLEventDrivenandReactiveProgrammingTopicCanonical%
        \item \PLEventDrivenandReactiveProgrammingTopicUsingA%
        \item \PLEventDrivenandReactiveProgrammingTopicExternally%
        \item \PLEventDrivenandReactiveProgrammingTopicSeparation%
\end{topics}
\begin{learningoutcomes}%
        \item \PLEventDrivenandReactiveProgrammingLOWriteEvent [\Usage] %
        \item \PLEventDrivenandReactiveProgrammingLOExplainWhyDriven [\Familiarity] %
        \item \PLEventDrivenandReactiveProgrammingLODescribeAn [\Familiarity] %
\end{learningoutcomes}%
\end{unit}

\begin{unit}{\PLLogicProgramming}{}{robsebesta, webber, vanroy}{12}{CS2, CS3, CS6}
\begin{topics}%
        \item \PLLogicProgrammingTopicCausal%
        \item \PLLogicProgrammingTopicUnification%
        \item \PLLogicProgrammingTopicBactracking%
        \item \PLLogicProgrammingTopicCuts%
\end{topics}
\begin{learningoutcomes}%
        \item \PLLogicProgrammingLOUseAToConventional [\Usage] %
        \item \PLLogicProgrammingLOUseAToAlgorithm [\Usage] %
\end{learningoutcomes}%
\end{unit}



\begin{coursebibliography}
\bibfile{Computing/CS/CS341}
\end{coursebibliography}

\end{syllabus}
