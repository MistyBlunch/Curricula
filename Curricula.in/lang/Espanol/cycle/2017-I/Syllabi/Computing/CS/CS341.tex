\begin{syllabus}

\course{CS341. Lenguajes de Programación}{Obligatorio}{CS341} % Common.pm

\begin{justification}
Los lenguajes de programación son el medio a través del cual los programadores describen con precisión los conceptos, 
formulan algoritmos y representan sus soluciones. Un científico de la computación trabajará con diferentes lenguajes, 
por separado o en conjunto. Los científicos de la computación deben entender los modelos de programación de los diferentes 
lenguajes, tomar decisiones de diseño basados en el lenguaje de programación y sus conceptos. El profesional a menudo 
necesitará aprender nuevos lenguajes y construcciones de programación y debe entender los fundamentos de como las 
características del lenguaje de programación estan definidas, compuestas e implementadas. El uso eficaz de los lenguajes 
de programación y la apreciación de sus limitaciones, también requiere un conocimiento básico de traducción de lenguajes 
de programación y su análisis de ambientes estáticos y dinámicos, así como los componentes de tiempo de ejecución tales 
como la gestión de memoria, entre otros detalles de relevancia.
\end{justification}

\begin{goals}
\item Capacitar a los estudiantes para entender los lenguajes de programación desde diferentes tipos de vista, según el 
modelo subyacente, los componentes fundamentales presentes en todo lenguaje de programación y como objetos formales 
dotados de una estructura y un significado según diversos enfoques.
\end{goals}

\begin{outcomes}{V1}
    \item \ShowOutcome{a}{2}
    \item \ShowOutcome{b}{2}
    \item \ShowOutcome{i}{2} 
    \item \ShowOutcome{j}{2} 
		
\end{outcomes}

\begin{competences}{V1}
    \item \ShowCompetence{C2}{a} 
    \item \ShowCompetence{CS1}{b}
    \item \ShowCompetence{CS2}{i}
    \item \ShowCompetence{CS3}{j}
    \item \ShowCompetence{CS6}{j}
\end{competences}

\begin{unit}{}{Evolución de los lenguajes de programación}{robsebesta, webber}{18}{C2, CS1, CS2}
\begin{topics}
	\item Historia de los Lenguajes de Programación
	\item \PLProgramRepresentationTopicPrograms
	\item \PLProgramRepresentationTopicData
	\item Estructura de un programa: Léxico, Sintáctico y Semántico
	\item BNF
	\item \PLLanguageTranslationandExecutionTopicInterpretation [\Familiarity]
\end{topics}
\begin{learningoutcomes}
	\item Reconocer el desarrollo histórico de los lenguajes de programación. [\Familiarity]
	\item Identificar los paradigmas que agrupan a la mayoría de lenguajes de programación existentes hoy en día. [\Familiarity]
	\item \PLProgramRepresentationLOExplainHowProcess [\Familiarity]
	\item \PLProgramRepresentationLODescribeAnTree [\Familiarity]
	\item \PLProgramRepresentationLOWriteAProcess [\Usage]	
	\item \PLLanguageTranslationandExecutionLODistinguishA [\Familiarity]
	\item Reconocer como funciona un programa a nivel de computador. [\Familiarity]
\end{learningoutcomes}
\end{unit}

\begin{unit}{\PLLanguagePragmatics}{}{robsebesta, webber, vanroy}{12}{C2, CS1, CS2}
\begin{topics}%
    \item \PLLanguagePragmaticsTopicPrinciples
    \item \PLLanguagePragmaticsTopicEvaluation
    \item \PLLanguagePragmaticsTopicEager
    \item \PLLanguagePragmaticsTopicDefining
    \item \PLLanguagePragmaticsTopicExternal
\end{topics}
\begin{learningoutcomes}%
    \item \PLLanguagePragmaticsLODiscussTheConcepts [\Usage]
    \item \PLLanguagePragmaticsLOUseCrisp [\Usage]
    \item \PLLanguagePragmaticsLOGiveAnWhose [\Usage]
    \item \PLLanguagePragmaticsLOShowUses [\Familiarity]
    \item \PLLanguagePragmaticsLODiscussTheAllowing [\Familiarity] %
\end{learningoutcomes}%
\end{unit}

\begin{unit}{\PLTypeSystems}{}{robsebesta, webber, vanroy}{18}{C2, CS1, CS2}
\begin{topics}%
    \item \PLTypeSystemsTopicCompositional
    \item \PLTypeSystemsTopicType
    \item \PLTypeSystemsTopicTypeSafety
    \item \PLTypeSystemsTopicTypeInference
    \item \PLTypeSystemsTopicStatic
\end{topics}
\begin{learningoutcomes}%
        \item \PLTypeSystemsLODefineAPrecisely [\Usage] %
        \item \PLTypeSystemsLOForVarious [\Familiarity] %
        \item \PLTypeSystemsLOPrecisely [\Familiarity] %
        \item \PLTypeSystemsLOProveType [\Usage] %
        \item \PLTypeSystemsLOImplementAType [\Usage] %
        \item \PLTypeSystemsLOExplainHowAndAlgorithms [\Familiarity] %
\end{learningoutcomes}%
\end{unit}

\begin{unit}{\PLObjectOrientedProgramming}{}{robsebesta, webber, vanroy}{12}{CS2, CS3, CS6}
\begin{topics}%
    \item \PLObjectOrientedProgrammingTopicObject
    \item \PLObjectOrientedProgrammingTopicDefinition
    \item \PLObjectOrientedProgrammingTopicSubclasses
    \item \PLObjectOrientedProgrammingTopicDynamic
    \item \PLObjectOrientedProgrammingTopicSubtyping
    \item \PLObjectOrientedProgrammingTopicObjectOriented
    \item \PLObjectOrientedProgrammingTopicUsing
\end{topics}
\begin{learningoutcomes}%
    \item \PLObjectOrientedProgrammingLODesignAndClass [\Usage]
    \item \PLObjectOrientedProgrammingLOUseSubclassing [\Usage]
    \item \PLObjectOrientedProgrammingLOCorrectly [\Usage]
    \item \PLObjectOrientedProgrammingLOCompareAndThe [\Assessment]
    \item \PLObjectOrientedProgrammingLOExplainTheObject [\Usage] 
    \item \PLObjectOrientedProgrammingLOUseObject [\Usage] 
    \item \PLObjectOrientedProgrammingLODefineAndAnd [\Usage] 
\end{learningoutcomes}%
\end{unit}

\begin{unit}{\PLFunctionalProgramming}{}{robsebesta, webber, vanroy}{18}{CS2, CS3, CS6}
\begin{topics}%
    \item \PLFunctionalProgrammingTopicEffect
    \item \PLFunctionalProgrammingTopicProcessing
    \item \PLFunctionalProgrammingTopicFirst
    \item \PLFunctionalProgrammingTopicFunction
    \item \PLFunctionalProgrammingTopicDefining
\end{topics}
\begin{learningoutcomes}%
        \item \PLFunctionalProgrammingLOWriteBasic [\Usage] %
        \item \PLFunctionalProgrammingLOWriteUseful [\Usage] %
        \item \PLFunctionalProgrammingLOCompareAndTheApproach [\Assessment] %
        \item \PLFunctionalProgrammingLOCorrectlyReason [\Usage] %
        \item \PLFunctionalProgrammingLOUseFunctional [\Usage] %
        \item \PLFunctionalProgrammingLODefineAndAndOn [\Usage] %
\end{learningoutcomes}%
\end{unit}

\begin{unit}{\PLEventDrivenandReactiveProgramming}{}{robsebesta}{12}{CS2, CS3, CS6}
\begin{topics}%
        \item \PLEventDrivenandReactiveProgrammingTopicEvents%
        \item \PLEventDrivenandReactiveProgrammingTopicCanonical%
        \item \PLEventDrivenandReactiveProgrammingTopicUsingA%
        \item \PLEventDrivenandReactiveProgrammingTopicExternally%
        \item \PLEventDrivenandReactiveProgrammingTopicSeparation%
\end{topics}
\begin{learningoutcomes}%
        \item \PLEventDrivenandReactiveProgrammingLOWriteEvent [\Usage] %
        \item \PLEventDrivenandReactiveProgrammingLOExplainWhyDriven [\Familiarity] %
        \item \PLEventDrivenandReactiveProgrammingLODescribeAn [\Familiarity] %
\end{learningoutcomes}%
\end{unit}

\begin{unit}{\PLLogicProgramming}{}{robsebesta, webber, vanroy}{12}{CS2, CS3, CS6}
\begin{topics}%
        \item \PLLogicProgrammingTopicCausal%
        \item \PLLogicProgrammingTopicUnification%
        \item \PLLogicProgrammingTopicBactracking%
        \item \PLLogicProgrammingTopicCuts%
\end{topics}
\begin{learningoutcomes}%
        \item \PLLogicProgrammingLOUseAToConventional [\Usage] %
        \item \PLLogicProgrammingLOUseAToAlgorithm [\Usage] %
\end{learningoutcomes}%
\end{unit}

\begin{coursebibliography}
\bibfile{Computing/CS/CS341}
\end{coursebibliography}

\end{syllabus}
