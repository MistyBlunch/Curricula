\begin{syllabus}

\course{CS343. Lenguajes de Programación}{Obligatorio}{CS343}

\begin{justification}
A pesar de que los algoritmos han sido diseñados y escritos por lo menos
desde el tiempo de Euclides; es que tan sólo en los últimos cincuenta años
(desde el desarrollo de la computadora digital) los métodos de expresar
algoritmos han sido objeto de un extenso estudio. En la actualidad existen
distintos paradigmas de programación, cientos de lenguajes de programación
en uso activo, muchos más en existencia y aún más por ser diseñados.
El propósto de este curso es el dar una introducción a los principios del
estudio de la programación, y brindar los fundamentos básicos en este
tópico. Al brindar un estudio exhaustivo de los principios del diseño de los
lenguajes de programación es que este curso pretende convertir al estudiante
en un mejor programador. Adicionalmente este curso es útil si se necesita
tomar la decisión acerca de que lenguaje de programación usar para un
proyecto, o si alguna vez necesita diseñar su propio lenguaje.
\end{justification}

\begin{goals}
\item Capacitar a los estudiantes para entender los lenguajes de programación desde 
diferentes tipos de vista, según el modelo subyacente, los componentes fundamentales 
presentes en todo lenguaje de programación y como objetos formales dotados de una 
estructura y un significado según diversos enfoques.
\end{goals}

\begin{outcomes}
\ExpandOutcome{a}{3}
\ExpandOutcome{b}{4}
\ExpandOutcome{i}{3}
\ExpandOutcome{j}{4}
\end{outcomes}

\begin{unit}{El desarrollo histórico de los lenguajes de programación}{hen96,rob05,rav96}{4}{2}
\begin{topics}
      \item Historia de los Lenguajes de Programación
      \item Paradigmas existentes
      \item Estructura de un programa: Léxico, Sintáctico y Semántico
      \item BNF
      \item Técnicas para la reducción de la complejidad en los programas
      \item Procesamiento de programas: Interpretación vs Compilación
   \end{topics}

   \begin{unitgoals}
      \item Reconocer el desarrollo histórico de los lenguajes de programación
      \item Identificar los paradigmas que agrupan a la mayoría de lenguajes de programación existentes hoy en día
      \item Discutir entre los distintos paradigmas y establecer sus ventajas y desventajas
      \item Establecer otros enfoques para la clasificación de los lenguajes de programación
      \item Diferenciar entre los distintos enfoques estructurales, desde el análisis léxico hasta el semántico
      \item Identificar el concepto de abstracción entre los distintos paradigmas
      \item Diferenciar entre la compilación y la interpretación en la ejecución de programas
      \item Reconocer como funciona un programa a nivel de computador
   \end{unitgoals}
\end{unit}

\begin{unit}{Lenguajes Imperativos}{hen96,rob05,rav96}{4}{4}
\begin{topics}
      \item Introducción
      \item Manejo de datos y tipos
      \item Asignaciones y Expresiones
      \item Flujos de control
      \item Componentes de un programa imperativo
      \item Ejemplos de programas imperativos
   \end{topics}

   \begin{unitgoals}
      \item Identificar los principios de la programación imperativa
      \item Determinar las bases del imperativismo:secuencialidad, selección y repetición
      \item Aprender cómo los lenguajes imperativos manejan datos y asigna valores
      \item Aprender el concepto de ortogonalidad en un lenguaje de programación
      \item Diseña e implementa un programa en un lenguaje de programación imperativo
   \end{unitgoals}
\end{unit}

\begin{unit}{Orientación a Objetos}{hen96,rob05,rav96}{8}{4}
\begin{topics}
      \item Introducción a los principios de la programación orientada a objetos
      \item Conceptos básicos: Clases, Herencia y Polimorfismo
      \item Binding Dinámico
      \item Semántica referencial
      \item Ejemplos de programas orientados a objetos
   \end{topics}

   \begin{unitgoals}
      \item Identificar los principios básicos en los cuales se basa la programación orientada a objetos
      \item Analiza como pasar del dominio de un problema a un modelado orientado a objetos
      \item Aprende como representar a nivel de lenguaje y a nivel de abstracción un caso problema
      \item Aprende la sintaxis de un lenguaje de programación orientado a objetos puro
      \item Implementa un programa en lenguaje de programación orientado a objetos
      \item Analiza los distintos tipos de herencia que presentan los lenguajes de programación orientados a este paradigma y examina sus ventajas y desventajas
   \end{unitgoals}
\end{unit}

\begin{unit}{Lenguajes Funcionales}{hen96,rob05,rav96}{4}{3}
\begin{topics}
    \item Introducción a los lenguajes funcionales
      \item Definición de función
      \item Listas
      \item Tipos y Polimorfismo
      \item Funciones de orden superior
      \item Lazy Evaluation
      \item Ejemplos de programas funcionales
   \end{topics}

   \begin{unitgoals}
      \item Reconocer los principios fundamentales del paradigma
      funcional
      \item Comparar las ventajas de la orientación funcional sobre
      otros esquemas
      \item Analiza el concepto de funciones y lo aplica en la
      solución de problemas
      \item Establece la facilidad del uso de lenguajes funcionales
      para casos en estructuras de datos y recursividad
      \item Investiga sobre el cálculo lambda
      \item Diseña e implementa programas en algn tipo de lenguaje
      funcional
   \end{unitgoals}
\end{unit}

\begin{unit}{Lenguajes Lógicos}{hen96,rob05,rav96}{4}{3}
\begin{topics}
      \item Principios
      \item Cláusulas de Horn
      \item Variables Lógicas
      \item Relaciones y Estructuras de Datos
      \item Control del orden de bsqueda
      \item Ejemplos de programas basados en el paradigma lógico

   \end{topics}

   \begin{unitgoals}
      \item Comprender el modo de operación de los lenguajes lógicos
      orientados al logro de metas
      \item Analizar el encadenamiento hacia adelante o hacia atrás
      \item Aprender un lenguaje orientado al paradigma lógico
      \item Diseñar e implementar programas en lenguajes de
      programación orientados a objetos
   \end{unitgoals}
\end{unit}

\begin{unit}{Otros Paradigmas}{hen96,mue04,rob05,rav96}{2}{2}
\begin{topics}
      \item Programación Paralela
      \item Programación Distribuida
      \item Crítica a la máquina de Von Newmann
   \end{topics}

   \begin{unitgoals}
      \item Identifica otros paradigmas presentes en nuestro medio
      \item Analiza si es que los paradigmas estudiados son novedosos o solamente una consecuencia de los principales paradigmas analizados
      \item Critica la máquina de Von Newmann en base a los conocimientos de su arquitectura
      \item Diseña e implementa programas en un lenguaje de programación basado en los paradigmas estudiados
   \end{unitgoals}
\end{unit}



\begin{coursebibliography}
\bibfile{Computing/CS/CS343}
\end{coursebibliography}

\end{syllabus}
