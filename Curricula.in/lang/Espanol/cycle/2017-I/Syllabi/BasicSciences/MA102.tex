\begin{syllabus}

\course{MA102. Matemática III}{Obligatorio}{MA102}

\begin{justification}
Este curso introduce los primeros conceptos del álgebra lineal, así como los métodos numéricos con un énfasis en la resolución de problemas
con el paquete de sotfware libe de código abierto Scilab.
La teoría matemática se limita a los fundamentos, mientras que la aplicación efectiva para la resolución de problemas es privilegiada. 
En cada tópico, se enseña unos cuantos métodos de de relevancia para la ingeniería. 
Los conocimientos sobre estos métodos prepara a los estudiantes para la búsqueda de alternativas más avanzadas, si se lo requiere.

% This course introduces first concepts of Linear Algebra as well as of Numerical Methods
% with an emphasis on problem-solving with the Scilab open-source computational
% package. Mathematical theory is restricted to fundamentals whereas effective pure and
% applied problem-solving is privileged. In each topic, only a few methods of relevance to
% everyday Engineering are taught. Knowledgeability about these methods prepares
% students for searching more advanced alternatives if and when necessary.

\end{justification}

\begin{goals}
\item Capacidad para aplicar los conocimientos sobre Matemáticas.
\item Capacidad para aplicar los conocimientos sobre Ingeniería.
\item Capacidad para aplicar los conocimientos, técnicas, habilidades y herramientas modernas de la ingeniería moderna para la práctica de la ingenieria.
 
%Capacity to apply knowledge of Mathematics
%Capacity to apply knowledge of Engineering
%Capacity to apply the knowledge, techniques, skills, and modern tools of modern Engineering necessary for the Engineering practice 

\end{goals}

\begin{outcomes}
    \item \ShowOutcome{a}{3}
    \item \ShowOutcome{j}{3}
\end{outcomes}

\begin{competences}
    \item \ShowCompetence{C1}{a} 
    \item \ShowCompetence{C20}{j} 
    \item \ShowCompetence{C24}{j} 
\end{competences}

\begin{unit}{}{Introducción}{Anton,Chapra}{18}{C1}
% begin{unit}{}{Introduction}{Anton,Chapra}{18}{C1}
  \begin{topics}
      \item Importancia del álgebra lineal y métodos numéricos. Ejemplos. %Importance of Linear Algebra and Numerical Methods. Examples.
   \end{topics}

   \begin{learningoutcomes}
      \item . %Be able to understand the basic concepts and importance of Linear Algebra and Numerical Methods
   \end{learningoutcomes}
\end{unit}

\begin{unit}{}{Álgebra lineal}{Anton,Chapra}{14}{C1}
% begin{unit}{}{Linear Algebra}{Anton,Chapra}{14}{C1}
   \begin{topics}
    \item . %Elementary matrix algebra and determinants
    \item . %Null space and exact solutions of systems of linear equations Ax=b:
	  \begin{subtopics}
	  \item . %Tridiagonal and triangular systems and Gaussian elimination with and without pivoting
	  \item . %LU factorization and Crout algorithm
	  \end{subtopics}
    \item . %Basics on eigenvalues and eigenvectors:
	  \begin{subtopics}
	  \item . %Characteristic polynomials
	  \item . %Algebraic and geometric multiplicities
	  \end{subtopics}
    \item . %Least squares estimation
    \item . %Linear transformations
    \end{topics}

   \begin{learningoutcomes}
      \item . %Understanding the basics concepts of Linear Algebra.
      \item . %Solve properly linear transformations problems.
      \end{learningoutcomes}
\end{unit}

\begin{unit}{}{Métodos Numéricos}{Anton,Chapra}{22}{C24}
% begin{unit}{}{Numerical methods}{Anton,Chapra}{22}{C24}
   \begin{topics}
    \item . %Basics on solutions of systems of linear equations Ax=b: Jacobi and Gauss Seidel methods
    \item . %Application of matrix factorizations to the solution of linear systems (singular value decomposition, QR, Cholesky) Numerical computation of null space, rank and condition number
    \item . %Root finding:
	  \begin{subtopics}
	  \item . %Bisection
	  \item . %Fixed-point iteration
	  \item . %Newton-Raphson methods
	  \end{subtopics}
    \item . %Basics on interpolation:
	  \begin{subtopics}
	  \item . %Newton and Lagrange polynomial interpolations
	  \item . %Spline interpolation
	  \end{subtopics}
    \item . %Basics on numerical differentiation and Taylor approximation
    \item . %Basics on numerical integration:
	  \begin{subtopics}
	  \item . %Trapezium, midpoint and `Simpson\textquotesingle s rule
	  \item . %Gaussian quadrature
	  \end{subtopics}
    \item . %Basics on numerical solutions to ODEs:
	  \begin{subtopics}
	  \item . %Finite differences; Euler and Runge-Kutta methods
	  \item . %Converting higher order ODEs into a system of low order ODEs
	  \item . %Runge-Kutta methods for systems of equations
	  \item . %Single shooting method
	  \end{subtopics}
    \item . %Short introduction to optimization techniques: overview on linear programming, bounded linear systems, quadratic programming, gradient descent.
    \end{topics}

   \begin{learningoutcomes}
      \item . %Understanding the basics concepts of Numerical Methods.
      \item . %Applying the most frequent methods for the resolution of mathematical problems.
      \item . %Implementing and applying numerical algorithms for the solution of mathematical problems using the Scilab open-source computational package.
      \item . %Applying Scilab for the solution of mathematical problems and for plotting graphs.
      \end{learningoutcomes}
\end{unit}



\begin{coursebibliography}
\bibfile{BasicSciences/MA102}
\end{coursebibliography}

\end{syllabus}

%\end{document}
