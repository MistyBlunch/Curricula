
\begin{syllabus}

\course{MA203. Estadística y Probabilidades}{Obligatorio}{MA203} % Common.pm

\begin{justification}
Provee de una introducción a la teoría de las probabilidades e inferencia estadística con aplicaciones, necesarias en el análisis de datos, diseño de modelos aleatorios y toma de decisiones.
\end{justification}

\begin{goals}
\item Capacidad para diseñar y conducir experimentos, así como para analizar e interpretar datos.
\item Capacidad para identificar, formular y resolver problemas reales.
\end{goals}

\begin{outcomes}{V1}
   \item \ShowOutcome{a}{2}
   \item \ShowOutcome{j}{3}
\end{outcomes}

\begin{competences}{V1}
    \item \ShowCompetence{C1}{a} 
    \item \ShowCompetence{CS2}{j}
\end{competences}


\begin{unit}{Tipo de variable}{}{Sheldon,Menden}{6}{C1}
\begin{topics}
      \item Tipo de variable: Continua, discreta.
   \end{topics}

   \begin{learningoutcomes}
      \item Clasificar las variables relevantes identificadas según su tipo: continuo (intervalo y razón), categórico (nominal, ordinario, dicotómico).
      \item Identificar las variables relevantes de un sistema utilizando un enfoque de proceso.
   \end{learningoutcomes}
\end{unit}

\begin{unit}{Estadísticas descriptiva}{}{Sheldon,Menden}{6}{C1}
\begin{topics}
      \item Tendencia Central (Media, mediana, modo)
      \item Dispersión (Rango, desviación estándar, cuartil)
      \item Gráficos: histograma, boxplot, etc .: Capacidad de comunicación.
   \end{topics}
   \begin{learningoutcomes}
      \item Utilizar medidas de tendencia central y medidas de dispersión para describir los datos recopilados.
      \item Utilizar gráficos para comunicar las características de los datos recopilados.
   \end{learningoutcomes}
\end{unit}

\begin{unit}{Estadística inferencial}{}{Sheldon,Menden}{6}{CS2}
\begin{topics}
      \item Determinación del tamaño de la muestra
      \item Intervalo de confianza
      \item Tipo I y error del tipo II
      \item Tipo de distribución
      \item Prueba de hipótesis (t-student, medias, proporciones y ANOVA)
      \item Relaciones entre variables: correlación, regresión.
   \end{topics}

   \begin{learningoutcomes}
      \item Proponer preguntas e hipótesis de interés.
      \item Analizar los datos recopilados utilizando diferentes herramientas estadísticas para responder preguntas de interés.
      \item Dibujar conclusiones basadas en el análisis realizado.
   \end{learningoutcomes}
\end{unit}





\begin{coursebibliography}
\bibfile{BasicSciences/MA203}
\end{coursebibliography}

\end{syllabus}
