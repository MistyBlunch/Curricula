\begin{syllabus}

\course{CB113. Termodin�mica}{Obligatorio}{CB113}

\begin{justification}
Este es un curso general introductorio en Ingenier�a termodin�mica. 
Se centra sobretodo en las propiedades termof�sicas, las leyes de la termodin�mica, 
el concepto del balance de masa y energ�a, y en los sistemas de conversi�n de energ�a est�ndar.
Aplicaciones ingenieriles y casos pertenecientes a diferentes carreras ser�n usados a lo largo del curso.
Capacidad de conocimiento sobre los diferentes t�picos del curso, establecer� las bases para los estudiantes
en buscar informaci�n m�s avanzada cuando sea necesario y poder participar en cursos m�s avanzados en termodin�mica.

% 
% This is a general, introductory course in Engineering Thermodynamics. It focus mostly
% on thermophysical properties of substances, the Laws of Thermodynamics, the concept
% of mass and energy balances, and on standard energy conversion systems. Engineering
% applications and cases pertaining to different careers will be used throughout course.
% Knowledgeability about the course topics will set the bases for students to search more
% advanced information when necessary and for participating in more advanced courses on
% Thermodynamics.

\end{justification}

\begin{goals}
\item Capacidad para aplicar los conocimientos de Ingenier�a.
\item Capacidad de comunicarse por escrito.
\item Capacidad para resolver problemas de Ingenier�a.

% - a3: Capacity to apply knowledge of Engineering (level 1);
% - g2: Capacity to communicate in writing (level 1);
% - e3: Capacity to solve Engineering problems (level 1).

\end{goals}

\begin{outcomes}
\ShowOutcome{a}{3}
\ShowOutcome{i}{3}
\ShowOutcome{j}{3}
\end{outcomes}

\begin{unit}{FI1 Introducci�n}{Serway2002,Alonso94}{0}{3}
\begin{topics}
      \item Importancia de la termodin�mica para las ciencias de la Ingenier�a.
      \item Concepto de equilibro (t�rmica, mec�nica y qu�mica).
      
% Importance of Thermodynamics for Engineering Sciences Concept of equilibrium (thermal, mechanical and
% chemical)

   \end{topics}

   \begin{learningoutcomes}
      \item Resolver problemas.
   \end{learningoutcomes}
\end{unit}

\begin{unit}{FI1 Propiedades termof�sicas}{}{0}{3}
\begin{topics}
      \item Evaluaci�n de las propiedades de las sustancias puras.
      \item Ecuaciones de estado y evaluaci�n de las propiedades mediante el modelo de los gases ideales.
      \item Sistemas cerrados, vol�menes de control y sistemas abiertos.
      \item Estado de las funciones (energ�a interna, entalp�a y entrop�a).
  
  
% Evaluating properties of pure substances
% Equations of state; evaluating properties using the Ideal Gas Model
% Evaluating properties using computer software (EES)
% Use of EES (Engineering Equation Solver) to compute properties 
% Closed systems, control volumes, open Systems
% State functions (internal energy, enthalpy and entropy)
  
  
   \end{topics}

   \begin{learningoutcomes}
      \item Resolver problemas.
   \end{learningoutcomes}
\end{unit}

\begin{unit}{FI1 1ra Ley y Procesos}{}{0}{3}
\begin{topics}
      \item 1ra Ley de la Termodin�mica.
      \item Balance de masa y energ�a de vol�menes de control.
      
% 1st Law of Thermodynamics
% Mass and energy balances of control volumes
      
      
   \end{topics}

   \begin{learningoutcomes}
      \item Resolver problemas.
   \end{learningoutcomes}
\end{unit}

\begin{unit}{FI1 2da Ley y Sistemas}{}{0}{3}
\begin{topics}
      \item Motores t�rmicos. Ciclo de Carnot.
      \item 2da Ley de la Termodin�mica.
      \item Eficiencia t�rmica. Coeficiente de rendimiento (COP).
      \item Entrop�a.
      
% Heat engines. The Carnot cycle.
% 2nd Law of Thermodynamics.
% Thermal efficiency. Coefficient of Performance (COP)
% Entropy
%       
   \end{topics}

   \begin{learningoutcomes}
      \item Resolver problemas.
   \end{learningoutcomes}
\end{unit}

\begin{unit}{FI1 Conversi�n de Energ�a}{}{0}{3}
\begin{topics}
      \item Conceptos b�sicos sobre los ciclos de potencia: Otto, Diesel, Brayton y Rankine.
      \item Ciclos combinados.
      \item Concept b�sicos sobre los ciclos de refrigeraci�n y bomba de calor.
     
% Basics on power cycles: Otto, Diesel, Brayton \& Rankine
% Combined cycles. Optimization of thermal power plant
% Basics on refrigeration and heat pump cycles
     
   \end{topics}

   \begin{learningoutcomes}
      \item Resolver problemas.
   \end{learningoutcomes}
\end{unit}



\begin{coursebibliography}
\bibfile{BasicSciences/CB111}
\end{coursebibliography}

\end{syllabus}
