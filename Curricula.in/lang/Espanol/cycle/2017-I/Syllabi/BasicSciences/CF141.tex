\begin{syllabus}

\course{ME0019. Física I}{Obligatorio}{ME0019} % Common.pm

\begin{justification}

Este curso es útil en esta carrera para que el alumno aprenda a mostrar un alto grado de dominio de las leyes del movimiento de la Física General.

\end{justification}

\begin{goals}
\item Capacitar y presentar al estudiante los principios básicos de la Física como ciencia natural abarcando sus tópicos más importantes y su relación con los problemas cotidianos.
\end{goals}

\begin{outcomes}
  \item \ShowOutcome{a}{2}
  \item \ShowOutcome{i}{2}
  \item \ShowOutcome{j}{2}
\end{outcomes}

\begin{competences}
    \item \ShowCompetence{C1}{a}
    \item \ShowCompetence{C20}{i,j}
\end{competences}

\begin{unit}{FI1. Introducción}{}{Serway2002,Alonso95}{4}{C1,C20}
\begin{topics}
      \item La investigación científica. El método científico.
      \item Concepto de Química. La Química en la actualidad.
      \item Materia. Clasificación y propiedades físicas, químicas, intensivas y extensivas.
      \item Modelo idealizado.
      \item Magnitudes físicas.
      \item Propiedades de los vectores.
      \item Componentes de un vector y vectores unitarios.
      \item Producto de vectores.
      \item Ejercicios y problemas.
   \end{topics}

   \begin{learningoutcomes}
      \item Entender y trabajar con las magnitudes físicas del SI.
      \item Abstraer de la naturaleza los conceptos físicos rigurosos y representarlos en modelos vectoriales.
      \item Entender y aplicar los conceptos vectoriales a problemas físicos reales.
   \end{learningoutcomes}
\end{unit}

\begin{unit}{FI2. Movimiento de partículas en una dimensión}{}{Serway2002,Alonso95}{2}{C1,C20}
\begin{topics}
      \item Desplazamiento, velocidad y rapidez.
      \item Velocidad instantánea.
      \item Aceleración media e instantánea.
      \item Movimiento con aceleración constante.
      \item Caída libre de los cuerpos.
      \item Ejercicios y problemas.
    \end{topics}
   \begin{learningoutcomes}
      \item Describir matemáticamente el movimiento mecánico de una partícula unidimensional como un cuerpo de dimensiones despreciables.
      \item Conocer y aplicar conceptos de magnitudes cinemáticas.
      \item Describir el comportamiento de movimiento de partículas, teórica y gráficamente.
      \item Conocer representaciones vectoriales de estos movimientos unidimensionales.
      \item Resolver problemas.
   \end{learningoutcomes}
\end{unit}

\begin{unit}{FI3. Movimiento de partículas en dos y tres dimensiones}{}{Serway2002,Alonso95}{4}{C1,C20}
\begin{topics}
      \item Desplazamiento y velocidad.
      \item El vector aceleración.
      \item Movimiento parabólico.
      \item Movimiento circular.
      \item Componentes tangencial y radial de la aceleración.
      \item Ejercicios y problemas
\end{topics}

   \begin{learningoutcomes}
      \item Describir matematicamente el movimiento mecánico de una partícula en dos y tres dimensiones como un cuerpo de dimensiones despreciables.
      \item Conocer y aplicar conceptos de magnitudes cinemáticas vectoriales en dos y tres dimensiones.
      \item Describir el comportamiento de movimiento de partículas teórica y gráficamente en dos y tres dimensiones.
      \item Conocer y aplicar conceptos del movimiento circular.
      \item Resolver problemas.
   \end{learningoutcomes}
\end{unit}

\begin{unit}{FI4. Leyes del movimiento}{}{Serway2002,Alonso95}{6}{C1,C20}
\begin{topics}
      \item Fuerza e interacciones.
      \item Primera ley de Newton.
      \item Masa inercial.
      \item Segunda ley de Newton.
      \item Peso.
      \item Diagramas de cuerpo libre.
      \item Tercera Ley de newton.
      \item Fuerzas de fricción.
      \item Dinámica del movimiento circular.
      \item Ejercicios y problemas.
   \end{topics}

   \begin{learningoutcomes}
      \item Conocer los conceptos de fuerza.
      \item Conocer las interacciones mas importantes de la naturaleza y representarlos en un diagrama de cuerpo libre.
      \item Conocer los conceptos de equilibrio estático.
      \item Conocer y aplicar las leyes del movimiento y caracterizarlos vectorialmente.
      \item Conocer y aplicar las leyes de Newton.
      \item Resolver problemas.
   \end{learningoutcomes}
\end{unit}

\begin{unit}{FI5. Trabajo y Energía}{}{Serway2002,Alonso95}{4}{C1,C20}
\begin{topics}
	\item Trabajo realizado por una fuerza constante.
	\item Trabajo realizado por fuerzas variables.
	\item Trabajo y energía cinética.
	\item Potencia.
	\item Energía potencial gravitatoria.
	\item Energía potencial elástica.
	\item Fuerzas conservativas y no conservativas.
	\item Principios de conservación de la energía.
	\item Ejercicios y problemas.
\end{topics}

   \begin{learningoutcomes}
      \item Establecer los conceptos de energía física. (Física clásica)
      \item Conocer algunas formas de energía.
      \item Establecer la relación entre trabajo y energía.
      \item Conocer y aplicar los conceptos de conservación de energía.
      \item Resolver problemas.
   \end{learningoutcomes}
\end{unit}

\begin{unit}{FI6. Momento lineal}{}{Serway2002,Alonso95}{3}{C1,C20}
\begin{topics}
      \item Momento lineal.
      \item Conservación del momento lineal.
      \item Centro de masa y de gravedad.
      \item Movimiento de un sistema de partículas.
      \item Ejercicios y problemas.
  \end{topics}

   \begin{learningoutcomes}
      \item Establecer los conceptos de momento lineal.
      \item Conocer los conceptos de conservación del momento lineal.
      \item Conocer el movimiento de un sistema de partículas.
      \item Resolver problemas.
   \end{learningoutcomes}
\end{unit}

\begin{unit}{FI7. Rotación de cuerpos rígidos}{}{Serway2002,Alonso95}{4}{C1,C20}
\begin{topics}
      \item Velocidad y aceleraciones angulares.
      \item Rotación con aceleración angular constante.
      \item Relación entre cinemática lineal y angular.
      \item Energía en el movimiento de rotación.
      \item Momento de torsión.
      \item Relación entre momento de torsión y aceleración angular.
      \item Ejercicios y problemas.
   \end{topics}

   \begin{learningoutcomes}
      \item Conocer los conceptos básicos de cuerpo rígido.
      \item Conocer y aplicar conceptos de rotación de cuerpo rígido.
      \item Conocer conceptos de torsión.
      \item Aplicar conceptos de energía al movimiento de rotación.
      \item Resolver problemas.
   \end{learningoutcomes}
\end{unit}

\begin{unit}{FI8. Dinámica del movimiento de rotación}{}{Serway2002,Alonso95}{3}{C1,C20}
\begin{topics}
      \item Momento de torsión y aceleración angular de un cuerpo rígido.
      \item Rotación de un cuerpo rígido sobre un eje móvil.
      \item Trabajo y potencia en el movimiento de rotación.
      \item Momento angular.
      \item Conservación del momento angular.
      \item Ejercicios y problemas.
    \end{topics}

   \begin{learningoutcomes}
      \item Conocer conceptos básicos de dinámica de rotación.
      \item Conocer y aplicar conceptos de torsión.
      \item Entender el momento angular y su conservación.
      \item Resolver problemas.
   \end{learningoutcomes}
\end{unit}



\begin{coursebibliography}
\bibfile{BasicSciences/CF141}
\end{coursebibliography}
\end{syllabus}
