\begin{syllabus}

\course{GH0017. Introducci�n al Quechua}{Electivos}{GH0017}

\begin{justification}
El curso de Quechua comunicativo permite acercar a los estudiantes al uso pr�ctico de la lengua andina en su variedad chanca. Esta es una de las variedades de mayor difusi�n y modelo para abordar otras variedades del denominado quechua sure�o o quechua II. Adem�s, se presenta sencilla en su aprendizaje por compartir sonidos con el castellano. Asimismo, el curso busca familiarizar al alumno con las estructuras b�sicas de esta lengua, as� como con la traducci�n y producci�n de textos. El objetivo �ltimo es proporcionar las herramientas b�sicas de aprendizaje de modo que el estudiante pueda expresarse en ella a un nivel b�sico y funcional, as� como conducir y desarrollar su propio aprendizaje de la lengua.
Consideramos que hablar quechua en ciertas situaciones donde los ingenieros UTEC tienen que desarrollarse es una ventaja important�sima: los hablantes nativos de quechua practican un trato diferenciado con las personas que lo hablan por sentir que se est� respetando su tradici�n y, a la vez, se est� haciendo un esfuerzo por entablar un di�logo en su propia lengua. Esto representa ventajas operativas muy puntuales en el trato y el acuerdo de intereses. 
\end{justification}

\begin{goals}
\item Otorgar herramientas b�sicas para presentarse y conversar en la lengua quechua, en la variedad chanca.
\item Acercar al estudiante a las estructuras b�sicas de la lengua con el fin de dirigir su estudio y auto aprendizaje.
\item Entrenar al alumno en la traducci�n y producci�n de textos en la lengua nativa.
\item Proporcionar herramientas para que el alumno desarrolle el conocimiento de esta lengua de manera individual.
\item Dar herramientas para reconocer la procedencia del quechua al cual se enfrentan a trav�s de elementos de an�lisis ling��stico
\end{goals}

\begin{outcomes}
    \item \ShowOutcome{f}{2}
    \item \ShowOutcome{n}{1}
    \item \ShowOutcome{�}{1}
\end{outcomes}

\begin{unit}{}{Semana 1.}{Zariquiey08}{12}{4}
   \begin{topics}
      \item Historia del quechua: breve panorama.
      \item Dialectolog�a.
      \item Saludos y preguntas b�sicas.
   \end{topics}

   \begin{learningoutcomes}
      \item .
   \end{learningoutcomes}
\end{unit}

\begin{unit}{}{Semana 2.}{Zariquiey08}{24}{3}
   \begin{topics}
      \item Sistema fonol�gico .
      \item Revisi�n de materiales sonoros.
   \end{topics}

   \begin{learningoutcomes}
      \item . 
      \end{learningoutcomes}
\end{unit}

\begin{unit}{}{Semana 3}{Zariquiey08}{24}{3}
   \begin{topics}
      \item Morfolog�a.
      \item Formulaci�n de preguntas. Frase nominal.
      \item Pedidos b�sicos en lengua quechua.
   \end{topics}

   \begin{learningoutcomes}
      \item .
     
   \end{learningoutcomes}
\end{unit}

\begin{unit}{}{Semana 4.}{Adelaar77}{30}{3}
   \begin{topics}
      \item Revisi�n de frase nominal y temas gramaticales.
   \end{topics}

   \begin{learningoutcomes}
      \item .
   \end{learningoutcomes}
\end{unit}

\begin{unit}{}{Semana 5.}{Adelaar77}{30}{3}
   \begin{topics}
      \item Frase verbal
      \item Uso de materiales multimedia
      \item Pedidos b�sicos
   \end{topics}

   \begin{learningoutcomes}
      \item .
   \end{learningoutcomes}
\end{unit}

\begin{unit}{}{Semana 6.}{Zariquiey08}{30}{3}
   \begin{topics}
      \item Frase verbal
      \item Uso de materiales multimedia
   \end{topics}

   \begin{learningoutcomes}
      \item .
   \end{learningoutcomes}
\end{unit}

\begin{unit}{}{Semana 7.}{Adelaar77}{30}{3}
   \begin{topics}
      \item Revisi�n de lo visto hasta este punto
   \end{topics}

   \begin{learningoutcomes}
      \item .
   \end{learningoutcomes}
\end{unit}

\begin{unit}{}{Semana 8.}{Adelaar77}{30}{3}
   \begin{topics}
      \item SEMANA DE EX�MENES PARCIALES
   \end{topics}

   \begin{learningoutcomes}
      \item .
   \end{learningoutcomes}
\end{unit}

\begin{unit}{}{Semana 9.}{Zariquiey08}{30}{3}
   \begin{topics}
      \item Preparaci�n de entrevistas.
   \end{topics}

   \begin{learningoutcomes}
      \item .
   \end{learningoutcomes}
\end{unit}

\begin{unit}{}{Semana 10.}{Adelaar77}{30}{3}
   \begin{topics}
      \item Morfolog�a deverbativa.
      \item Morfolog�a denominativa
   \end{topics}

   \begin{learningoutcomes}
      \item .
   \end{learningoutcomes}
\end{unit}

\begin{unit}{}{Semana 11.}{Adelaar77}{30}{3}
   \begin{topics}
      \item Ejercicios de traduci�n.
      \item Ejercicios de conversaci�n.
   \end{topics}

   \begin{learningoutcomes}
      \item . 
   \end{learningoutcomes}
\end{unit}

\begin{unit}{}{Semana 12.}{Zariquiey08}{30}{3}
   \begin{topics}
      \item Preparaci�n de entrevistas.
   \end{topics}

   \begin{learningoutcomes}
      \item .
   \end{learningoutcomes}
\end{unit}

\begin{unit}{}{Semana 13.}{Adelaar77}{30}{3}
   \begin{topics}
      \item Trabajo en grupo: redacci�n de un guion.
   \end{topics}

   \begin{learningoutcomes}
      \item .
   \end{learningoutcomes}
\end{unit}

\begin{unit}{}{Semana 14.}{Zariquiey08}{30}{3}
   \begin{topics}
      \item Preparaci�n y presentaci�n de exposiciones.
   \end{topics}

   \begin{learningoutcomes}
      \item .
   \end{learningoutcomes}
\end{unit}

\begin{unit}{}{Semana 15.}{Adelaar77}{30}{3}
   \begin{topics}
      \item Preparaci�n y presentaci�n de exposiciones.
   \end{topics}

   \begin{learningoutcomes}
      \item .
   \end{learningoutcomes}
\end{unit}

\begin{unit}{}{Semana 16.}{Adelaar77}{30}{3}
   \begin{topics}
      \item SEMANA DE EX�MENES FINALES.
   \end{topics}

   \begin{learningoutcomes}
      \item .
   \end{learningoutcomes}
\end{unit}



\begin{coursebibliography}
\bibfile{GeneralEducation/GH0017}
\end{coursebibliography}

\end{syllabus}
