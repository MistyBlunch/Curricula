\begin{syllabus}

\course{GH0006. Laboratorio de Comunicación II}{Obligatorio}{GH0006}

\begin{justification}
Este laboratorio está orientado a consolidar las habilidades comunicativas del estudiante, tanto a nivel oral como escrito en el marco de la disciplina que se estudia. En particular, el estudiante fortalecerá sus capacidades expositivas al ejercitarse en toda la primera parte del curso en la escritura de un tipo de texto que
desarrollará a lo largo de su carrera como ingeniero: los informes de laboratorio. Reflexionará sobre la situación retórica que enfrenta al escribir este tipo de texto: quién será su lector, cuál es la intención comunicativa de ese texto y el tema sobre el que está escribiendo.
En una segunda parte, el curso se presenta como un espacio de discusión sobre el discurso argumentativo y de lectura crítica de textos argumentativos, para que el alumno reflexione, conozca y emplee las herramientas comunicativas para producir textos argumentativos formales. En este sentido, el curso se orienta hacia la producción
permanente de textos escritos y orales, por lo que el alumno participará no solo en foros de discusión sino que se espera que sea capaz de debatir con sus compañeros sobre un tema propuesto por el profesor. En suma, el curso busca consolidar las competencias de lectura, análisis y elaboración de textos escritos y orales, tanto expositivos como argumentativos.
\end{justification}

\begin{goals}
\item Desarrollar habilidades que les permitan a los estudiantes mejorar sus capacidades comunicativas, tanto orales como escritas.
\item Comprender y producir textos expositivos en los que informen sobre la aplicación del conocimiento teórico en un experimento o contexto diferente.
\item Comprender y producir textos argumentativos orales y escritos.
\item Se capaz de debatir empleando argumentos sólidos.
\item Emplear adecuadamente y reflexivamente la información obtenida en diferentes fuentes.
\item Mostrar apertura y respeto para escuchar la diversidad de opiniones o puntos de vista de los compañeros de clase.
\end{goals}

\begin{outcomes}
   \item \ShowOutcome{a}{3}
   \item \ShowOutcome{i}{2}
   \item \ShowOutcome{j}{4}
\end{outcomes}

\begin{competences}
    \item \ShowCompetence{C17}{f,h,n}
    \item \ShowCompetence{C20}{f,n}
    \item \ShowCompetence{C24}{f,h}
\end{competences}

\begin{unit}{Semana 1A.}{}{Cassany08}{12}{4}
   \begin{topics}
      \item Presentación del curso: sílabo .
      \item Diferencias entre texto expositivo y texto argumentativo. 
      \item Características del texto argumentativo .
   \end{topics}
   \begin{learningoutcomes}
      \item . 
   \end{learningoutcomes}
\end{unit}

\begin{unit}{Semana 1B}{}{Cassany08}{24}{3}
   \begin{topics}
      \item Características del texto expositivo.
   \end{topics}

   \begin{learningoutcomes}
      \item . 
      \item .
      \item . 
      \end{learningoutcomes}
\end{unit}

\begin{unit}{Semana 2A.}{}{Cassany08}{12}{4}
   \begin{topics}
      \item ¿Qué es un informe de laboratorio? 
   \end{topics}
   \begin{learningoutcomes}
      \item . 
   \end{learningoutcomes}
\end{unit}

\begin{unit}{Semana 2B}{}{Cassany08}{24}{3}
   \begin{topics}
      \item Informe de laboratorio: desarrollo del laboratorio o metodología.
   \end{topics}

   \begin{learningoutcomes}
      \item . 
      \item .
      \item . 
      \end{learningoutcomes}
\end{unit}

\begin{unit}{Semana 3A.}{}{Cassany08}{12}{4}
   \begin{topics}
      \item . Informe de laboratorio: resultados de laboratorio y aplicaciones. 
   \end{topics}
   \begin{learningoutcomes}
      \item . 
   \end{learningoutcomes}
\end{unit}

\begin{unit}{Semana 3B}{}{Cassany08}{24}{3}
   \begin{topics}
      \item Informe de laboratorio: Introducción y conclusiones. 
   \end{topics}

   \begin{learningoutcomes}
      \item . 
      \item .
      \item . 
      \end{learningoutcomes}
\end{unit}

\begin{unit}{Semana 4A.}{}{Cassany08}{12}{4}
   \begin{topics}
      \item Citado, referencias parentéticas y construcción de bibliografía I. 
   \end{topics}
   \begin{learningoutcomes}
      \item . 
   \end{learningoutcomes}
\end{unit}

\begin{unit}{Semana 4B}{}{Cassany08}{24}{3}
   \begin{topics}
      \item Citado, referencias parentéticas y construcción de bibliografía I.
   \end{topics}

   \begin{learningoutcomes}
      \item . 
      \item .
      \item . 
      \end{learningoutcomes}
\end{unit}

\begin{unit}{Semana 5A.}{}{Córdova09}{12}{4}
   \begin{topics}
      \item Características de oralidad.
   \end{topics}
   \begin{learningoutcomes}
      \item . 
   \end{learningoutcomes}
\end{unit}

\begin{unit}{Semana 5B}{}{Córdova09}{24}{3}
   \begin{topics}
      \item Ejercicio en Clase.
   \end{topics}

   \begin{learningoutcomes}
      \item . 
      \item .
      \item . 
      \end{learningoutcomes}
\end{unit}

\begin{unit}{Semana 6A.}{}{Córdova09}{12}{4}
   \begin{topics}
      \item Preparación para la exposición oral. 
   \end{topics}
   \begin{learningoutcomes}
      \item . 
   \end{learningoutcomes}
\end{unit}

\begin{unit}{Semana 6B}{}{Córdova09}{24}{3}
   \begin{topics}
      \item Exposiciones orales.
   \end{topics}

   \begin{learningoutcomes}
      \item . 
      \item .
      \item . 
      \end{learningoutcomes}
\end{unit}

\begin{unit}{Semana 7A.}{}{Córdova09}{12}{4}
   \begin{topics}
      \item Exposiciones orales.
   \end{topics}
   \begin{learningoutcomes}
      \item . 
   \end{learningoutcomes}
\end{unit}

\begin{unit}{Semana 7B}{}{Córdova09}{24}{3}
   \begin{topics}
      \item Exposiciones orales.
   \end{topics}

   \begin{learningoutcomes}
      \item . 
      \item .
      \item . 
      \end{learningoutcomes}
\end{unit}

\begin{unit}{Semana 8A.}{}{Córdova09}{12}{4}
   \begin{topics}
      \item Semana de Parciales -Libre
   \end{topics}
   \begin{learningoutcomes}
      \item . 
   \end{learningoutcomes}
\end{unit}

\begin{unit}{Semana 8B}{}{Córdova09}{24}{3}
   \begin{topics}
      \item Semana de Parciales -Libre
   \end{topics}

   \begin{learningoutcomes}
      \item . 
      \item .
      \item . 
      \end{learningoutcomes}
\end{unit}

\begin{unit}{Semana 9A.}{}{Córdova09}{12}{4}
   \begin{topics}
      \item Presentación de un texto argumentativo y características de la argumentación.
      \item Delimitación de tema y postura

   \end{topics}
   \begin{learningoutcomes}
      \item . 
   \end{learningoutcomes}
\end{unit}

\begin{unit}{Semana 9B}{}{Córdova09}{24}{3}
   \begin{topics}
      \item ¿Cómo se construye un argumento?
   \end{topics}

   \begin{learningoutcomes}
      \item . 
      \item .
      \item . 
      \end{learningoutcomes}
\end{unit}

\begin{unit}{Semana 10A.}{}{Miranda02}{12}{4}
   \begin{topics}
      \item ¿Cómo se construye un argumento?
   \end{topics}
   \begin{learningoutcomes}
      \item . 
   \end{learningoutcomes}
\end{unit}

\begin{unit}{Semana 10B}{}{Miranda02}{24}{3}
   \begin{topics}
      \item  Argumento pragmático.
   \end{topics}

   \begin{learningoutcomes}
      \item . 
      \item .
      \item . 
      \end{learningoutcomes}
\end{unit}

\begin{unit}{Semana 11A.}{}{Miranda02}{12}{4}
   \begin{topics}
      \item Ejercicio en clase: Redacción de texto argumentativo .
   \end{topics}
   \begin{learningoutcomes}
      \item . 
   \end{learningoutcomes}
\end{unit}

\begin{unit}{Semana 11B}{}{Miranda02}{24}{3}
   \begin{topics}
      \item Citado, referencias parentéticas y construcción de bibliografía (II)
   \end{topics}

   \begin{learningoutcomes}
      \item . 
      \item .
      \item . 
      \end{learningoutcomes}
\end{unit}

\begin{unit}{Semana 12A.}{}{Miranda02}{12}{4}
   \begin{topics}
      \item Exposición oral .
   \end{topics}
   \begin{learningoutcomes}
      \item . 
   \end{learningoutcomes}
\end{unit}

\begin{unit}{Semana 12B}{}{Miranda02}{24}{3}
   \begin{topics}
      \item Exposición oral .
   \end{topics}

   \begin{learningoutcomes}
      \item . 
      \item .
      \item . 
      \end{learningoutcomes}
\end{unit}

\begin{unit}{Semana 13A.}{}{Miranda02}{12}{4}
   \begin{topics}
      \item Contraargumentación.
   \end{topics}
   \begin{learningoutcomes}
      \item . 
   \end{learningoutcomes}
\end{unit}

\begin{unit}{Semana 13B}{}{Miranda02}{24}{3}
   \begin{topics}
      \item Contraargumentación:Producción.
   \end{topics}

   \begin{learningoutcomes}
      \item . 
      \item .
      \item . 
      \end{learningoutcomes}
\end{unit}

\begin{unit}{Semana 14A.}{}{Miranda02}{12}{4}
   \begin{topics}
      \item Solidez argumentativa de la contraargumentación. 
   \end{topics}
   \begin{learningoutcomes}
      \item . 
   \end{learningoutcomes}
\end{unit}

\begin{unit}{Semana 14B}{}{Miranda02}{24}{3}
   \begin{topics}
      \item Solidez argumentativa de la contraargumentación.
   \end{topics}

   \begin{learningoutcomes}
      \item . 
      \item .
      \item . 
      \end{learningoutcomes}
\end{unit}

\begin{unit}{Semana 15A.}{}{Miranda02}{12}{4}
   \begin{topics}
      \item Debates.
   \end{topics}
   \begin{learningoutcomes}
      \item . 
   \end{learningoutcomes}
\end{unit}

\begin{unit}{Semana 15B}{}{Miranda02}{24}{3}
   \begin{topics}
      \item Debates.
   \end{topics}

   \begin{learningoutcomes}
      \item . 
      \item .
      \item . 
      \end{learningoutcomes}
\end{unit}

\begin{unit}{Semana 16A.}{}{Miranda02}{12}{4}
   \begin{topics}
      \item Finales.
   \end{topics}
   \begin{learningoutcomes}
      \item . 
   \end{learningoutcomes}
\end{unit}

\begin{unit}{Semana 16B}{}{Miranda02}{24}{3}
   \begin{topics}
      \item Finales.
   \end{topics}

   \begin{learningoutcomes}
      \item . 
      \item .
      \item . 
      \end{learningoutcomes}
\end{unit}



\begin{coursebibliography}
\bibfile{GeneralEducation/GH0006}
\end{coursebibliography}

\end{syllabus}
