\begin{syllabus}

\course{CB101. �lgebra y Geometr�a}{Obligatorio}{CB101}

\begin{justification}
Este curso est� dise�ado para proporcionar a los estudiantes una s�lida comprensi�n del proceso de innovaci�n dentro de una empresa. Se centra en la aplicaci�n de las habilidades de innovaci�n empresarial en una empresa bien establecida. Esto se conoce como Intrapreneurship.
Es el tercero de un conjunto de tres cursos dise�ados para acompa�ar a los estudiantes a medida que transforman una idea en un negocio o empresa potencial. El estudiante experimentar� el proceso desde la fase de ideaci�n hasta la revisi�n de las estrategias de negocios actuales.
El material visto en este curso responde a 2 preguntas principales: "�Qu� debe hacer?" Y "�C�mo debe hacerlo?". 

\end{justification}

\begin{goals}
\item Identificar c�mo se relaciona la innovaci�n con el proceso emprendedor e intraempresarial
\item Familiarizarse con las herramientas de innovaci�n y practicar c�mo hacer uso de ellas.
\item Aprender a integrar la innovaci�n en el ciclo econ�mico
\item Comprender la importancia de la estrategia y la implementaci�n y c�mo una idea debe ir acompa�ada de un plan de implementaci�n efectivo
\item An�lisis de la informaci�n
\item Interpretaci�n de informaci�n y resultados.
\item Trabajo en equipo.
\item �tica.
\item Comunicaci�n oral.
\item Comunicaci�n escrita
\item Comunicaci�n gr�fica
\item Entiendimiento de la necesidad de aprender de forma continua
\end{goals}

\begin{outcomes}
\ExpandOutcome{a}{2}
\ExpandOutcome{c}{3}
\ExpandOutcome{e}{2}
\end{outcomes}

\begin{unit}{Sistemas de coordenadas. La recta.}{Morales13}{12}{4}
   \begin{topics}
      \item 
      \item 
   \end{topics}
   \begin{unitgoals}
      \item 
   \end{unitgoals}
\end{unit}

\begin{coursebibliography}
\bibfile{GeneralEducation/GH0011}
\end{coursebibliography}

\end{syllabus}
