\begin{syllabus}

\course{CB101. �lgebra y Geometr�a}{Obligatorio}{CB101}

\begin{justification}
A trav�s de este curso, el alumno mejorar� y fortalecer� sus capacidades para comunicarse tanto a nivel oral como escrito en un contexto acad�mico. Para ello, el alumno se ejercitar� en la composici�n de textos, tomando en cuenta las exigencias propias de un lenguaje formal acad�mico: caracter�sticas de la redacci�n acad�mica (reglas de puntuaci�n, ortograf�a, competencia l�xico gramatical, normativa) y empleo correcto de la informaci�n. A su vez, el curso promueve una lectura comprensiva que no se limita al nivel descriptivo, sino que abarca tambi�n lo conceptual y metaf�rico, pues solo de ese modo el estudiante desarrollar� su capacidad cr�tica y anal�tica. El estudiante afrontar� lecturas acad�micas y de divulgaci�n cient�fica que le permitir�n distinguir los objetivos planteados en los distintos tipos de textos, y reconocer al texto oral y escrito como una unidad coherente y cohesionada en cuanto a forma y contenido. Alcanzados estos objetivos, el estudiante comprender� que las habilidades comunicativas orales y escritas son competencias centrales de su vida universitaria y, posteriormente, de su vida profesional. 
\end{justification}

\begin{goals}
\item Con este curso el estudiante desarrolla y fortalece sus habilidades comunicativas orales y escritas en el marco de un contexto acad�mico. Adem�s, comprende conceptual y metaf�ricamente textos expositivos, e identifica los objetivos, jerarqu�a de las ideas y estructura de dichos textos. Al finalizar el curso, el estudiante es capaz de producir textos expositivos descriptivos e informativos. As� mismo, desarrolla su capacidad de apertura y tolerancia hacia la diversidad de puntos de vista gracias al continuo trabajo grupal, autoevaluaciones y evaluaciones de pares que enfrentar� a lo largo del ciclo en el curso. 
\end{goals}

\begin{outcomes}   XYZ 
\ExpandOutcome{a}{3}
\ExpandOutcome{i}{2}
\ExpandOutcome{j}{4}
\end{outcomes}

\begin{unit}{Sistemas de coordenadas. La recta.}{Lehmann05}{12}{4}
   \begin{topics}
      \item 
      \item 
   \end{topics}
   \begin{unitgoals}
      \item 
   \end{unitgoals}
\end{unit}

\begin{unit}{C�nicas y Coordenadas polares}{Lehmann05}{24}{3}
   \begin{topics}
      \item 
      \item 
   \end{topics}

   \begin{unitgoals}
      \item 
      \item
      \item 
      \end{unitgoals}
\end{unit}

\begin{unit}{Sistemas de ecuaciones. Matrices y determinantes}{Strang03,Grossman96}{24}{3}
   \begin{topics}
      \item 
      \item 
      \item 
      \end{topics}

   \begin{unitgoals}
      \item 
      \item 
      \item 
     
   \end{unitgoals}
\end{unit}

\begin{unit}{Vectores en $R^2$ y vectores en $R^3$}{Grossman96}{30}{3}
   \begin{topics}
      \item 
      \item 
   \end{topics}

   \begin{unitgoals}
      \item 
      \item 
      \item 
   \end{unitgoals}
\end{unit}



\begin{coursebibliography}
\bibfile{GeneralEducation/GH0005}
\end{coursebibliography}

\end{syllabus}
