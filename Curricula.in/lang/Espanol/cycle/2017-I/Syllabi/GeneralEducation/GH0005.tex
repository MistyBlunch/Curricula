\begin{syllabus}

\course{CB101. Álgebra y Geometría}{Obligatorio}{CB101}

\begin{justification}
A través de este curso, el alumno mejorará y fortalecerá sus capacidades para comunicarse tanto a nivel oral como escrito en un contexto académico. Para ello, el alumno se ejercitará en la composición de textos, tomando en cuenta las exigencias propias de un lenguaje formal académico: características de la redacción académica (reglas de puntuación, ortografía, competencia léxico gramatical, normativa) y empleo correcto de la información. A su vez, el curso promueve una lectura comprensiva que no se limita al nivel descriptivo, sino que abarca también lo conceptual y metafórico, pues solo de ese modo el estudiante desarrollará su capacidad crítica y analítica. El estudiante afrontará lecturas académicas y de divulgación científica que le permitirán distinguir los objetivos planteados en los distintos tipos de textos, y reconocer al texto oral y escrito como una unidad coherente y cohesionada en cuanto a forma y contenido. Alcanzados estos objetivos, el estudiante comprenderá que las habilidades comunicativas orales y escritas son competencias centrales de su vida universitaria y, posteriormente, de su vida profesional. 
\end{justification}

\begin{goals}
\item Con este curso el estudiante desarrolla y fortalece sus habilidades comunicativas orales y escritas en el marco de un contexto académico. Además, comprende conceptual y metafóricamente textos expositivos, e identifica los objetivos, jerarquía de las ideas y estructura de dichos textos. Al finalizar el curso, el estudiante es capaz de producir textos expositivos descriptivos e informativos. Así mismo, desarrolla su capacidad de apertura y tolerancia hacia la diversidad de puntos de vista gracias al continuo trabajo grupal, autoevaluaciones y evaluaciones de pares que enfrentará a lo largo del ciclo en el curso. 
\end{goals}

\begin{outcomes}   
\ExpandOutcome{a}{3}
\ExpandOutcome{i}{2}
\ExpandOutcome{j}{4}
\end{outcomes}

\begin{unit}{Semana 1A.}{Cassany93}{12}{4}
   \begin{topics}
      \item Presentación del curso.
   \end{topics}
   \begin{unitgoals}
      \item
   \end{unitgoals}
\end{unit}

\begin{unit}{Semana 1B.}{Cassany93}{24}{3}
   \begin{topics}
      \item Aproximación a algunas características de la escritura formal.
   \end{topics}

   \begin{unitgoals}
      \item 
      \item
      \item 
      \end{unitgoals}
\end{unit}

\begin{unit}{Semana 2A.}{Cassany93}{12}{4}
   \begin{topics}
      \item Características de la escritura académica I.
   \end{topics}
   \begin{unitgoals}
      \item 
   \end{unitgoals}
\end{unit}

\begin{unit}{Semana 2B.}{Cassany93}{24}{3}
   \begin{topics}
      \item Características de la escritura académica II .
   \end{topics}

   \begin{unitgoals}
      \item 
      \item
      \item 
      \end{unitgoals}
\end{unit}

\begin{unit}{Semana 3A.}{Cassany93}{12}{4}
   \begin{topics}
      \item Estrategias de lectura: subrayado y sumillado I.
   \end{topics}
   \begin{unitgoals}
      \item 
   \end{unitgoals}
\end{unit}

\begin{unit}{Semana 3B.}{Cassany93}{24}{3}
   \begin{topics}
      \item Estrategias de lectura: subrayado y sumillado II.
   \end{topics}

   \begin{unitgoals}
      \item 
      \item
      \item 
      \end{unitgoals}
\end{unit}

\begin{unit}{Semana 4A.}{Cassany93}{12}{4}
   \begin{topics}
      \item Estructura del texto.
   \end{topics}
   \begin{unitgoals}
      \item 
   \end{unitgoals}
\end{unit}

\begin{unit}{Semana 4B.}{Cassany93}{24}{3}
   \begin{topics}
      \item Estructura de párrafos .
   \end{topics}

   \begin{unitgoals}
      \item 
      \item
      \item 
      \end{unitgoals}
\end{unit}

\begin{unit}{Semana 5A.}{Moliner98}{12}{4}
   \begin{topics}
      \item Características del párrafo.
   \end{topics}
   \begin{unitgoals}
      \item 
   \end{unitgoals}
\end{unit}

\begin{unit}{Semana 5B.}{Moliner98}{24}{3}
   \begin{topics}
      \item Esquema de síntesis: esquema y resumen .
   \end{topics}

   \begin{unitgoals}
      \item 
      \item
      \item 
      \end{unitgoals}
\end{unit}

\begin{unit}{Semana 6A.}{Moliner98}{12}{4}
   \begin{topics}
      \item Revisión de Ejercicios .
   \end{topics}
   \begin{unitgoals}
      \item 
   \end{unitgoals}
\end{unit}

\begin{unit}{Semana 6B.}{Moliner98}{24}{3}
   \begin{topics}
      \item Texto argumentativo vs. expositivo.
   \end{topics}

   \begin{unitgoals}
      \item 
      \item
      \item 
      \end{unitgoals}
\end{unit}

\begin{unit}{Semana 7A.}{Moliner98}{12}{4}
   \begin{topics}
      \item Proceso de redacción: delimitación de tema y esquema de producción.
      \item Proceso redacción: introducción y cierre.
   \end{topics}
   \begin{unitgoals}
      \item 
   \end{unitgoals}
\end{unit}

\begin{unit}{Semana 7B.}{Moliner98}{24}{3}
   \begin{topics}
      \item Proceso de redacción: tema, esquema producción (partes y subpartes) . 
   \end{topics}

   \begin{unitgoals}
      \item 
      \item
      \item 
      \end{unitgoals}
\end{unit}

\begin{unit}{Semana 8A.}{Moliner98}{12}{4}
   \begin{topics}
      \item Entrega Virtual (CANVAS) de Avance 1 .
   \end{topics}
   \begin{unitgoals}
      \item 
   \end{unitgoals}
\end{unit}

\begin{unit}{Semana 8B.}{Moliner98}{24}{3}
   \begin{topics}
      \item Entrega Virtual (CANVAS) de Avance 1 .
   \end{topics}

   \begin{unitgoals}
      \item 
      \item
      \item 
      \end{unitgoals}
\end{unit}

\begin{unit}{Semana 9A.}{Moliner98}{12}{4}
   \begin{topics}
      \item Proceso de redacción: tema, esquema producción (partes y subpartes).
   \end{topics}
   \begin{unitgoals}
      \item 
   \end{unitgoals}
\end{unit}

\begin{unit}{Semana 9B.}{Moliner98}{24}{3}
   \begin{topics}
      \item Citas: función y tipos / Bibliografía 
   \end{topics}

   \begin{unitgoals}
      \item 
      \item
      \item 
      \end{unitgoals}
\end{unit}


\begin{unit}{Semana 10A.}{Seco02}{12}{4}
   \begin{topics}
      \item Tipos de Párrafos.
   \end{topics}
   \begin{unitgoals}
      \item 
   \end{unitgoals}
\end{unit}

\begin{unit}{Semana 10B.}{Seco02}{24}{3}
   \begin{topics}
      \item Aproximación a características de la exposición oral.
   \end{topics}

   \begin{unitgoals}
      \item 
      \item
      \item 
      \end{unitgoals}
\end{unit}


\begin{unit}{Semana 11A.}{Seco02}{12}{4}
   \begin{topics}
      \item Tipos de párrafos: comparativo.
   \end{topics}
   \begin{unitgoals}
      \item 
   \end{unitgoals}
\end{unit}

\begin{unit}{Semana 11B.}{Seco02}{24}{3}
   \begin{topics}
      \item Características de la exposición oral y tipos de párrafo.
   \end{topics}

   \begin{unitgoals}
      \item 
      \item
      \item 
      \end{unitgoals}
\end{unit}


\begin{unit}{Semana 12A.}{Seco02}{12}{4}
   \begin{topics}
      \item Redacción de texto completo I .
   \end{topics}
   \begin{unitgoals}
      \item 
   \end{unitgoals}
\end{unit}

\begin{unit}{Semana 12B.}{Seco02}{24}{3}
   \begin{topics}
      \item Redacción de texto completo  II.
   \end{topics}

   \begin{unitgoals}
      \item 
      \item
      \item 
      \end{unitgoals}
\end{unit}


\begin{unit}{Semana 13A.}{Seco02}{12}{4}
   \begin{topics}
      \item Asesoria trabajo final .
   \end{topics}
   \begin{unitgoals}
      \item 
   \end{unitgoals}
\end{unit}

\begin{unit}{Semana 13B.}{Seco02}{24}{3}
   \begin{topics}
      \item Exposiciones.
   \end{topics}

   \begin{unitgoals}
      \item 
      \item
      \item 
      \end{unitgoals}
\end{unit}


\begin{unit}{Semana 14A.}{Seco02}{12}{4}
   \begin{topics}
      \item Exposiciones.
   \end{topics}
   \begin{unitgoals}
      \item 
   \end{unitgoals}
\end{unit}

\begin{unit}{Semana 14B.}{Seco02}{24}{3}
   \begin{topics}
      \item Exposiciones.
   \end{topics}

   \begin{unitgoals}
      \item 
      \item
      \item 
      \end{unitgoals}
\end{unit}

\begin{unit}{Semana 15A.}{Seco02}{12}{4}
   \begin{topics}
      \item Retroalimentación de exposiciones.
   \end{topics}
   \begin{unitgoals}
      \item 
   \end{unitgoals}
\end{unit}

\begin{unit}{Semana 15B.}{Seco02}{24}{3}
   \begin{topics}
      \item COEVALUACIONES de exposiciones.
   \end{topics}

   \begin{unitgoals}
      \item 
      \item
      \item 
      \end{unitgoals}
\end{unit}

\begin{unit}{Semana 16A.}{Seco02}{12}{4}
   \begin{topics}
      \item El CURSO NO TIENE EXAMEN FINAL: ES LA ENTREGA DE SU TRABAJO POR ESCRITO.
   \end{topics}
   \begin{unitgoals}
      \item 
   \end{unitgoals}
\end{unit}

\begin{unit}{Semana 16B.}{Seco02}{24}{3}
   \begin{topics}
      \item El CURSO NO TIENE EXAMEN FINAL: ES LA ENTREGA DE SU TRABAJO POR ESCRITO.
   \end{topics}

   \begin{unitgoals}
      \item 
      \item
      \item 
      \end{unitgoals}
\end{unit}


\begin{coursebibliography}
\bibfile{GeneralEducation/GH0005}
\end{coursebibliography}

\end{syllabus}
