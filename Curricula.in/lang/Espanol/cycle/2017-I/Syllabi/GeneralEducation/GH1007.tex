\begin{syllabus}

\course{GH1007. Introducción al Desarrollo de Empresas}{Obligatorio}{GH1007} % Common.pm

\begin{justification}
Este curso tiene como objetivo proporcionar a los estudiantes una experiencia práctica de la vida  real en los  primeros pasos dentro de un ciclo de vida de negocios, a través del cual una idea se transforma en un modelo de negocio formal.
Es el primero de un conjunto de tres cursos diseñados para acompañar a los estudiantes a medida que transforman una idea en un negocio o negocio prospectivo, desde la idea  hasta la revisión de la estrategia empresarial actual.
\end{justification}

\begin{goals}
  \item Capacidad de Análisis de la información.
  \item Interpretación de información y resultados.
  \item Capacidad de Trabajo en equipo.
  \item Ética.
  \item Comunicación oral.
  \item Comunicación escrita.
  \item Comunicación gráfica.
  \item Entender la necesidad de aprender de forma continua.
\end{goals}

\begin{outcomes}
  \item \ShowOutcome{d}{2} 
  \item \ShowOutcome{e}{2} 
  \item \ShowOutcome{f}{2} 
  \item \ShowOutcome{n}{2} 
  \item \ShowOutcome{o}{2} 
\end{outcomes}

\begin{competences}
    \item \ShowCompetence{C10}{d,n,o}
    \item \ShowCompetence{C17}{f}
    \item \ShowCompetence{C18}{e}
\end{competences}

\begin{unit}{Introducción al Desarrollo de Empresas.}{}{Osterwalder10}{12}{4}
   \begin{topics}
      \item El ciclo de vida empresarial: desde la idea hasta la revisión de su estrategia.
      \item El proceso de ideación y la visión del cliente.
      \item Cómo construir y mantener equipos eficaces?
      \item Running LEAN: lo básico.
      \item Diseño de un modelo de negocio: herramientas de diseño y Canvas.
      \item Generación de Modelos de Negocio: Modelo de Negocio Canvas (Osterwalder).
      \item Venture Engineering:utilizando las habilidades de la informática para construir un modelo de negocio efectivo.
      \item Herramientas de investigación de mercado primario y nichos de mercado.
      \item La Importancia del Capital: Humano, Financiero e Intelectual
      \item Técnicas de monetización y financiamiento.
      \item Comunicación eficaz: crear una presentación de un modelo de negocio de impacto.
   \end{topics}
   \begin{learningoutcomes}
      \item Transformar una idea inicial de negocio o un proceso de innovación en un modelo de negocio factible.
   \end{learningoutcomes}
\end{unit}



\begin{coursebibliography}
\bibfile{GeneralEducation/GH1007}
\end{coursebibliography}

\end{syllabus}
