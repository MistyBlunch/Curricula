\begin{syllabus}

\course{CB101. Álgebra y Geometría}{Obligatorio}{CB101}

\begin{justification}
El curso de Preparación para la práctica preprofesional I, es el primero de dos cursos del área de desarrollo de competencias personales y profesionales. Este curso brinda
oportunidades al estudiante para que se autoevalúe y reconozca las oportunidades de mejora, sentirse capaz de realizar cada uno de los retos que se le proponga a nivel
personal y profesional y de realizar un adecuado análisis de su situación, ejecución y propuesta de resolución.
El aprendizaje a través de la experiencia, le ayudará a evaluar desde su propia perspectiva, las diferentes formas de pensar y la capacidad de aportar individualmente o en equipo al logro de un determinado objetivo; a través del intercambio de ideas, la
evaluación de propuestas y la ejecución de la alternativa de solución idónea.

\end{justification}

\begin{goals}
\item Comprende las responsabilidades profesional y ética
\item Capacidad de comunicación oral
\item Capacidad de comunicación escrita
\item Reconoce la necesidad del aprendizaje permanente
\end{goals}

\begin{outcomes}
\ShowOutcome{d}{2} % Multidiscip teams
\ShowOutcome{e}{2} % ethical, legal, security and social implications
\ShowOutcome{f}{2} % communicate effectively
\ShowOutcome{n}{2} % Apply knowledge of the humanities
\ShowOutcome{o}{2} % TASDSH
\end{outcomes}

\begin{competences}
    \item \ShowCompetence{C10}{d,n,o}
    \item \ShowCompetence{C17}{f}
    \item \ShowCompetence{C18}{f}
    \item \ShowCompetence{C21}{e}
\end{competences}

\begin{unit}{Semana 1.}{Robbins05}{12}{4}
   \begin{topics}
      \item Introducción al curso. Presentación de la metodología a aplicar (tipos de evaluación, cronograma de clases, talleres)
      \item Presentación a las competencias que se buscan desarrollar(liderazgo, trabajo en equipo, pensamiento crítico, entre otros). Teoría de las competencias y lo que quiere el mercado.
      \item Envío del DISC - online.
   \end{topics}
   \begin{unitgoals}
      \item .
   \end{unitgoals}
\end{unit}

\begin{unit}{Semana 2.}{Gomez09}{24}{3}
   \begin{topics}
      \item Repaso de los compromisos y acuerdos de la clase .
      \item Assessment center de autoevaluación.
      \item Dinámica sobre el autoconocimiento, la identificación de FODA personal y visión de futuro personal
   \end{topics}

   \begin{unitgoals}
      \item .
      \end{unitgoals}
\end{unit}

\begin{unit}{Semana 3.}{Robbins05}{24}{3}
   \begin{topics}
      \item Teoría. Personal branding. Plan de posicionamiento en el mercado (como voy hacer que mis competencias puedan ingresar al mercado).
      \item Cómo uno se debe comunicar, la utilización de la voz para potenciar sus habilidades y conseguir seguridad y eficacia en su comunicación.
      \item Teoría ¿Qué es un CV? ¿Cómo crear un CV innovador? Implementación de proyectos; actualización de datos, armado de portafolio de proyectos; comunicación virtual.
      \item Creación de un CV por grupo.
   \end{topics}

   \begin{unitgoals}
      \item .
   \end{unitgoals}
\end{unit}

\begin{unit}{Semana 4}{Robbins05}{30}{3}
   \begin{topics}
      \item Charla: Expectativas del mercado laboral. ¿Qué busca y quiere el mercado?
      \item Tipos de entrevistas y evaluaciones en el proceso de reclutamiento y selección. Uso de estrategias de persuasión; formas y técnicas exitosas para entrevistas, tips y recomendaciones.
      \item Entrega del Reto 1: Envío del CV
      \item Charla VOLCAN: “Tips de entrevistas y evaluaciones de reclutamiento y selección en Jueves del Conocimiento
      \item Tarea. Después de conocerse y saber lo que quiere el mercado, se crean los elementos para diseñar la propia estrategia de cada alumno
   \end{topics}

   \begin{unitgoals}
      \item .
   \end{unitgoals}
\end{unit}

\begin{unit}{Semana 5}{Gomez09}{24}{3}
   \begin{topics}
      \item Conversatorio de la charla de Volcan y de la clase anterior.
      \item Plataformas virtuales de empleo: revisión de las principales plataformas virtuales (CSM), correcto uso de la Bolsa UTEC.
      \item Linkedin como creador de relaciones: introducción a la red social; utilidad y trascendencia en la actualidad; reglas de uso y herramientas de LinkedIn. Exposición de los estudiantes sobre LinkedIn y herramientas similares
      \item Explicación del networkingUTEC.   
   \end{topics}

   \begin{unitgoals}
      \item .
   \end{unitgoals}
\end{unit}

\begin{unit}{Semana 6.}{Robbins05}{24}{3}
   \begin{topics}
      \item Networking UTEC: R eto 2: Consigue una entrevista. Envía tu CV a 40 empresas. Tómate una foto con tu vestimenta
      
   \end{topics}

   \begin{unitgoals}
      \item .
   \end{unitgoals}
\end{unit}

\begin{unit}{Semana 7.}{Bolles15}{24}{3}
   \begin{topics}
      \item Encuesta sobre el feedback sobre el Networking UTEC.
      \item Dinámica – Reto 3: Círculo de Entrevistas por competencias con profesionales
   \end{topics}

   \begin{unitgoals}
      \item 
      \item 
      \item 
     
   \end{unitgoals}
\end{unit}

\begin{unit}{Semana 8.}{Gomez09}{24}{3}
   \begin{topics}
      \item PARCIALES ( Entrevistas por competencias. Revisión del CV y LinkedIn) 
   \end{topics}

   \begin{unitgoals}
      \item 
      \item 
      \item 
     
   \end{unitgoals}
\end{unit}

\begin{unit}{Semana 9.}{Robbins05}{24}{3}
   \begin{topics}
      \item Entrega y discusión de los resultados del assessment center.
      \item Assessment center en clase, con la aplicación de casos reales, en la medición de las competencias : Proactividad, análisis de problemas, pensamiento analítico y planificación y organización, trabajo en equipo y liderazgo, adaptabilidad, comunicación asertiva, ética
    \end{topics}

   \begin{unitgoals}
      \item 
      \item 
      \item 
     
   \end{unitgoals}
\end{unit}

\begin{unit}{Semana 10.}{Bolles15}{24}{3}
   \begin{topics}
      \item Conversatorio sobre los resultados de la clase anterior
      \item Taller fuera del salón: Proactividad, análisis de problemas, pensamiento analítico y planificación y organización, trabajo en equipo y liderazgo, adaptabilidad, comunicación asertiva, ética. Se tendrá el feedback.
   \end{topics}

   \begin{unitgoals}
      \item .
   \end{unitgoals}
\end{unit}

\begin{unit}{Semana 11.}{Gomez09}{24}{3}
   \begin{topics}
      \item Charla: Derechos y obligaciones laborales de los practicantes.
   \end{topics}

   \begin{unitgoals}
      \item .
   \end{unitgoals}
\end{unit}

\begin{unit}{Semana 12.}{Robbins05}{24}{3}
   \begin{topics}
      \item Entrevista a expertos: El mundo real laboral desde la visión del área de recursos humanos -RRHH, con jefes corporativos de selección como invitados
   \end{topics}

   \begin{unitgoals}
      \item .
   \end{unitgoals}
\end{unit}

\begin{unit}{Semana 13.}{Bolles15}{24}{3}
   \begin{topics}
      \item Conversatorio y presentación de los alumnos sobre la entrevista a expertos y temas relacionados.
      \item Reforzamiento de las evaluaciones de reclutamiento y selección.
      \item Dinámica de reclutamiento: Reto 4: Cómo te va.
      \end{topics}

   \begin{unitgoals}
      \item .
   \end{unitgoals}
\end{unit}

\begin{unit}{Semana 14.}{Robbins05}{24}{3}
   \begin{topics}
      \item Caso de Estudio
      \item Competencias: Planificación y organización y confianza en sí mismo, asociado a la resolución de problemas.
      \item Feedback de resultados sobre la dinámica de refuerzo.
   \end{topics}

   \begin{unitgoals}
      \item .
   \end{unitgoals}
\end{unit}

\begin{unit}{Semana 15.}{Gomez09}{24}{3}
   \begin{topics}
      \item Reto 5: Assessment center de consolidación de competencias.
      \item Feedback de resultados finales.
   
   \end{topics}

   \begin{unitgoals}
      \item . 
     
   \end{unitgoals}
\end{unit}

\begin{unit}{Semana 16}{Bolles15}{24}{3}
   \begin{topics}
      \item Examenes Finales 
   \end{topics}

   \begin{unitgoals}
      \item 
      \item 
      \item 
     
   \end{unitgoals}
\end{unit}

\begin{coursebibliography}
\bibfile{GeneralEducation/GH2015}
\end{coursebibliography}

\end{syllabus}
