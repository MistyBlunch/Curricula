\begin{syllabus}

\course{GH0010. Ética y Tecnología}{Obligatorio}{GH0010}

\begin{justification}
Este curso busca proporcionar a los y las estudiantes ciertos marcos referenciales con los cuales analizar las disyuntivas que se pueden presentar en su ejercicio profesional. El curso pone en práctica constante el razonamiento crítico y responsable de los  y las estudiantes, siendo esta una competencia fundamental para los procesos de toma de decisión que asumiremos como profesionales y ciudadanos.
\end{justification}

\begin{goals}
\item Introducir a los estudiantes al pensamiento crítico y ético aplicado a su campo profesional.
\item Fortalecer en el estudiante la capacidad de pensar interdisciplinariamente.
\item Desarrollar la competencia de mirar un fenómeno desde varias disciplinas y perspectivas genera en la persona empatía y respeto a la diversidad de opinión.
\item Capacidad de trabajo en equipo.
\item Capacidad para identificar problemas
\item Comprender las responsabilidades profesional y ética.
\item Capacidad de comunicación oral
\item Comprende el impacto de las soluciones de la ingeniería en un contexto global, económico, ambiental y de la sociedad-
\item Tiene interés por conocer sobre temas actuales de la sociedad peruana y del mundo.
\item Capacidad de comunicación escrita.
\end{goals}

\begin{outcomes}
\ShowOutcome{d}{2} % Multidiscip teams
\ShowOutcome{e}{2} % ethical, legal, security and social implications
\ShowOutcome{f}{2} % communicate effectively
\ShowOutcome{n}{2} % Apply knowledge of the humanities
\ShowOutcome{o}{2} % TASDSH
\end{outcomes}

\begin{competences}
    \item \ShowCompetence{C10}{d,n,o}
    \item \ShowCompetence{C17}{f}
    \item \ShowCompetence{C18}{f}
    \item \ShowCompetence{C21}{e}
\end{competences}

\begin{unit}{}{Ética, ciencia y tecnología.}{Garcia06}{12}{4}
   \begin{topics}
      \item Definición y alcance de la ética Pensamiento crítico /  argumentación ética
      \item Ciencia y Tecnología, ¿son las ingenierías y la tecnología cuestiones objetivas? 
      \item Tecnología: concepto y límites
      \item Importancia de la ética en las ciencias e ingeniería .

   \end{topics}
   \begin{learningoutcomes}
      \item .
   \end{learningoutcomes}
\end{unit}

\begin{unit}{}{Traditional ethical values and norms vs. new technological context}{Garcias06}{24}{3}
   \begin{topics}
      \item Consideraciones éticas en temas de autoría y derechos de uso en la era digital. Derechos de autor/ copyright/ piratería digital  Caso: Free software movement

   \end{topics}

   \begin{learningoutcomes}
      \item .
      \end{learningoutcomes}
\end{unit}

\begin{unit}{}{Responsabilidad en la ciencia e ingeniería}{Alvarado05}{24}{3}
   \begin{topics}
      \item Alcance del concepto  Responsabilidad en la ciencia (Imperative of Responsability)
      \item Introducción al tema Responsabilidad / libertad 
      \end{topics}

   \begin{learningoutcomes}
      \item .
     
   \end{learningoutcomes}
\end{unit}

\begin{unit}{}{Bioética }{Alvarado05}{30}{3}
   \begin{topics}
      \item  Concepto central. Orígenes. Casos significativos (primeros caso de genoma, Comité de Bioétia en EEUU)
      \item  Caso: Transhumanism
      \item  Caso: Derecho a morir. Debate en torno al retiro de la ventilación artificial.
   \end{topics}

   \begin{learningoutcomes}
      \item . 
   \end{learningoutcomes}
\end{unit}

\begin{unit}{}{Robot Uprising}{Alvarado05}{30}{3}
   \begin{topics}
      \item  Caso análisis: 2021 ¿humanos sin trabajo? 
      \item  Caso: Legalización el matrimonio humano/ robots 
   \end{topics}

   \begin{learningoutcomes}
      \item .
   \end{learningoutcomes}
\end{unit}

\begin{unit}{}{ Tecnología e Ingeniería Sustentable}{Alvarado05}{30}{3}
   \begin{topics}
      \item Reflexión desde el contexto peruano acerca de la responsabilidad en las ciencias e Ingeniería. 
      \item Casos peruano: Obras de ingeniería vs poblaciones
      \item Caso : Energías renovables y sustentable en el Perú 
   \end{topics}

   \begin{learningoutcomes}
      \item . 
   \end{learningoutcomes}
\end{unit}

\begin{unit}{}{ Ciudadanía y ejercicio de la justicia en la era digital}{Alvarado05}{30}{3}
   \begin{topics}
      \item Introducción al tema de ciudadanía en la era digital
      \item Tecnología,  nuevos activismos y ciudadanía
   \end{topics}

   \begin{learningoutcomes}
      \item .
   \end{learningoutcomes}
\end{unit}

\begin{unit}{}{ Hackctivismo}{Alvarado05}{30}{3}
   \begin{topics}
      \item Caso: Anonnymus
   \end{topics}

   \begin{learningoutcomes}
      \item .
   \end{learningoutcomes}
\end{unit}

\begin{unit}{}{Ciberfeminismo}{Alvarado05}{30}{3}
   \begin{topics}
      \item Ciudadanía y género en la era digital Ciberfeminismo y tecnofeminismo:  Caso: ‘90: Mujeres en red y E-leusis

   \end{topics}

   \begin{learningoutcomes}
      \item .
   \end{learningoutcomes}
\end{unit}



\begin{coursebibliography}
\bibfile{GeneralEducation/GH0010}
\end{coursebibliography}

\end{syllabus}
