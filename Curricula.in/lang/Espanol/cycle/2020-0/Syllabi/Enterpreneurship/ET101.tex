\begin{syllabus}

\course{ET1001. Formación de Empresas de Base Tecnológica I}{Obligatorio}{ET1001}
% Source file: ../Curricula.in/lang/Espanol/cycle/2019-II/Syllabi/Enterpreneurship/ET101.tex

\begin{justification}
Este es el primer curso dentro del área formación de empresas de
base tecnológica, tiene como objetivo dotar al futuro profesional 
de conocimientos, actitudes y aptitudes que le
permitan elaborar un plan de negocio para una empresa de base tecnológica.
El curso está dividido en las siguientes unidades:
Introducción, Creatividad, De la idea a la oportunidad, el modelo Canvas, Customer Development y Lean Startup, Aspectos Legales y Marketing, Finanzas de la empresa y Presentación.

Se busca aprovechar el potencial creativo e innovador y el esfuerzo de los alumnos en la creación de nuevas empresas.
\end{justification}

\begin{goals}
\item Que el alumno conozca como elaborar un plan de negocio para dar inicio a una empresa de base tecnológica.
\item Que el alumno sea capaz de realizar, usando modelos de negocio, la concepción y presentación de una propuesta de negocio.
\end{goals}

\begin{outcomes}{V1}
\ShowOutcome{d}{3}
\ShowOutcome{f}{3}
\ShowOutcome{m}{2}
\end{outcomes}

\begin{unit}{Introducción}{}{byers10,osterwalder10,garzozi14}{5}{1}
\begin{topics}
      \item Emprendedor, emprendedurismo e innovación tecnológica
      \item Modelos de negocio
      \item Formación de equipos
   \end{topics}

   \begin{learningoutcomes}
      \item Identificar caracterTecnologíasticas de los emprendedores
      \item Introducir modelos de negocio 
   \end{learningoutcomes}
\end{unit}

\begin{unit}{Creatividad}{}{byers10,blank12,garzozi14}{5}{1}
\begin{topics}
      \item Visión
      \item Misión
      \item La Propuesta de valor
      \item Creatividad e invención
      \item Tipos y fuentes de innovación
      \item Estrategia y Tecnología
      \item Escala y ámbito
   \end{topics}

   \begin{learningoutcomes}
      \item Plantear correctamente la vision y misión de empresa
	  \item Caracterizar una propuesta de valor innovadora
      \item Identificar los diversos tipos y fuentes de innovación
   \end{learningoutcomes}
\end{unit}

\begin{unit}{De la Idea a la Oportunidad}{}{byers10,osterwalder10,ries11,garzozi14}{5}{1}
\begin{topics}
      \item Estrategia de la Empresa
      \item Barreras 
      \item Ventaja competitiva sostenible
      \item Alianzas
      \item Aprendizaje organizacional
      \item Desarrollo y diseño de productos
   \end{topics}

   \begin{learningoutcomes}
      \item Conocer estrategias empresariales
      \item Caracterizar barreras y ventajas competitivas 
       
    \end{learningoutcomes}
\end{unit}

\begin{unit}{El Modelo Canvas}{}{osterwalder10,blank12,garzozi14}{20}{3}
	\begin{topics}
      \item Creación de un nuevo negocio
      \item El plan de negocio 
      \item Canvas
      \item Elementos del Canvas
   \end{topics}

   \begin{learningoutcomes}
      \item Conocer los elementos del modelo Canvas
      \item Elaborar un plan de negocio basado en el modelo Canvas 
    \end{learningoutcomes}
\end{unit}

\begin{unit}{Customer Development y Lean Startup}{}{blank12,ries11,garzozi14}{20}{3}
	\begin{topics}
      \item Aceleración versus incubación  
      \item Customer Development
      \item Lean Startup 
   \end{topics}

   \begin{learningoutcomes}
      \item Conocer y aplicar el modelo Customer Development
      \item Conocer y aplicar el modelo Lean Startup
    \end{learningoutcomes}
\end{unit}

\begin{unit}{Aspectos Legales y Marketing}{}{byers10,ries11,congreso96, congreso97,garzozi14}{5}{1}
	\begin{topics}
	  \item Aspectos Legales y tributarios para la constitución de la empresa
      \item Propiedad intelectual
      \item Patentes
      \item Copyrights y marca registrada
      \item Objetivos de marketing  y segmentos de mercado
      \item Investigación de mercado y búsqueda de clientes
   \end{topics}

   \begin{learningoutcomes}
      \item Conocer los aspectos legales necesarios para la formación de una empresa tecnológica
      \item Identificar segmentos de mercado y objetivos de marketing    
   \end{learningoutcomes}
\end{unit}

\begin{unit}{Finanzas de la Empresa}{}{byers10,blank12,garzozi14}{5}{1}
	\begin{topics}
      \item Modelo de costos
      \item Modelo de utilidades
      \item Precio
      \item Plan financiero
      \item Formas de financiamiento
      \item Fuentes de capital
      \item Capital de riesgo
   \end{topics}

   \begin{learningoutcomes}
      \item Definir um modelo de costos y utilidades
      \item Conocer las diversas fuentes de financiamento
   \end{learningoutcomes}
\end{unit}

\begin{unit}{Presentación}{}{byers10,blank12,garzozi14}{5}{1}
	\begin{topics}
      \item The Elevator Pitch
      \item Presentación
      \item Negociación
    \end{topics}

   \begin{learningoutcomes}
      \item Conocer las diversas formas de presentar propuestas de negocio
      \item Realizar la presentación de una propuesta de negocio
   \end{learningoutcomes}
\end{unit}

\begin{coursebibliography}
\bibfile{Enterpreneurship/ET101}
\end{coursebibliography}

\end{syllabus}
