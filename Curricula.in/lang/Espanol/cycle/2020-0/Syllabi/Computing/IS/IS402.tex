\begin{syllabus}

\course{IS4002. Proyecto de Final de Carrera I}{Obligatorio}{IS4002}
% Source file: ../Curricula.in/lang/Espanol/cycle/2020-I/Syllabi/Computing/IS/IS402.tex

\begin{justification}
Este curso tiene por objetivo que el alumno pueda realizar un estudio del estado del arte de un que el alumno ha elegido como tema para su tesis.
\end{justification}

\begin{goals}
\item Que el alumno realice una investigación inicial en un tema especifico realizando el estudio del estado del arte del tema elegido.
\item Que el alumno muestre dominio en el tema de la línea de investigación elegida.
\item Que el alumno elija un docente que domine el de investigación elegida como asesor. 
\item Los entregables de este curso son:
	\begin{description}
		\item [Avance parcial:] Bibliografía sólida y avance de un Reporte Técnico.
		\item [Final:] Reporte Técnico con experimentos preliminares comparativos que demuestren que el alumno ya conoce las técnicas existentes en el área de su proyecto y elegir a un docente que domine el área de su proyecto como asesor de su proyecto.
	\end{description}
\end{goals}

\begin{outcomes}{V1}
\item \ShowOutcome{a}{2}
\item \ShowOutcome{b}{3}
\item \ShowOutcome{c}{2}
\item \ShowOutcome{e}{3}
\item \ShowOutcome{f}{2}
\item \ShowOutcome{h}{2}
\item \ShowOutcome{i}{3}
\item \ShowOutcome{l}{2}
\end{outcomes}

\begin{outcomes}{V2}
\item \ShowOutcome{1}{2}
\item \ShowOutcome{2}{2}
\item \ShowOutcome{3}{2}
\item \ShowOutcome{6}{2}
\end{outcomes}

\begin{competences}{V1}
\item \ShowCompetence{C1}{a,b,c} 
\item \ShowCompetence{C20}{e,f,g}
\item \ShowCompetence{CS2}{h,i,l}
\end{competences}

\begin{competences}{V2}
\item \ShowCompetence{C1}{1,2} 
\item \ShowCompetence{C20}{3,6}
\item \ShowCompetence{CS2}{6}
\end{competences}

\begin{unit}{Levantamiento del estado del arte}{}{ieee,acm,citeseer}{60}{C1,C20,CS2}
  \begin{topics}
      \item Realizar un estudio profundo del estado del arte en un determinado tópico del área de Computación.
      \item Redacción de artículos técnicos en computación.
  \end{topics}
  \begin{learningoutcomes}
      \item Hacer un levantamiento bibliográfico del estado del arte del tema escogido (esto significa muy probablemente 1 o 2 capítulos de marco teórico además de la introducción que es el capítulo I de la tesis) [\Usage]
      \item Redactar un documento en latex en formato articulo (\emph{paper}) con mayor calidad que en Proyecto I (dominar tablas, figuras, ecuaciones, índices, bibtex, referencias cruzadas, citaciones, pstricks) [\Usage]
      \item Tratar de hacer las presentaciones utilizando prosper [\Usage]
      \item Mostrar experimentos básicos [\Usage]
      \item Elegir un asesor que domine el área de investigación realizada [\Usage]
   \end{learningoutcomes}
\end{unit}

\begin{coursebibliography}
\bibfile{Computing/CS/CS401}
\end{coursebibliography}

\end{syllabus}
