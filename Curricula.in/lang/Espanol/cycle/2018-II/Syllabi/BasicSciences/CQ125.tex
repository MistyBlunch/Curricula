\begin{syllabus}

\course{CQ125. Laboratorio de QuTecnologíamica 1}{Obligatorio}{CQ125} % Common.pm

\begin{justification}
En este curso se promueve el desarrollo y fortalecimiento de las siguientes competencias del perfil de egreso de Estudios Generales Ciencias: comunicación, A1; autoaprendizaje, B1; desempeño personal y académico, C3 y aprender a aprender, D2. La dinátrabajo en el laboratorio de química implica en primer lugar la conformación de grupos permanentes de estudiantes que trabajaráááááááááárativa durante la realización de las experiencias a lo largo del semestre. Para cada sesión se considera una etapa de preparación al desarrollo experimental mediante la cual el estudiante debe revisar los conceptos y procedimientos que se aplicaráná áráráeáiárátá áentes (libros de texto, videos, págiás á iáerát,átcá, án á qá se promueve el desarrollo de sus propios estilos de aprendizaje. Durante el desarrollo de cada sesión el estudiante realizará opáaciáes ásicáparál táajoáperántaáapliándo ás noáas de seguridad pertinentes, para finalmente elaborar conjuntamente con sus compañeros de grupo un reporte escrito siguiendo las pautas de un informe científico.

Al término del semestre el estudiante:
\end{justification}

\begin{goals}

\item Manipularáreza los implementos báááááááááátorio de química y obtendrá átá áeáaásánáaábárvación de un proceso experimental. 
\item Aplicaráe seguridad establecidas para la manipulación de sustancias químicas y procedimientos bááááááááááio.
\item Realizarááááááááo de los datos obtenidos de sus observaciones y formulará áaáxáiácá³áaáaáiáde ello.
\item Elaborará escritos de las observaciones realizadas, de los datos registrados y de los resultados obtenidos; y, tendrááááááááááica frente al procedimiento realizado para poder proponer posibles fuentes de error y sugerencias para mejorarlo. 
\item Mostrarádes para el trabajo en equipo.
\item Valoraráoración como un medio de obtener mejores aprendizajes.
\end{goals}

\begin{outcomes}
\item \ShowOutcome{d}{2}
\item \ShowOutcome{h}{2}
\end{outcomes}

\begin{competences}
    \item \ShowCompetence{C20}{d,h}
\end{competences}

\begin{unit}{CapTecnologíatulo 1: Identificación de sustancias usando procedimientos físicos y quí­micos (10 horas)}{}{}{4}{C20}

Conceptuales

\begin{topics}
      \item Identifica y diferencia cambios físicos y cambios químicos.
      \item Identifica los tipos y componentes de una mezcla
      \item Escribe e interpreta una ecuación química.
      \item Interpreta los resultados experimentales obtenidos en base a la teorTecnologíaa.
   \end{topics}
   
   Procedimentales

\begin{topics}
      \item Realiza cá moles obtenidas en una reacción. 
      \item Prepara soluciones sencillas.
      \item Determina y expresa la concentración de la solución obtenida.
      \item Sigue pautas establecidas para el trabajo científico y el desempeño en el laboratorio.
   \end{topics}
   
   Actitudinales

\begin{topics}
      \item Conoce y cumple las normas de seguridad en el laboratorio.
      \item Trabaja en equipo.
      \item Agudiza su capacidad de observación.
      \item Mantiene el orden y la limpieza de los materiales recibidos.
   \end{topics}

   \begin{learningoutcomes}
      \item 
   \end{learningoutcomes}
\end{unit}

\begin{unit}{CapTecnologíatulo 2: Configuración electrónica y Propiedades periódicas (10 horas)}{}{}{4}{C20}

Conceptuales

\begin{topics}
      \item Explica la relación entre el color de luz emitida y configuración electrónica de algunas sales metá
      \item Explica la reactividad frente al agua de algunos metales en función de la configuración electrónica.
   \end{topics}
   
   Procedimentales

\begin{topics}
      \item Sigue pautas establecidas para el trabajo científico y el desempeño en el laboratorio.
      \item Manipula sustancias químicas y materiales aplicando las normas correspondientes de seguridad en el laboratorio.
      \item Elabora un reporte escrito de la experiencia, siguiendo formato establecido.
   \end{topics}
   
   Actitudinales

\begin{topics}
      \item Comprende mááááááá³n.
      \item Trabaja de manera organizada y responsable.
      \item Relaciona los conceptos teóricos con actividades prá
      \item Busca información complementaria que permita ampliar sus conocimientos.
     \item Reconoce la importancia del trabajo en equipo.
   \end{topics}

\end{unit}

\begin{unit}{CapTecnologíatulo 3: Experimentando con gases)}{}{}{4}{C20}

Conceptuales

\begin{topics}
      \item Explica el comportamiento gaseoso aplicando las leyes de Boyle y Charles.
      \item Explica algunas propiedades químicas del $CO_2$(g).
   \end{topics}
   
   Procedimentales

\begin{topics}
      \item Demuestra experimentalmente la ley de Boyle realizando la recolección y el tratamiento de datos pertinentes y elaborando e interpretando grás V y P vs 1/V.
      \item Aplica el procedimiento de generación de gases en jeringas.
      \item Manipula adecuadamente instrumentos de medición volumétrica. 
      \item Elabora un reporte escrito de la experiencia, siguiendo formato establecido.
   \end{topics}
   
   Actitudinales

\begin{topics}
      \item Muestra interés y compromiso con su aprendizaje cumpliendo con la preparación de las actividades previas a la experiencia.
      \item Conoce y cumple las normas de seguridad en el laboratorio.
      \item Trabaja en equipo, de manera responsable y solidaria
      \item Demuestra disposición a cambiar o modificar sus puntos de vista si es necesario.
     \item Asume responsabilidad por su propio aprendizaje al mismo tiempo que sobre el de sus compañeros de equipo.
      \item Escucha las opiniones de los demá sean divergentes.
     \item Trabaja de manera organizada.
   \end{topics}

   \begin{learningoutcomes}
      \item 
   \end{learningoutcomes}
\end{unit}

\begin{unit}{CapTecnologíatulo 4: Un problema de seguridad industrial: reacciones REDOX y el método del ion - electrón}{}{}{6}{C20}

Conceptuales

\begin{topics}
      \item  Explica las observaciones de los resultados de ensayos correspondientes a reacciones de óxido-reducción, señalando al agente oxidante, agente reductor, la especie oxidada y la especie reducida, en cada caso.
   \end{topics}
   
   Procedimentales

\begin{topics}
      \item Aplica el método del ión-electrón para realizar el balance de las reacciones de óxido-reducción ensayadas, asTecnología como de los ejercicios que se les proponga.
      \item Elabora un reporte escrito de la experiencia, siguiendo formato establecido.
   \end{topics}
   
   Actitudinales

\begin{topics}
      \item Muestra interés y compromiso con su aprendizaje cumpliendo con la preparación de las actividades previas a la experiencia.
      \item Conoce y cumple las normas de seguridad en el laboratorio.
      \item Trabaja en equipo.
      \item Agudiza su capacidad de observación.
     \item Mantiene el orden y la limpieza de los materiales recibidos.
   \end{topics}

   \begin{learningoutcomes}
      \item 
   \end{learningoutcomes}
   
\end{unit}

\begin{unit}{CapTecnologíatulo 5:  EstequiometrTecnologíaa: Titulación REDOX}{}{}{4}{C20}

Conceptuales

\begin{topics}
      \item  Explica las observaciones de los resultados de ensayos correspondientes a reacciones de óxido-reducción, señalando al agente oxidante, agente reductor, la especie oxidada y la especie reducida, en cada caso.
   \end{topics}
   
   Procedimentales

\begin{topics}
      \item Aplica el método del ión-electrón para realizar el balance de las reacciones de óxido-reducción ensayadas, asTecnología como de los ejercicios que se les proponga.
      \item Elabora un reporte escrito de la experiencia, siguiendo formato establecido.
   \end{topics}
   
   Actitudinales

\begin{topics}
      \item Muestra interés y compromiso con su aprendizaje cumpliendo con la preparación de las actividades previas a la experiencia.
      \item Conoce y cumple las normas de seguridad en el laboratorio.
      \item Trabaja en equipo.
      \item Agudiza su capacidad de observación.
     \item Mantiene el orden y la limpieza de los materiales recibidos.
   \end{topics}

   \begin{learningoutcomes}
      \item 
   \end{learningoutcomes}
   
\end{unit}

\begin{unit}{CapTecnologíatulo 5:  CalorimetrTecnologíaa a presión constante: determinación del calor de neutralización.}{}{}{4}{C20}

Conceptuales

\begin{topics}
      \item  Explica el proceso de neutralización ábase incluyendo el calor involucrado.
   Aplica el concepto de reactivo limitante en el anáequiométrico de una reacción.
	\item Explica el procedimiento experimental para realizar mediciones calorimétricas a presión constante.
	\item Compara los resultados experimentales obtenidos con los valores teóricos y señala las posibles fuentes de error 
   \end{topics}
   
   Procedimentales

\begin{topics}
      \item Determina el calor de neutralización de ártes con bases fuertes, realizando mediciones y aplicando relaciones calorimétricas a presión constante, asTecnología como el concepto de reactivo limitante.
      \item Manipula adecuadamente instrumentos bálaboratorio: instrumentos de medición volumétrica, balanza, termómetro.
      \item Elabora un reporte escrito de la experiencia, siguiendo formato establecido..
   \end{topics}
   
   Actitudinales

\begin{topics}
      \item Muestra interés y compromiso con su aprendizaje cumpliendo con la preparación de las actividades previas a la experiencia.
	  \item Conoce y cumple las normas de seguridad en el laboratorio.
	  \item Trabaja en equipo, de manera responsable y solidaria.
	  \item Demuestra disposición a cambiar o modificar sus puntos de vista si es necesario.
	  \item Asume responsabilidad por su propio aprendizaje al mismo tiempo que sobre el de sus compañeros de equipo.
	  \item Escucha las opiniones de los demá sean divergentes.
	  \item Trabaja de manera organizada.
   \end{topics}

   \begin{learningoutcomes}
      \item 
   \end{learningoutcomes}
   
\end{unit}



\begin{coursebibliography}
\bibfile{BasicSciences/CQ125}
\end{coursebibliography}

%BROWN, T. L.; LE MAY Jr., H. E.; BURSTEN, B. E.
%2009	QuTecnologíamica, la ciencia central. 11.a ed. México: Pearson Educación.
%BROWN, L.; HOLME T.
%2015	Chemistry for Engineering Students, 3rd edition, Brooks/Cole-Cengage Learning.
%CHANG, R.
%2010	QuTecnologíamica. 10.a ed. México: Mc Graw-Hill.
%KELTER, P.; MOSHER, M.; SCOTT, A.
%2007	Chemistry: The Practical Science. Brooks Cole, 2007.
%SILBERBERG, M.S.
%2009	Chemistry: The Molecular Nature of Matter and Change. 5th ed. New York: McGraw-Hill.
%ZUMDAHL, S.
%2007	Fundamentos de QuTecnologíamica, 5a edición. McGraw-Hill.

\end{syllabus}
