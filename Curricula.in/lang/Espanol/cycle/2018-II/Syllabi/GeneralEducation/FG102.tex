\begin{syllabus}

\course{FG102. MetodologTecnologíaa del Estudio}{Obligatorio}{FG102} % Common.pm

\begin{justification}
Los alumnos en formación profesional necesitan mejorar su actitud frente al trabajo y exigencia académicos. Además conviene que entiendan el proceso mental que se da en el ejercicio del estudio para lograr el aprendizaje; asTecnología  sabrán dónde y cómo hacer los ajustes más convenientes a sus necesidades. Asimismo, requieren dominar variadas formas de estudiar, para que puedan seleccionar las estrategias  más convenientes a su personal estilo de aprender y a la naturaleza de cada asignatura. De igual modo conocer y usar  maneras de buscar información académica y realizar trabajos creativos de tipo académico formal, asTecnología podrán  aplicarlos a su trabajo universitario, haciendo exitoso su esfuerzo.
\end{justification}

\begin{goals}
\item Desarrollar en el estudiante actitudes y habilidades que promuevan la autonomTecnologíaa en el aprendizaje, el buen desempeño académico y su formación como persona y profesional.
\end{goals}

\begin{outcomes}{V1}
    \item \ShowOutcome{d}{2}
    \item \ShowOutcome{h}{2}
    \item \ShowOutcome{l}{1}
\end{outcomes}

\begin{competences}{V1}
    \item \ShowCompetence{C19}{h}
    \item \ShowCompetence{C24}{h,d}
\end{competences}

\begin{unit}{}{Primera Unidad: La universidad, trabajo intelectual y organización}{BibliografTecnologíaa}{12}{C19, C24}
\begin{topics}
        \item El subrayado.
        \item Toma de puntes.
        \item La vocación, hábitos de la vida universitaria.
        \item Interacción humana.
        \item La voluntad como requisito para el aprendizaje.
        \item La plantificación y el tiempo
\end{topics}
\begin{learningoutcomes}
        \item Analizar la documentación normativa de la Universidad valorando su importancia para la  convivencia y desempeño académico. [\Usage]
        \item Comprender y valorar la exigencia de la vida universitaria como parte de la formación personal y profesional.[\Usage]
        \item Planificar adecuadamente el tiempo  en función de sus metas personales y académicas.[\Usage]
        \item Elaborar un plan de mejora personal a partir del conocimiento de sTecnología mismo.[\Usage]
\end{learningoutcomes}
\end{unit}

\begin{unit}{}{Segunda Unidad}{Rodriguez, Pereza,Quintana}{12}{C19,C24}
\begin{topics}
   \item Resumen. Notas al margen. Nemotecnias.
   \item Procesos mentales: Simples, complejos. Fundamentos del aprendizaje significativo.
   \item Los pasos o factores para el aprendizaje. Leyes del aprendizaje. Cuestionario de estilos de aprendizaje Identificación del estilo de aprendizaje personal
   \item La lectura académica. Niveles de  análisis de un texto: idea central, idea principal e ideas secundarias. El modelo de Meza de Vernet.
   \item Exámenes: Preparación. Pautas y estrategias para antes, durante y después de un examen. Inteligencia emocional y exámenes.
   \item Las fuentes de información. Aparato crTecnologíatico: concepto y finalidad. Normas Vancouver. Referencias y citas.
\end{topics}
\begin{learningoutcomes}
        \item Identificar los procesos mentales relacionándolos con el aprendizaje [\Usage].
        \item Comprender el proceso del aprendizaje para determinar el estilo propio e incorporarlo en su actividad académica [\Usage].
        \item Desarrollar estrategias para el análisis de textos potenciando la comprensión lectora [\Usage].
        \item Diseñar un programa estratégico para afrontar con éxito los exámenes[\Usage].
\end{learningoutcomes}
\end{unit}

\begin{unit}{}{Tercera Unidad}{Chaveza, Flores}{12}{C24}
\begin{topics}
        \item Los mapas conceptuales. CaracterTecnologíasticas y elementos.
        \item Los derechos de autor y el plagio. Derechos personales o morales. Derechos patrimoniales. ``Copyrigth''.
        \item Autoestima, Inteligencia Emocional, Asertividad y Resiliencia. Conceptos, desarrollo y fortalecimiento.
        \item Aparato crTecnologíatico: Normas Vancouver. Aplicación práctica.
        \item Generación de ideas. Estrategias para organizar las ideas, redacción y revisión.
\end{topics}
\begin{learningoutcomes}
        \item Aplicar las técnicas de estudio atendiendo a sus particularidades y adecuándolas a las distintas situaciones que demanda el aprendizaje [\Usage].
        \item Reconocer la importancia del respeto a la propiedad Intelectual [\Usage].
        \item Reconocer la importancia de la Inteligencia Emocional, la conducta asertiva, la autoestima y la resiliencia valorándolas como fortalezas para el desempeño universitario [\Usage].
\end{learningoutcomes}
\end{unit}

\begin{unit}{}{Cuarta Unidad}{Rodriguez, Chaveza}{12}{C19}
\begin{topics}
        \item Cuadro Sinóptico. Los mapas mentales. Practicas con la temática del curso.
        \item El método personal de estudio.
        \item El aprendizaje cooperativo: definición, los grupos de estudio, organización, roles de los miembros.
        \item Pautas para conformar grupos eficientes y armónicos.
        \item El método personal de estudio.Reforzamiento de técnicas de estudio.
        \item Presentación y exposición de trabajos de producción intelectual.
        \item El debate y la argumentación.
\end{topics}
\begin{learningoutcomes}
        \item Aplicar las técnicas de estudio atendiendo a sus particularidades y adecuándolas a las distintas situaciones que demanda el aprendizaje [\Usage].
        \item Asumir el manejo de conductas y actitudes para el aprendizaje cooperativo y el desempeño en los equipos de trabajo [\Usage].
        \item Formular un proyecto de método personal de estudio, de acuerdo a su estilo y necesidades, que incluya técnicas y estrategias [\Usage].
\end{learningoutcomes}
\end{unit}

\begin{coursebibliography}
\bibfile{GeneralEducation/FG101}
\end{coursebibliography}

\end{syllabus}
