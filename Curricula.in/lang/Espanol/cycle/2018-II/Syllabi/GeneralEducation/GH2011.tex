\begin{syllabus}

\course{GH0011. Innovación y Desarrollo de Productos}{Obligatorio}{GH0011} % Common.pm

\begin{justification}
Este curso está diseñado para proporcionar a los estudiantes una sólida comprensión del proceso de innovación dentro de una empresa. Se centra en la aplicación de las habilidades de innovación empresarial en una empresa bien establecida. Esto se conoce como Intrapreneurship.
Es el tercero de un conjunto de tres cursos diseñados para acompañar a los estudiantes a medida que transforman una idea en un negocio o empresa potencial. El estudiante experimentará el proceso desde la fase de ideación hasta la revisión de las estrategias de negocios actuales.
El material visto en este curso responde a 2 preguntas principales: "?` Qué debe hacer?" Y "?` Cómo debe hacerlo?". 

\end{justification}

\begin{goals}
\item Identificar cómo se relaciona la innovación con el proceso emprendedor e intraempresarial
\item Familiarizarse con las herramientas de innovación y practicar cómo hacer uso de ellas.
\item Aprender a integrar la innovación en el ciclo económico.
\item Comprender la importancia de la estrategia y la implementación y cómo una idea debe ir acompañada de un plan de implementación efectivo
\item Análisis de la información
\item Interpretación de información y resultados.
\item Trabajo en equipo.
\item Ética.
\item Comunicación oral.
\item Comunicación escrita
\item Comunicación gráfica
\item Entiendimiento de la necesidad de aprender de forma continua
\end{goals}

\begin{outcomes}{V1}
    \item \ShowOutcome{d}{2} % Multidiscip teams
    \item \ShowOutcome{e}{2} % ethical, legal, security and social implications
    \item \ShowOutcome{f}{2} % communicate effectively
    \item \ShowOutcome{n}{2} % Apply knowledge of the humanities
    \item \ShowOutcome{o}{2} % TASDSH
\end{outcomes}

\begin{competences}{V1}
    \item \ShowCompetence{C10}{d,n,o}
    \item \ShowCompetence{C17}{f}
    \item \ShowCompetence{C18}{f}
    \item \ShowCompetence{C21}{e}
\end{competences}

\begin{unit}{Innovación y Desarrollo de Productos}{}{Morales13}{12}{4}
   \begin{topics}
      \item Creatividad:entendiendo cómo funciona nuestro cerebro.
      \item Innovación: ?` Quién ,Qué,Por qué, Cuándo,Dónde ?
      \item Los grandes mitos de la innovación
      \item Estrategias de innovación : cómo introducir la innovación en una empresa.
      \item El proceso de innovación.
      \item Implementando y gestionando la innovación.
      \item Corporate spinouts.
      \item Emprendedores e intra emprendedores.
      \item Economía circular.
      \item Huella de carbon.
      \item Eco eficiencia.
      \item Desarrollo de producto.
   \end{topics}
   \begin{learningoutcomes}
      \item Los estudiantes habrán adquirido un conjunto de herramientas para ayudarles a lo largo del proceso de innovación , incluyendo también las estrategias de gestión de la innovación.
   \end{learningoutcomes}
\end{unit}

\begin{coursebibliography}
\bibfile{GeneralEducation/GH2011}
\end{coursebibliography}

\end{syllabus}
