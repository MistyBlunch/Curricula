\begin{syllabus}

\course{FG104. Introducción a la Filosofía}{Obligatorio}{FG104}

\begin{justification}
Para que haya verdadera vida universitaria ésta debe ser académica y por tanto filosófica. Toda ciencia particular necesita como fundamento un conocimiento que la sobrepasa, llevándonos a las puertas de la Filosofía. Toda disciplina se fundamenta en una concepción ética, antropológica y metafísica, de lo cual resulta fundamental que los alumnos se interioricen en los temas filosóficos básicos.
El curso corresponde al área de Filosofía y Teología y es parte de la formación general que ofrece la universidad. Es de carácter teórico y permite a los alumnos por lo menos dos cosas. En primer lugar, como todo saber filosófico, ayuda a cultivar el deseo desinteresado por saber, orientado a la búsqueda de la verdad en sí misma. En segundo lugar, permite adquirir los hábitos intelectuales y morales necesarios para introducirse en la reflexión filosófica. Abarca los siguientes contenidos: Aproximación a la noción de Filosofía”, El hombre”, El conocimiento humano”, El obrar humano” y El ser”.
\end{justification}

\begin{goals}
\item Conocer y saber los contenidos fundamentales de la Filosofía para iniciarse en la reflexión de la misma y expresarlos de manera clara, ordenada y precisa, valorándola como un conocimiento sapiencial que permite entender las cuestiones fundamentales que lo orienten al desarrollo personal y social.[\Familiarity]
\end{goals}

\begin{outcomes}
    \item \ShowOutcome{ñ}{2}
\end{outcomes}
\begin{competences}
    \item \ShowCompetence{C22}{ñ}
    \item \ShowCompetence{C24}{ñ}
\end{competences}

\begin{unit}{}{Aproximación a la noción de Filosofía}{Reale2007,Pieper1998,Yepez2000,Pieper1989,Verneaux1980,Casaubon2006,Fraile2005}{12}{C22,C24}
\begin{topics}
	\item Capítulo 1 
		\subitem Importancia de la filosofía.
		\subitem Filosofía: definición etimológica y real.
		\subitem El asombro como comienzo del filosofar.
		\subitem El ocio como condición para la filosofía.
		\subitem Aproximación histórica.
		\subitem La filosofía como sabiduría natural.
		\subitem Condiciones morales del filosofar.
	\item Capítulo 2
		\subitem Filosofía y conocimiento vulgar.
		\subitem Filosofía y ciencias particulares.
		\subitem Filosofía y teología.
\end{topics}
\begin{learningoutcomes}
	\item Comprender y valorar la naturaleza de la filosofía.[\Familiarity]
	\item Distinguir el conocimiento filosófico de otras formas de conocer la realidad.[\Familiarity]
\end{learningoutcomes}
\end{unit}

\begin{unit}{}{El hombre}{Zanotti1987,Garcia2011}{9}{C22,C24}
\begin{topics}
	\item Capítulo 1
		\subitem Características generales.
		\subitem Definiciones.
		\subitem Objetos.
		\subitem Métodos.
		\subitem Relación con otros saberes.
	\item Capítulo 2
		\subitem El cuerpo humano.
		\subitem La persona, alguien corporal. 
		\subitem Otras visiones. 
		\subitem Cómo es el cuerpo humano.
		\subitem La espiritualidad humana. 
		\subitem Potencias espirituales.
			\subsubitem Inteligencia.
			\subsubitem Voluntad.
	\item Capítulo 3
		\subitem La persona humana. 
		\subitem Definición. 
		\subitem Unidad sustancial del cuerpo y el espíritu.
\end{topics}

\begin{learningoutcomes}
	\item Comprender las nociones fundamentales de la antropología realista.[\Familiarity]
	\item Entender al hombre como un ser corpóreo-espiritual.[\Familiarity]
	\item Reconocer y valorar al hombre como un ser personal.[\Familiarity]
\end{learningoutcomes}
\end{unit}

\begin{unit}{}{El conocimiento humano}{Zanotti1987,Casaubon2006,Garcia2011}{9}{C22,C24}
\begin{topics}
	\item Capítulo 1
		\subitem Discusión con el escepticismo y el relativismo.
	\item Capítulo 2
		\subitem Noción de inteligencia.
		\subitem Relación de la inteligencia con los sentidos. 
		\subitem Intencionalidad. 
		\subitem Posición de Kant.
	\item Capítulo 3
		\subitem La verdad: lógica y ontológica.
		\subitem Intuición y razonamiento.
	\item Capítulo 4
		\subitem Las ciencias particulares. 
		\subitem Posición de D. Hume
		\subitem Relación fe-razón.
\end{topics}

\begin{learningoutcomes}
	\item Comprender las dificultades que plantea el conocimiento humano y las diversas soluciones que se dan a las mismas.[\Familiarity]
	\item Reconocer los conceptos fundamentales del realismo gnoseológico.[\Familiarity]
	\item Entender y valorar la diferencia y complementariedad entre el conocimiento científico, filosófico y teológico.[\Familiarity]
\end{learningoutcomes}
\end{unit}

\begin{unit}{}{El obrar humano}{Aristoteles1985,Tomas2001}{9}{C22,C24}
\begin{topics}
	\item Capítulo 1
		\subitem Características generales: etimología, objeto, tipo de conocimiento.
	\item Capítulo 2
		\subitem El bien. El fin último. La felicidad.
		\subitem Virtudes.
	\item Capítulo 3
		\subitem Criterios de moralidad.
		\subitem Fuentes de la moralidad.
		\subitem Relativismo ético.
\end{topics}

\begin{learningoutcomes}
	\item Comprender las nociones esenciales de la ética filosófica desde sus fundamentos.[\Familiarity	]
	\item Valorar y asumir la vida virtuosa como camino a la felicidad.[\Familiarity]
	\item Reconocer a la ética como una ciencia que se refiere a la vida de cada uno en concreto.[\Familiarity]
\end{learningoutcomes}
\end{unit}

\begin{unit}{}{El ser}{Gomez2006,Alvira1993}{12}{C22,C24}
\begin{topics}
	\item Capítulo 1
		\subitem La metafísica como estudio del ser.
	\item Capítulo 2
		\subitem Los trascendentales
		\subitem La estructura del ente finito. 
			\subsubitem Sustancia-Accidente.
			\subsubitem Materia-Forma.
			\subsubitem Acto-Potencia.
			\subsubitem Esencia-Acto de ser.
	\item Capítulo 3
		\subitem La causalidad.
		\subitem La existencia de Dios.
		\subitem La creación y sus implicancias.
\end{topics}

\begin{learningoutcomes}
	\item Conocer las características fundamentales de la metafísica y valorar su primacía en el pensamiento filosófico.[\Familiarity]
	\item Conocer las nociones metafísicas fundamentales.[\Familiarity]
	\item Reconocer la posibilidad de de acceder filosóficamente a Dios como creador.[\Familiarity]
\end{learningoutcomes}
\end{unit}

\begin{coursebibliography}
\bibfile{GeneralEducation/FG104}
\end{coursebibliography}

\end{syllabus}
