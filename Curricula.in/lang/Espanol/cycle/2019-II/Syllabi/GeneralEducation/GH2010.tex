\begin{syllabus}

\course{GH0010. Ética y Tecnología}{Obligatorio}{GH0010}
% Source file: ../Curricula.in/lang/Espanol/cycle/2019-II/Syllabi/GeneralEducation/GH2010.tex

\begin{justification}
Este curso busca proporcionar a los y las estudiantes ciertos marcos referenciales con los cuales analizar las disyuntivas que se pueden presentar en su ejercicio profesional. 
El curso pone en práctica constante el razonamiento crítico y responsable de los  y las estudiantes, siendo esta una competencia fundamental para los procesos de toma de decisión que asumiremos como profesionales y ciudadanos.
\end{justification}
\begin{goals}
\item Introducir a los estudiantes al pensamiento crítico y ético aplicado a su campo profesional.
\item Desarrollar la competencia de mirar un fenómeno desde varias disciplinas y perspectivas genera en la persona empatía y respeto a la diversidad de opinión.
\item Capacidad de trabajo en equipo.
\item Capacidad para identificar problemas.
\item Capacidad de comunicación oral.
\item Tiene interés por conocer sobre temas actuales de la sociedad peruana y del mundo.
\item Capacidad de comunicación escrita.
\end{goals}

\begin{outcomes}{V1}
    \item \ShowOutcome{d}{2}
    \item \ShowOutcome{e}{2}
    \item \ShowOutcome{f}{2}
    \item \ShowOutcome{n}{2}
    \item \ShowOutcome{o}{2}
\end{outcomes}

\begin{competences}{V1}
    \item \ShowCompetence{C10}{d,n,o}
    \item \ShowCompetence{C17}{f}
    \item \ShowCompetence{C18}{f}
    \item \ShowCompetence{C21}{e}
\end{competences}

\begin{unit}{Ética, ciencia y tecnología.}{}{Garcia06}{12}{C10}
   \begin{topics}
      \item Definición y alcance de la ética Pensamiento crítico /  argumentación ética.
      \item Ciencia y Tecnología , ?`Son las ingenierías y la tecnología cuestiones objetivas? 
      \item Tecnología: concepto y límites.
      \item Importancia de la ética en las ciencias e ingeniería .
   \end{topics}
   \begin{learningoutcomes}
      \item Fortalecer en el estudiante la capacidad de pensar interdisciplinariamente..
   \end{learningoutcomes}
\end{unit}

\begin{unit}{Responsabilidad en la ciencia e ingeniería}{}{Alvarado05}{24}{C17,C21}
   \begin{topics}
      \item Alcance del concepto  Responsabilidad en la ciencia (Imperative of Responsability)
      \item Introducción al tema Responsabilidad / libertad 
      \end{topics}

   \begin{learningoutcomes}
      \item  Comprender las responsabilidades profesionales y éticas.
   \end{learningoutcomes}
\end{unit}

\begin{unit}{Ciudadanía y ejercicio de la justicia en la era digital}{}{Alvarado05}{30}{C17,C21}
   \begin{topics}
      \item Introducción al tema de ciudadanía en la era digital
      \item Tecnología,  nuevos activismos y ciudadanía
   \end{topics}

   \begin{learningoutcomes}
      \item Comprende el impacto de las soluciones de la ingeniería en un contexto global, económico, ambiental y de la sociedad.
   \end{learningoutcomes}
\end{unit}

\begin{coursebibliography}
\bibfile{GeneralEducation/GH2010}
\end{coursebibliography}

\end{syllabus}
