\begin{syllabus}

\course{. }{}{} % Common.pm

\begin{justification}
La Sociologí­a permite un acercamiento vivencial a la realidad, lo que facilita su reconocimiento análisis y comprensión.
\end{justification}

\begin{goals}
\item Valorar el hecho social como un factor de comprensión e interpretación de la realidad social. [\Familiarity]
\item Interpelar la realidad desde una visión unitaria a partir de la perspectiva teológica y filosófica. [\Familiarity]
\item Superar el planteamiento teórico por formas de servicio desinteresado que expresen el sentido de la verdad. [\Familiarity]
\end{goals}

\begin{outcomes}
    \item \ShowOutcome{e}{1}
    \item \ShowOutcome{n}{1}
    \item \ShowOutcome{ñ}{2}
\end{outcomes}

\begin{competences}
    \item \ShowCompetence{C21}{e}
    \item \ShowCompetence{C22}{n} 
    \item \ShowCompetence{C24}{ñ}
\end{competences}

\begin{unit}{}{La Sociologí­a como Disciplina}{Macionis1999,pablo1998,Mendoza1990}{6}{C21,C22,C24}
\begin{topics}
    \item Su Naturaleza y su Objeto.
    \item Su relación con las otras Ciencias.
    \item Las Técnicas de Investigación Sociológica.
    \item Perspectivas sociológicas más recientes.
\end{topics}
\begin{learningoutcomes}
    \item Acercarse, describir y explicar los hechos sociales. [\Familiarity]
    \item Indagar la conexión existente entre Sociologí­a y Sociedad. [\Familiarity]
\end{learningoutcomes}
\end{unit}

\begin{unit}{}{Cultura y Sociedad}{Pablo2001,Doig1996,Giddens2002,Mendoza1990}{9}{C21,C22,C24}
\begin{topics}
    \item Conceptos.
    \item Valores y Normas.
    \item La diversidad cultural.
    \item La socialización.
    \item Los roles sociales.
    \item La identidad.
    \item Interacción y vida cotidiana.
    \item Tipos de sociedad.
    \item El cambio social.
    \item Grupos sociales.
    \begin{subtopics}
	    \item Primarios.
	    \item Secundarios.
    \end{subtopics}
\end{topics}

\begin{learningoutcomes}
    \item Estudiar la importancia e influencia de la cultura en la vida social humana. [\Familiarity]
    \item Analizar lo que son los grupos sociales.[\Familiarity]
    \item Entender el significado de grupo, categorTecnologíaas, reuniones o aglomeraciones y organizaciones sociales. [\Familiarity]
\end{learningoutcomes}
\end{unit}

\begin{unit}{}{La familia}{Guerra2004,Benedicto2009,Morande1999}{12}{C21,C22,C24}
\begin{topics}
    \item Familia.
    \item La familia como fenómeno generalizado.
    \item Funciones insustituibles.
    \item Familia y polí­ticas públicas.
    \item Relaciones constitutivas.
    \item Relaciones de reciprocidad.
\end{topics}
\begin{learningoutcomes}
	\item Pensar en la realidad de la familia desde las relaciones constitutivas y de reciprocidad. Analizar su funcionalidad insustituible. [\Familiarity]
\end{learningoutcomes}
\end{unit}

\begin{unit}{}{Bien Común: fin y función de la sociedad}{Leon2002,Pablo2010}{6}{C21,C22,C24}
\begin{topics}
    \item Los elementos que constituyen el Bien Común.
    \item Los ordenamientos de la Organización Social.
    \item Caracterí­sticas del Bien Común.
    \item Principios morales del Bien Común.
\end{topics}
\begin{learningoutcomes}
    \item Conocer las obligaciones que tenemos con el Bien Común. [\Familiarity]
\end{learningoutcomes}
\end{unit}

\begin{unit}{}{Desviación y control social}{Macionis1999,Giddens2002,Figari1996}{3}{C21,C22,C24}
\begin{topics}
    \item Desviación, delito  y control social.
    \item Teorí­as sobre el delito.
    \item El sistema de control social.
    \item El conflicto.
    \begin{subtopics}
	    \item Percepción.
	    \item Componentes.
    \end{subtopics}
    \item Introducción a la Resolución del Conflicto.
\end{topics}
\begin{learningoutcomes}
    \item Analizar las normas que guí­an el rango de las actividades humanas.[\Familiarity]
\end{learningoutcomes}
\end{unit}

\begin{unit}{}{Clase, Estratificación y Desigualdad}{Macionis1999,Giddens2002}{6}{C21,C22,C24}
\begin{topics}
    \item Teorí­as sobre la clase y la  Estratificación.
    \item La movilidad social.
\end{topics}
\begin{learningoutcomes}
	\item Distinguir la Desigualdad Social y sus principales teorí­as.[\Familiarity]
\end{learningoutcomes}
\end{unit}

\begin{unit}{}{Pobreza, Bienestar y Exclusión Social}{Macionis1999,Giddens2002}{12}{C21,C22,C24}
\begin{topics}
    \item Pobreza.
    \begin{subtopics}
	    \item Tipos.
    \end{subtopics}
    \item Bienestar.
    \begin{subtopics}
	    \item Asistencia Social.
	    \item Estado.
    \end{subtopics}
    \item Exclusión Social.
    \begin{subtopics}
	    \item Formas.
    \end{subtopics}
\end{topics}
\begin{learningoutcomes}
	\item Abordar la pobreza desde dos enfoques: absoluta y relativa. [\Familiarity]
	\item Analizar la pobreza y la movilidad social. [\Familiarity]
	\item Centrar el concepto de exclusión social analizando sus factores condicionantes.[\Familiarity]
\end{learningoutcomes}
\end{unit}



\begin{coursebibliography}
\bibfile{GeneralEducation/FG206}
\end{coursebibliography}

\end{syllabus}
