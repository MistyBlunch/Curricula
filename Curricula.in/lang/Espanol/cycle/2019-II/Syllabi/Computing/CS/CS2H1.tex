\begin{syllabus}

\course{CS2H01. Interacción Humano Computador}{Obligatorio}{CS2H01}
% Source file: ../Curricula.in/lang/Espanol/cycle/2019-I/Syllabi/Computing/CS/CS2H1.tex

\begin{justification}
El lenguaje ha sido una de las creaciones más significativas de la humanidad. Desde el lenguaje corporal y gestual, 
pasando por la comunicación verbal y escrita, hasta códigos simbólicos icónicos y otros, ha posibilitado interacciones complejas 
entre los seres humanos y facilitado considerablemente la comunicación de información. 
Con la invención de dispositivos automáticos y semiautomáticos, entre los que se cuentan las computadoras, 
la necesidad de lenguajes o interfaces para poder interactuar con ellos, ha cobrado gran importancia. 

La usabilidad del software, aunada a la satisfacción del usuario y su incremento de productividad, depende de la eficacia de la Interfaz Usuario-Computador.
Tanto es así­, que a menudo la interfaz es el factor más importante en el éxito o el fracaso de cualquier sistema computacional. 
El diseño e implementación de adecuadas Interfaces Humano-Computador, que además de cumplir los requisitos técnicos y la 
lógica transaccional de la aplicación, considere las sutiles implicaciones psicológicas, culturales y estéticas de los usuarios, 
consume buena parte del ciclo de vida de un proyecto software, y requiere habilidades especializadas, 
tanto para la construcción de las mismas, como para la realización de pruebas de usabilidad.
\end{justification}

\begin{goals}
\item Conocer y aplicar criterios de usabilidad y accesibilidad al diseño y construcción de interfaces humano-computador, buscando siempre que la tecnología se adapte a las personas y no las personas a la tecnología.
\item Que el alumno tenga una visión centrada en la experiencia de usuario al aplicar apropiados enfoques conceptuales y tecnológicos.
\item Entender como la tecnologica emergente hace posible nuevos estilos de interacción. 
\item Determinar los requerimientos básicos a nivel de interfaces, hardware y software para la construcción de ambientes inmersivos.
\end{goals}

\begin{outcomes}{V1}
    \item \ShowOutcome{b}{1}
    \item \ShowOutcome{c}{3}
    \item \ShowOutcome{d}{2}
    \item \ShowOutcome{o}{1}
\end{outcomes}

\begin{outcomes}{V2}
    \item \ShowOutcome{1}{2}
    \item \ShowOutcome{2}{2}
    \item \ShowOutcome{4}{2}
    \item \ShowOutcome{5}{2}
\end{outcomes}

\begin{competences}{V1}
    \item \ShowCompetence{CS8}{b} 
    \item \ShowCompetence{C7}{c}
    \item \ShowCompetence{C9}{o}
    \item \ShowCompetence{C15}{d}
    \item \ShowCompetence{CS10}{d}
\end{competences}

\begin{competences}{V2}
    \item \ShowCompetence{CS8}{1,2} 
    \item \ShowCompetence{C7}{2}
    \item \ShowCompetence{C9}{4}
    \item \ShowCompetence{C15}{5}
    \item \ShowCompetence{CS10}{5}
\end{competences}

\begin{unit}{\HCIFoundations}{}{Dix2004,Stone2005, Sharp2011}{8}{CS8}
\begin{topics}
    \item \HCIFoundationsTopicContexts
    \item \HCIFoundationsTopicUsability
    \item \HCIFoundationsTopicProcesses
    \item \HCIFoundationsTopicPrinciples
    \item \HCIFoundationsTopicDifferent
\end{topics}
\begin{learningoutcomes}
    \item \HCIFoundationsLODiscussWhy [\Familiarity]
    \item \HCIFoundationsLODefineA [\Familiarity]
    \item \HCIFoundationsLOSummarizeTheOf [\Familiarity]
      \item \HCIFoundationsLODevelop	[\Familiarity]
\end{learningoutcomes}
\end{unit}

\begin{unit}{Factores Humanos}{}{Dix2004,Stone2005, Sharp2011, Mathis2011, Donald2004}{8}{CS8}
\begin{topics}%
    \item \HCIFoundationsTopicCognitive
    \item \HCIFoundationsTopicPhysical
    \item \HCIFoundationsTopicAccessibility
    \item \HCIFoundationsTopicInterfaces
\end{topics}
\begin{learningoutcomes}
    \item \HCIFoundationsLOCreateAnd [\Familiarity]
\end{learningoutcomes}
\end{unit}

\begin{unit}{\HCIUsercentereddesignandtesting}{}{Dix2004,Stone2005, Sharp2011, Mathis2011, Buxton2007}{16}{C7, CS8, CS10}
\begin{topics}%
    \item \HCIUsercentereddesignandtestingTopicApproaches
    \item \HCIUsercentereddesignandtestingTopicFunctionality
    \item \HCIUsercentereddesignandtestingTopicTechniques
    \item \HCIUsercentereddesignandtestingTopicTechniquesAnd
    \item \HCIDesigningInteractionTopicTask
    \item \HCIDesignorientedHCITopicConsideration
    \begin{subtopics}
	\item Sketching
	\item Diseño participativo
    \end{subtopics}
    \item \HCIUsercentereddesignandtestingTopicPrototyping
    \item \HCIDesigningInteractionTopicLow
    \item \HCIDesigningInteractionTopicQuantitative
    \item \HCIUsercentereddesignandtestingTopicEvaluation
    \item \HCIUsercentereddesignandtestingTopicEvaluationWith
    \item \HCIUsercentereddesignandtestingTopicChallenges
    \item \HCIUsercentereddesignandtestingTopicReporting
    \item \HCIUsercentereddesignandtestingTopicInternationalization
\end{topics}
\begin{learningoutcomes}
	\item \HCIDesigningInteractionLOConduct[\Familiarity]
	\item \HCIDesigningInteractionLOForAn [\Familiarity]
	\item \HCIDesigningInteractionLODiscussAt [\Familiarity]
	\item \HCIUsercentereddesignandtestingLOExplainHowDesign [\Familiarity]
	\item \HCIUsercentereddesignandtestingLOUseLo [\Usage]
	\item \HCIUsercentereddesignandtestingLOChooseAppropriate [\Assessment]
	\item \HCIUsercentereddesignandtestingLOUseATechniques [\Assessment]
	\item \HCIUsercentereddesignandtestingLOCompareThe [\Assessment]
\end{learningoutcomes}
\end{unit}

\begin{unit}{\HCIDesigningInteraction}{}{Dix2004,Stone2005, Sharp2011, Johnson2010, Mathis2011, Leavitt2006}{8}{CS8, CS15}
\begin{topics}%
    \item \HCIDesigningInteractionTopicPrinciplesOf
    \item \HCIDesigningInteractionTopicElements
    \item \HCIDesigningInteractionTopicHandling
    \item \HCIDesigningInteractionTopicUser
    \item \HCIProgrammingInteractiveSystemsTopicPresenting
    \item \HCIProgrammingInteractiveSystemsTopicInterface
    \item \HCIProgrammingInteractiveSystemsTopicWidget
    \item \HCIUsercentereddesignandtestingTopicInternationalization
    \item \HCIProgrammingInteractiveSystemsTopicChoosing
\end{topics}
\begin{learningoutcomes}
    \item \HCIDesigningInteractionLOCreateATogether [\Usage]
\end{learningoutcomes}
\end{unit}

\begin{unit}{\HCINewInteractiveTechnologies}{}{Dix2004,Stone2005, Sharp2011, Wigdor2011, Mathis2011}{8}{C9}
\begin{topics}%
	\item \HCINewInteractiveTechnologiesTopicChoosing
	\item \HCINewInteractiveTechnologiesTopicApproachesTo
	\item \HCIMixedAugmentedandVirtualRealityTopicOutput
	\item \HCIMixedAugmentedandVirtualRealityTopicSystem
\end{topics}
\begin{learningoutcomes}
	\item \HCINewInteractiveTechnologiesLODescribeWhen [\Familiarity]
	\item \HCINewInteractiveTechnologiesLOUnderstandThe [\Familiarity]
	\item \HCINewInteractiveTechnologiesLODiscussTheDisadvantages [\Usage]
	\item \HCIMixedAugmentedandVirtualRealityLODescribeTheRealized [\Familiarity]
	\item \HCIMixedAugmentedandVirtualRealityLODescribeTheDifferentTechnologies [\Familiarity]
	\item \HCIMixedAugmentedandVirtualRealityLODetermineThe [\Assessment]
\end{learningoutcomes}
\end{unit}

\begin{unit}{\HCICollaborationandcommunication}{}{Dix2004,Stone2005, Sharp2011}{8}{CS8, CS9}
\begin{topics}%
	\item \HCICollaborationandcommunicationTopicAsynchronous
	\item \HCICollaborationandcommunicationTopicSocial
	\item \HCICollaborationandcommunicationTopicOnline
	\item \HCICollaborationandcommunicationTopicOnlineCommunities
	\item \HCICollaborationandcommunicationTopicSoftware
	\item \HCICollaborationandcommunicationTopicSocialPsychology
\end{topics} 
\begin{learningoutcomes}
	\item \HCICollaborationandcommunicationLODescribeTheSynchronous[\Familiarity]
	\item \HCICollaborationandcommunicationLOCompareTheIn[\Familiarity]
	\item \HCICollaborationandcommunicationLODiscussSeveral[\Usage]
	\item \HCICollaborationandcommunicationLODiscussTheIn [\Assessment]
\end{learningoutcomes}
\end{unit}

\begin{coursebibliography}
\bibfile{Computing/CS/CS2H1}
\end{coursebibliography}

\end{syllabus}

