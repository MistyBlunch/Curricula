\begin{syllabus}

\curso{CQ121. Química General I}{Obligatorio}{CQ121}

\begin{justification}
Este curso es útil en esta carrera para que el alumno aprenda a mostrar un alto grado de dominio de las leyes de la Química General.
\end{justification}

\begin{goals}
\item Capacitar y presentar al estudiante los principios básicos de la química como ciencia natural abarcando sus tópicos más importantes y su relación con los problemas cotidianos.
\end{goals}

\begin{outcomes}
\ExpandOutcome{a}
\ExpandOutcome{i}
\ExpandOutcome{j}
\end{outcomes}

\begin{unit}{QU1. Termodinámica}{Raymond99,Kennet92}{2}
\begin{topics}
      \item Sistemas termodinámicos y su clasificación. Variables termodinámicas y funciones de estado.
      \item Estados de un sistema. Estados de equilibrio. Variables extensivas e intensivas.
      \item Equilibrios térmicos. Principio cero de la termodinámica.
      \item Primer principio de la termodinámica. Capacidad calorífica. Procesos reversibles y trabajo máximo.
      \item Energía interna de los gases ideales. Transformaciones adiabáticas. Termoquímica. Ley de Lavoisier y La Place, Ley de Hess. Ley de Kirchhoff.
      \item Segunda Ley de la termodinámica. Entropía. Eficiencia de un ciclo reversible.
	\item Energía libre. Tercera ley de la termodinámica.
   \end{topics}

   \begin{unitgoals}
      \item Entender y trabajar con los principios de la Termodinámica.
      \item Abstraer de la naturaleza los conceptos de las transformaciones de los gases.
   \end{unitgoals}
\end{unit}

\begin{unit}{QU2. Equilibrio Químico}{Raymond99,Kennet92}{4}
\begin{topics}
      \item Concepto. Constante de equilibrio.
      \item Ley de acción de las masas.
      \item Equilibrios homogéneos. Equilibrios heterogéneos. Equilibrios múltiples.
      \item Factores que afectan el equilibrio químico. Principio de Le Chatelier.
    \end{topics}
   \begin{unitgoals}
      \item Describir, conocer y aplicar los conceptos del equilibrio químico.
      \item Resolver problemas.
   \end{unitgoals}
\end{unit}

\begin{unit}{QU3. Estudios que Contribuyeron al Desarrollo de la Teoría del Átomo}{Raymond99}{4}
\begin{topics}
      \item Propiedades de las ondas.
      \item Radiación electromagnética. Característica. Espectros.
      \item Teoría Cuántica de Max Planck.
      \item Efecto fotoeléctrico.
      \item Relación entre la materia y energía.
      \item Rayos X, Rayos catódicos y rayos canales.
      \item Ejercicios y problemas
\end{topics}

   \begin{unitgoals}
      \item Describir el comportamiento y características de las ondas.
      \item Entender qualitativa y quantitativamente el comportamiento corpuscular de las ondas electromagnéticas.
      \item Resolver problemas.
   \end{unitgoals}
\end{unit}

\begin{unit}{QU4. Teorías del Átomo}{Babor83,Kennet92}{6}
\begin{topics}
      \item Postulados de Dalton. Modelo atómico de Thompson.
      \item Experimento de Rutherford, Modelo atómico de Rutherford. Inconsistencia.
      \item Modelo atómico de Bohr. Espectro de emisión del átomo de hidrógeno.
      \item Teoría atómica moderna. Dualidad de la materia.
      \item Principio de incertidumbre de Heisenberg.
      \item Orbitales atómicos. Ecuación de Schrodinguer.
      \item Descripción mecánico cuántica del átomo de hidrogeno Números cuánticos.
      \item Configuración electrónica. Principio de exclusión de Pauli.
      \item Regla de Hund. Excepciones.
      \item Paramagnetismo y diamagnetismo. Efecto pantalla.
      \item Ejercicios y problemas.
   \end{topics}

   \begin{unitgoals}
      \item Conocer e interpretar los modelos atómicos clásicos.
      \item Entender los fundamentos de la teoría atómica moderna.
      \item Conocer los conceptos básicos de la mecánica cuántica.
      \item Resolver problemas.
   \end{unitgoals}
\end{unit}

\begin{unit}{QU5. Tabla Periódica}{Kennet92,Mahan92}{4}
\begin{topics}
	\item Ley periódica.
	\item Descripción de la tabla periódica. Periodo y grupo. Ubicación de un elemento.
	\item Propiedades periódicas: Radio atómico, radio iónico, energía de ionización, afinidad electrónica. Electronegatividad.
	\item Variación de las propiedades químicas.
	\item Ejercicios y problemas.
   \end{topics}

   \begin{unitgoals}
      \item Entender la estructura de la tabla periódica.
      \item Conocer las propiedades de los elementos.
      \item Resolver problemas.
   \end{unitgoals}
\end{unit}

\begin{unit}{QU6. Enlace Químico}{Mahan92,Ander83}{3}
   \begin{topics}
	\item Teoría de  la valencia. Evolución.
	\item Regla del octeto.
	\item Teoría de Lewis.
	\item Enlace iónico y electrovalente.
	\item Formación del par iónico entre los elementos $s$ y los elementos $p$. Las energías iónicas de las redes cristalinas.
	\item Ciclo de Born Haber.
	\item Enlace covalente. Compartición de pares de electrones.
	\item Carga formal y estructura de Lewis. Concepto de resonancia.
	\item Excepciones a la regla del octeto. Fuerzas en enlace covalente.
	\item Teoría de la repulsión de pares electrónicos del nivel de valencia (RPENV).
	\item Concepto de hibridación. Hibridación sp, sp2, sp3 y otros tipos de hibridación.
	\item Teoría del orbital molecular.
	\item Ejercicios y problemas.
   \end{topics}

   \begin{unitgoals}
      \item Conocer y entender las teorías de valencia y de enlaces químicos.
      \item Conocer y entender la teoría del orbital molecular.
      \item Resolver problemas.
   \end{unitgoals}
\end{unit}

\begin{unit}{QU7. Gases}{Ander83,Masterton98}{4}
\begin{topics}
      \item Definición. Presión de un gas.
      \item Leyes de los gases: de Boyle, Gay-Lussac y Charles. Ecuación de un gas ideal.
      \item Ley de presiones parciales de Dalton.
      \item Teoría cinética de los gases. Distribución de velocidades moleculares. Trayectoria libre media.
      \item Ley de Graham de la difusión y efusión.
      \item Gases reales. Ecuación de Van der Waals.
      \item Ejercicios y problemas.
   \end{topics}

   \begin{unitgoals}
      \item Conocer los conceptos básicos de los gases ideales.
      \item Entender y aplicar la teoría cinética de los gases.
      \item Conocer conceptos de difusión y efusión de gases.
      \item Entender los conceptos de gases reales.
      \item Resolver problemas.
   \end{unitgoals}
\end{unit}

\begin{unit}{QU8. Fuerzas Intermoleculares y Líquidos}{Masterton98,Babor83}{3}
\begin{topics}
      \item Definición. La evaporación y la presión de vapor en el estado de equilibrio.
      \item Medida de la presión de vapor y del calor de vaporización. Punto de ebullición y calor latente de vaporización.
      \item Fuerzas intermoleculares; fuerzas dipolo-dipolo, ion-dipolo, disperso, fuerza y radio de van der Waals. Enlace de hidrógeno.
      \item Viscocidad. Tensión superficial y acción capilar.
      \item Cambios de fase.
      \item Ejercicios y problemas.
    \end{topics}

   \begin{unitgoals}
      \item Conocer conceptos básicos de las fuerzas intermoleculares.
      \item Conocer y aplicar conceptos de vaporización y ebullición.
      \item Conocer y aplicar conceptos de tensión superficial y cambios de fase.
      \item Resolver problemas.
   \end{unitgoals}
\end{unit}

\begin{unit}{QU9. Sólidos}{Masterton98,Babor83}{3}
\begin{topics}
      \item Definición. Empaquetación de esferas. Eficiencia de empaquetamiento. Empaquetamiento compacto.
      \item Empleo de los Rayos X en el estudio de la estructura de los cristales.
      \item Clases de estructuras cristalinas: cristales iónicos. Covalentes, moleculares, metálicos. Enlace metálico Cristales amorfos.
      \item Cambios de fase. Equilibrio líquido-vapor. Calor de vaporización y punto de ebullición.
      \item Equilibrio líquido-sólido. Equilibrio sólido-vapor. Diagrama de fase  del agua y del dióxido de carbono.
      \item Ejercicios y problemas.
    \end{topics}

   \begin{unitgoals}
      \item Conocer conceptos básicos de las estructuras cristalinas de sólidos.
      \item Conocer y aplicar conceptos de cambios de fase y de equilibrio.
      \item Resolver problemas.
   \end{unitgoals}
\end{unit}

\begin{unit}{QU10. Disoluciones}{Masterton98,Babor83}{3}
\begin{topics}
      \item Definición. Visión molecular del proceso de disolución.
      \item Disoluciones de líquidos en líquidos. Disoluciones de sólidos en líquidos.
      \item Unidades de concentración: porcentaje en masa, fracción molar, molaridad, molalidad Normalidad.
      \item Efecto de la temperatura en la solubilidad, la solubilidad de los sólidos y la temperatura, cristalización fraccionada.
      \item La solubilidad de los gases y la temperatura. Efecto  de la presión en la solubilidad de los gases.
      \item Propiedades coligativas de las soluciones. Dispersiones coloidales.
      \item Ejercicios y problemas.
    \end{topics}

   \begin{unitgoals}
      \item Conocer conceptos básicos de las disoluciones moleculares.
      \item Conocer y aplicar conceptos de concentración y solubilidad.
      \item Resolver problemas.
   \end{unitgoals}
\end{unit}

\begin{unit}{QU11. Estequiometría}{Masterton98,Babor83}{3}
\begin{topics}
      \item Reacción química. Expresiones de las reacciones químicas en forma de ecuaciones. Características de una ecuación química.
      \item Tipos de reacciones químicas: Precipitación, ácido-base, óxido-reducción. Cantidad de reactivos y productos.
      \item Relaciones estequiométricas: moles, masa y volumen.
      \item Leyes ponderales y volumétricas.
      \item Reactivo limitante. Rendimiento de las reacciones.
      \item Ejercicios y problemas.
    \end{topics}

   \begin{unitgoals}
      \item Conocer conceptos básicos de las reacciones químicas.
      \item Conocer y aplicar las leyes ponderales y volumétricas.
      \item Resolver problemas.
   \end{unitgoals}
\end{unit}

\begin{coursebibliography}
\bibfile{BasicSciences/CQ121}
\end{coursebibliography}
\end{syllabus}
