\begin{syllabus}

\course{CB112. Física II}{Obligatorio}{CB112}

\begin{justification}
Mostrar un alto grado de dominio de las leyes del movimiento ondulatorio,
la naturaleza de la luz y las leyes de la óptica geométrica.
Utilizando  adecuadamente los conceptos de movimiento, energía y óptica
geométrica en la resolución de problemas de la vida cotidiana y poseer
capacidad y habilidad en la interpretación de los fenomenos ondulatorios,
que contribuyen en la elaboración de soluciones eficientes y útiles en
diferentes áreas de las ciencias de la computación.
\end{justification}

\begin{goals}
\item Que el alumno aprenda y domine las leyes del movimiento ondulatorio
\item Que el alumno aprenda a aplicar principios de la fisica para edsarrollar modelos computacionales
\end{goals}

\begin{outcomes}
\ExpandOutcome{a}{3}
\ExpandOutcome{i}{3}
\ExpandOutcome{j}{3}
\end{outcomes}

\begin{unit}{FI1 Movimiento oscilatorio}{Serway2002,Alonso94}{0}{3}
\begin{topics}
      \item Introducción.
      \item Movimiento armónico simple.
      \item Sistema masa-resorte.
      \item Energía del oscilador simple.
      \item El péndulo.
      \item Comparación del movimiento armónico simple con el movimiento circular.
      \item Ejercicios y problemas.
   \end{topics}

   \begin{unitgoals}
      \item Explicar, analizar y caracterizar el movimiento oscilatorio a partir del MAS.
      \item Resolver problemas.
   \end{unitgoals}
\end{unit}

\begin{unit}{FI2 Movimiento ondulatorio}{Serway2002,Alonso94}{0}{3}
\begin{topics}
      \item Introducción.
        \item Variables básicas del movimiento ondulatorio.
        \item Dirección del desplazamiento de las partículas.
        \item Ondas viajeras unidimensionales.
        \item Superposición e interferencia.
        \item La rapidez de ondas en las cuerdas.
        \item Reflexión y transmisión.
        \item Ondas senoidales.
        \item Ejercicios y problemas.
   \end{topics}
   \begin{unitgoals}
      \item Explicar, encontrar y caracteriza mediante problemas de la vida cotidiana el movimiento ondulatorio, así como, la reflexión y transmisión de ondas en el espacio.
      \item Resolver problemas.
   \end{unitgoals}
\end{unit}

\begin{unit}{FI3 Ondas sonoras}{Serway2002,Alonso94}{0}{3}
\begin{topics}
      \item  Introducción.
        \item Rapidez de las ondas sonoras.
        \item Ondas sonoras periódicas.
        \item Intensidad de las ondas sonoras periódicas.
        \item Ondas planas y esféricas.
        \item El efecto Doppler
        \item Ejercicios y problemas.
   \end{topics}
   \begin{unitgoals}
      \item Explicar, encontrar y caracterizar mediante problemas de la vida cotidiana los fenomenos acústicos.
      \item Resolver problemas.
   \end{unitgoals}
\end{unit}

\begin{unit}{FI4 Naturaleza de la luz y las leyes de la óptica geométrica}{Serway2002,Alonso94}{0}{3}
\begin{topics}
      \item La naturaleza de la luz.
        \item Mediciones de la velocidad de la luz.
        \item La aproximación de rayos en óptica geométrica.
        \item Reflexión y Refracción.
        \item Dispersión y prismas.
        \item El principio de Huygens.
        \item Reflexión interna total.
        \item Ejercicios y problemas.
   \end{topics}

   \begin{unitgoals}
      \item Explicar, encontrar y caracterizar mediante problemas de la vida cotidiana el fenómeno de la Luz.
      \item Resolver problemas.
      \end{unitgoals}
\end{unit}

\begin{unit}{FI5 Óptica geométrica}{Serway2002,Alonso94}{0}{3}
\begin{topics}
      \item Imagenes formadas por espejos planos.
        \item Imagenes formadas por espejos esféricos.
        \item Imagenes formadas por refracción.
        \item Lentes delgados.
        \item Aberraciones de lentes.
        \item Ejercicios y problemas.
      \end{topics}

   \begin{unitgoals}
      \item Explicar, encontrar y caracterizar mediante problemas de la vida cotidiana las imagenes formadas por espejos planos y esféricos.
      \item Resolver problemas.
   \end{unitgoals}
\end{unit}

\begin{unit}{FI6 Interferencia de ondas luminosas}{Serway2002,Alonso94}{0}{3}
\begin{topics}
      \item Condiciones para la interferencia.
        \item Experimento de la doble rendija.
        \item Distribución de intensidad del patrón
              de interferencia de doble rendija
        \item Adición fasorial de ondas.
        \item Cambio de fase debido a la reflexión.
        \item Interferencia en películas delgadas.
        \item Ejercicios y problemas
   \end{topics}
   \begin{unitgoals}
      \item Aplicar, encontrar y teorizar, el concepto de interferencia y
            principios relativos a la interferncia de ondas luminosas.
      \item Resolver problemas.
   \end{unitgoals}
\end{unit}

\begin{unit}{FI7 Difracción y Polarización}{Serway2002,Alonso94}{0}{3}
\begin{topics}
      \item Introducción a la difracción.
      \item Difracción de una sola rendija.
      \item Resolución de abertura circular y de una sola rendija.
      \item La rejilla de difracción.
      \item Difracción de rayos X mediante cristales.
      \item Polarización de ondas luminosas.
      \item Ejercicios y problemas.
   \end{topics}

   \begin{unitgoals}
      \item Analizar, encontrar e interpretar la difracción y polarización de la luz.
      \item Resolver problemas.
   \end{unitgoals}
\end{unit}



\begin{coursebibliography}
\bibfile{BasicSciences/CB111}
\end{coursebibliography}

\end{syllabus}
