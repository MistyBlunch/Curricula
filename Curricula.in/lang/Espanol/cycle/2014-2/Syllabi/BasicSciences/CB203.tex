% Responsable : Luis Díaz Basurco
% Sumilla de  : Estadística y Probabilidades
% Versión     : 1

\begin{syllabus}

\course{CB203. Estadística y Probabilidades}{Obligatorio}{CB203}

\begin{justification}
Provee de una introducción a la teoría de las probabilidades e inferencia estadística con aplicaciones, necesarias en el análisis de datos, diseño de modelos aleatorios y toma de decisiones.
\end{justification}

\begin{goals}
\item Que el alumno aprenda a utilizar las herramientas de la estadística para tomar decisiones ante situaciones de incertidumbre.
\item Que el alumno aprenda a obtener conclusiones a partir de datos experimentales.
\item Que el alumno pueda extraer conslusiones útiles sobre la totalidad de una población basándose en información. recolectada
\end{goals}

\begin{outcomes}
\ExpandOutcome{a}{3}
\ExpandOutcome{i}{3}
\ExpandOutcome{j}{4}
\end{outcomes}

\begin{unit}{Estadística descriptiva}{Mendenhall97}{10}{3}
\begin{topics}
      \item Presentación de datos
      \item Medidas de localización central
      \item Medidas de dispersión
   \end{topics}

   \begin{unitgoals}
      \item Presentar resumir y describir datos.
   \end{unitgoals}
\end{unit}

\begin{unit}{Probabilidades}{Meyer70}{10}{3}
\begin{topics}
      \item Espacios muestrales y eventos
      \item Axiomas y propiedades de probabilidad
      \item Probabilidad condicional
      \item Independencia,
      \item Teorema de Bayes
   \end{topics}
   \begin{unitgoals}
      \item Identificar espacios aleatorios
      \item diseñar  modelos probabilísticos
      \item Identificar eventos como resultado de un experimento aleatorio
      \item Calcular la probabilidad de ocurrencia de un evento
      \item Hallar la probabilidad usando condicionalidad, independencia y Bayes
   \end{unitgoals}
\end{unit}

\begin{unit}{Variable aleatoria}{Meyer70,Devore98}{10}{4}
\begin{topics}
      \item Definición y tipos de variables aleatorias
      \item Distribución de probabilidades
      \item Funciones densidad
      \item Valor esperado
      \item Momentos
   \end{topics}

   \begin{unitgoals}
      \item Identificar variables aleatorias que describan un espacio muestra
      \item Construir la distribución o función de densidad.
      \item Caracterizar distribuciones o funciones densidad conjunta.
   \end{unitgoals}
\end{unit}

\begin{unit}{Distribución de probabilidad discreta y continua}{Meyer70,Devore98}{10}{3}
\begin{topics}
      \item Distribuciones de probabilidad básicas
      \item Densidades de probabilidad básicas
      \item Funciones de variable aleatoria
   \end{topics}

   \begin{unitgoals}
      \item Calcular probabilidad de una variable aleatoria con distribución o función densidad
      \item Identificar la distribución o función densidad que describe un problema aleatorio
      \item Probar propiedades de distribuciones o funciones de densidad
   \end{unitgoals}
\end{unit}

\begin{unit}{Distribución de probabilidad conjunta}{Meyer70,Devore98}{10}{3}
\begin{topics}
      \item Variables aleatorias distribuidas conjuntamente
      \item Valores esperados, covarianza y correlación
      \item Las estadísticas y sus distribuciones
      \item Distribución de medias de muestras
      \item Distribución de una combinación lineal

   \end{topics}
   \begin{unitgoals}
      \item Encontrar la distribución conjunta de dos variables aleatorias discretas o continuas
      \item Hallar las distribuciones marginales o condicionales de variables aleatorias conjuntas
      \item Determinar dependencia o independencia de variables aleatorias
      \item Probar propiedades que son consecuencia del teorema  del límite central
   \end{unitgoals}
\end{unit}

\begin{unit}{Inferencia estadística}{Meyer70,Devore98}{10}{3}
\begin{topics}
      \item Estimación estadística
      \item Prueba de hipótesis
      \item Prueba de hipótesis usando ANOVA
   \end{topics}

   \begin{unitgoals}
      \item Probar si un estimador es insesgado, consistente o suficiente
      \item Hallar intervalo intervalos de confianza para estimar parámetros
      \item Tomar decisiones de parámetros en base a pruebas de hipótesis
      \item Probar hipótesis usando ANOVA
   \end{unitgoals}
\end{unit}



\begin{coursebibliography}
\bibfile{BasicSciences/CB203}
\end{coursebibliography}

\end{syllabus}
