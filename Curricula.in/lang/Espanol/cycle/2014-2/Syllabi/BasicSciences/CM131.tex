\begin{syllabus}

\curso{CM131. Cálculo Diferencial}{Obligatorio}{CM131}

\begin{justification}
Conceptos básicos de lógica matemática, métodos de demostración. Definición axiomática de los números reales, El axioma del supremo, densidad de los racionales. Funciones reales: funciones biyectivas, funciones monótonas, composición e inversa de una función. Sucesiones reales: limite de sucesiones, sucesiones monótonas.  Limites de funciones, limites trigonométricos, limites y el infinito, asíntotas  oblicuas. Continuidad, el teorema del valor intermedio, el teorema de los valores extremos, Derivadas: regla de la cadena, derivación implícita, la diferencial., el teorema del valor medio, criterios de la primera y segunda derivada, concavidad y puntos de inflexión, el teorema del valor medio generalizado, las reglas de L'hospital, el método de Newton para hallar raíces y el  teorema de Taylor  con resto diferencial.
\end{justification}

\begin{goals}
\item Al finalizar el curso el alumno comprenderá los fundamentos del Cálculo Diferencial, y habrá adquirido habilidades que le permitan usar los conceptos estudiados, en el desarrollo de otras asignaturas, así como también en la solución de problemas vinculados a su especialidad 
\end{goals}

\begin{outcomes}
\ExpandOutcome{a}
\ExpandOutcome{i}
\ExpandOutcome{j}
\end{outcomes}

\begin{unit}{Introducción a la Lógica Proposicional}{Hasser97,Spivak96}{4}
\begin{topics}
      \item Disyunción y conjunción de proposiciones.
      \item Negación.
      \item Implicación, equivalencia.
      \item Reglas de Inferencia y demostraciones.
   \end{topics}

   \begin{unitgoals}
      \item Conocer y aplicar conceptos de Lógica Proposicional.
	\item Resolver problemas.
   \end{unitgoals}
\end{unit}

\begin{unit}{Propiedades Básicas de los Números}{Helfgott89}{8}
\begin{topics}
	\item Axiomas de Cuerpo
	\item Axioma de orden
	\item Conjuntos Acotados
	\item Axioma del supremo  (completitud)
	\item Propiedades del supremo y del ínfimo
	\item Representación decimal de los números reales
	\item Fracciones continuas
\end{topics}
\begin{unitgoals}
	\item Describir matemáticamente las propiedades básicas de los Números
	\item Resolver problemas
\end{unitgoals}
\end{unit}

\begin{unit}{Funciones}{Bartle90}{12}
\begin{topics}
      \item Definición. Dominio. Rango. Imagen. Pre imagen
      \item Operaciones de funciones: Suma, resta, multiplicación, división y composición
      \item Funciones monótonas. Funciones inyectivas,  suryectivas y biyectivas
      \item Función inversa. Gráfica de funciones
      \item Modelación con funciones
\end{topics}

   \begin{unitgoals}
      \item Describir matemáticamente las funciones
      \item Conocer y aplicar las operaciones de funciones
	\item Resolver problemas
   \end{unitgoals}
\end{unit}

\begin{unit}{Límites y Continuidad}{Leithold82}{12}
\begin{topics}
      \item El \'limite de una función. Límites laterales. Límites  infinito. Teoremas sobre límites y aplicaciones
      \item Asíntotas horizontales, verticales y oblicuas a las gráficas de una función
      \item Continuidad. El Teorema del Valor Intermedio. Teorema del Cero. Tipos de discontinuidad
      \item Funciones acotadas Teorema fundamental de las funciones continuas
	\end{topics}

   \begin{unitgoals}
      \item Describir matemáticamente los límites y continuidad de funciones
	\item Resolver problemas
   \end{unitgoals}
\end{unit}

\begin{unit}{La Derivada}{Edwards96}{12}
\begin{topics}
	\item La derivada de una función en un punto. Interpretaciones geométrica y física de la derivada
	\item Regla de derivación. La regla de la cadena
	\item Derivación implícita. Derivada de la función inversa
	\item Funciones derivables en un intervalo
	\item Representación paramétrica de una curva. Curvas diferenciables
\end{topics}

\begin{unitgoals}
	\item Describir matemáticamente la derivada
	\item Conocer y aplicar conceptos de derivada en la solución de problemas
\end{unitgoals}
\end{unit}

\begin{unit}{Aplicaciones de La Derivada}{Stewart99}{8}
\begin{topics}
	\item Extremos locales y globales de una función. Funciones crecientes y decrecientes
	\item Puntos críticos. Los Teoremas de Rolle y del Valor Medio
	\item La regla de L'Hospital para el cálculo de límites
	\item Criterio de la primera derivada. Concavidad. Puntos de inflexión
	\item Criterio de la segunda derivada. Asíntotas oblicuas. Trazo del gráfico de una función
	\item Problemas de Optimización. La derivada como razón de cambio instantáneo
	\item Velocidades relacionadas. La derivada en otras ciencias
	\item Diferenciales. Aproximación, usando diferenciales
\end{topics}

\begin{unitgoals}
	\item Conocer y aplicar conceptos de derivada
	\item Resolver problemas
\end{unitgoals}
\end{unit}

\begin{coursebibliography}
\bibfile{BasicSciences/CM131}
\end{coursebibliography}
\end{syllabus}


