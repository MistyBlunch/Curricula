\begin{syllabus}

\curso{CM254. Estructuras Discretas}{Obligatorio}{CM254}

\begin{justification}
Las estructuras discretas son fundamentales para la ciencia de la computación. Es evidente que las estructuras discretas son usadas en las áreas de estructura de datos y algoritmos, sin embargo son también importantes en otras, como por ejemplo en la verificación, en criptografía y métodos formales.

Para entender las técnicas computacionales avanzadas, los estudiantes deberán tener un fuerte conocimiento de las diversas estructuras discretas, estructuras que serán implementadas y usadas en laboratorio en el lenguaje de programación.

El álgebra abstracta tiene un lado práctico que explotaremos para comprender en profundidad temas de computación como criptografía y álgebra relacional.
\end{justification}

\begin{goals}
\item Desarrollar Operaciones asociadas con conjuntos, funciones y relaciones.
\item Relacionar ejemplos prácticos al modelo apropiado de conjunto, función o relación.
\item Conocer las diferentes técnicas de conteo más utilizadas.
\item Describir como las herramientas formales de lógica simbólica son utilizadas.
\item Describir la importancia y limitaciones de la lógica de predicados.
\item Bosquejar la estructura básica y dar ejemplos de cada tipo de prueba descrita en esta unidad.
\item Relacionar las ideas de inducción matemática con la recursividad y con estructuras definidas recursivamente.
\item Enunciar, identificar y habituarse a los conceptos más importantes de Conjuntos Parcialmente Ordenados y Látices
\item Analizar, comentar y aceptar las nociones básicas de álgebras Booleanas.
\item Que el alumno sea capaz de modelar problemas de ciencia de la computación usando grafos y árboles relacionados con estructuras de datos
\item Que el alumno aplicar eficientemente estrategias de recorrido para poder buscar datos de una manera óptima
\item Conocer las técnicas y métodos de encriptación de datos.
\end{goals}

\begin{outcomes}
\ExpandOutcome{a}
\ExpandOutcome{b}
\ExpandOutcome{j}
\end{outcomes}

\begin{unit}{\DSONEDef}{Kolman97,Grassmann97,Johnsonbaugh99}{13}
\begin{topics}
	\item  \DSONETopicFunciones
	\item  Conjuntos producto, especificación de relaciones.
	\item  \DSONETopicRelaciones
	\item  Clases de equivalencia, operaciones entre relaciones.
	\item  \DSONETopicConjuntos
	\item  \DSONETopicPrincipio
	\item  \DSONETopicCardinalidad
\end{topics}

\begin{unitgoals}
	\item \DSONEObjONE
	\item \DSONEObjTWO
	\item \DSONEObjTHREE
	\item \DSONEObjFOUR
\end{unitgoals}
\end{unit}

\begin{unit}{\DSTWODef}{Grassmann97,Iranzo05,Paniagua03,Johnsonbaugh99}{14}
\begin{topics}
         \item \DSTWOTopicLogica
         \item \DSTWOTopicConectivos
         \item \DSTWOTopicTablas
         \item \DSTWOTopicFormas
         \item \DSTWOTopicValidacion
         \item \DSTWOTopicLogicade
     \item \DSTWOTopicCuantificacion
         \item \DSTWOTopicModus
         \item \DSTWOTopicLimitaciones
   \end{topics}

   \begin{unitgoals}
      \item \DSTWOObjONE
         \item \DSTWOObjTWO
         \item \DSTWOObjTHREE
         \item \DSTWOObjFOUR
   \end{unitgoals}
\end{unit}

\begin{unit}{\DSTHREEDef}{Scheinerman01,Brassard97,Kolman97,Johnsonbaugh99}{14}
\begin{topics}
      \item \DSTHREETopicNociones
      \item \DSTHREETopicEstructura
      \item \DSTHREETopicPruebas
      \item \DSTHREETopicPruebasy
      \item \DSTHREETopicPruebaspor
      \item \DSTHREETopicPruebasporcontradiccion
      \item \DSTHREETopicInduccion
      \item \DSTHREETopicInduccionfuerte
      \item \DSTHREETopicDefiniciones
      \item \DSTHREETopicEl
   \end{topics}

   \begin{unitgoals}
      \item \DSTHREEObjONE
      \item \DSTHREEObjTWO
      \item \DSTHREEObjTHREE
      \item \DSTHREEObjFOUR
   \end{unitgoals}
\end{unit}

\begin{unit}{\ARONEDef}{Kolman97, Grimaldi97, Gersting87}{19}
\begin{topics}
      \item Conjuntos Parcialmente Ordenados.
      \item Elementos extremos de un conjunto parcialmente ordenado.
      \item Látices.
      \item Álgebras Booleanas.
      \item Funciones Booleanas.
      \item \ARONETopicExpresiones
      \item \ARONETopicBloques
   \end{topics}
   \begin{unitgoals}
      \item \DSTHREEObjONE
      \item \DSTHREEObjTWO
      \item \DSTHREEObjTHREE
   \end{unitgoals}
\end{unit}

\begin{unit}{\DSFOURDef}{Grimaldi97}{25}
   \begin{topics}
	 \item 	\DSFOURTopicArgumentos
	 \item 	\DSFOURTopicPermutaciones
	 \item 	\DSFOURTopicPrincipio
	 \item 	\DSFOURTopicSolucion 
   \end{topics}

   \begin{unitgoals}
	\item \DSFOURObjONE
	\item \DSFOURObjTWO
	\item \DSFOURObjTHREE 
	\item \DSONEObjFOUR 
   \end{unitgoals}
\end{unit}

\begin{unit}{\DSCINCODef}{Johnsonbaugh99}{25}
   \begin{topics}
	 \item \DSCINCOTopicArboles
	 \item \DSCINCOTopicGrafos 
	 \item \DSCINCOTopicGrafosdirigidos
	 \item \DSCINCOTopicArbolesde 
	 \item \DSCINCOTopicEstrategias 
   \end{topics}

   \begin{unitgoals}
	 \item \DSCINCOObjONE
	 \item \DSCINCOObjTWO
	 \item \DSCINCOObjTHREE
	 \item \DSCINCOObjFOUR
   \end{unitgoals}
\end{unit}

\begin{unit}{\DSSEISDef}{Micha98,Rosen2004}{10}
   \begin{topics}
      \item \DSSEISTopicEspacios
      \item \DSSEISTopicProbabilidad 
      \item \DSSEISTopicVariables 
   \end{topics}

   \begin{unitgoals}
      \item \DSSEISObjONE
      \item \DSSEISObjTWO
      \item \DSSEISObjTHREE
      \item \DSSEISObjFOUR
   \end{unitgoals}
\end{unit}

\begin{unit}{\ALNUEVEDef}{Grimaldi97, Scheinerman01}{20}
   \begin{topics}
         \item  \ALNUEVEDef
         \item  \ALNUEVETopicRevision
         \item  \ALNUEVETopicCriptografia
        \item   \ALNUEVETopicCriptografiade
        \item   \ALNUEVETopicFirmas
        \item   \ALNUEVETopicProtocolos
        \item   \ALNUEVETopicAplicaciones
   \end{topics}

   \begin{unitgoals}
         \item \ALNUEVEObjONE
         \item \ALNUEVEObjTWO
          \item \ALNUEVEObjTHREE
   \end{unitgoals}
\end{unit}

\begin{unit}{\IMFOURDef}{Grassmann97}{20}
   \begin{topics}
         \item \IMFOURDef
         \item \IMFOURTopicMapeo
         \item \IMFOURTopicEntidad
         \item \IMFOURTopicAlgebra
   \end{topics}

   \begin{unitgoals}
         \item \IMFOURObjONE
         \item \IMFOURObjTWO
         \item \IMFOURObjTHREE
         \item \IMFOURObjFOUR
         \item \IMFOURObjCINCO
   \end{unitgoals}
\end{unit}

\begin{unit}{Teoría de Números}{Grimaldi97, Scheinerman01}{20}
   \begin{topics}
      \item Teoría de los números
     \item Aritmética  Modular
      \item Teorema del Residuo Chino
       \item Factorización
      \item Grupos, teoría de la codificación y método de enumeración de Polya
      \item Cuerpos finitos y diseños combinatorios
   \end{topics}

   \begin{unitgoals}
      \item Establecer la importancia de la teoría de números en la criptografía
      \item Utilizar las propiedades de las estructuras algebraicas en el estudio de la teoría algebraica de códigos
   \end{unitgoals}
\end{unit}

\begin{coursebibliography}
\bibfile{BasicSciences/CM254}
\end{coursebibliography}

\end{syllabus}

%\end{document}
