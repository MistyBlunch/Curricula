\begin{syllabus}

\course{HU302. Visión Cristiana de nuestro tiempo}{Obligatorio}{FG302}

\begin{justification}
Parte fundamental de la formación universitaria consiste en tener una cada vez mayor actitud situada, es decir, una conciencia clara y profunda del ``aquí y ahora'' del entorno cultural.  Una constructiva actitud crítica se presenta como imprescindible en este sentido, a la par de un sentido de mayor responsabilidad.  La perspectiva que brota de la antropología cristiana hará posible un juicio más profundo acerca de lo que acontece.  Una mirada mas amplia de la realidad es de ayuda para una mayor eficacia profesional.
\end{justification}

\begin{goals}
\item Que el estudiante conozca y valore con mayor profundidad algunas tendencias culturales relevantes del mundo actual, que influyen directamente sobre su entorno personal, familiar y laboral, y sea capaz de argumentar una posición crítica e integral frente a ellas, redactando principalmente artículos de opinión.
\end{goals}

\begin{outcomes}
\ExpandOutcome{FH}{2}
\ExpandOutcome{f}{1}
\end{outcomes}

\begin{unit}{Apuntes sobre nuestro tiempo. Taller: ``Aproximación a los medios de comunicación y Teoría y Géneros del Periodismo''}{Figari98,Ugarte,Marin}{15}{3}
\begin{topics}
	\item Lenguaje, Homogeneización y Globalización
	\item Aproximación a los medios de comunicación social
	\item Nociones básicas de periodismo.
	\item Los géneros periodísticos Informativo, Interpretativo, Opinión.
	\item El lenguaje periodístico. Cualidades del lenguaje periodístico.
\end{topics}
\begin{unitgoals}
	\item Brindar elementos de juicio sobre algunos hechos resaltantes de nuestro tiempo y a partir de allí generar la necesidad de una visión cristiana (integral) de nuestra realidad. 	
	\item Complementar esto con una aproximación reflexiva y crítica al mundo de los medios de comunicación en Arequipa, el Perú y el mundo, y revisar teoría y géneros del periodismo.
\end{unitgoals}
\end{unit}

\begin{unit}{Feminismo e ideología de ``género''. Taller: ``Palabra, oración y párrafo''}{Conferencia,Barreiro,Avila,Buendia,Diario}{15}{3}
\begin{topics}
	\item Teoría: 
		\subitem Definición del término género. El feminismo de género. Roles socialmente construidos. El objetivo: deconstruir la sociedad. Primer blanco, la Familia. Salud y Derechos Sexuales Reproductivos. Ataque a la Religión.
		\subitem Perspectiva de género. Sus consecuencias. Propuestas para la promoción de la mujer.
		\subitem Manipulación del lenguaje.

	\item Taller de Redacción:
 		\subitem La palabra (palabra corta, palabra conocida, extranjerismos, localismos, palabra precisa, falsas sinónimas, palabras propias).
 		\subitem La frase (frase corta).
 		\subitem La oración (oración corta, la claridad, más puntos y menos comas, voz activa).
 		\subitem El párrafo (ritmo, impacto). 
 		\subitem Esquemas para buscar y ordenar ideas.
 		\subitem El proceso de redacción.
\end{topics}
\begin{unitgoals}
	\item Identificar los orígenes ideológicos de la Perspectiva de Género. Investigar su influencia en el mundo y especialmente en el Perú. 	
	\item Profundizar en los principales puntos de su agenda y las consecuencias de su aplicación en la realidad. 	
	\item Contrastar la perspectiva de género con una visión cristiana de la mujer, la familia, el matrimonio y la defensa de la vida.
	\item Redactar un artículo periodístico sobre las consecuencias de la influencia del feminismo y la ideología de género en el mundo. 	
	\item Dominar los aspectos básicos de la redacción, construyendo oraciones y párrafos con exactitud, originalidad, concisión y claridad.

\end{unitgoals}
\end{unit}

\begin{unit}{Nuevos desafíos bioéticos. Taller: ``El Periodismo de Opinión''}{Congregacion1,Congregacion2,GonzalesS,Grijelmo}{15}{3}
\begin{topics}
	\item Teoría
 		\subitem Aspectos antropológicos, teológicos y éticos de la vida y la procreación. 
 		\subitem Problemas relativos a la procreación. Técnicas de ayuda a la fertilidad, Fecundación in vitro, eliminación voluntaria y congelamiento de embriones.
 		\subitem Manipulación del embrión o del patrimonio genético humano. Terapia génica, clonación humana, uso terapéutico de las células troncales, etc.
	\item Taller de Redacción
 		\subitem El género de opinión. (El comentario, la columna, editorial, el artículo).
 		\subitem El artículo de fondo: Definición, clasificación, estructura, estilo, sugerencias para la redacción.
\end{topics}
\begin{unitgoals}
	\item Emplear principios antropológicos, teológicos y éticos, para valorar las nuevas aplicaciones científicas y médicas sobre la salud, la vida humana y la procreación.
	\item Redactar un artículo periodístico de opinión que proponga una valoración crítica de los problemas bioéticos de nuestro tiempo. Aplicar estructura, estilo y sugerencias en la escritura del género de opinión.
\end{unitgoals}
\end{unit}

\begin{unit}{La Doctrina Social de la Iglesia, Comunidad Política y Economía. Taller: ``El Estilo de Redacción''}{Pontificio,MorandeP,Fontana,MorandeP2,Grijelmo}{15}{3}
\begin{topics}
	\item Teoría:
 		\subitem Comunidad política, persona humana y pueblo. La autoridad política. Los valores y el sistema de la democracia. Política, bien común y justicia social. Política y solidaridad. Política y subsidiariedad.
 		\subitem La económica y la persona. Moral y economía. La acción del Estado. El desarrollo integral y solidario.
	\item Taller de Redacción:
 		\subitem El estilo es la claridad. Ordenación lógica. El estilo es: sorpresa, humor, ironía, vocabulario, paradoja, ritmo, adjetivo, metáfora, sonido, ambiente, orden, remate.

\end{topics}
\begin{unitgoals}
	\item Comprender la misión de la comunidad política desde los principios de la DSI, y así estar en capacidad de analizar los principales desafíos de la actividad socio-política y económica (nacional e internacional) que están más directamente relacionados con los principios del bien común y la justicia social, la solidaridad y la subsidiariedad.
	\item Demostrar esta capacidad de análisis e interpretación en la composición de un artículo de opinión sobre aspectos socio-políticos y económicos nacionales. Y desarrollar mejores habilidades expositivas y argumentativas.
\end{unitgoals}
\end{unit}



\begin{coursebibliography}
\bibfile{GeneralEducation/FG101}
\end{coursebibliography}
\end{syllabus}
