\begin{syllabus}

\course{FG201. Artes Plásticas}{Electivo}{FG201}

\begin{justification}
El curso es importante para la formación, analTecnologíatica, crTecnologíatica y practica de la Historia del Arte, el curso de una visión de acontecimientos artTecnologíasticos resultados de experiencias humanas que nos muestra el pasado y nos proyecta al futuro.
\end{justification}

\begin{goals}
\item Conocer los acontecimientos de la Historia Universal y tener un estudio critico.
\item Proporcionar al estudiante una aproximación a la realidad en el estudio de la Historia del Arte.
\item Proporcionar al estudiante el manejo apropiado de los materiales en pintura.
\end{goals}

\begin{outcomes}
\ExpandOutcome{FH}{3}
\end{outcomes}

\begin{unit}{}{}{9}{5}
\begin{topics}
	\item OrTecnologíagenes de la Comunicación Visual.
	\item Problemas de Comunicación Visual.
	\item El centro más potente.
	\item La gravedad.
	\item Elementos del Diseño: Espacio, punto, forma, equilibrio, simetrTecnologíaa, proporción.
	\item Elementos compositivos en una obra artTecnologíastica.
	\item Análisis Visual y crTecnologíatico de la misma.
	\item Estética e iconografTecnologíaa.
\end{topics}
\begin{unitgoals}
	\item Comprensión de los términos Arte y Estética.
	\item Entender a la comunicación, como base en las Artes visuales.

\end{unitgoals}
\end{unit}

\begin{unit}{}{}{9}{5}
\begin{topics}
	\item Introducción.
	\item Historia.
	\item El ojo, la visión - La luz.
	\item La memoria visual. (PsicologTecnologíaa).
	\item El recuerdo del Color.
	\item CTecnologíarculo cromático.
	\item Grises (escala de 3 a 10)
	\item Mezclas=ArmonTecnologíaa en Pintura.
\end{topics}
\begin{unitgoals}
	\item Conocer y sensibilizar el uso del color.
	\item Conocer: TeorTecnologíaas sobre el color y su consecutiva aplicación al entorno.
\end{unitgoals}
\end{unit}

\begin{unit}{}{}{9}{1}
\begin{topics}
	\item Historia del Arte: Concepto, crTecnologíatica y apreciación.
	\item Introducción, conceptos.
	\item Cultura, civilización e historia.
	\item La prehistoria: Arte PaleolTecnologíatico, NeolTecnologíatico.
	\item Chauvet y Caral.
	\item Mesopotamia y Persia.
	\item Asirios y Babilonios: Escultura - Pintura.
	\item Antiguo Egipto: Arquitectura, pintura y escultura.
\end{topics}
\begin{unitgoals}
	\item Desarrollar las diferentes épocas de la Historia Humana; las manifestaciones artTecnologíasticas y como estas influencian a través de los siglos.
	\item Formar modelos de apreciación a través de los conocimientos adquiridos.
\end{unitgoals}
\end{unit}

\begin{unit}{}{}{6}{1}
\begin{topics}
	\item Arte Grecia y Roma.
	\item Edad Media: Arte Bizantino, Románico y Gótico.
	\item Renacimiento.
	\item El Barroco e influencia en Perú.
	\item Impresionismo, fotografTecnologíaa y cine.
\end{topics}

\begin{unitgoals}
	\item Conocer las diferentes manifestaciones del Arte durante estos últimos tres milenios y el legado de nuestros antepasados; revalorarlos conociendo su historia y proyectarla al presente y futuro.
\end{unitgoals}
\end{unit}

\begin{unit}{}{}{6}{3}
\begin{topics}
	\item Siglo XX Cubismo (Guernica)
	\item Perú: Arte Inca y Arte Colonial.
	\item Proyecto de investigación.
	\item Exposición del Proyecto.
\end{topics}
\begin{unitgoals}
	\item Desarrollar un proyecto de investigación grupal, análisis conceptual, técnico, etc. de una obra de un pintor universal con trayectoria.
\end{unitgoals}
\end{unit}



\begin{coursebibliography}
\bibfile{GeneralEducation/FG101}
\end{coursebibliography}
\end{syllabus}
