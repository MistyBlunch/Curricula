\begin{syllabus}

\course{FG220. Análisis de la Realidad Peruana}{Obligatorio}{FG220}

\begin{justification}
Las perspectivas dominantes durante el último siglo en el análisis de la realidad nacional están evidenciando claros signos de agotamiento y caducidad para plantear soluciones plausibles al dilema cultural del Perú. Se hace cada vez más evidente que las aproximaciones ilustradas, en cualquiera de sus  vertientes -socialista, liberal y/o indigenista- tienen graves limitaciones para contener la realidad peruana en toda su complejidad, más aún, proponer soluciones de unidad, solidaridad e integración para los pueblos del Perú. 
En este contexto, el presente curso responde a la exigencia de ensayar renovados enfoques que contribuyan a la reconciliación y la integración (en la diversidad) de la sociedad peruana. Desde este curso se pretende generar una reflexión e investigación crítica que luego pueda ser desdoblada sobre la actividad profesional y la sociedad en general.
\end{justification}

\begin{goals}
\item Analizar y comprender la situación actual del Perú desde una perspectiva histórica y sociológica de modo que el alumno asuma desde su trabajo profesional la corresponsabilidad de construir una sociedad peruana más justa, integrada y reconciliada.
\end{goals}

\begin{outcomes}
\ExpandOutcome{HU}{4}
\ExpandOutcome{FH}{6}
\end{outcomes}

\begin{unit}{Orígenes de la peruanidad y la formación de la conciencia nacional}{VAB1965,Busto,Morande,Messori96,Estenos}{15}{6}
\begin{topics}
	\item Aproximaciones críticas a la realidad peruana: Víctor Andrés Belaúnde, José Carlos Mariátegui, Víctor Raúl Haya de la Torre.
	\item Aspectos conceptuales relevantes para el análisis: Cultura, Identidad, Nación, Sociedad y Estado. 
	\item El imperio de los Incas. Repaso de  aspectos socio-culturales más importantes. 
	\item Conquista española. ¿Encuentro o choque de las culturas? Hacia una comprensión integral del fenómeno. Debate conceptual. 
	\item Virreinato. Repaso de  aspectos socio-culturales más importantes. Surgimiento de la identidad nacional peruana al calor de la Fe Católica.  

\end{topics}
\begin{unitgoals}
	\item Comprender adecuadamente el proceso histórico que determina el nacimiento de nuestra identidad nacional a partir de la síntesis cultural del virreinato.
\end{unitgoals}
\end{unit}

\begin{unit}{Procesos de integración y desintegración nacional del siglo XIX}{delaPuente1970,PersonaCultura,BASADRE,Mariategui}{12}{6}
\begin{topics}
	\item La independencia del Perú y la fundación del Estado Peruano.
	\item Primeros cambios culturales: Inicio del proceso secularizador de la cultura. Primera República y Militarismo. Repaso de  aspectos socio-culturales más importantes.
	\item Prosperidad Falaz.  Repaso de  aspectos socio-culturales más importantes. 
\end{topics}
\begin{unitgoals}
	\item Identificar adecuadamente los procesos históricos de integración y desintegración nacional en el siglo XIX.
\end{unitgoals}
\end{unit}

\begin{unit}{Procesos de integración y desintegración nacional del siglo XX}{PEASE1999,Barnechea}{6}{6}
\begin{topics}
	\item Principales ideologías políticas en el siglo XX en contrapunto con los principios de la Doctrina Social de la Iglesia.
	\item Análisis del aspecto simbólico ideacional y socio-cultural más importante del siglo XX. 

\end{topics}
\begin{unitgoals}
	\item Identificar adecuadamente los procesos históricos de integración y desintegración nacional en el siglo XX. 
\end{unitgoals}
\end{unit}

\begin{unit}{El Perú en la actualidad: análisis del sector político, social y económico}{Contreras2002,Estenos1,Estenos2,BenedictoXVI}{9}{6}
\begin{topics}
	\item Análisis del sistema político peruano.
	\item Balance del Estado Peruano: centralismo y corrupción.
	\item Marginación y exclusión: proceso de migración y pobreza en el Perú.
	\item Violencia social, terrorismo y narcotráfico en el Perú.
	\item Transformaciones culturales de la sociedad peruana: la educación como crisis y posibilidad.
	\item Bases para un desarrollo social basado en la promoción  humana.
	\item Conclusiones finales

	\item Análisis del aspecto simbólico ideacional y socio-cultural más importante del siglo XX. 

\end{topics}
\begin{unitgoals}
	\item Conocer y analizar la situación actual de la política, económica y social en el Perú, su problemática y posibilidades de solución.
\end{unitgoals}
\end{unit}



\begin{coursebibliography}
\bibfile{GeneralEducation/FG220}
\end{coursebibliography}
\end{syllabus}
