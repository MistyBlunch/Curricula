\begin{syllabus}

\course{IS351. Ética Profesional}{Obligatorio}{FG211}

\begin{justification}
La moralidad y la ética no pueden separarse del individuo y como consecuencia
no pueden separarse de su actuar diario y profesional, existe una necesidad 
mayor de éstas en el mundo globalizado y contemporáneo en el que estamos inmersos, 
pues el mercado no va a solucionar los problemas éticos por sí solo.

El curso esta diseñado como una introducción a los elementos de la computación forense. 
En particular, se busca entender con detalle la manera como se generan las posibles 
fallas de seguridad, donde se pueden encontrar evidencias del hecho y estrategias 
generales de manejo de las mismas. En este contexto, el curso ofrece un marco 
conceptual de análisis que oriente a los participantes ante una situación en la 
que se encuentre comprometida la seguridad informática de una organización.

Como contribución a la formación humana, los participantes conprenderán que la 
realización personal implica discernimiento constante para el buen ejercicio 
de la libertad en la consecución del bien personal y social.

Contribución a la formación profesional: Comprender la carrera profesional 
elegida como una actitud de servicio y como contribución en la edificación 
de la sociedad, actividad en la que podremos construir y cualificar personalmente 
la sociedad que deseamos heredar a nuestros hijos, viviendo cotidianamente 
en el actuar profesional los valores que podrían hacer la diferencia en un futuro.

Los estudiantes una vez concluido el curso, tendrán elementos para abordar 
posibles situaciones de fraude realizados a través de computadores, 
identificar evidencia digital relevante y presentar una aproximación desde la 
perspectiva legal en conjunción con las especificaciones técnicas que reviste 
el hecho.
\end{justification}

\begin{goals}
\item Desarrollar criterios éticos y morales, con énfasis en las nociones de prudencia y equidad para aplicarlos en el ejercicio profesional.
\end{goals}

\begin{outcomes}
\ExpandOutcome{e}{2}
\ExpandOutcome{g}{4}
\ExpandOutcome{TASDSH}{3}
\end{outcomes}

\begin{unit}{El profesional frente a la sociedad}{Mendell2004,Mohay}{8}{2}
\begin{topics}
      \item {La objetividad moral y la formulación de principios morales.}
      \item {El profesional y sus valores.}
      \item {La conciencia moral de la persona.}
      \item {El bien común y el principio de subsidiariedad.}
      \item {Principios morales y propiedad privada.}
      \item {Justicia, algunos conceptos básicos.}
\end{topics}
\end{unit}

\begin{unit}{Profesión y moralidad}{Mendell2004,Mohay}{8}{2}
\begin{topics}
      \item {El secreto profesional.}
      \item {La obligación moral de comunicar la verdad.}
      \item {Ética y medio ambiente.}
      \item {Principios generales sobre la colaboración en hechos inmorales.}
      \item {El profesional frente al soborno, ?`víctima o colaboración?.}
      \item {La profesión: una vocación de servicio.}
\end{topics}
\end{unit}

\begin{unit}{Modelo OSI/ISO: Análisis de seguridad de TCP/IP}{Mendell2004,Mohay}{6}{3}
\begin{topics}
      \item {Arquitectura TCP/IP.}
      \item {Enrutamiento y seguridad en enrutadores.}
      \item {Protocolo TCP en detalle.}
      \item {Servicios de Seguridad en los niveles del OSI, para TCP/IP.}
\end{topics}
\end{unit}

\begin{unit}{Aplicación de las técnicas forenses a la computación}{Mendell2004,Mohay}{6}{3}
\begin{topics}
      \item {Introducción a las ciencias forenses en computación.}
      \item {Componentes de los sistemas de cómputo.}
      \item {Procesamiento y revisión de la evidencia.}
\end{topics}
\end{unit}

\begin{unit}{Evidencia digital en redes}{Mendell2004,Mohay}{8}{4}
\begin{topics}
      \item {Análisis de tráfico de red. Network Intrusion Detection.}
      \item {Registros y rastros a través del Modelo OSI/ISO.}
      \item {Análisis de incidentes de seguridad en redes.}
\end{topics}
\end{unit}

\begin{unit}{Investigando un cybercrimen}{Mendell2004,Mohay}{8}{3}
\begin{topics}
      \item {Definición de cybercrimen.}
      \item {Modelo de investigación.}
      \item {Herramientas forenses.}
      \item {Participantes y resultados.}
\end{topics}
\end{unit}

\begin{unit}{Consideraciones legales}{Mendell2004,Mohay}{8}{4}
\begin{topics}
      \item {Leyes internacionales sobre los cybercrimenes.}
      \item {El discurso probatorio en informática.}
      \item {Implicaciones en Colombia.}
\end{topics}
\end{unit}

\begin{unit}{Tendencias futuras}{Mendell2004,Mohay}{8}{2}
\begin{topics}
      \item {Digital Evidence Response Team.}
      \item {Preparación forense de redes.}
      \item {Análisis forenses de sistemas inalámbricos.}
      \item {Anonimato y seguimiento.}
      \item {Retos y problemáticas legales.}
\end{topics}
\end{unit}

\begin{coursebibliography}
\bibfile{GeneralEducation/FG211}
\end{coursebibliography}

\end{syllabus}

%\end{document}
