\begin{syllabus}

\course{IS321. Auditoría de Sistemas}{Obligatorio}{IS321}

\begin{justification}
La profesión de auditoría y control de SI continúa evolucionando. Las universidades e
instituciones educativas deben entender las necesidades de la comunidad profesional
para proporcionar al mercado graduados que posean las destrezas requeridas y el
conocimiento que los profesionales necesitan. El modelo académico que propone
ISACA en esta área, proporciona a las universidades un marco conceptual básico
sobre de la educación requerida para desarrollar las destrezas necesarias en la
profesión.

Los auditores de sistemas de información necesitan enfrentar el ritmo de cambios
rápidos de tecnología y ponerse al día regularmente con conocimiento técnico
competente. Eventos recientes, regulaciones gubernamentales y cambios en los
procesos de negocios han afectado el rol de la auditoría de SI y la metodología que los
auditores usan sean específicas o de manera integral. Por eso, la profesión de
Sistemas de Información, en el área de Auditoría y control de SI, debe entender las
nuevas tecnologías, ser capaz de determinar su impacto en el proceso de control y los
procedimientos de auditoría y comunicar claramente que las herramientas y técnicas
de recolección de evidencia han sido desarrolladas. El modelo académico no sólo
toma en consideración los desafíos tecnológicos, sino también los asuntos
relacionados con el mejoramiento de las habilidades orales y escritas.

En el ambiente de negocios basado en información, se encuentra la gran demanda de
las carreras que en ellas tienen inmersas las sumillas de auditoría de SI y Seguridad
de la información. El especialista en SI y el auditor de SI deben recibir capacitación
continua para actualizar sus conocimientos, destrezas y habilidades.

Las universidades con el plan de estudios apropiado pueden educar candidatos que
pueden ser empleados en auditoría y control de SI. Una universidad proactiva que
patrocina un plan de estudios en auditoría y control de SI como lo promueve ISACA es
muy apreciada por los profesionales que desean cambiar sus carreras o actualizar sus
destrezas para el mejoramiento en su labor ya que este modelo proporciona un
razonable y comprensivo conjunto de temas relacionados con un programa ideal de
auditoría y control de SI. Adicionalmente, el modelo puede servir para aquellos que
están interesados en la educación de auditoría de SI así como las instituciones
educativas alrededor del mundo que están desarrollando un plan de estudios en
auditoría y control de SI.
\end{justification}

\begin{goals}
\item Acercar al alumno a conceptos y técnicas usados en auditoría de sistemas de información.
\item Dar al alumno las herramientas necesarias para administrar y llevar a cabo una auditoría de sistemas de información.
\end{goals}

\begin{outcomes}
\ExpandOutcome{a}
\ExpandOutcome{b}
\ExpandOutcome{c}
\ExpandOutcome{d}
\ExpandOutcome{g}
\ExpandOutcome{i}
\ExpandOutcome{j}
\ExpandOutcome{k}
\end{outcomes}

\begin{unit}{Conocimiento de la Función de la Auditoría en los Sistemas de Información}{ISACA2008,ITAuditing2007,Cascarino2007,ITG2005}{3}{1}
    \begin{topics}
		\item Leyes y regulaciones: estatutos de la auditoría.
		\item Naturaleza de la auditoría: demanda por auditorías (Ej., teoría de agencia, hipótesis del seguro, hipótesis de la información).
		\item Naturaleza de auditorías de SI: la necesidad del control y la auditoría en los SI computarizados.
		\item Tipos de auditoría y de auditores: SI, externa, interna, gobierno/sector público.
		\item Responsabilidad y autoridad del auditor de SI: estatutos de la auditoría; subcontratación de los servicios de auditoría de SI.
		\item Regulación y control de la auditoría de SI: estándares de ISACA, pautas, código de ética profesional; leyes; regulaciones.
    \end{topics}
    \begin{unitgoals}
	\item Conocer la función de un auditor de sistemas de información.
    \end{unitgoals}
\end{unit}

\begin{unit}{Conceptos Fundamentales de Auditoría}{ISACA2008,ITAuditing2007,Cascarino2007,ITG2005}{2}{1}
    \begin{topics}
		\item Materialidad: aplicación de materialidad a las auditorías de SI en comparación a la materialidad en auditorías de estados financieros
		\item Evidencia: tipos de evidencia; significado de la evidencia suficiente, confiable y pertinente.
		\item Independencia: necesidad de independencia en actitud y apariencia; situaciones que pueden deteriorar la independencia.
		\item Riesgo de auditoría: riesgo inherente, riesgo de control y riesgo de detección.
		\item SI: responsabilidades generales en la auditoría ante el fraude.
		\item Certeza.
    \end{topics}
    \begin{unitgoals}
	\item Introducir los conceptos funcamientales de auditoría.
    \end{unitgoals}
\end{unit}

\begin{unit}{Estándares y Directrices de Auditoría de SI}{ISACA2008,ITAuditing2007,Cascarino2007,ITG2005}{2}{1}
\begin{topics}
\item Conocimiento del Código de Ética Profesional de ISACA
\item Revisión de los estándares y pautas vigentes de ISACA para auditorías de SI
\item Estándares y pautas específicos para cada región/país: AITP, ACM, AICPA, IIA, ISO COCO, AGA NIA (IFAC)
\item Prácticas y técnicas en la auditoría de SI
\end{topics}
\begin{unitgoals}
\item Estudiar los estándares conocidos de auditoría de SI.
\end{unitgoals}
\end{unit}

\begin{unit}{Conceptos de Controles Internos}{ISACA2008,ITAuditing2007,Cascarino2007,ITG2005}{2}{1}
\begin{topics}
\item Relevancia, estructura e indicadores de un gobierno efectivo de TI para organizaciones y auditores de SI; estructura de gobierno de TI
\item Objetivos de control interno; control interno y la documentación de SI, i.e., COSO, COCO, KING, Ley Sarbanes-Oxley del 2002, SAS94
\item Clasificación de Controles: preventivo, detección y compensatorio/correctivo 
\item Controles generales: de organización, de seguridad, de funcionamiento general y de recuperación del desastre, desarrollo, documentación
\item Controles de aplicación: objetivos de control; clasificaciones de las aplicaciones control, ej., computarizado/manual; entrada/procesamiento/salida; preventivo/de \item detección/correctivo, pistas de auditoría intervención
\item COBIT: La estructura y la importancia de COBIT para las organizaciones y los auditores de SI
\end{topics}

\begin{unitgoals}
\item Revisar los recursos y procesos de control.
\end{unitgoals}
\end{unit}

\begin{unit}{Proceso de Planificación de Auditoría}{ISACA2008,ITAuditing2007,Cascarino2007,ITG2005}{2}{1}
\begin{topics}
\item Planificación estratégica de la auditoría Carta de entendimiento: propósito y contenido de las cartas de entendimiento.
\item Valoración de riesgo: auditoría basada en riesgo; métodos de valoración de riesgo.
\item Estándares: AS-NZ 4360, CRAMM.
\item Evaluación preliminar de controles internos: recopilación de la información y evaluación de las técnicas de control.
\item Plan de auditoría, programa y alcance: pruebas de controles vs. pruebas sustantivas, aplicación de la valoración de riesgo a la auditoría.
\item Clasificación, alcance de las auditorías: e.g., financiera, operacional, general, aplicación, OS, física, lógica.
\end{topics}

\begin{unitgoals}
\item Conocer el proceso de planificación de una auditoría.
\end{unitgoals}
\end{unit}

\begin{unit}{Administración de la Auditoría}{ISACA2008,ITAuditing2007,Cascarino2007,ITG2005}{2}{1}
\begin{topics}
\item Distribución/categorización/planificación/ ejecución/reasignación de los recursos.
\item Evaluación de la calidad de la auditoría y la revisión entre colegas.
\item Identificación de la mejor práctica.
\item CIS Desarrollo de la carrera de auditoría.
\item Planificación de la trayectoria de la carrera.
\item Valoración del desempeño.
\item Consejería y retroalimentación del desempeño.
\item Capacitación (interna/externa).
\item Desarrollo profesional (certificaciones, participación profesional, etc.).
\item Evidencia: suficiente, confiable, pertinente, y útil.
\item Técnicas de recopilación de evidencia, e.g., observar, preguntar, entrevistar y probar.
\item Pruebas de control vs. pruebas sustantivas: naturaleza de las pruebas de control y las pruebas sustantivas y sus diferencias, tipos de pruebas de control, tipos de pruebas sustantivas.
\end{topics}
\begin{unitgoals}
\item Estudiar las herramientas y técnicas para llevar a cabo la administración de una auditoría.
\end{unitgoals}
\end{unit}

\begin{unit}{Proceso de Obtención de  Evidencia en la Auditoría}{ISACA2008,ITAuditing2007,Cascarino2007,ITG2005}{2}{1}
\begin{topics}
\item Muestreo: conceptos de muestreo, enfoques estadísticos y no estadísticos, diseño y selección de muestras, evaluación de los resultados de la muestra.
\item Técnicas de auditoría computarizadas (Computer-assisted audit techniques CAATs): necesidad, tipos, planificación y uso de CAATs, enfoque de la auditoría continua en línea.
\item Documentación: relación con evidencia de la auditoría; usos de la documentación; contenido mínimo; custodia, retención y recuperación.
\item Análisis: evaluar la materialidad de los resultados e identificar las condiciones reportables y alcanzar las conclusiones.
\end{topics}
\begin{unitgoals}
\item Estudiar técnicas de muestreo, documentación y análisis para auditoría.
\end{unitgoals}
\end{unit}

\begin{unit}{Seguimiento del Informe de Auditoría}{ISACA2008,ITAuditing2007,Cascarino2007,ITG2005}{2}{1}
\begin{topics}
\item Forma y contenido del informe de auditoría: propósito; estructura y contenido; estilo; usuario; tipo de opinión; consideración de acontecimientos subsecuentes.
\item Acciones de la gerencia para poner recomendaciones en ejecución.
\end{topics}
\begin{unitgoals}
\item Conocer cómo se debe llevar el seguimiento de informes de auditoría.
\end{unitgoals}
\end{unit}

\begin{unit}{Gerencia de SI/TI}{ISACA2008,ITAuditing2007,Cascarino2007,ITG2005}{2}{1}
\begin{topics}
\item Gerencia de proyecto TI.
\item Gerencia de riesgo económico, social, cultural, gerencia de riesgo tecnológico.
\item Software de gerencia de control de calidad.
\item Administración de la infraestructura de TI y arquitectura de TI, administración de la configuración.
\item Administración de la entrega de TI (operaciones) y de apoyo (mantenimiento).
\item Medida y divulgación de funcionamiento: cuadro de mando integral.
\item Contratación externa.
\item Aseguramiento de la calidad.
\item Acercamiento técnico-social y cultural a la gerencia.
\end{topics}
\begin{unitgoals}
\item Revisar conceptos de gerencia de sistemas de información y de tecnologías de información.
\end{unitgoals}
\end{unit}

\begin{unit}{Planificación Estratégica de SI/TI}{ISACA2008,ITAuditing2007,Cascarino2007,ITG2005}{2}{1}
\begin{topics}
\item SI/IT planificación estratégica. Estrategias competitivas e inteligencia de negocios: enlace con la estrategia corporativa.
\item Marco y aplicaciones de los sistemas de información: tipos de SI - gerencia del conocimiento, sistemas de apoyo para las decisiones ; clasificación de los sistemas de información.
\item Gerencia de recursos humanos de TI, políticas de los empleados, acuerdos y contratos.
\item Segregación de tareas.
\item SI/TI entrenamiento y educación.
\end{topics}

\begin{unitgoals}
\item Revisar conceptos de planificación estratégica de SI/TI.
\end{unitgoals}
\end{unit}

\begin{unit}{Asuntos Gerenciales de SI/TI}{ISACA2008,ITAuditing2007,Cascarino2007,ITG2005}{2}{1}
\begin{topics}
\item Asuntos legales relacionados a la introducción de TI a la empresa (internacional y local).
\item Asuntos de propiedad intelectual en el espacio cibernético: marcas registradas, copyright y patentes.
\item Problemas éticos.
\item Privacidad.
\item Gobierno de TI.
\item Mantenimiento de SI/TI.
\end{topics}

\begin{unitgoals}
\item Revisar conceptos de gerencia de SI/TI.
\end{unitgoals}
\end{unit}

\begin{unit}{Herramientas de Apoyo para la gestión}{ISACA2008,ITAuditing2007,Cascarino2007,ITG2005}{2}{1}
\begin{topics}
\item COBIT. Pautas gerenciales para gerentes de SI/TI.
\item COBIT. uso de auditorías como apoyo para el ciclo del negocio.
\item Estándares Internacionales - ISO-I7799, Estándares de Privacidad, COCO, COSO, Cadbury, King, ITIL.
\item Revisiones de control de cambios.
\end{topics}

\begin{unitgoals}
\item Estudiar herramientas de apoyo para la gestión y asi tener en cuenta para cuando se requiera auditar.
\end{unitgoals}
\end{unit}

\begin{unit}{Técnicas}{ISACA2008,ITAuditing2007,Cascarino2007,ITG2005}{2}{1}
\begin{topics}
\item Revisiones operacionales.
\item Revisiones de ISO 9000.
\end{topics}
\begin{unitgoals}
\item Estudiar técnicas operativas para la gestión de SI/TI.
\end{unitgoals}
\end{unit}

\begin{unit}{Infraestructura Técnica}{ISACA2008,ITAuditing2007,Cascarino2007,ITG2005}{6}{1}
\begin{topics}
\item Arquitectura y estándares de TI.
\item Hardware: todo el equipo de TI incluyendo la unidad central, las mini computadoras, clientes/servidores, los enrutadores, los interruptores, las comunicaciones, las PC, etc.
\item Software: sistemas operacionales, programas de utilidades, bases de datos, etc.
\item Red: el equipo y los servicios de comunicaciones dedicados para proporcionar las redes, red relacionada al hardware, red relacionada al software, el uso de los proveedores que proporcionan servicios de comunicación, etc.
\item Controles fundamentales.
\item Seguridad / pruebas y validación.
\item Herramientas de evaluación y supervisión de desempeño.
\item Gobierno de TI. Mantenimiento y Funcionamiento.
\item Supervisión de controles de TI y herramientas de evaluación, como vigilancia de sistemas de control de acceso o vigilancia de incursión con sistemas de detección.
\item Gerencia de recursos de información e infraestructura: software de gerencia de empresas.
\item Gerencia de centros de servicio y estándares/guías de las operaciones: COBIT, ITIL, ISO 17799.
\item Asuntos y consideraciones de centro de servicio vs. infraestructuras técnicas propietarias .
\item Sistemas abiertos.
\item Gerencia de cambio/Implementación de nuevos sistemas: organización de las herramientas usadas para controlar la introducción de productos nuevos al ambiente del centro de servicio, etc.
\end{topics}

\begin{unitgoals}
\item Conocer la infraestructura técnica relevante para una auditoría.
\end{unitgoals}
\end{unit}

\begin{unit}{Gerencia de Centros de Servicio}{ISACA2008,ITAuditing2007,Cascarino2007,ITG2005}{4}{1}
\begin{topics}
\item Gerencia de Seguridad
\item Gerencia de Recurso/configuración: cumplimiento con organización/TI estándares operacionales, políticas y procedimientos (uso correcto del lenguaje de computadoras)
\item Gerencia de problemas e incidentes 
\item Planificación y estimación de capacidad
\item Gerencia de la distribución de sistemas automatizados
\item Administración del lanzamiento y versiones de sistemas automatizados
\item Gerencia de proveedores
\item Enlaces con clientes
\item Administración del nivel de servicios
\item Contingencia/ Respaldos y administración de la recuperación
\item Gerencia del centro de llamadas
\item Gerencia de las operaciones de la infraestructura (central y distribuida)
\item Administración de redes
\item Gerencia de riesgo
\item Principios claves de gerencia
\end{topics}
\begin{unitgoals}
\item Conocer cómo se debe gerenciar un centro de servicios.
\end{unitgoals}
\end{unit}


\begin{unit}{Planificación de SI}{ISACA2008,ITAuditing2007,Cascarino2007,ITG2005}{2}{1}
\begin{topics}
\item Componentes para manejar SI (datos-procesos-tecnologías-organización); entendiendo a los tenedores y sus requerimientos
\item Métodos de planificación de SI: investigación del sistema, oportunidades de proceso de integración/reingeniería, evaluación del riesgo, análisis de costo/beneficio, gravamen de riesgo; análisis y diseño de los sistemas orientados a objetos
\item Integración de los usos de la empresa del software de ERP
\end{topics}
\begin{unitgoals}
\item Revisar conceptos de planificación de SI.
\end{unitgoals}
\end{unit}

\begin{unit}{Uso y Gerencia de Información}{ISACA2008,ITAuditing2007,Cascarino2007,ITG2005}{2}{1}
\begin{topics}
\item Supervisión del funcionamiento del porcentaje de disponibilidad contra acuerdos del porcentaje de disponibilidad, calidad del servicio, de la disponibilidad, del tiempo de reacción, de la seguridad y de los controles, proceso la integridad, aislamiento, remedios, cumpliendo con los niveles de servicio acordados (SLAs por sus siglas en
inglés)
\item Datos e información: analizar, evaluar y diseñar la arquitectura de información (i.e., rol de las bases de datos y Gerencia de sistemas de bases de datos incluyendo sistemas de gerencia del conocimiento, almacenes de datos)
\item Datos y arquitectura del uso (Modelo de SI, los modelos del negocio, los procesos y las soluciones); análisis, evaluación y diseño de los procesos del negocio de la entidad y los modelos del negocio
\item Gerencia de Información (administración de datos, funciones y administración de bases de datos, roles y responsabilidades de DBA)
\item Tecnología de base de datos como herramienta para el auditor
\item Estructura de datos y lenguaje básico SQL
\end{topics}
\begin{unitgoals}
\item Revisar conceptos de uso y gerencia de información.
\end{unitgoals}
\end{unit}

\begin{unit}{Desarrollo, Adquisición y Mantenimiento de SI}{ISACA2008,ITAuditing2007,Cascarino2007,ITG2005}{2}{1}
\begin{topics}
\item Gerencia de proyecto de los sistemas de información: planificación, organización, despliegue del recurso humano, control del proyecto, supervisión y ejecución
\item Métodos tradicionales para el desarrollo del ciclo de vida del sistema; analizar, evaluar y diseñar las fases del desarrollo del ciclo de vida de un sistema (SDLC)
\item Acercamientos para el desarrollo del sistema: paquetes de software, prototipo, reingeniería de proceso del negocio, herramientas CASE.
\item Mantenimiento de sistemas y procedimientos para el control de cambios para modificaciones de sistemas
\item Problemas de riesgo y control, analizar y evaluar características y riesgos del proyecto
\end{topics}
\begin{unitgoals}
\item Revisar conceptos de desarrollo, adquisición y mantenimiento de SI.
\end{unitgoals}
\end{unit}

\begin{unit}{Impacto de TI en los Negocios}{ISACA2008,ITAuditing2007,Cascarino2007,ITG2005}{2}{1}
\begin{topics}
\item Contratación externa de Procesos de Negocios (Business Process Outsourcing BPO)
\item Aplicación de los problemas y de las tendencias del comercio electrónico
\end{topics}
\begin{unitgoals}
\item Discutir el impacto de TI en los negocios.
\end{unitgoals}
\end{unit}

\begin{unit}{Desarrollo del Software}{ISACA2008,ITAuditing2007,Cascarino2007,ITG2005}{2}{1}
\begin{topics}
\item Separación de la especificación e implementación en la programación
\item Metodología de la especificación de requisitos
\item Diseño del algoritmo; clasificación y búsqueda de algoritmos
\item Manejo de archivos
\item Listas encadenadas y árboles binarios
\item Creación y manipulación de la base de datos
\item Principios del buen diseño de la pantalla y del informe
\item Alineamiento del lenguaje de programación
\end{topics}
\begin{unitgoals}
\item Revisar conceptos de desarrollo de software para tener en cuenta en la forma de auditar.
\end{unitgoals}
\end{unit}

\begin{unit}{Auditoría y Desarrollo de Controles de Aplicación}{ISACA2008,ITAuditing2007,Cascarino2007,ITG2005}{4}{1}
\begin{topics}
\item Controles de entrada/origen
\item Procedimientos del control de proceso
\item Controles de salida
\item Documentación del sistema de aplicación
\item Pistas de Auditoría
\end{topics}
\begin{unitgoals}
\item Estudiar conceptos que permitan desarrollar controles de aplicación y determinar pistas de auditoria que ayuden a identificar riesgos.
\end{unitgoals}
\end{unit}



\begin{coursebibliography}
\bibfile{Computing/IS/IS}
\end{coursebibliography}

\end{syllabus}
