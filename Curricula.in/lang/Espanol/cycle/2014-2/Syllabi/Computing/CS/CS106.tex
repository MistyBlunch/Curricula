\begin{syllabus}

\course{CS106. Estructuras Discretas II}{Obligatorio}{CS106}

\begin{justification}
Para entender las t�cnicas computacionales avanzadas, los estudiantes deber�n tener un fuerte conocimiento de las
diversas estructuras discretas, estructuras que ser�n implementadas y usadas en laboratorio en el lenguaje de programaci�n.
\end{justification}

\begin{goals}
\item Que el alumno sea capaz de modelar problemas de ciencia de la computaci�n usando grafos y �rboles relacionados con estructuras de datos
\item Que el alumno aplicar eficientemente estrategias de recorrido para poder buscar datos de una manera �ptima
\end{goals}

\begin{outcomes}
\ExpandOutcome{a}{3}
\ExpandOutcome{b}{4}
\ExpandOutcome{i}{3}
\ExpandOutcome{j}{3}
\end{outcomes}

\begin{unit}{\DSBasicCountingDef}{Grimaldi97}{25}{4}
   \DSBasicCountingAllTopics
   \DSBasicCountingAllObjectives
\end{unit}

\begin{unit}{\DSGraphsAndTreesDef}{Johnsonbaugh99}{25}{3}
   \DSGraphsAndTreesAllTopics
   \DSGraphsAndTreesAllObjectives
\end{unit}

\begin{unit}{\DSDiscreteProbabilityDef}{Micha98,Rosen2004}{10}{4}
   \DSDiscreteProbabilityAllTopics
   \DSDiscreteProbabilityAllObjectives
\end{unit}



\begin{coursebibliography}
\bibfile{Computing/CS/CS105}
\end{coursebibliography}

\end{syllabus}

%\end{document}
