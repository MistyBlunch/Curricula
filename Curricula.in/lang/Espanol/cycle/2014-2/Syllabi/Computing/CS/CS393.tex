\begin{syllabus}

\course{CS393. Métodos Formales}{Electivo}{CS393}

\begin{justification}
Los desarrollo de software, en gran medida, aún es una actividad artesanal lo que implica que muchas veces no es posible entregar el software correcto, en el tiempo y presupuestos planeados. Los métodos formales intentan dar rigidez y solidez matemática, a todo el proceso de desarrollo de software, en la búsqueda de la producción de software de calidad.
\end{justification}

\begin{goals}
\item Crear especificaciones y diseños matemáticamente precisos utilizando lenguajes de especificación formales. Analizar las propiedades de las especificaciones y diseños formales.
\item Aplicar las técnicas formales de verificación a los segmentos de software con complejidad baja. Discutir y analizar los tipos de modelos existentes para Métodos Formales.
\item Discutir el papel de la verificación de las técnicas formales en el contexto de la validación y prueba de software. Aprender a utilizar los diferentes lenguajes de especificación formal para la especificación y validación de requisitos. Analizar las propiedades de las especificaciones y diseños formales.
\item Utilizar herramientas para transformar especificaciones y diseños. Explicar las ventajas y desventajas potenciales de usar lenguajes de especificación formal. Crear y evaluar aserciones (pre y post condiciones e invariantes), para una variedad de situaciones que se extienden de simples a complejas.
\item Con un lenguaje de especificación formal común, formular la especificación de un sistema de software simple y demostrar las ventajas de una perspectiva de calidad.
\end{goals}

\begin{outcomes}
\ExpandOutcome{a}{4}
\ExpandOutcome{b}{3}
\ExpandOutcome{c}{3}
\ExpandOutcome{d}{3}
\ExpandOutcome{i}{3}
\ExpandOutcome{j}{3}
\ExpandOutcome{l}{3}
\end{outcomes}

\begin{unit}{\SEFormalMethodsDef}{Baugh92}{14}{3}
    \SEFormalMethodsAllTopics
    \SEFormalMethodsAllObjectives
\end{unit}

\begin{unit}{Métodos y Fundamentos Matematicos}{Baugh92}{12}{3}
\begin{topics}
      \item Métodos de construcción formal.
      \item Fundamentos matemáticos.
      \begin{inparaenum}
         \item Grafos y árboles.
         \item Autómata finito, expresiones regulares.
         \item Gramáticas.
         \item Precisión numérica, exactitud, y errores.
      \end{inparaenum}
   \end{topics}
   \begin{unitgoals}
      \item Crear especificaciones y diseños matemáticamente precisos utilizando. lenguajes de especificación formales.
      \item Analizar las propiedades de las especificaciones y diseños formales.
   \end{unitgoals}
\end{unit}

\begin{unit}{Modelamiento}{Baugh92}{12}{3}
   \begin{topics}
      \item Introducción a los modelos matemáticos y lenguajes de especificación.
      \item Tipos de modelos.
      \item Modelamiento de comportamiento.
   \end{topics}
   \begin{unitgoals}
      \item Aplicar las técnicas formales de verificación a los segmentos de software con complejidad baja.
      \item Discutir y analizar los tipos de modelos existentes para Métodos Formales.
   \end{unitgoals}
\end{unit}

\begin{unit}{Especificacion de Requerimientos}{Hinchey96}{12}{4}
   \begin{topics}
      \item Documentación y especificación de requerimientos.
      \begin{inparaenum}
         \item Lenguajes de especificación (OCL, Z, etc.).
      \end{inparaenum}
      \item Validación de requerimientos.
   \end{topics}
   \begin{unitgoals}
      \item Discutir el papel de la verificación de las técnicas formales en el contexto de la validación y prueba de software.
      \item Aprender a utilizar los diferentes lenguajes de especificación formal para la especificación y validación de requisitos.
      \item Analizar las propiedades de las especificaciones y diseños formales
   \end{unitgoals}
\end{unit}

\begin{unit}{Diseño}{Guttag91}{12}{3}
   \begin{topics}
      \item Diseño detallado.
      \item Notaciones de diseño y herramientas de soporte.
      \begin{inparaenum}
         \item Análisis de diseño formal.
      \end{inparaenum}
      \item Evaluación de diseño.
      \begin{inparaenum}
         \item Técnicas de evaluación.
      \end{inparaenum}
   \end{topics}
   \begin{unitgoals}
       \item Utilizar herramientas para transformar especificaciones y diseños.
       \item Explicar las ventajas y desventajas potenciales d eusar lenguajes de especificación formal.
       \item Crear y evaluar aserciones (pre y post condiciones e invariantes), para una variedad de situacioines que se extienden de simples a complejas.
   \end{unitgoals}
\end{unit}

\begin{unit}{Evolución}{Jacky96}{12}{4}
   \begin{topics}
      \item Actividades de evolución.
      \begin{inparaenum}
         \item Refabricación.
         \item Transformación de programas.
      \end{inparaenum}
   \end{topics}
   \begin{unitgoals}
      \item Con un lenguaje de especificación formal común, formular la especificación de un sistema de software simple y demostrar las ventajas de una perspectiva de calidad.
   \end{unitgoals}
\end{unit}



\begin{coursebibliography}
\bibfile{Computing/CS/CS393}
\end{coursebibliography}

\end{syllabus}
