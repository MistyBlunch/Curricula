\begin{syllabus}

\course{CS392. Tópicos en Ingeniería de Software}{Electivo}{CS392}

\begin{justification}
Este curso es útil en la formación profesional para que el alumno tenga contacto con tópicos especializados  en el área de Ingeniería de Software.
\end{justification}

\begin{goals}
\item Los alumnos deben conocer los modelos de confiabilidad de software y la aplicación de los métodos de análisis probabilísticas para un sistema de software.
\item Los alumnos deben identificar y aplicar la redundancia y tolerancia a fallas para un sistema de software.
\item El alumno aplicará las diferentes técnicas de verificación formal a software con baja complejidad.
\item Los alumnos utilizarán un lenguaje de especificación formal para especificar un sistema de software y demostrará los beneficios de una perspectiva de calidad.
\item Familiarizar a los alumnos con los principios reconocidos para la construcción de componentes de software de alta calidad.
\item Aplicar métodos orientados a componentes para el diseño de un software.
\end{goals}

\begin{outcomes}
\ExpandOutcome{b}{4}
\ExpandOutcome{c}{4}
\ExpandOutcome{d}{4}
\ExpandOutcome{e}{4}
\ExpandOutcome{g}{3}
\ExpandOutcome{h}{4}
\ExpandOutcome{i}{3}
\ExpandOutcome{k}{5}
\ExpandOutcome{l}{3}
\end{outcomes}

\begin{unit}{\SESoftwareReliabilityDef}{Dean96,Baugh92,Lau03,Sitaraman00,Neufleder93,peled2001}{11}{4}
   \SESoftwareReliabilityAllTopics
   \SESoftwareReliabilityAllObjectives
\end{unit}

\begin{unit}{\SEFormalMethodsDef}{Dean96,Baugh92,Lau03,Sitaraman00,Neufleder93,peled2001}{12}{2}
   \SEFormalMethodsAllTopics
   \SEFormalMethodsAllObjectives
\end{unit}

\begin{unit}{\SEComponentBasedComputingDef}{Dean96,Baugh92,Lau03,Sitaraman00,Neufleder93,peled2001}{19}{4}
   \SEComponentBasedComputingAllTopics
   \SEComponentBasedComputingAllObjectives
\end{unit}



\begin{coursebibliography}
\bibfile{Computing/CS/CS392}
\end{coursebibliography}

\end{syllabus}
