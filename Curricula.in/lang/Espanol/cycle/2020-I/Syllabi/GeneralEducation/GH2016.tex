\begin{syllabus}

\course{GH0016. Liderazgo y Negociación}{Obligatorio}{GH0016}
% Source file: ../Curricula.in/lang/Espanol/cycle/2020-I/Syllabi/GeneralEducation/GH2016.tex

\begin{justification}
Este curso está diseñado para ayudar a los estudiantes a desarrollar habilidades relacionadas al liderazgo, asTecnología como también, introducirlos en los elementos esenciales de una negociación, de modo que ganen experiencia y confianza para desenvolverse en sus ámbitos laborales.
El curso busca entrenar a los estudiantes en las herramientas necesarias para formar profesionales Tecnologíantegros y resueltos, capaces de enfrentar retos a nivel individual y colectivo. El aprendizaje se da a través de la experiencia y el pensamiento crTecnologíatico. Para ello, se expondrán a los estudiantes a diferentes temas y actividades que permitan distintas formas pensar y de tomar decisiones. Siempre sobre la base de tres ejes: las reglas éticas, la
constante auto evaluación y la planificación de actividades.
\end{justification}

\begin{goals}
\item Capacidad de analizar información
\item Capacidad para identificar problemas
\item Capacidad para formular alternativas de solución
\item Comprende las responsabilidades profesional y ética
\item Capacidad de liderar un equipo
\end{goals}

\begin{outcomes}{V1}
    \item \ShowOutcome{d}{2} % Multidiscip teams
    \item \ShowOutcome{e}{2} % ethical, legal, security and social implications
    \item \ShowOutcome{f}{2} % communicate effectively
    \item \ShowOutcome{n}{2} % Apply knowledge of the humanities
    \item \ShowOutcome{o}{2} % TASDSH
\end{outcomes}

\begin{competences}{V1}
    \item \ShowCompetence{C10}{d,n,o}
    \item \ShowCompetence{C17}{f}
    \item \ShowCompetence{C18}{f}
    \item \ShowCompetence{C21}{e}
\end{competences}

\begin{unit}{Liderazgo y Negociación.}{}{Caravedo11,Robbins04}{12}{4}
   \begin{topics}
      \item Introducción al Liderazgo y Negociación
      \item Introducción al Liderazgo y autoevaluación de competencias
      \item Liderazgo: Influencia y Motivación
      \item Entrevistas por Competencias
      \item Comunicación LTecnologíader
      \item Aplicación vivencial de competencias lTecnologíaderes
      \item Liderazgo y Negociación
      \item Negociación Bipartidaria
      \item Negociación y Trabajo en equipo
      \item Negociación de Beneficios
      \item Negociación y Ética
      
   \end{topics}
   \begin{learningoutcomes}
      \item Desarrollo del potencial de liderazgo a través de estudios de casos, dinámicas y assessment center en clases con coaches especializados.
   \end{learningoutcomes}
\end{unit}

\begin{coursebibliography}
\bibfile{GeneralEducation/GH2016}
\end{coursebibliography}

\end{syllabus}
