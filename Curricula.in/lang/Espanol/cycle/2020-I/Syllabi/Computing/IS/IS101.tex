\begin{syllabus}

\course{IS1001. Introducción a la Gestión Empresarial}{Obligatorio}{IS1001}
% Source file: ../Curricula.in/lang/Espanol/cycle/2020-I/Syllabi/Computing/IS/IS101.tex

\begin{justification}
Proporciona una introducción a la teoría organizacional y el comportamiento de grupos e individuos dentro de las organizaciones, incluidos los 
procesos de trabajo en equipo / formación de equipos, y desarrollará estos conceptos con referencia particular a los fundamentos de la gestión dentro del contexto empresarial. Los estudios de caso desarrollarán una comprensión de la aplicación práctica de los conceptos teóricos.
\end{justification}

\begin{goals}
\item Proveer al alumno de los conceptos básicos de Tecnologías de la Información y software de aplicación.
\item Entender como la información es utilizada en las organizaciones y como la Tecnología de la Información permite aumento de calidad y ventajas competitivas.
\end{goals}

\begin{outcomes}{V1}
   \item \ShowOutcome{a}{1}
   \item \ShowOutcome{b}{1}
   \item \ShowOutcome{c}{1}
   \item \ShowOutcome{d}{1}
   \item \ShowOutcome{g}{1}
   \item \ShowOutcome{i}{1}
   \item \ShowOutcome{j}{1}
   \item \ShowOutcome{k}{1}
\end{outcomes}

\begin{unit}{\LUFIVEDef}{}{\LUFIVEBib}{10}{1}
   \begin{topics}
      \item \OMCTWOTopicTWOxTWOxONEOH
      \item \TDSONETopicTHREExONExONE
      \item \TDSONETopicTHREExONExTWO
      \item \TDSONETopicTHREExONExTHREE
      \item \TDSONETopicTHREExONExFOUR
      \item \TDSONETopicTHREExONExFIVE
   \end{topics}
	\LUFIVEGoal
\end{unit}

\begin{coursebibliography}
\bibfile{Computing/IS/IS}
\end{coursebibliography}

\end{syllabus}
