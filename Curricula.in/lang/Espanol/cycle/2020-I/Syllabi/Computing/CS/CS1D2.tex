\begin{syllabus}

\course{CS1D02. Estructuras Discretas II}{Obligatorio}{CS1D02}
% Source file: ../Curricula.in/lang/Espanol/cycle/2020-I/Syllabi/Computing/CS/CS1D2.tex

\begin{justification}
Para entender las técnicas computacionales avanzadas, los estudiantes deberán tener un fuerte conocimiento de las
diversas estructuras discretas, estructuras que serán implementadas y usadas en laboratorio en el lenguaje de programación.
\end{justification}

\begin{goals}
\item Que el alumno sea capaz de modelar problemas de ciencia de la computación usando grafos y árboles relacionados con estructuras de datos
\item Que el alumno aplique eficientemente estrategias de recorrido para poder buscar datos de una manera óptima
\end{goals}

\begin{outcomes}{V1}
    \item \ShowOutcome{a}{1}
    \item \ShowOutcome{b}{2}	
    \item \ShowOutcome{i}{1}
\end{outcomes}

\begin{outcomes}{V2}
    \item \ShowOutcome{1}{2}
    \item \ShowOutcome{2}{2}	
    \item \ShowOutcome{6}{2}
\end{outcomes}

\begin{competences}{V1}
    \item \ShowCompetence{C1}{a}
    \item \ShowCompetence{C5}{b}
    \item \ShowCompetence{CS2}{i}
\end{competences}

\begin{competences}{V2}
    \item \ShowCompetence{C1}{1}
    \item \ShowCompetence{C5}{2}
    \item \ShowCompetence{CS2}{6}
\end{competences}

\begin{unit}{Lógica Digital y Representación de Datos}{}{Rosen2007,Grimaldi03}{10}{C1,C20}
    \begin{topics}
         \item Retículo: Tipos y propiedades.
         \item Álgebras booleanas.
         \item Funciones y expresiones booleanas.
         \item Representación de las funciones booleanas: Disjuntiva normal y conjuntiva normal.
         \item Puertas Lógicas.
         \item Minimización del Circuito.
    \end{topics}
     \begin{learningoutcomes}
         \item Explicar la importancia del álgebra booleana como una unificación de la teoría de conjuntos y la lógica proposicional [\Assessment].
         \item Explicar las estructuras algebraicas del retículo y sus tipos [\Assessment].
         \item Explicar la relación entre el retículo y el conjunto de ordenadas y el uso prudente para demostrar que un conjunto es un retículo [\Assessment].
         \item Explicar las propiedades que satisfacen un álgebra booleana [\Assessment].
         \item Demostrar si una terna formada por un conjunto y dos operaciones internas es o no Álgebra booleana [\Assessment].
         \item Encuentra las formas canónicas de una función booleana  [\Assessment].
         \item Representar una función booleana como un circuito booleano usando puertas lógica[\Assessment].
         \item Minimizar una función booleana [\Assessment].
     \end{learningoutcomes}
 \end{unit}

\begin{unit}{\DSBasicsofCounting}{}{Grimaldi97}{40}{C1} 
    \begin{topics}
        \item \DSBasicsofCountingTopicCounting
	\item \DSBasicsofCountingTopicThePigeonhole
	\item \DSBasicsofCountingTopicPermutations
	\item \DSBasicsofCountingTopicSolving
	\item \DSBasicsofCountingTopicBasic
   \end{topics}
   \begin{learningoutcomes}
	\item \DSBasicsofCountingLOApplyCounting [\Familiarity]
	\item \DSBasicsofCountingLOApplyThe[\Familiarity]
	\item \DSBasicsofCountingLOComputePermutations[\Familiarity]
	\item \DSBasicsofCountingLOMap[\Familiarity]
	\item \DSBasicsofCountingLOSolveA[\Familiarity]
	\item \DSBasicsofCountingLOAnalyzeA[\Familiarity]
	\item \DSBasicsofCountingLOPerformComputations[\Familiarity]
   \end{learningoutcomes}
\end{unit}

\begin{unit}{\DSGraphsandTrees}{}{Johnsonbaugh99}{40}{C1}
    \begin{topics}
        \item \DSGraphsandTreesTopicTrees
	\item \DSGraphsandTreesTopicUndirected
	\item \DSGraphsandTreesTopicDirected
	\item \DSGraphsandTreesTopicWeighted
	\item \DSGraphsandTreesTopicSpanning
	\item \DSGraphsandTreesTopicGraph
   \end{topics}
   \begin{learningoutcomes}
	\item \DSGraphsandTreesLOIllustrate[\Familiarity]
	\item \DSGraphsandTreesLODemonstrateDifferent[\Familiarity]
	\item \DSGraphsandTreesLOModel[\Familiarity]
	\item \DSGraphsandTreesLOShowHow[\Familiarity]
	\item \DSGraphsandTreesLOExplainHowA [\Familiarity]
	\item \DSGraphsandTreesLODetermineIf [\Familiarity]
   \end{learningoutcomes}
\end{unit}

\begin{coursebibliography}
\bibfile{Computing/CS/CS1D1}
\end{coursebibliography}

\end{syllabus}
