\begin{syllabus}

\course{CS391. Ingeniería de Software III}{Obligatorio}{CS391}
% Source file: ../Curricula.in/lang/Espanol/cycle/2020-I/Syllabi/Computing/CS/CS391.tex

\begin{justification}
El desarrollo de software requiere del uso de mejores prácticas de desarrollo, gestión de proyectos de TI, manejo de equipos 
y uso eficiente y racional de frameworks de aseguramiento de la calidad, estos elemento son pieza clave y transversal durante 
todo el proceso productivo.

La construcción de software contempla la implementación y uso de procesos, métodos, modelos y herramientas que permitan lograr 
la realización de los atributos de calidad de un producto.
\end{justification}

\begin{goals}
  \item Comprender y poner en práctica los conceptos fundamentales sobre la gestión de proyectos y manejo de equipos de software.
  \item Comprender los fundamentos de la gestión de proyectos, incluyendo su definición, alcance, y la necesidad de gestión de proyectos en la organización moderna.
  \item Los alumnos deben comprender los conceptos fundamentales de CMMI, PSP, TSP para que sean adoptados en los proyectos de software.
  \item Describir y comprender los modelos de aseguramiento de la calidad como marco clave para el éxitos de los proyectos de TI.
\end{goals}

\begin{outcomes}{V1}
    \item \ShowOutcome{c}{2}
    \item \ShowOutcome{d}{2}
    \item \ShowOutcome{i}{2}
    \item \ShowOutcome{j}{3}
    \item \ShowOutcome{m}{3}
    \item \ShowOutcome{o}{2}
\end{outcomes}

\begin{outcomes}{V2}
    \item \ShowOutcome{1}{2}
    \item \ShowOutcome{5}{2}
    \item \ShowOutcome{6}{2}
\end{outcomes}

\begin{competences}{V1}
    \item \ShowCompetence{C7}{c} 
    \item \ShowCompetence{C11}{i,k}
    \item \ShowCompetence{C12}{j,m} 
    \item \ShowCompetence{C13}{c,i,m} 
    \item \ShowCompetence{C18}{d} 
    \item \ShowCompetence{C19}{j} 
    \item \ShowCompetence{CS6}{c,i,m} 
    \item \ShowCompetence{CS7}{d,i,o} 
    \item \ShowCompetence{CS9}{c,d,m}  
\end{competences}

\begin{competences}{V2}
    \item \ShowCompetence{C7}{1} 
    \item \ShowCompetence{C11}{1,6}
    \item \ShowCompetence{C12}{6} 
    \item \ShowCompetence{C13}{1,6} 
    \item \ShowCompetence{C18}{5} 
    \item \ShowCompetence{C19}{5} 
    \item \ShowCompetence{CS6}{6} 
    \item \ShowCompetence{CS7}{5,6} 
    \item \ShowCompetence{CS9}{6}  
\end{competences}

\begin{unit}{\SESoftwareEvolution}{}{Pressman2014,Sommerville2010}{12}{C7, C11, C12, CS6}
\begin{topics}
    \item \SESoftwareEvolutionTopicSoftware
    \item \SESoftwareEvolutionTopicSoftwareEvolution
    \item \SESoftwareEvolutionTopicCharacteristics
    \item \SESoftwareEvolutionTopicReengineering
    \item \SESoftwareEvolutionTopicSoftwareReuse
\end{topics}
\begin{learningoutcomes}
    \item \SESoftwareEvolutionLOIdentifyTheAssociatedEvolution [\Familiarity] %
    \item \SESoftwareEvolutionLOEstimateTheA [\Usage] %
    \item \SESoftwareEvolutionLOUseRefactoring [\Usage] %
    \item \SESoftwareEvolutionLODiscussTheEvolving [\Familiarity] %
    \item \SESoftwareEvolutionLOOutlineTheRegression [\Familiarity] %
    \item \SESoftwareEvolutionLODiscussTheDisadvantagesTypes [\Familiarity] %
\end{learningoutcomes}
\end{unit}

\begin{unit}{\SESoftwareProjectManagement}{}{Pressman2014,Sommerville2010}{10}{C18, C19, CS7, CS9}
\begin{topics}
	\item \SESoftwareProjectManagementTopicTeam
	\item \SESoftwareProjectManagementTopicEffort
	\item \SESoftwareProjectManagementTopicRisk
	\item \SESoftwareProjectManagementTopicTeamManagement
	\item \SESoftwareProjectManagementTopicProject
\end{topics}
\begin{learningoutcomes}%Usage Familiarity Assessment
	\item \SESoftwareProjectManagementLODiscussCommon [\Familiarity] %
	\item \SESoftwareProjectManagementLOCreateAndAgenda [\Usage] %
	\item \SESoftwareProjectManagementLOIdentifyAndRoles [\Usage] %
	\item \SESoftwareProjectManagementLOUnderstandTheAnd [\Usage] %
	\item \SESoftwareProjectManagementLOApplyAStrategy [\Usage] %
	\item \SESoftwareProjectManagementLOUseAn [\Usage] %
	\item \SESoftwareProjectManagementLOListSeveral [\Familiarity] %
	\item \SESoftwareProjectManagementLODescribeTheRiskSoftware [\Familiarity] %
	\item \SESoftwareProjectManagementLODescribeDifferent [\Familiarity] %
	\item \SESoftwareProjectManagementLODemonstrateThrough [\Usage] %
	\item \SESoftwareProjectManagementLODescribeHowOfAffects [\Familiarity] %
	\item \SESoftwareProjectManagementLOCreateAIdentifying [\Usage] %
	\item \SESoftwareProjectManagementLOAssessAnd [\Usage] %
	\item \SESoftwareProjectManagementLOUsing [\Familiarity] %
\end{learningoutcomes}
\end{unit}

\begin{unit}{\SESoftwareProjectManagement}{}{Pressman2014,Sommerville2010}{8}{C18, C19, CS7, CS9}
\begin{topics}
	\item \SESoftwareProjectManagementTopicSoftwareMeasurement
	\item \SESoftwareProjectManagementTopicSoftwareQuality
	\item \SESoftwareProjectManagementTopicRiskS
	\item \SESoftwareProjectManagementTopicSystem
\end{topics}
\begin{learningoutcomes}%Usage Familiarity Assessment
	\item \SESoftwareProjectManagementLOTrack [\Usage] %
	\item \SESoftwareProjectManagementLOCompareSimple [\Usage] %
	\item \SESoftwareProjectManagementLOUseATool [\Usage] %
	\item \SESoftwareProjectManagementLODescribeTheRiskThe [\Assessment] %
	\item \SESoftwareProjectManagementLOIdentifyRisks [\Familiarity] %
	\item \SESoftwareProjectManagementLOExplainHowDecisions [\Usage] %
	\item \SESoftwareProjectManagementLOIdentifySecurity [\Usage] %
	\item \SESoftwareProjectManagementLODemonstrateA [\Usage] %
	\item \SESoftwareProjectManagementLOApplyTheOf [\Usage] %
	\item \SESoftwareProjectManagementLOConductAAnalysis [\Usage] %
	\item \SESoftwareProjectManagementLOIdentifyAndOfFor [\Usage] %
\end{learningoutcomes}
\end{unit}

\begin{unit}{\SESoftwareProcesses}{}{Pressman2014,Sommerville2010}{12}{C7, C13, C19, CS6, CS7}
\begin{topics}
    \item \SESoftwareProcessesTopicSystem
    \item \SESoftwareProcessesTopicIntroduction
    \item \SESoftwareProcessesTopicProgramming
    \item \SESoftwareProcessesTopicEvaluation
    \item \SESoftwareProcessesTopicSoftware
    \item \SESoftwareProcessesTopicProcess
    \item \SESoftwareProcessesTopicSoftwareProcess
    \item \SESoftwareProcessesTopicSoftwareProcessMeasurements
\end{topics}
\begin{learningoutcomes}
    \item \SESoftwareProcessesLODescribeHowInteract [\Usage] 
    \item \SESoftwareProcessesLODescribeTheAndSeveral [\Usage] 
    \item \SESoftwareProcessesLODescribeTheThat [\Usage] 
    \item \SESoftwareProcessesLODifferentiateAmong [\Usage] 
    \item \SESoftwareProcessesLODescribeHowThe [\Usage] 
    \item \SESoftwareProcessesLOExplainTheAAndExample [\Usage] 
    \item \SESoftwareProcessesLOCompareSeveral [\Usage] 
    \item \SESoftwareProcessesLODefineSoftware [\Usage] 
    \item \SESoftwareProcessesLODescribeTheFundamental [\Usage] 
    \item \SESoftwareProcessesLOCompareSeveralModels [\Usage] 
    \item \SESoftwareProcessesLOAssess [\Usage] 
    \item \SESoftwareProcessesLOExplainTheProcess [\Usage] 
    \item \SESoftwareProcessesLODescribeSeveralForControlling [\Usage] 
    \item \SESoftwareProcessesLOUseProject  [\Usage] 
\end{learningoutcomes}
\end{unit}

\begin{unit}{Estándares ISO/IEC}{}{Sommerville2010,Pressman2014}{6}{C7, C13, C19, CS6, CS7}
\begin{topics}
    \item ISO 9001:2001.
    \item ISO 9000-3.
    \item ISO/IEC 9126.
    \item ISO/IEC 12207.
    \item ISO/IEC 15939.
    \item ISO/IEC 14598.
    \item ISO/IEC 15504-SPICE.
    \item IT Mark.
    \item SCRUM.
    \item SQuaRE.
    \item CISQ.
\end{topics}
\begin{learningoutcomes}%Usage Familiarity Assessment
    \item Aprender y aplciar correctamente normas y estandares internacionales. [\Usage]
\end{learningoutcomes}
\end{unit}

\begin{coursebibliography}
\bibfile{Computing/CS/CS392}
\end{coursebibliography}

\end{syllabus}
