\begin{syllabus}

\course{FG101. Comunicación}{Obligatorio}{FG101}

\begin{justification}
Justificación ...
% Para lograr una eficaz comunicación en el ámbito personal y profesional, es prioritario el manejo adecuado de la Lengua en forma oral y escrita. Se justifica, por lo tanto, que los alumnos de la Universidad Católica San Pablo conozcan, comprendan y apliquen los aspectos conceptuales y operativos de su idioma, para el desarrollo de sus habilidades comunicativas fundamentales: Escuchar, hablar, leer y escribir.
% En consecuencia el ejercicio permanente y el aporte de los fundamentos contribuyen grandemente en la formación académica y, en el futuro, en el desempeño de su profesión
\end{justification}

\begin{goals}
\item Goal 1
% \item Desarrollar capacidades comunicativas a través de la teorTecnologíaa y práctica del lenguaje que ayuden al estudiante a superar las exigencias académicas del pregrado y contribuyan a su formación humanTecnologíastica y como persona humana.
\end{goals}

\begin{outcomes}
   \item \ShowOutcome{f}{2}
   \item \ShowOutcome{h}{2}
   \item \ShowOutcome{n}{2}
\end{outcomes}

\begin{competences}
    \item \ShowCompetence{C17}{f,h,n}
    \item \ShowCompetence{C20}{f,n}
    \item \ShowCompetence{C24}{f,h}
\end{competences}

\begin{unit}{}{Primera Unidad}{Real}{16}{C17,C20}
\begin{topics}
      \item text 1
      \item text 2
      \item ...
\end{topics}

\begin{learningoutcomes}
   \item LO 1 [\Usage].
   \item LO 2 [\Usage].
   \item LO 3 [\Usage].
\end{learningoutcomes}
\end{unit}

\begin{coursebibliography}
\bibfile{GeneralEducation/FG}
\end{coursebibliography}

\end{syllabus}
