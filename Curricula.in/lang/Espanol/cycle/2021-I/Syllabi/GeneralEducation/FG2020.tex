\begin{syllabus}

\course{FG2020. Derecho y Bioseguridad }{Electivo}{FG2020}
% Source file: ../Curricula.in/lang/Espanol/cycle/2021-I/Syllabi/GeneralEducation/FG2020.tex

\begin{justification}
Expone a los estudiantes el concepto de bioseguridad y su relación con el uso seguro de los recursos biológicos y genéticos en el laboratorio. Así como, su implicancia con las leyes y normas que regulan este tipo de procesos, por lo que se presentarán casos para analizar y profundizar en el tema.
\end{justification}

\begin{goals}
\item Capacidad de interpretar información.
\end{goals}

\begin{outcomes}{V1}
    \item \ShowOutcome{d}{2}
    \item \ShowOutcome{e}{2}
    \item \ShowOutcome{n}{2}
    
\end{outcomes}

\begin{competences}{V1}
    \item \ShowCompetence{C10}{d,n}
    \item \ShowCompetence{C17}{d}
    \item \ShowCompetence{C18}{n}
    \item \ShowCompetence{C21}{e}
\end{competences}

\begin{unit}{Culturas de Gobernanza y Distribución de Poder}{}{Lessig15}{12}{4}
   \begin{topics}
      \item ?`Cómo se relaciona la economía con la política?.
      \item El rol de las Instituciones.
      \item Análisis de casos.
   \end{topics}
   \begin{learningoutcomes}
      \item Desarrollo del innterés por conocer sobre temas actuales en la sociedad peruana y el mundo.
   \end{learningoutcomes}
\end{unit}

\begin{coursebibliography}
\bibfile{GeneralEducation/FG2020}
\end{coursebibliography}

\end{syllabus}
