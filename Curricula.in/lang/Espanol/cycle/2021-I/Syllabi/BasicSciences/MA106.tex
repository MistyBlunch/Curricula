\begin{syllabus}

\course{EG0001. Métodos Numéricos}{Obligatorio}{EG0001}
% Source file: ../Curricula.in/lang/Espanol/cycle/2021-I/Syllabi/BasicSciences/MA106.tex

\begin{justification}
Este curso introduce los primeros conceptos del álgebra lineal, asTecnología como los métodos numéricos con un énfasis en la resolución de problemas
con el paquete de sotfware libe de código abierto Scilab.
La teorTecnologíaa matemática se limita a los fundamentos, mientras que la aplicación efectiva para la resolución de problemas es privilegiada. 
En cada tópico, se enseña unos cuantos métodos de de relevancia para la ingeniería. 
Los conocimientos sobre estos métodos prepara a los estudiantes para la búsqueda de alternativas más avanzadas, si se lo requiere.
\end{justification}

\begin{goals}
\item Capacidad para aplicar los conocimientos sobre Matemáticas.
\item Capacidad para aplicar los conocimientos sobre Ingeniería .
\item Capacidad para aplicar los conocimientos, técnicas, habilidades y herramientas modernas de la ingeniería moderna para la práctica de la ingenieria.
\end{goals}

\begin{outcomes}{V1}
    \item \ShowOutcome{a}{3}
    \item \ShowOutcome{j}{3}
\end{outcomes}

\begin{competences}{V1}
    \item \ShowCompetence{C1}{a} 
    \item \ShowCompetence{C20}{j} 
    \item \ShowCompetence{C24}{j} 
\end{competences}

\begin{unit}{Introducción}{}{Anton,Chapra}{18}{C1}
  \begin{topics}
      \item Importancia del álgebra lineal y métodos numéricos. Ejemplos.
   \end{topics}

   \begin{learningoutcomes}
      \item Ser capaz de entender los conceptos básicos y la importancia de Álgebra Lineal y Métodos Numéricos.
   \end{learningoutcomes}
\end{unit}

\begin{unit}{Álgebra lineal}{}{Anton,Chapra}{14}{C1}
   \begin{topics}
    \item Álgebra matricial elemental y determinantes.
    \item Espacio nulo y soluciones exactas de sistemas de ecuaciones lineales Ax = b:
	  \begin{subtopics}
	    \item Sistemas tridiagonal y triangular y eliminación gaussiana con y sin giro.
	    \item Factorización LU y algoritmo Crout.
	  \end{subtopics}
    \item Conceptos básicos sobre valores propios y vectores propios
	  \begin{subtopics}
	    \item Polinomios caracterTecnologíasticos.
	    \item Multiplicaciones algebraicas y geométricas.
	  \end{subtopics}
    \item Estimación de mTecnologíanimos cuadrados.
    \item Transformaciones lineales.
    \end{topics}

   \begin{learningoutcomes}
      \item Comprender los conceptos básicos del Álgebra Lineal.
      \item Resolver problemas de transformaciones lineales.
   \end{learningoutcomes}
\end{unit}

\begin{unit}{Métodos Numéricos}{}{Anton,Chapra}{22}{C24}
   \begin{topics}
    \item Fundamentos de soluciones de sistemas de ecuaciones lineales Ax = b: métodos de Jacobi y Gauss Seidel
    \item Aplicación de factorizaciones de matriz a la solución de sistemas lineales (descomposición de valores singulares, QR, Cholesky) Cálculo numérico del espacio nulo, rango y número de condición
    \item Conclusión de la raTecnologíaz:
	  \begin{subtopics}
	    \item Bisección.
	    \item Iteración de punto fijo.
	    \item Métodos de Newton-Raphson.
	  \end{subtopics}
    \item Fundamentos de la interpolación:
	  \begin{subtopics}
	    \item Interpolaciones polinomiales de Newton y Lagrange.
	    \item Interpolación de spline.
	    \end{subtopics}
    \item Fundamentos de la diferenciación numérica y la aproximación de Taylor.
    \item Aspectos básicos de la integración numérica:
	  \begin{subtopics}
	    \item Trapecio, punto medio y regla de Simpson 
	    \item Cuadratura gaussiana
	  \end{subtopics}
    \item Conceptos básicos sobre las soluciones numéricas a las EDOs:
	  \begin{subtopics}
	    \item Diferencias finitas; Métodos de Euler y Runge-Kutta
	    \item Convertir ODEs de orden superior en un sistema de ODEs de bajo orden.
	    \item Métodos de Runge-Kutta para sistemas de ecuaciones
	    \item Método simple.XYZ
	  \end{subtopics}
    \item Breve introducción a las técnicas de optimización: visión general sobre la programación lineal, sistemas lineales acotados, programación cuadrática, descenso gradiente.
    \end{topics}

   \begin{learningoutcomes}
      \item Comprender los conceptos básicos de los métodos numéricos.
      \item Aplicar los métodos más frecuentes para la resolución de problemas matemáticos.
      \item Implementación y aplicación de algoritmos numéricos para la solución de problemas matemáticos utilizando el paquete computacional Scilab open-source.
      \item Aplicación de Scilab para la solución de problemas matemáticos y para trazar  graficas.
      \end{learningoutcomes}
\end{unit}

\begin{coursebibliography}
\bibfile{BasicSciences/MA102}
\end{coursebibliography}

\end{syllabus}
