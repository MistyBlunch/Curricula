\begin{syllabus}

\course{IS3108. Implicaciones de la Digitalización}{Obligatorio}{IS3108}
% Source file: ../Curricula.in/lang/Espanol/cycle/2020-I/Syllabi/Computing/IS/IS318.tex

\begin{justification}
	Entender las implicancias de la digitalización de datos, información y comunicaciones en las organizaciones y en la sociedad es de extrema importancia. Estas implicancias deben ser examindas considerando asuntos éticos tales como la privacidad, accesibilidad, propiedad y exactitud de la información. La proliferación del crimen informático así­ como los ambientes legales y regulatorios tienen que ser examinados. Así­ mismo, las ramificaciones de la digitalización en la form en que afectan a individuos, organizaciones y sociedad deben ser discutidas. Finalmente, un examen de los impactos de la globalización, fuerza de trabajo tecnológica y división digital es necesario.
	\end{justification}
	
	\begin{goals}
	\item Ganar un entendimiento profundo de la influencia de incrementar la digitalización en organizaciones y sociedad.
	\item Entender como la digitalización de la información y la proliferación de redes globales crean nuevas relaciones entre organizaciones, nuevas amenazas y nuevas formas de trabajar.
	\item Examinar las carácteristicas de la era de la información y explorar las implicaciones de asuntos éticos emergentes tales como la privacidad, propuedad, exactitud y accesibilidad de la información.
	\item Examinar lo que constituye un ambiente digital seguro.
	\end{goals}
	
	\begin{outcomes}{V1}
		\item \ShowOutcome{b}{1}
		\item \ShowOutcome{e}{1}
		\item \ShowOutcome{g}{1}
		\item \ShowOutcome{h}{1}
		\item \ShowOutcome{i}{1}
	\end{outcomes}
	
	\begin{unit}{\LUTWOSIXDef}{}{\LUTWOSIXBib}{35}{1}
	   \begin{topics}
		   \item \OMCEIGHTTopicTWOxEIGHTxONEOH
		\begin{subtopics}
			\item \OMCEIGHTTopicTWOxEIGHTxONEOHxONE
			\item \OMCEIGHTTopicTWOxEIGHTxONEOHxTWO
			\item \OMCEIGHTTopicTWOxEIGHTxONEOHxTHREE
			\item \OMCEIGHTTopicTWOxEIGHTxONEOHxFOUR
			\item \OMCEIGHTTopicTWOxEIGHTxONEOHxFIVE
			\item \OMCEIGHTTopicTWOxEIGHTxONEOHxSIX
			\item \OMCEIGHTTopicTWOxEIGHTxONEOHxSEVEN
			\item \OMCEIGHTTopicTWOxEIGHTxONEOHxEIGHT
			\item \OMCEIGHTTopicTWOxEIGHTxONEOHxNINE
		\end{subtopics}
	   \end{topics}
		\LUTWOSIXGoal
	\end{unit}
	
	\begin{unit}{\LUTWOSEVENDef}{}{\LUTWOSEVENBib}{40}{1}
	   \begin{topics}
		   \item \OMCTWOTopicTWOxTWOxONESIX
			\begin{subtopics}
				\item \OMCTWOTopicTWOxTWOxONESIXxONE
				\item \OMCTWOTopicTWOxTWOxONESIXxTWO
				\item \OMCTWOTopicTWOxTWOxONESIXxTHREE
				\item \OMCTWOTopicTWOxTWOxONESIXxFOUR
				\item \OMCTWOTopicTWOxTWOxONESIXxFIVE
				\item \OMCTWOTopicTWOxTWOxONESIXxSIX
				\item \OMCTWOTopicTWOxTWOxONESIXxSEVEN
				\item \OMCTWOTopicTWOxTWOxONESIXxEIGHT
			\end{subtopics}
	   \end{topics}
		\LUTWOSEVENGoal
	\end{unit}
	
	\begin{coursebibliography}
	\bibfile{Computing/IS/IS}
	\end{coursebibliography}
	
	\end{syllabus}
	