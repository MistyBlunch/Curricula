\begin{syllabus}

\course{IS319. Interacción Humano Computador}{Obligatorio}{IS319}
% Source file: ../Curricula.in/lang/Espanol/cycle/2020-I/Syllabi/Computing/CS/CS250W.tex

\begin{justification}
	El lenguaje ha sido una de las creaciones más significativas de la humanidad. Desde el lenguaje corporal y gestual, pasando por la comunicación verbal y escrita, hasta códigos simbólicos icónicos y otros, ha posibilitado interacciones complejas entre los seres humanos y facilitado considerablemente la comunicación de información. Con la invención de dispositivos automáticos y semiautomáticos, entre los que se cuentan las computadoras, la necesidad de "lenguajes" o "interfaces" para poder interactuar con ellos, ha cobrado gran importancia.
	La usabilidad del software, aunada a la satisfacción del usuario y su incremento de productividad, depende de la eficacia de la Interfaz Usuario-Computador. Tanto es así­, que a menudo la interfaz es el factor más importante en el éxito o el fracaso de cualquier sistema computacional.
	El diseño e implementación de adecuadas Interfaces Humano-Computador, que además de cumplir los requisitos técnicos y la lógica transaccional de la aplicación, considere las sutiles implicaciones psicológicas, culturales y estéticas de los usuarios, consume buena parte del ciclo de vida de un proyecto software, y requiere habilidades especializadas, tanto para la construcción de las mismas, como para la realización de pruebas de usabilidad.
	\end{justification}
	
	\begin{goals}
	\item Conocer y aplicar criterios de usabilidad y accesibilidad al diseño y construcción de interfaces humano-computador, buscando siempre que la tecnología se adapte a las personas y no las personas a la tecnología.
	\end{goals}
	
	\begin{outcomes}{V1}
		\item \ShowOutcome{b}{3}
		\item \ShowOutcome{c}{3}
		\item \ShowOutcome{d}{3}
		\item \ShowOutcome{e}{4}
		\item \ShowOutcome{g}{3}
		\item \ShowOutcome{TASDSH}{4}
	\end{outcomes}
	
	\begin{unit}{\HCIFoundations}{}{Smith2006HCI, Baecker2000HCI}{6}{3}
		\HCIFoundationsAllTopics
		\HCIFoundationsAllLearningOutcomes
	\end{unit}
	
	\begin{unit}{\PLObjectOrientedProgramming}{}{Pressman2007HCI}{1}{3}
		\PLObjectOrientedProgramming
	\end{unit}
	
	%GLA Unidad eliminada a pedido del docente
	%\begin{unit}{\PFEventDrivenProgrammingDef}{}{Wirfs2002HCI}{1}{3}
	%    \PFEventDrivenProgrammingAllTopics
	%    \PFEventDrivenProgrammingAllObjectives
	%\end{unit}
	
	\begin{unit}{\HCIUsercentereddesignandtesting}{}{Smith2006HCI, Sharp2009HCI, Constantine2004HCI, Baecker2000HCI}{5}{4}
		\HCIUsercentereddesignandtestingAllTopics
		\HCIUsercentereddesignandtestingAllLearningOutcomes
	\end{unit}
	
	
	\begin{unit}{\HCIDesigningInteraction}{}{Baecker2000HCI, Apple2009HCI, Sharp2009HCI, Wirfs2002HCI}{4}{3}
		\HCIDesigningInteractionAllTopics
		\HCIDesigningInteractionAllLearningOutcomes
	\end{unit}
	
	\begin{unit}{\ARInterfacingandcommunication}{}{Baecker2000HCI, Apple2009HCI, Constantine2002HCI, Loranger2002HCI}{6}{4}
		\ARInterfacingandcommunicationAllTopics
		\ARInterfacingandcommunicationAllLearningOutcomes
	\end{unit}
	
	
	\begin{unit}{\IMMultimediaSystems}{}{Smith2006HCI, Baecker2000HCI}{4}{3}
		\IMMultimediaSystemsAllTopics
		\IMMultimediaSystemsAllLearningOutcomes
	\end{unit}
	
	\begin{unit}{\SPProfessionalCommunication}{}{Baecker2000HCI}{4}{3}
		\SPProfessionalCommunicationAllTopics
		\SPProfessionalCommunicationAllLearningOutcomes
	\end{unit}
	
	
	\begin{unit}{\HCIHumanfactorsandsecurity}{}{Baecker2000HCI}{4}{3}
		\HCIHumanfactorsandsecurityAllTopics
		\HCIHumanfactorsandsecurityAllLearningOutcomes
	\end{unit}
	
	\begin{coursebibliography}
	\bibfile{Computing/CS/CS250W}
	\end{coursebibliography}
	\end{syllabus}
	