\begin{syllabus}

\course{CB1005. Estructuras Discretas I}{Obligatorio}{CB1005}
% Source file: ../Curricula.in/lang/Espanol/cycle/2020-I/Syllabi/Computing/CS/CS105.tex

\begin{justification}
   Las estructuras discretas son fundamentales para la ciencia de la
   computación. Es evidente que las estructuras discretas son usadas en
   las áreas de estructura de datos y algoritmos , sin embargo son
   también importantes en otras, como por ejemplo en la
   verificación, en criptografía y métodos formales.
   \end{justification}
   
   \begin{goals}
   \item Desarrollar Operaciones asociadas con conjuntos, funciones y relaciones.
   \item Relacionar ejemplos prácticos al modelo apropiado de conjunto, función o relación.
   \item Conocer las diferentes técnicas de conteo más utilizadas.
   \item Describir como las herramientas formales de lógica simbólica son utilizadas.
   \item Describir la importancia y limitaciones de la lógica de predicados.
   \item Bosquejar la estructura básica y dar ejemplos de cada tipo de prueba descrita en esta unidad.
   \item Relacionar las ideas de inducción matemática con la recursividad y con estructuras definidas recursivamente.
   \item Enunciar, identificar y habituarse a los conceptos más importantes de Conjuntos Parcialmente Ordenados y Látices
   \item Analizar, comentar y aceptar las nociones básicas de Álgebras Booleanas.
   \end{goals}
   
   \begin{outcomes}{V1}
      \item \ShowOutcome{a}{3}
      \item \ShowOutcome{i}{3}
      \item \ShowOutcome{j}{3}
   \end{outcomes}
   
   \begin{unit}{\DSSetsRelationsandFunctions}{}{Kolman97,Grassmann97,Johnsonbaugh99}{13}{4}
       \DSSetsRelationsandFunctionsTopicSets
       \DSSetsRelationsandFunctionsAllLearningOutcomes
   \end{unit}
   
   \begin{unit}{\DSBasicLogic}{}{Grassmann97,Iranzo05,Paniagua03,Johnsonbaugh99}{14}{4}
       \DSBasicLogicAllTopics
       \DSBasicLogicAllLearningOutcomes
   \end{unit}
   
   \begin{unit}{\DSProofTechniques}{}{Scheinerman01,Brassard97,Kolman97,Johnsonbaugh99}{14}{4}
      \DSProofTechniquesAllTopics
      \DSProofTechniquesAllLearningOutcomes
   \end{unit}
   
   \begin{unit}{\ARDigitallogicanddigitalsystems}{}{Kolman97, Grimaldi97, Gersting87}{19}{3}
      \ARDigitallogicanddigitalsystemsAllTopics
      \ARDigitallogicanddigitalsystemsAllLearningOutcomes
   \end{unit}
   
   \begin{coursebibliography}
   \bibfile{Computing/CS/CS105}
   \end{coursebibliography}
   
   \end{syllabus}
   
   %\end{document}
   