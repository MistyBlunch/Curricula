\begin{syllabus}

\course{CB211. Investigaci�n Operativa II}{Obligatorio}{CB211}

\begin{justification}
Este curso es importante en la medida que proporciona modelos de optimizaci�n �tiles para la toma de decisiones en negocios.
\end{justification}

\begin{goals}
\item Reconocer, modelar, resolver, implementar e interpretar modelos de optimizaci�n no lineales y estoc�sticos en problemas reales.
\end{goals}

\begin{outcomes}
\ExpandOutcome{a}{1}
\ExpandOutcome{b}{1}
\ExpandOutcome{c}{1}
\ExpandOutcome{g}{1}
\ExpandOutcome{j}{1}
\end{outcomes}

\begin{unit}{Programaci�n no lineal}{Taha2004,Hillier2006}{12}{1}
   \begin{topics}
      \item Funciones convexas y c�ncavas.
      \item Soluciones de PNL con una variable.
      \item Maximizaci�n y minimizaci�n no restringida con varias variables.
      \item M�todos de ascenso no escalonado.
      \item Condiciones de Kuhn-Tucker.
   \end{topics}

   \begin{unitgoals}
      \item resolver problemas de optimizaci�n no lineal.
   \end{unitgoals}
\end{unit}

\begin{unit}{Toma de decisiones bajo incertidumbre}{Wayne2005,Hillier2002IO}{12}{1}
   \begin{topics}
      \item Teor�a de la utilidad.
      \item Arboles de decisi�n.
      \item Toma de decisi�n con objetivos m�ltiples.
   \end{topics}

   \begin{unitgoals}
      \item Presentar los fundamentos de la teor�a de decisiones en condiciones de incertidumbre.
   \end{unitgoals}
\end{unit}

\begin{unit}{Teor�a de juegos}{Taha2004,Hillier2006}{12}{1}
   \begin{topics}
      \item Juegos de suma cero.
      \item Juegos de suma no constante para dos personas.
      \item Introducci�n a la teor�a de juegos para n personas.
   \end{topics}

   \begin{unitgoals}
      \item Resolver problemas de toma de decisiones donde dos o m�s personas que deciden se enfretan a un conflicto de intereses.
   \end{unitgoals}
\end{unit}

\begin{unit}{Cadenas de Markov}{Wayne2005,Hillier2002IO}{12}{1}
   \begin{topics}
      \item Procesos estoc�ticos.
      \item Cadenas de Markov.
      \item Clasificaci�n de los estados de una cadena de Markov.
	  \item Propiedades a largo de las cadenas de Markov.
   \end{topics}

   \begin{unitgoals}
      \item Dar elementos de toma de decisiones para variables aleatorias que varian el tiempo.
   \end{unitgoals}
\end{unit}

\begin{unit}{Teor�a de colas}{Taha2004,Hillier2006}{12}{1}
   \begin{topics}
      \item Estructura b�sicas de los modelos de colas.
      \item Ejemplos de sistemas de colas reales.
      \item Proceso de nacimiento y muerte.
      \item Modelos de colas basados en el proceso de nacimiento y muerte.
      \item Modelos de colas con distribuciones no exponenciales.
      \item Modelos de colas con disciplina de prioridades.
   \end{topics}

   \begin{unitgoals}
      \item Responder preguntas relacionadas a modelos de espera.
   \end{unitgoals}
\end{unit}

\begin{unit}{Simulaci�n}{Taha2004,Hillier2006}{10}{1}
   \begin{topics}
      \item Tipos comunes de aplicaciones.
      \item Generaci�n de n�meros aleatorios.
      \item Estudio de casos.
      \item Simulaci�n con hojas de c�lculos.
      \item T�cnicas de reducci�n de varianza.
      \item M�todo regenerativo de an�lisis estad�stico.
   \end{topics}

   \begin{unitgoals}
      \item Poder hacer un an�lisis de riesgo ante situaciones de alta complejidad.
   \end{unitgoals}
\end{unit}



\begin{coursebibliography}
\bibfile{BasicSciences/CB210}
\end{coursebibliography}

\end{syllabus}
