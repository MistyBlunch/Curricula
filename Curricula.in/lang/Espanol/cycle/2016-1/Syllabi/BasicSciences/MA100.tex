\begin{syllabus}

\course{MA100. Matem�tica I}{Obligatorio}{MA100}

\begin{justification}
Un aspecto muy importante en el nivel universitario lo constituye el c�lculo diferencial,  aspecto que constituye la piedra angular de las posteriores asignaturas de matem�ticas as� como de la utilidad de la matem�tica en la soluci�n de problemas aplicados a la ciencia y la tecnolog�a. Cualquier profesional con rango universitario debe por lo tanto tener conocimiento amplio de esta asignatura, pues se convertir� en su punto de partida para los intereses de su desarrollo profesional; as� tambi�n ser� soporte para no tener dificultades en las asignaturas de matem�tica y f�sica de toda la carrera.
\end{justification}

\begin{goals}
\item Asimilar y manejar los conceptos de funci�n, sucesi�n y relacionarlos con los de l�mites y continuidad.
\item Describir, analizar, dise�ar y formular modelos continuos que dependan de una variable.
\item Conocer y manejar la propiedades del c�lculo diferencial y aplicarlas a la resoluci�n de problemas.
\end{goals}

\begin{outcomes}
    \item \ShowOutcome{a}{3}
    \item \ShowOutcome{j}{3}
\end{outcomes}

\begin{competences}
    \item \ShowCompetence{C1}{a}
    \item \ShowCompetence{C20}{j}
    \item \ShowCompetence{C24}{j}
\end{competences}

\begin{unit}{}{N�meros reales y funciones}{Leithold2000,Stewart,ThomasFinney}{20}{C1}
   \begin{topics}
      \item N�meros reales
      \item Funciones de variable real
   \end{topics}

   \begin{learningoutcomes}
      \item Comprender la importancia del sistema de los n�meros reales (construcci�n), manipular los axiomas algebraicos y de orden [\Assessment].
      \item Comprender el concepto de funci�n. Manejar dominios, operaciones, gr�ficas, inversas [\Assessment].
      \end{learningoutcomes}
\end{unit}

\begin{unit}{}{Sucesiones num�ricas de n�meros reales}{Leithold2000,ThomasFinney}{10}{C20}
   \begin{topics}
      \item Sucesiones
      \item Covergencia
      \item L�mites. Operaciones con sucesiones
   \end{topics}

   \begin{learningoutcomes}
      \item Entender el concepto de sucesi�n y su importancia [\Assessment].
      \item Conecer los principales tipos de sucesiones, manejar sus propiedades [\Assessment].
      \item Manejar y calcular l�mites de sucesiones [\Assessment].
      \end{learningoutcomes}
\end{unit}

\begin{unit}{}{L�mites de funciones y continuidad}{Leithold2000,Leithold2000,Stewart}{20}{C1}
   \begin{topics}
      \item L�mites
      \item Continuidad
      \item Aplicaciones de funciones continuas. Teorema del valor intermedio
   \end{topics}

   \begin{learningoutcomes}
      \item Comprender el concepto de l�mite. calcular l�mites [\Assessment].
      \item Analizar la continuidad de una funci�n [\Assessment].
      \item Aplicar el teorema del valor intermedio [\Assessment].
      \end{learningoutcomes}
\end{unit}

\begin{unit}{}{Diferenciaci�n}{Leithold2000,ThomasFinney,Stewart}{22}{C20}
   \begin{topics}
      \item Definici�n. reglas de derivaci�n
      \item Incrementos y diferenciales
      \item Regla de la cadena. Derivaci�n impl�cita
   \end{topics}

   \begin{learningoutcomes}
      \item Comprender el concepto de derivada e interpretarlo [\Assessment].
      \item Manipular las reglas de derivaci�n [\Assessment].
      \end{learningoutcomes}
\end{unit}

\begin{unit}{}{Aplicaciones}{Stewart,Leithold2000}{20}{C24}
   \begin{topics}
      \item Funciones crecientes, decrecientes
      \item Extremos de funciones
      \item Raz�n de cambio
      \item L�mites infinitos
      \item Teorema de Taylor
   \end{topics}

   \begin{learningoutcomes}
      \item Utilizar la derivada para hallar extremos de funciones [\Assessment].
      \item Resolver problemas aplicativos [\Assessment].
      \item Utilizar el Teorema de Taylor [\Assessment].
   \end{learningoutcomes}
\end{unit}



\begin{coursebibliography}
\bibfile{BasicSciences/MA100}
\end{coursebibliography}

\end{syllabus}

%\end{document}
