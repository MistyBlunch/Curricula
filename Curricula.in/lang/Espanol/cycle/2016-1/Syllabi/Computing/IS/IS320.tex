\begin{syllabus}

\course{IS320. Seguridad de la Información}{Obligatorio}{IS320}

\begin{justification}
Actualmente, cada ente que cuenta con un servicio informático,necesita estar salvaguardado para asegurar la continuidad de las operaciones o tareas que se realiza, un ejemplo de esto es la tecnología inalámbrica que alcanza a las comunicaciones móviles, y los propios satélites, yendo un poco mas cerca ahora  a las organizaciones modernas que necesitan dotar sus sistemas e infraestructuras informáticas de políticas y medidas de protección adecuadas, que garanticen el continuo desarrollo y sostenibilidad de sus actividades. Además, es imprescindible disponer de un adecuado sistema de protección en los sistemas informáticos que asegure desde la privacidad de los datos hasta la seguridad en las transacciones de información, pasando por el control de acceso, los protocolos de comunicación, las transferencias de datos, etc.

Estudiar y planificar estos procesos es el objetivo principal de este curso, con el que se pretende conseguir convertir al estudiante en un auténtico experto en seguridad en la gestión de la misma, haciendo frente a una de las profesiones más demandadas y competitivas del mercado laboral actual.

La presente asignatura, recoge los últimos estudios realizados sobre la actividad de seguridad de la información a nivel mundial, ha sido diseñada de acuerdo a la temática basada en conceptos fundamentales, metodologías y diversos estándares nacionales e internacionales que ayudan a salvaguardar el servicio informático;  con la finalidad de establecer un marco de referencia, en el proceso formativo del estudiante de  Sistemas de Información, para el desarrollo y mejor desempeño en el ejercicio de sus funciones como oficial de seguridad en un primer momento de la información llegando al final a ser un gestor de la misma, surtiendo y generando así nuevo conocimiento. Este curso permitirá una sólida formación que le prepare para enfrentarse a la realidad, utilizando de manera efectiva sus procesos cognitivos y estrategias de solución para mejorar en la salvaguarda del servicio informático en una determinada entidad, desempeñando un rol en el desarrollo de la sociedad.

Debido a que el uso de diferentes tecnologías como el internet va en aumento, las empresas permiten a sus socios y proveedores acceder a sus sistemas de información. Por lo tanto, es fundamental saber qué recursos de la compañía necesitan protección para así controlar el acceso al sistema y los derechos de los usuarios a los diferentes Sistemas de información.  Por ello la seguridad informática, continúa evolucionando. Las universidades e instituciones educativas deben entender las necesidades de la comunidad profesional para proporcionar al mercado graduados que posean las destrezas requeridas y el conocimiento que los profesionales necesitan. El modelo académico que propone  ISACA en esta área, proporciona a las universidades un marco conceptual básico sobre de la educación requerida para desarrollar las destrezas necesarias en la profesión.
\end{justification}

\begin{goals}
\item Acercar al alumno a conceptos y técnicas usados en la Seguridad de Sistemas de Información.
\item Asimila y crea conciencia de la necesidad de salvaguardar los entes informáticos dentro de la organización y fuera de la misma.
\item Conoce las principales certificaciones existentes, las diferentes metodologías, los estándares que están relacionados a la seguridad informática y sus requisitos junto con las exigencias para convertirse en un administrador de la seguridad de información internacional.
\end{goals}

\begin{outcomes}
\ExpandOutcome{a}{1}
\ExpandOutcome{b}{1}
\ExpandOutcome{c}{1}
\ExpandOutcome{d}{1}
\ExpandOutcome{g}{1}
\ExpandOutcome{i}{1}
\ExpandOutcome{j}{1}
\ExpandOutcome{k}{1}
\end{outcomes}

\begin{unit}{Gerencia de la Seguridad de los Activos de Información}{ISACA2008,ITAuditing2007,Cascarino2007,ITG2005}{2}{1}
\begin{topics}
\item Tecnología informatica y conceptos básicos de seguridad
\item Conceptos de seguridad de TI 
\item Necesidad de asegurar recursos de TI
\item Política para seguridad de activos de TI 
\item Gerencia en la seguridad de activos de TI 
\item Entrenamiento
\end{topics}

\begin{unitgoals}
\item Cómocer cómo gerenciar la seguridad de los activos de información.
\end{unitgoals}
\end{unit}

\begin{unit}{Seguridad Lógica de Tecnología Informatica}{ISACA2008,ITAuditing2007,Cascarino2007,ITG2005}{2}{1}
\begin{topics}
\item Componentes de la seguridad lógica de TI; problemas en la lógica del control de acceso
\item Software de control de acceso
\item Riesgos lógicos de seguridad, consideraciones de control y auditoría (auditoría de acceso lógico, prueba de seguridad)
\item Características, herramientas y procedimientos lógicos de la seguridad
\end{topics}

\begin{unitgoals}
\item Revisar conceptos de seguridad lógica de TI
\end{unitgoals}
\end{unit}

\begin{unit}{Seguridad Aplicada de TI: Recursos de Alta Tecnología}{ISACA2008,ITAuditing2007,Cascarino2007,ITG2005}{2}{1}
\begin{topics}
\item Comunicaciones y seguridad de la red: principios de la seguridad de la red, cliente y servidor, del Internet y de servicios tele-basados, de sistemas de la seguridad del cortafuego y de otros recursos de la protección de la conectividad (criptografía, firmas digitales, políticas de gerencia dominantes), sistemas de detección de intrusosintrusión, Cobita, revisiones de sistema
\item Instalaciones de seguridad de la unidad central
\item Uso de la base de datos y seguridad básicos del sistema
\item Seguridad en el proceso del desarrollo y del mantenimiento del sistema
\end{topics}
\begin{unitgoals}
\item Revisar conceptos de seguridad aplicada de TI.
\end{unitgoals}
\end{unit}

\begin{unit}{Seguridad Física y Ambiental}{ISACA2008,ITAuditing2007,Cascarino2007,ITG2005}{2}{1}
\begin{topics}
\item Problemas y exposiciones ambientales: conceptos en la seguridad física de TI
\item Exposiciones y controles físicos del acceso
\end{topics}
\begin{unitgoals}
\item Revisar conceptos de seguridad física y ambiental.
\end{unitgoals}
\end{unit}

\begin{unit}{Protección de la Arquitectura y Activos de la TI}{ISACA2008,ITAuditing2007,Cascarino2007,ITG2005}{2}{1}
\begin{topics}
\item Apoyo y compromiso de la gerencia con el proceso
\item Preparación y documentación del plan
\item Aprobación de la gerencia y distribución del plan
\item Prueba, mantenimiento y revisión del plan; entrenamiento
\item Rol del auditor
\item Provisiones de los respaldos
\item Planificación de la continuidad del negocio
\item Análisis del impacto del negocio
\end{topics}
\begin{unitgoals}
\item Revisar conceptos de arquitectura y activos de TI.
\end{unitgoals}
\end{unit}

\begin{unit}{Seguros}{ISACA2008,ITAuditing2007,Cascarino2007,ITG2005}{2}{1}
\begin{topics}
\item Descripción de los seguros
\item Artículos que pueden ser asegurados
\item Tipos de cobertura de seguro
\item Valoración de activos: equipo, gente, proceso de la información y tecnología
\end{topics}
\begin{unitgoals}
\item Conocer seguros y artículos asegurables.
\end{unitgoals}
\end{unit}

\begin{unit}{Desarrollo, Adquisición y Mantenimiento de SI}{ISACA2008,ITAuditing2007,Cascarino2007,ITG2005}{2}{1}
\begin{topics}
\item Gerencia de proyecto de los sistemas de información: planificación, organización, despliegue del recurso humano, control del proyecto, supervisión y ejecución
\item Métodos tradicionales para el desarrollo del ciclo de vida del sistema; analizar, evaluar y diseñar las fases del desarrollo del ciclo de vida de un sistema (SDLC)
\item Acercamientos para el desarrollo del sistema: paquetes de software, prototipo, reingeniería de proceso del negocio, herramientas CASE.
\item Mantenimiento de sistemas y procedimientos para el control de cambios para modificaciones de sistemas
\item Problemas de riesgo y control, analizar y evaluar características y riesgos del proyecto
\end{topics}
\begin{unitgoals}
\item Revisar conceptos de desarrollo, adquisición y mantenimiento de SI.
\end{unitgoals}
\end{unit}

\begin{unit}{Infraestructura Técnica}{ISACA2008,ITAuditing2007,Cascarino2007,ITG2005}{6}{1}
\begin{topics}
\item Arquitectura y estándares de TI
\item Hardware: todo el equipo de TI incluyendo la unidad central, las mini computadoras, clientes/servidores, los enrutadores, los interruptores, las comunicaciones, las PC, etc.
\item Software: sistemas operacionales, programas de utilidades, bases de datos, etc.
\item Red: el equipo y los servicios de comunicaciones dedicados para proporcionar las redes, red relacionada al hardware, red relacionada al software, el uso de los proveedores que proporcionan servicios de comunicación, etc.
\item Controles fundamentales
\item Seguridad / pruebas y validación
\item Herramientas de evaluación y supervisión de desempeño
\item Gobierno de TI. Mantenimiento y Funcionamiento
\item Supervisión de controles de TI y herramientas de evaluación, como vigilancia de sistemas de control de acceso o vigilancia de incursión con sistemas de detección
\item Gerencia de recursos de información e infraestructura: software de gerencia de empresas
\item Gerencia de centros de servicio y estándares/guías de las operaciones: COBIT, ITIL, ISO 17799
\item Asuntos y consideraciones de centro de servicio vs. infraestructuras técnicas propietarias 
\item Sistemas abiertos
\item Gerencia de cambio/Implementación de nuevos sistemas: organización de las herramientas usadas para controlar la introducción de productos nuevos al ambiente del centro de servicio, etc.
\end{topics}
\begin{unitgoals}
\item Conocer la infraestructura técnica relevante para una auditoría.
\end{unitgoals}
\end{unit}

\begin{unit}{Gerencia de Centros de Servicio}{ISACA2008,ITAuditing2007,Cascarino2007,ITG2005}{4}{1}
\begin{topics}
\item Gerencia de Seguridad
\item Gerencia de Recurso/configuración: cumplimiento con organización/TI estándares operacionales, políticas y procedimientos (uso correcto del lenguaje de computadoras)
\item Gerencia de problemas e incidentes 
\item Planificación y estimación de capacidad
\item Gerencia de la distribución de sistemas automatizados
\item Administración del lanzamiento y versiones de sistemas automatizados
\item Gerencia de proveedores
\item Enlaces con clientes
\item Administración del nivel de servicios
\item Contingencia/ Respaldos y administración de la recuperación
\item Gerencia del centro de llamadas
\item Gerencia de las operaciones de la infraestructura (central y distribuida)
\item Administración de redes
\item Gerencia de riesgo
\item Principios claves de gerencia
\end{topics}
\begin{unitgoals}
\item Conocer cómo se debe gerenciar un centro de servicios.
\end{unitgoals}
\end{unit}

\begin{unit}{Herramientas de Apoyo para la seguridad de información}{ISACA2008,ITAuditing2007,Cascarino2007,ITG2005}{2}{1}
\begin{topics}
\item COBIT. Pautas gerenciales para gerentes de SI/TI
\item COBIT. uso de auditorías como apoyo para el ciclo del negocio
\item Estándares Internacionales - ISO-I7799, Estándares de Privacidad, COCO, COSO, Cadbury, King, ITIL
\item Revisiones de control de cambios
\end{topics}

\begin{unitgoals}
\item Estudiar herramientas de apoyo para la auditoría.
\end{unitgoals}
\end{unit}



\begin{coursebibliography}
\bibfile{Computing/IS/IS}
\end{coursebibliography}

\end{syllabus}
