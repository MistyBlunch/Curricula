\begin{syllabus}

\course{CS356. Programaci�n de Video Juegos}{Electivo}{CS356}

\begin{justification}
La industria de los video juegos ha tenido uncrecimiento exponencial en las �ltimas dos d�cadas y puede ser aplicada a 
diversas �reas del conocimiento humano.

El potencial que ofrece esta �rea para un egresado es muy amplio y como tal se considera como un �rea cr�tica para el 
desarrollo de la industria del software.
\end{justification}

\begin{goals}
\item Que el alumno conozca las t�cnicas fundamentales que permiten la creaci�n de video juegos.
\item Que el alumno construya videos juegos de complejidad media incorporando conceptos de Inteligencia Artificial.
\end{goals}

\begin{outcomes}
\ExpandOutcome{a}{4}
\ExpandOutcome{b}{4}
\ExpandOutcome{i}{4}
\ExpandOutcome{j}{4}
\end{outcomes}

\begin{unit}{\GVAdvancedTechniquesDef}{Foley90,HearnAndBaker94}{8}{3}
        \GVAdvancedTechniquesAllTopics
        \GVAdvancedTechniquesAllObjectives
\end{unit}

\begin{unit}{\GVVisualizationDef}{Foley90,HearnAndBaker94}{4}{3}
        \GVVisualizationAllTopics
        \GVVisualizationAllObjectives
\end{unit}

\begin{unit}{\HCFoundationsDef}{Baecker2000HCI}{4}{4}
    \HCFoundationsAllTopics
    \HCFoundationsAllObjectives
\end{unit}

\begin{unit}{\GVAdvancedRenderingDef}{Foley90,HearnAndBaker94}{10}{3}
        \GVAdvancedRenderingAllTopics
        \GVAdvancedRenderingAllObjectives
\end{unit}

\begin{unit}{\GVGameEngineProgrammingDef}{DS4GameProgrammers,C4programmers2006,DSAndAlgortihms4Games}{26}{3}
    \GVGameEngineProgrammingAllTopics
    \GVGameEngineProgrammingAllObjectives
\end{unit}



\begin{coursebibliography}
\bibfile{Computing/CS/CS255}
\end{coursebibliography}

\end{syllabus}
