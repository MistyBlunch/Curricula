\begin{syllabus}

\course{CS2B1. Desarrollo Basado en Plataformas}{Obligatorio}{CS2B1}

\begin{justification}
El mundo ha cambiado debido al uso de la web y tecnologías relacionadas, el acceso rápido, oportuno y personalizado de la 
información, a través de la tecnología web, ubícuo  y pervasiva; han cambiado la forma de ?`cómo hacemos las cosas?, ?`cómo pensamos? y ?`cómo la industria se desarrolla?.

Las tecnologías web, ubicuo  y pervasivo se basan en el desarrollo de servicios web, aplicaciones web y aplicaciones móviles, 
las cuales son necesarias entender la arquitectura, el diseño, y la implementación de servicios web, aplicaciones web y aplicaciones móviles.
\end{justification}

\begin{goals}
    \item Que el alumno sea capaz de diseño e implementación de servicios, aplicaciones web utilizando herramientas y lenguajes como HTML, CSS, 
    JavaScript (incluyendo AJAX) , back-end scripting y una base de datos, a un nivel intermedio.
    \item Que el alumno sea capaz de desarrollar aplicaciones móviles, administración de servidores web en un sistema Unix y una introducción a la seguridad web, a un nivel intermedio.
\end{goals}


\begin{outcomes}
    \item \ShowOutcome{c}{2}
    \item \ShowOutcome{d}{2}
    \item \ShowOutcome{g}{2}
    \item \ShowOutcome{i}{2}
    \item \ShowOutcome{o}{2}
\end{outcomes}

\begin{competences}
    \item \ShowCompetence{C1}{c,d,i}
    \item \ShowCompetence{C6}{c,d,i}
    \item \ShowCompetence{CS8}{g,o}
\end{competences}


\begin{unit}{\PBDIntroduction}{}{grove2009web,annuzzi2013introduction}{5}{CS8}
\begin{topics}%
    \item \PBDIntroductionTopicOverview
    \item \PBDIntroductionTopicProgramming
    \item \PBDIntroductionTopicOverviewOf
    \item \PBDIntroductionTopicProgrammingUnder
\end{topics}
\begin{learningoutcomes}
    \item \PBDIntroductionLODescribeHowDevelopment [\Familiarity]
    \item \PBDIntroductionLOListCharacteristics [\Familiarity]
    \item \PBDIntroductionLOWriteAnd [\Familiarity]
    \item \PBDIntroductionLOListTheDisadvantages [\Familiarity]
\end{learningoutcomes}
\end{unit}



\begin{unit}{\PBDWebPlatforms}{}{grove2009web}{5}{C1,C6}
\begin{topics}%
    \item \PBDWebPlatformsTopicWeb
    \item \PBDWebPlatformsTopicWebPlatform
    \item \PBDWebPlatformsTopicSoftware
    \item \PBDWebPlatformsTopicWebStandards
\end{topics}
\begin{learningoutcomes}
    \item \PBDWebPlatformsLODesignAndSimple [\Familiarity]
    \item \PBDWebPlatformsLODescribeTheTheOn [\Familiarity]
    \item \PBDWebPlatformsLOCompareAndProgramming [\Familiarity]
    \item \PBDWebPlatformsLODescribeTheSoftware [\Familiarity]
    \item \PBDWebPlatformsLODiscussHowImpact [\Familiarity]
    \item \PBDWebPlatformsLOReview [\Familiarity]
\end{learningoutcomes}
\end{unit}


\begin{unit}{Desarrollo de servicios y aplicaciones web}{}{grove2009web}{25}{C1,C6}
   \begin{topics}
    \item Describir, identificar y depurar problemas relacionados con el desarrollo de aplicaciones web
    \item Diseño y desarrollo de aplicaciones web interactivas usando este tipo de incrustar scripts en lenguaje python
    \item Utilice MySQL para la gestión de datos y manipular MySQL con python
    \item Diseño y desarrollo de aplicaciones web asíncronos utilizando técnicas Ajax
    \item Uso del lado del cliente dinámico lenguaje de script Javascript y del lado del servidor lenguaje de scripting python con Ajax
    \item Aplicar las tecnologías XML / JSON para la gestión de datos con Ajax
    \item Utilice marco, los servicios y APIs web Ajax y aplicar los patrones de diseño para el desarrollo de aplicaciones web
   \end{topics}
   \begin{learningoutcomes}
      \item Del lado del servidor lenguaje de scripting python: variables, tipos de datos, operaciones, cadenas, 
            funciones, sentencias de control, matrices, archivos y el acceso a directorios, mantener el estado. [\Usage]
      \item Enfoque de programación web usando python incrustado. [\Usage]
      \item El acceso y la manipulación de MySQL. [\Usage]
      \item El enfoque de desarrollo de aplicaciones web Ajax. [\Usage]
      \item DOM y CSS utilizan en JavaScript. [\Usage]
      \item Tecnologías de actualización de contenido asíncrono. [\Usage]
      \item Objetos XMLHttpRequest utilizar para comunicarse entre clientes y servidores. [\Usage]
      \item XML y JSON. [\Usage]
      \item XSLT y XPath como mecanismos para transformar documentos XML. [\Usage]
      \item Servicios web y APIs (especialmente Google Maps). [\Usage]
      \item Marcos Ajax para el desarrollo de aplicaciones web contemporánea. [\Usage]
      \item Los patrones de diseño utilizados en aplicaciones web. [\Usage]
   \end{learningoutcomes}
\end{unit}

\begin{unit}{\PBDMobilePlatforms}{}{annuzzi2013introduction}{5}{C1,C6}
\begin{topics}%
    \item \PBDMobilePlatformsTopicMobile
    \item \PBDMobilePlatformsTopicChallenges
    \item \PBDMobilePlatformsTopicLocation
    \item \PBDMobilePlatformsTopicPerformance
    \item \PBDMobilePlatformsTopicMobilePlatform
    \item \PBDMobilePlatformsTopicEmerging
\end{topics}
\begin{learningoutcomes}
    \item \PBDMobilePlatformsLODesignAndMobile [\Familiarity]
    \item \PBDMobilePlatformsLODiscussTheMobile [\Familiarity]
    \item \PBDMobilePlatformsLODiscussThePower [\Familiarity]
    \item \PBDMobilePlatformsLOCompareAndProgrammingPurpose [\Familiarity]
\end{learningoutcomes}
\end{unit}

\begin{unit}{}{Mobile Applications for Android Handheld Systems}{annuzzi2013introduction}{25}{C1,C6}
\begin{topics}
    \item The Android Platform
    \item The Android Development Environment
    \item Application Fundamentals
    \item The Activity Class
    \item The Intent Class
    \item Permissions
    \item The Fragment Class
    \item User Interface Classes
    \item User Notifications
    \item The BroadcastReceiver Class
    \item Threads, AsyncTask \& Handlers
    \item Alarms
    \item Networking (http class)
    \item Multi-touch \& Gestures
    \item Sensors
    \item Location \& Maps
\end{topics}

\begin{learningoutcomes}
    \item Los estudiantes identifican software necesario y lo instalan en sus ordenadores personales. 
          Los estudiantes realizan varias tareas para familiarizarse con la plataforma Android y Ambiente para el Desarrollo. [\Usage]
    \item Los estudiantes construyen aplicaciones que trazan los métodos de devolución de llamada de ciclo de 
          vida emitidas por la plataforma Android y que demuestran el comportamiento de Android cuando los cambios de configuración de 
          dispositivos (por ejemplo, cuando el dispositivo se mueve de vertical a horizontal y viceversa). [\Usage]
    \item Los estudiantes construyen aplicaciones que requieren iniciar múltiples actividades a través de ambos métodos estándar y personalizados. [\Usage]
    \item Los estudiantes construyen aplicaciones que requieren permisos estándar y personalizados. [\Usage]
    \item Los estudiantes construyen una aplicación que utiliza una única base de código, sino que crea diferentes interfaces de 
          usuario dependiendo del tamaño de la pantalla de un dispositivo. [\Usage]
    \item Los estudiantes construyen un gestor de listas de tareas pendientes utilizando los elementos de la interfaz de 
          usuario discutidos en clase. La aplicación permite a los usuarios crear nuevos elementos y para mostrarlos en un ListView. [\Usage]
    \item Los estudiantes construyen una aplicación que utiliza la información de ubicación para recoger latitud, longitud de los lugares que visitan. [\Usage]
\end{learningoutcomes}
\end{unit}




\begin{coursebibliography}
\bibfile{Computing/CS/CS2B1}
\end{coursebibliography}

\end{syllabus}
