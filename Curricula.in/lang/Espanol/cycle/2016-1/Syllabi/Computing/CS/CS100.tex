\begin{syllabus}

\course{CS100. Introducción de Ciencia de la Computación}{Obligatorio}{CS100}

\begin{justification}
La Ciencia de la Computación es un campo de estudio enorme con muchas especialidades y aplicaciones. Este curso brindará a sus participantes, una visión panorámica de la informática y mostrará sus campos más representativos, como son: Algoritmos, Estructuras de de Datos, Sistemas Operativos, Bases de Datos, etc.
\end{justification}

\begin{goals}
\item Brindar un panorama del área del conocimiento que es cubierta en la ciencia de la computación.
\end{goals}

\begin{outcomes}
    \item \ShowOutcome{a}{1}
    \item \ShowOutcome{b}{1}
    \item \ShowOutcome{e}{1}
    \item \ShowOutcome{g}{1}
    \item \ShowOutcome{h}{1}
\end{outcomes}

\begin{competences} 
    \item \ShowCompetence{C24}{h} 
    \item \ShowCompetence{C10}{g} 
    \item \ShowCompetence{C2}{b}
    \item \ShowCompetence{CS4}{a}
\end{competences}

\begin{unit}{}{Introducción}{brookshear}{2}{C24}
    \begin{topics}
	\item Introducción a la computación.
	\item Historia de la computación.
   \end{topics}
   \begin{learningoutcomes}
      \item  Incentivar a los alumnos el estudio de Computacion como una ciencia. [\Familiarity]
   \end{learningoutcomes}
\end{unit}

\begin{unit}{\DSBasicLogic}{}{brookshear}{2}{C24}
   \begin{topics}
      \item \DSBasicLogicTopicPropositional
      \item \DSBasicLogicTopicLogical
      \item \DSBasicLogicTopicTruth
      \item \DSBasicLogicTopicNormal
   \end{topics}
   \begin{learningoutcomes}
      \item \DSBasicLogicLOConvertLogical[\Familiarity]
      \item \DSBasicLogicLOApplyFormal [\Familiarity]
   \end{learningoutcomes}
\end{unit}



\begin{coursebibliography}
\bibfile{Computing/CS/CS100}
\end{coursebibliography}
\end{syllabus}
%\end{document}
