\begin{syllabus}

\course{CS101F. Introducci�n a la Programaci�n}{Obligatorio}{CS101F}

\begin{justification}
Este es el primer curso en la secuencia de los cursos introductorios a la inform�tica. En este curso se pretende cubrir los conceptos se�alados por la \textit{Computing Curricula} IEEE-CS/ACM 2008, bajo el enfoque \textit{functional-first}.

La programaci�n es uno de los pilares de la inform�tica; cualquier profesional del �rea, necesitar� programar para concretizar sus modelos y propuestas.

Este curso introducir� a los participantes en los conceptos fundamentales de este arte. Lo t�picos incluyen tipos de datos, estructuras de control, funciones, listas, recursividad y la mec�nica de la ejecuci�n, prueba y depuraci�n.

El curso tambi�n ofrecer� una introducci�n al contexto hist�rico y social de la inform�tica y una revisi�n del �mbito de esta disciplina.
\end{justification}

\begin{goals}
\item Introducir los conceptos fundamentales de programaci�n y estructuras de datos utilizando un lenguaje funcional.
\item Desarrollar su capacidad de abstracci�n, utilizar un lenguaje de programaci�n funcional.
\end{goals}

\begin{outcomes}
\ExpandOutcome{a}{3}
\ExpandOutcome{b}{3}
\ExpandOutcome{c}{3}
\ExpandOutcome{i}{3}
\ExpandOutcome{k}{2}
\end{outcomes}

\begin{unit}{\SPHistoryOfComputingDef}{brookshear,thompson2011,guttag2013,zelle2010}{4}{2}
    \SPHistoryOfComputingAllTopics
    \SPHistoryOfComputingAllObjectives
\end{unit}

\begin{unit}{\PLOverviewDef}{thompson2011,guttag2013,zelle2010}{1}{2}
   \begin{topics}
      \item \PLOverviewTopicHistory
      \item Paradigmas de programaci�n.
   \end{topics}

   \begin{unitgoals}
      \item \PLOverviewObjONE
      \item \PLOverviewObjTWO
   \end{unitgoals}
\end{unit}

\begin{unit}{\PLDeclarationsAndTypesDef}{thompson2011,guttag2013,zelle2010}{1}{3}
    \begin{topics}%
	\item \PLDeclarationsAndTypesTopicThe%
	\item \PLDeclarationsAndTypesTopicOverview%
    \end{topics}%
    \PLDeclarationsAndTypesAllObjectives
\end{unit}

\begin{unit}{\PFFundamentalConstructsDef}{thompson2011,guttag2013,zelle2010}{2}{3}
  \PFFundamentalConstructsAllTopics
  \PFFundamentalConstructsAllObjectives
\end{unit}

\begin{unit}{\PLFunctionalProgrammingDef}{thompson2011,guttag2013,zelle2010}{1}{4}
    \begin{topics}%
	\item \PLFunctionalProgrammingTopicOverview%
	\item \PLFunctionalProgrammingTopicRecursion%
	\item \PLFunctionalProgrammingTopicPragmatics%
    \end{topics}%
   \PLFunctionalProgrammingAllObjectives
\end{unit}

\begin{unit}{\PFRecursionDef}{thompson2011,guttag2013,zelle2010}{6}{4}
    \begin{topics}%
	\item \PFRecursionTopicTheconcept%
	\item \PFRecursionTopicRecursive%
	\item \PFRecursionTopicSimple%
	\item \PFRecursionTopicDiveAndConquer%
    \end{topics}%

    \begin{unitgoals}%
	\item \PFRecursionObjONE%
	\item \PFRecursionObjTWO%
	\item \PFRecursionObjTHREE%
	\item \PFRecursionObjFOUR%
	\item \PFRecursionObjFIVE%
	\item \PFRecursionObjSIX%
	\item \PFRecursionObjEIGHT%
    \end{unitgoals}%
\end{unit}

\begin{unit}{\ALFundamentalAlgorithmsDef}{thompson2011,guttag2013,zelle2010}{4}{4}
    \begin{topics}%
	\item \ALFundamentalAlgorithmsTopicSimple%
	\item \ALFundamentalAlgorithmsTopicSequential%
	\item \ALFundamentalAlgorithmsTopicQuadratic%
	\item \ALFundamentalAlgorithmsTopicBinary%
	\item \ALFundamentalAlgorithmsTopicDepth%
    \end{topics}%
    \ALFundamentalAlgorithmsAllObjectives
\end{unit}

\begin{unit}{\PLAbstractionMechanismsDef}{thompson2011,guttag2013,zelle2010}{4}{3}
   \begin{topics}
      \item \PLAbstractionMechanismsTopicProcedures%
      \item \PLAbstractionMechanismsTopicParameterization%
      \item \PLAbstractionMechanismsTopicType%
      \item \PLAbstractionMechanismsTopicModules%
   \end{topics}
   \begin{unitgoals}
      \item \PLAbstractionMechanismsObjONE%
      \item \PLAbstractionMechanismsObjTWO%
      \item \PLAbstractionMechanismsObjTHREE%
   \end{unitgoals}
\end{unit}

\begin{unit}{\PFAlgorithmsAndProblemSolvingDef}{thompson2011,guttag2013,zelle2010}{10}{4}
    \PFAlgorithmsAndProblemSolvingAllTopics
    \PFAlgorithmsAndProblemSolvingAllObjectives
\end{unit}

\begin{unit}{\PLVirtualMachinesDef}{thompson2011,guttag2013,zelle2010}{1}{2}
   \begin{topics}
      \item \PLVirtualMachinesTopicETheconcept
   \end{topics}
   \begin{unitgoals}
      \item \PLVirtualMachinesObjONE
   \end{unitgoals}
\end{unit}

\begin{unit}{\PLObjectOrientedProgrammingDef}{thompson2011,guttag2013,zelle2010}{4}{3}
    \begin{topics}%
	\item \PLObjectOrientedProgrammingTopicClasses%
	\item \PLObjectOrientedProgrammingTopicPolymorphism%
	\item \PLObjectOrientedProgrammingTopicClasshierarchies%
    \end{topics}%

    \begin{unitgoals}%
	\item \PLObjectOrientedProgrammingObjFOUR%
	\item \PLObjectOrientedProgrammingObjFIVE%
    \end{unitgoals}%
\end{unit}

\begin{unit}{\SEUsingAPIsDef}{thompson2011,guttag2013,zelle2010}{2}{3}
   \begin{topics}
      \item \SEUsingAPIsTopicProgramming
   \end{topics}

   \begin{unitgoals}
      \item \SEUsingAPIsObjONE
   \end{unitgoals}
\end{unit}



\begin{coursebibliography}
\bibfile{Computing/CS/CS101F}
\end{coursebibliography}

\end{syllabus}
