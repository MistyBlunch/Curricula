\begin{syllabus}

\course{FG204A. Teología II}{Obligatorio}{FG204A}

\begin{justification}
La Universidad Católica San Pablo busca ofrecer una visión de la persona humana y del mundo iluminada por el Evangelio y, consiguientemente, por la fe en Cristo-Logos, como centro de la creación y de la historia. El estudio de la teología es fundamental para dicha comprensión de Dios, del hombre y del cosmos.
La Teología permite al creyente en Cristo conocer y comprender mejor su fe. Al no creyente, la comprensión de la cosmovisión que ha forjado la cultura occidental en la cual ha nacido, vive y desarrollará su propia vida, así como abrirse al conocimiento de Dios desde Jesucristo y su Iglesia.
El curso de Teología II le permitirá al alumno  adentrarse en la comprensión de los contenidos y consecuencias del Dogma cristiano.
\end{justification}

\begin{goals}
\item Conocer y comprender el Cristianismo en cuanto religión revelada desde las razones en las que se apoya, mostrando su credibilidad, a fin de ofrecer al creyente razones que motivan su opción de fe y presentar a quien no lo es razones para creer. [\Usage]
\end{goals}

\begin{outcomes}
    \item \ShowOutcome{n}{1}
    \item \ShowOutcome{ñ}{3}
    \item \ShowOutcome{o}{3}
\end{outcomes}

\begin{competences}
    \item \ShowCompetence{C17}{n,ñ}
    \item \ShowCompetence{C20}{n,ñ,o}
    \item \ShowCompetence{C21}{n,ñ}
    \item \ShowCompetence{C22}{n,ñ}
    \item \ShowCompetence{C24}{ñ}
\end{competences}

\begin{unit}{}{Dios en sí}{pablo1998creo,pozo1968,ratzinger2005,Ibanez2005}{9}{C20,C21,C22,C24}
\begin{topics}
	\item Dios Uno y Trino.
	    \begin{subtopics}
		\item Dios plenitud del Ser.
		\item Los atributos divinos.
		\item Dios es plenitud del Amor.
		\item Dios Padre, Hijo y Espíritu Santo, Comunión de Amor.
	    \end{subtopics}
\end{topics}
\begin{learningoutcomes}
	\item Que el  alumno reflexione sobre el problema de Dios para la humanidad y los atributos del Ser Divino. [\Familiarity]
\end{learningoutcomes}
\end{unit}

\begin{unit}{}{Dios creador}{pablo1998creo,Catecismo,guardini2006}{6}{C20,C21,C22,C24}
\begin{topics}
	\item La pregunta sobre la creación.
	      \begin{subtopics}
		\item La revelación de la creación como obra de Dios.
		\item Creación obra de la Trinidad.
		\item Características.
		\item Errores acerca de la creación.
		\item El hombre Señor de la Creación.
		\item La Providencia.
	      \end{subtopics}
\end{topics}
\begin{learningoutcomes}
	\item Que el  alumno reflexione sobre Dios Creador como fundamento de toda la realidad.[\Familiarity]
\end{learningoutcomes}
\end{unit}

\begin{unit}{}{El pecado y la reconciliación en Cristo}{Ratzinger2007,Catecismo,DelaPotterie1997,ratzinger2005}{12}{C20,C21,C22,C24}
\begin{topics}
	\item El Problema del Mal.
	    \begin{subtopics}
		\item El Demonio.
		\item El Pecado Original Originante.
		\item El Pecado Original Originado.
	    \end{subtopics}
	\item Cristo Reconciliador
	      \begin{subtopics}
		\item Términos Soteriológicos.
		\item Ciclo reconciliador.
		\item Reconciliación: Nueva Creación.
	      \end{subtopics}
\end{topics}
\begin{learningoutcomes}
	\item Que el alumno reflexione sobre el contenido de la fe en Jesucristo que parte de lo que Él dijo de sí mismo y es recogido por sus primeros testigos. [\Familiarity]
\end{learningoutcomes}
\end{unit}

\begin{unit}{}{La gracia y las realidades últimas}{Collantes2011,Catecismo,Aquino1943}{18}{C20,C21,C22,C24}
\begin{topics}
	\item La Gracia .
	      \begin{subtopics}
		\item Santidad.
		\item Naturaleza de la Gracia.
		\item La acción de la Gracia.
		\item Los sacramentos.
		\item La oración.
		\item La liturgia.
	      \end{subtopics}
	\item María en el misterio de Cristo y de la Iglesia.
	      \begin{subtopics}
		\item Misión de la Santísima Virgen en la economía de la salvación.
		\item La Santísima Virgen y la Iglesia.
	      \end{subtopics}
	\item La Vida eterna.
	      \begin{subtopics}
		\item Escatología intermedia.
		\item La Resurrección de la Carne.
		\item El infierno.
		\item La Vida Eterna.
	      \end{subtopics}
\end{topics}
\begin{learningoutcomes}
	\item Que el alumno reflexione sobre la gracia y las realidades últimas de la fe. [\Usage]
\end{learningoutcomes}
\end{unit}



\begin{coursebibliography}
\bibfile{GeneralEducation/FG204}
\end{coursebibliography}
\end{syllabus}
