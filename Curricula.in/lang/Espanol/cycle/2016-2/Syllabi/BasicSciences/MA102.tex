\begin{syllabus}

\course{MA102. Cálculo I}{Obligatorio}{MA102}

\begin{justification}
Estudia la integral de funciones en una variable, series numéricas y de funciones así como una introducción a las ecuaciones diferenciales, base para los siguientes cursos de Análisis Matemático y Física.
\end{justification}

\begin{goals}
\item Comprender el concepto de integral, calcular integrales y aplicar la integral a la resolución de problemas.
\item Manejar, manipular las sucesiones y series. Determinar la convergencia de una serie numérica y de funciones.
\item Comprender el concepto de ecuación diferencial, resolver ecuaciones y aplicarlas (como modelos) a la resolución de problemas.
\end{goals}

\begin{outcomes}
    \item \ShowOutcome{a}{3}
    \item \ShowOutcome{j}{3}
\end{outcomes}

\begin{competences}
    \item \ShowCompetence{C1}{a} 
    \item \ShowCompetence{C20}{j} 
    \item \ShowCompetence{C24}{j} 
\end{competences}

\begin{unit}{}{Integración}{Apostol97,Simmons95}{18}{C1}
   \begin{topics}
      \item Integral definida
      \item Integral indefinida
   \end{topics}

   \begin{learningoutcomes}
      \item Comprender el proceso de deducción de la integral definida y su relación con el cocepto de área [\Usage].
      \item Calcular integrales definidas [\Usage].
      \item Asimilar el Teorema fundamental del cálculo. Manejar los métodos de integración [\Usage].
      \item Aplicar la integral a problemas [\Usage].
   \end{learningoutcomes}
\end{unit}

\begin{unit}{}{Funciones trascendentes}{Apostol97,Simmons95}{14}{C1}
   \begin{topics}
      \item Función logaritmo
      \item Función exponencial
      \item Funciones trigonométricas e inversas
      \item Derivación e integración
      \item Regla de L'Hopital
   \end{topics}

   \begin{learningoutcomes}
      \item Conocer las funciones trascendentes y su importancia. Calcular derivadas e integrales [\Usage].
      \item Manejar y ejecutar aplicaciones de las funciones trascendentes [\Usage].
      \end{learningoutcomes}
\end{unit}

\begin{unit}{}{Integrales Impropias. Sucesiones y series}{Apostol97,Bartle76,Simmons95}{22}{C24}
   \begin{topics}
      \item Integrales impropias
      \item Sucesiones
      \item Series.
      \item Criterios de convergencia
   \end{topics}

   \begin{learningoutcomes}
      \item Manejar el concepto de integral impropia, calcular integrales [\Usage].
      \item Conocer y manejar los diferentes series. Determinar la convergencia de una serie [\Usage].
      \item Manejar los criterios de convergencia [\Usage].
      \end{learningoutcomes}
\end{unit}

\begin{unit}{}{Sucesiones y Series de funciones}{Apostol97,Simmons95,Bartle76}{18}{C20}
   \begin{topics}
      \item Convergencia uniforme y puntual
      \item Series de potencias. Series de Taylor
      \item Integración de series
   \end{topics}

   \begin{learningoutcomes}
      \item Asimilar y comprender los conceptos de convergencia puntual y uniforme [\Usage].
      \item Aproximar funciones mediante series de potencias. Manejar y utilizar las series de Taylor [\Usage].
      \end{learningoutcomes}
\end{unit}

\begin{unit}{}{Introducción a las Ecuaciones diferenciales}{Apostol97}{18}{C1}
   \begin{topics}
      \item Ecuaciones diferenciales de primer orden
      \item Ecuaciones lineales de segundo orden
   \end{topics}

   \begin{learningoutcomes}
      \item Comprender el concepto de ecuación diferencial y su aplicabilidad en las ciencias [\Usage].
      \item Resolver ecuaciones diferenciales de primer orden y segundo orden [\Usage].
      \item Aplicar ecuaciones diferenciales a la resolución de problemas [\Usage].
      \end{learningoutcomes}
\end{unit}



\begin{coursebibliography}
\bibfile{BasicSciences/MA102}
\end{coursebibliography}

\end{syllabus}

%\end{document}
