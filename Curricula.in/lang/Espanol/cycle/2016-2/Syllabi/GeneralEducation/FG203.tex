\begin{syllabus}

\course{FG203. Oratoria}{Obligatorio}{FG203}

\begin{justification}
En la sociedad competitiva como la nuestra,  se exige que la persona sea un comunicador eficaz y  sepa utilizar sus potencialidades a fin de resolver problemas y enfrentar los desafíos del mundo moderno dentro de la actividad laboral, intelectual y social. Tener el conocimiento no basta, lo importante es saber comunicarlo y en la medida que la persona sepa emplear sus facultades comunicativas, derivará en éxito o fracaso aquello que tenga que realizar en su desenvolvimiento personal y profesional. Por ello es necesario para lograr un buen decir, recurrir a conocimientos, estrategias y recursos, que debe tener todo orador, para llegar con claridad, precisión y convicción al interlocutor
\end{justification}

\begin{goals}
\item Al término del curso, el alumno será capaz de organizar y asumir la palabra desde la perspectiva del orador, en cualquier situación, en forma más correcta, coherente  y adecuada, mediante el uso de conocimientos y habilidades lingüísticas, buscando en todo momento su realización personal y social  a través de su expresión, teniendo como base  la verdad y la preparación constante.
\end{goals}

\begin{outcomes}
    \item \ShowOutcome{f}{2}
    \item \ShowOutcome{n}{2}
    \item \ShowOutcome{ñ}{2}
\end{outcomes}

\begin{competences}
    \item \ShowCompetence{C17}{f,n,ñ}
    \item \ShowCompetence{C24}{f,n,ñ}
\end{competences}

\begin{unit}{}{Primera Unidad: Generalidades de la Oratoria}{Monroe,Rodriguez}{3}{C24}
\begin{topics}
	\item La Oratoria
	\item La función de la palabra.
	\item El proceso de la comunicación.
	\item Bases racionales y emocionales de la oratoria
		\begin{subtopics}
			\item La expresión oral en la participación.
		\end{subtopics}

	\item Fuentes de conocimiento para la oratoria: niveles de cultura general.
\end{topics}
\begin{learningoutcomes}
	\item Comprensión:  interpretar, ejemplificar y generalizar las bases de la oratoria como fundamento teórico  y  práctico. [\Usage].
\end{learningoutcomes}
\end{unit}

\begin{unit}{}{Segunda Unidad: El Orador}{Rodriguez}{4}{C17}
\begin{topics}
	\item Cualidades de un buen orador.
	\item Normas para primeros discursos.
	\item El cuerpo humano como instrumento de comunicación:
		\begin{subtopics}
			\item La expresión  corporal en el discurso
			\item La voz en el discurso.
	   	\end{subtopics}
	\item Oradores con historia y su ejemplo.
\end{topics}
\begin{learningoutcomes}
	\item Comprensión: Interpretar, ejemplificar y generalizar
conocimientos y habilidades de la comunicación oral mediante la experiencia de grandes oradores y la suya propia. [\Usage].
	\item Aplicación: Implementar, usar, elegir y desempeñar los conocimientos adquiridos para  expresarse en público en forma eficiente, inteligente y agradable. [\Usage].
\end{learningoutcomes}
\end{unit}

% begin{unit}{}{Tercera Unidad: Elementos técnicos y complementarios del orado}{Rodriguez}{1}{C17}
% begin{topics}
% 	\item Las fichas, apuntes, citas.
% 	\item Recursos técnicos.
% end{topics}
% begin{learningoutcomes}
% 	\item Conocimiento y aplicación: reconocer y utilizar material de apoyo en forma adecuada y correcta para hacer más eficiente su discurso. [\Usage].
% end{learningoutcomes}
% end{unit}
% 
% begin{unit}{}{Cuarta Unidad: El Discurso}{Rodriguez,MonroeEhninger76, Altamirano}{9}{C17, C24}
% begin{topics}
% 	\item El discurso - Los primeros discursos en clase.
% 	\item Clases de discurso.
% 	\item El propósito del discurso. Discursos informativos. Discursos persuasivos. Discursos sociales, de entretenimiento.
% 	\item El auditorio.	
% 	\item Análisis de discursos famosos.
% end{topics}
% begin{learningoutcomes}
% 	\item Síntesis:  crear, elaborar hipótesis, discernir y experimentar al producir sus propios discursos de manera correcta, coherente y oportuna teniendo en cuenta su propósito y hacia quién los dirige.  [\Usage].
% 	\item Evaluación: ponderar, juzgar, relacionar y apoyar sus propios discursos y los de sus compañeros [\Usage].
% end{learningoutcomes}
% end{unit}
% 
% begin{unit}{}{Quinta Unidad: Formas deliberativas en público}{Rodriguez, MonroeEhninger76, McEntee, Fonseca}{6}{C17}
% begin{topics}
% 	\item La participación
% 		 begin{subtopics}
% 			\item La argumentación: organización, tipos de razonamiento.
% 	   	end{subtopics}
% 	\item Formas de grupo.
% 		 begin{subtopics}
% 			\item La mesa redonda.
% 			\item El conversatorio.
% 			\item El simposio.
% 			\item El foro.
% 			\item Panel de discusión.
% 			\item El debate.
% 	   	end{subtopics}
% 	\item Práctica.
% end{topics}
% begin{learningoutcomes}
% 	\item Comprensión: Interpretar, ejemplificar y generalizar
% conocimientos y habilidades de la comunicación oral mediante el conocimiento y aplicación de formas deliberativas orales. [\Usage].
% 	\item Aplicación: Implementar, usar, elegir y desempeñar los conocimientos adquiridos para  expresarse en público en forma eficiente, inteligente y agradable. [\Usage].
% 	\item Síntesis: Crear, elaborar hipótesis, discernir y experimentar al producir sus propios discursos de manera correcta, coherente y oportuna teniendo en cuenta su propósito y hacia quién los dirige.  [\Usage].
% 	\item Evaluación: El alumno puede ponderar, juzgar, relacionar y apoyar sus propios discursos y los de sus compañeros. [\Usage].
% end
% end{learningoutcomes}
% end{unit}
% 
% begin{unit}{}{Sexta Unidad: Discurso Final}{Altamirano, Rodriguez, MonroeEhninger76}{6}{C17}
% begin{topics}
% 	\item Redacción de discursos: Plan de redacción.
% 		begin{subtopics}
% 			\item Planteamiento del tema
% 			\item Selección de información
% 			\item Análisis del auditorio
% 			\item Elaboración del esquema
% 	   	end{subtopics}
% 	\item Sustentación del discurso: Debate.
% end{topics}
% begin{learningoutcomes}
% 	\item Aplicación: Implementar, usar, elegir y desempeñar los conocimientos adquiridos para  elaborar y expresar  en público un discurso en  forma eficiente, inteligente y agradable. [\Usage].
% 	\item Síntesis: Crear, elaborar hipótesis, discernir y experimentar al producir sus propios discursos de manera correcta, coherente y oportuna teniendo en cuenta su propósito y hacia quién los dirige [\Usage].
% 	\item Evaluación:  ponderar, juzgar, relacionar y apoyar sus propios discursos y los de sus compañeros [\Usage].
% end{learningoutcomes}
% end{unit}



\begin{coursebibliography}
\bibfile{GeneralEducation/FG203}
\end{coursebibliography}
\end{syllabus}
