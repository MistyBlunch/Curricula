\begin{syllabus}

\course{CS402. Proyecto de Final de Carrera I}{Obligatorio}{CS402}

\begin{justification}
Este curso tiene por objetivo que el alumno pueda realizar un estudio del estado del arte de un que el alumno ha elegido como tema para su tesis.
\end{justification}

\begin{goals}
\item Que el alumno realice una investigaci�n inicial en un tema especifico realizando el estudio del estado del arte del tema elegido.
\item Que el alumno muestre dominio en el tema de la l�nea de investigaci�n elegida.
\item Que el alumno elija un docente que domine el de investigaci�n elegida como asesor. 
\item Los entregables de este curso son:
	\begin{description}
		\item [Avance parcial:] Bibliograf�a s�lida y avance de un Reporte T�cnico.
		\item [Final:] Reporte T�cnico con experimentos preliminares comparativos que demuestren que el alumno ya conoce las t�cnicas existentes en el �rea de su proyecto y elegir a un docente que domine el �rea de su proyecto como asesor de su proyecto.
	\end{description}
\end{goals}

\begin{outcomes}
\item \ShowOutcome{a}{2}
\item \ShowOutcome{b}{3}
\item \ShowOutcome{c}{2}
\item \ShowOutcome{e}{3}
\item \ShowOutcome{f}{2}
\item \ShowOutcome{h}{2}
\item \ShowOutcome{i}{3}
\item \ShowOutcome{l}{2}
\end{outcomes}

\begin{competences}
\item \ShowCompetence{C1}{a,b,c} 
\item \ShowCompetence{C20}{e,f,g}
\item \ShowCompetence{CS2}{h,i,l}
\end{competences}

\begin{unit}{}{Levantamiento del estado del arte}{ieee,acm,citeseer}{60}{C1,C20,CS2}
  \begin{topics}
      \item Realizar un estudio profundo del estado del arte en un determinado t�pico del �rea de Computaci�n.
      \item Redacci�n de art�culos t�cnicos en computaci�n.
  \end{topics}
  \begin{learningoutcomes}
      \item Hacer un levantamiento bibliogr�fico del estado del arte del tema escogido (esto significa muy probablemente 1 o 2 cap�tulos de marco te�rico adem�s de la introducci�n que es el cap�tulo I de la tesis) [\Usage]
      \item Redactar un documento en latex en formato articulo (\emph{paper}) con mayor calidad que en Proyecto I (dominar tablas, figuras, ecuaciones, �ndices, bibtex, referencias cruzadas, citaciones, pstricks) [\Usage]
      \item Tratar de hacer las presentaciones utilizando prosper [\Usage]
      \item Mostrar experimentos b�sicos [\Usage]
      \item Elegir un asesor que domine el �rea de investigaci�n realizada [\Usage]
   \end{learningoutcomes}
\end{unit}



\begin{coursebibliography}
\bibfile{Computing/CS/CS401}
\end{coursebibliography}

\end{syllabus}
