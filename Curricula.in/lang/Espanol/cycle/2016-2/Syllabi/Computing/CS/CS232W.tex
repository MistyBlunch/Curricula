\begin{syllabus}

\course{CS232W. Programación de Dispositivos Móviles}{Obligatorio}{CS232W}

\begin{justification}
El siempre creciente desarrollo de las tecnologías de comunicación y la
información hace que exista una marcada tendencia a  establecer medios de 
comunicación más simples y eficientes. De esta forma es que las soluciones
móbiles aparecen como respuesta a esta nueva tendencia.

En este curso se brindará a los participantes una introducción a los
problemas que conlleva la comunicación usando dispositivos móviles, a través del
estudio e implementación de aplicativos; tomando como referencia otros aplicativos
móbiles creados por diferentes grupos de investigación, y también de la industria.

\end{justification}

\begin{goals}
      \item Explorar problemas de investigación en computación móvil.
      \item Conocer tecnologías usadas para computación móvil.
      \item Entender y construir sistemas que soporten la computación móvil.
      \item Comprender las razones por las que dispositivos móviles sean convertido ubicuos, y
      \item Evaluar y proponer aplicaciones cuya solución es apropiada a la computación móvil.

\end{goals}

\begin{outcomes}
\ExpandOutcome{b}{3}
\ExpandOutcome{c}{4}
\ExpandOutcome{e}{3}
\ExpandOutcome{g}{3}
\ExpandOutcome{i}{3}
\ExpandOutcome{j}{4}
\end{outcomes}

\begin{unit}{Mobilidad y Manejo de Localidad}{AGGL:2005}{8}{4}
   \begin{topics}
     \item Definiciones y visiones sobre mobilidad.
     \item Historia de la computación ubicua.
     \item Sistemas ubicuos.
     \item Localidad.
     \item Context aware computing.
   \end{topics}

  \begin{unitgoals}
     \item Conocer los conceptos relaciones con la computación móvil.
     \item Comprender nuevas tendencias en la computación ubicua.
  \end{unitgoals}

\end{unit}

\begin{unit}{Manejo de datos en ambientes móviles}{Pitoura:1997:DMM:550358}{10}{2}
   \begin{topics}
     \item Privacidad en Ubiquitous Computing.
     \item Manejo de datos en ambientes móviles.
     \item Manejo de recursos.
   \end{topics}

   \begin{unitgoals}
     \item Comparar el manejo de datos en sistemas convencionales con el manejo de datos de sistemas móviles y/o ubicuos.
     \item Evaluar las ventajas y desventajas del manejo de recursos en dispositivos móviles.
  \end{unitgoals}
\end{unit}

\begin{unit}{Mobile Ad Hoc y Sensor Networks}{AGGL:2005}{8}{2}
        \NCMobileComputingAllTopics
        \NCMobileComputingAllObjectives
\end{unit}

\begin{unit}{Aplicaciones de computación móvil y ubicua}{Krumm:2009:UCF:1803789}{20}{6}
   \begin{topics}
     \item Áreas de aplicación.
     \item Procesamiento de sensores y datasets.
     \item Mobile social networking.
   \end{topics}
   \begin{unitgoals}
     \item Conocer los tipos de aplicaciones que pueden usarse en diferentes áreas de la industria.
     \item Evaluar formas de procesamiento de señales de dispositivos móviles para generar datasets, y posteriomente poder analizarlos.
  \end{unitgoals}

\end{unit}



\begin{coursebibliography}
\bibfile{Computing/CS/CS232W}
\end{coursebibliography}

\end{syllabus}

%\end{document}
