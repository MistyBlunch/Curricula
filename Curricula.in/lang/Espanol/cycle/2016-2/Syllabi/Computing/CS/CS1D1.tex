\begin{syllabus}

\course{CS1D1. Estructuras Discretas I}{Obligatorio}{CS1D1}

\begin{justification}
Las estructuras discretas proporcionan los fundamentos teóricos necesarios para la computación.
Dichos fundamentos no son sólo útiles para desarrollar la computación desde un punto de vista teórico
como sucede en el curso de teoría de la computación,  sino que también son útiles para la práctica de la
computación; en particular en se aplica en áreas como verificación, criptografía, métodos formales, etc.
\end{justification}

\begin{goals}
\item Aplicar adecuadamente conceptos de la matemática finita (conjuntos, relaciones, funciones) para representar datos de problemas reales.
\item Modelar situaciones reales descritas en el lenguaje natural, usando lógica proposicional y lógica de predicados.
\item Determinar las propiedades abstractas de las relaciones binarias.
\item Escoger el método de demostración más adecuado para determinar la veracidad de una proposición y construir argumentos matemáticos correctos.
\item Interpretar las soluciones matemáticas para un problema y determinar su fiabilidad, ventajas y desventajas.
\item Expresar el funcionamiento de un circuito electrónico simple usando el álgebra de Boole.
\end{goals}

\begin{outcomes}
    \item \ShowOutcome{a}{2}
    \item \ShowOutcome{i}{3}
    \item \ShowOutcome{j}{2}
\end{outcomes}

\begin{competences}
    \item \ShowCompetence{C1}{a}
    \item \ShowCompetence{C20}{i,j}
\end{competences}

 \begin{unit}{\DSSetsRelationsandFunctions}{}{Grimaldi03,Rosen2007}{13}{C1,C20}
   \begin{topics}
        \item \DSSetsRelationsandFunctionsTopicSets
        \item \DSSetsRelationsandFunctionsTopicRelations
        \item \DSSetsRelationsandFunctionsTopicFunctions
   \end{topics}
   \begin{learningoutcomes}
	\item \DSSetsRelationsandFunctionsLOExplainWith [\Assessment]
	\item \DSSetsRelationsandFunctionsLOPerformThe [\Assessment]
	\item \DSSetsRelationsandFunctionsLORelate [\Assessment]
   \end{learningoutcomes}
 \end{unit}

 \begin{unit}{\DSBasicLogic}{}{Rosen2007,Grimaldi03}{14}{C1,C20}
   \begin{topics}
    % KU: L?gica b?sica
        \item \DSBasicLogicTopicPropositional%
        \item \DSBasicLogicTopicLogical%
        \item \DSBasicLogicTopicTruth%
        \item \DSBasicLogicTopicNormal%
        \item \DSBasicLogicTopicValidity%
        \item \DSBasicLogicTopicPropositionalInference%
        \item \DSBasicLogicTopicPredicate%
        \item \DSBasicLogicTopicLimitations%
   \end{topics}
   \begin{learningoutcomes}
	\item \DSBasicLogicLOConvertLogical [\Usage ]
	\item \DSBasicLogicLOApplyFormal [\Usage ]
	\item \DSBasicLogicLOUseThe [\Usage]
	\item \DSBasicLogicLODescribeHowCan [\Familiarity]
	\item \DSBasicLogicLOApplyFormalAnd [\Usage ]
	\item \DSBasicLogicLODescribeTheLimitationsAnd [\Usage]
   \end{learningoutcomes}
 \end{unit}

\begin{unit}{\DSProofTechniques}{}{Rosen2007, Epp10, Scheinerman12}{14}{C1,C20}
\begin{topics}%
	% KU: T?cnicas de demostraci?n
        \item \DSProofTechniquesTopicNotions%
        \item \DSProofTechniquesTopicThe%
        \item \DSProofTechniquesTopicDirect%
        \item \DSProofTechniquesTopicDisproving%
        \item \DSProofTechniquesTopicProof%s
        \item \DSProofTechniquesTopicInduction%
        \item \DSProofTechniquesTopicStructural%
        \item \DSProofTechniquesTopicWeak%
        \item \DSProofTechniquesTopicRecursive%
        \item \DSProofTechniquesTopicWell%
\end{topics}

\begin{learningoutcomes}
    %% itemizar cada learning outcomes [nivel segun el curso]
	\item \DSProofTechniquesLOIdentifyTheUsed [\Assessment]
	\item \DSProofTechniquesLOOutline [\Usage ]
	\item \DSProofTechniquesLOApplyEach [\Usage ]
	\item \DSProofTechniquesLODetermineWhich [\Assessment]
	\item \DSProofTechniquesLOExplainTheIdeas [\Familiarity ]
	\item \DSProofTechniquesLOExplainTheWeak [\Assessment]
	\item \DSProofTechniquesLOStateThe [\Familiarity]
\end{learningoutcomes}
\end{unit}

\begin{unit}{}{Lógica Digital y Representación de Datos}{Rosen2007,Grimaldi03}{19}{C1,C20}
   \begin{topics}
	\item Ordenes Parciales y Conjuntos Parcialmente Ordenados.
 	\item Elementos extremos de un conjunto parcialmente ordenado.
	\item Retículas: Tipos y propiedades.
	\item Álgebras Booleanas
	\item Funciones y expresiones Boolenas
	\item Representación de Funciones Booleanas: Forma Normal Disyuntiva y Conjuntiva
	\item Puertas Lógicas
	\item Minimización de Circuitos
   \end{topics}
   \begin{learningoutcomes}
	\item Explicar la importancia del álgebra de Boole como unificación de la teoría de conjuntos y la lógica proposicional [\Assessment].
	\item Conocer las estructuras algebraicas de retículo y sus tipos [\Assessment].
	\item Explicar la relación entre retículo y conjunto parcialmente ordenado y saber utilizarlo para demostrar que un conjunto es un retículo [\Assessment].
	\item Conocer las propiedades que satisface un álgebra de Boole  [\Assessment].
	\item Demostrar si una terna formada por un conjunto y dos operaciones internas es o no álgebra de Boole [\Assessment].
	\item Encontrar las formas canónicas de una función booleana  [\Assessment].
	\item Representar una función booleana como un circuito booleano usando puertas lógicas  [\Assessment].
	\item Minizar una función booleana [\Assessment].
    \end{learningoutcomes}
 \end{unit}



\begin{coursebibliography}
\bibfile{Computing/CS/CS1D1}
\end{coursebibliography}

\end{syllabus}

%\end{document}
