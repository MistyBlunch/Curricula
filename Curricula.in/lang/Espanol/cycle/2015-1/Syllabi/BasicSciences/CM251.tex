\begin{syllabus}

\curso{CM251. Álgebra Lineal}{Obligatorio}{CM251}

\begin{justification}
En este curso se estudiarán los espacios vectoriales, determinantes, transformaciones lineales, Álgebra multilineal.
Autovalores y formas canónicas, operadores sobre espacios con producto interno formas bilineales y cuadráticas. Geometría Afín y Transformaciones afines.
\end{justification}

\begin{goals}
\item  Lograr que el alumno asimile los conceptos básicos sobre espacios vectoriales, transformaciones lineales, matrices, así como determinantes y sus aplicaciones
\item  Dotar al estudiante de los conocimientos básicos de temas de Álgebra Lineal que son de utilidad para el estudio de otros cursos y sus aplicaciones
\end{goals}

\begin{outcomes}
\ExpandOutcome{a}
\ExpandOutcome{i}
\ExpandOutcome{j}
\end{outcomes}

\begin{unit}{Espacios Vectoriales en general}{Halmos58}{6}
   \begin{topics}
         \item  Definición y ejemplos
	 \item  Subespacios, sus propiedades. Suma y suma directa
         \item  Independencia lineal, base y dimensión
	 \item  Producto interno. Bases ortogonales; ortogonalización de Gram-Schmidt
         \item  Distancia de un punto a una variedad lineal. (Aplicación a la Geometría)
         \item  El espacio cociente
   \end{topics}

   \begin{unitgoals}
         \item  Entender los conceptos y características de los espacios vectoriales
         \item  Resolver problemas
   \end{unitgoals}
\end{unit}

\begin{unit}{Transformaciones lineales}{Halmos58,Hoffman71}{8}
   \begin{topics}
         \item  Definición y ejemplos.
	 \item  Teorema fundamental de las transformaciones lineales y sus consecuencias.
         \item  Álgebra de las transformaciones lineales. Espacio de las transformaciones lineales. Espacio dual.
	 \item  Matrices. Sus operaciones. Rango e inversa. Matriz asociada a una transformación lineal. Matrices equivalentes y semejantes.
         \item  Autovalores y autovectores. Forma triangular. Teorema de Cayley-Hamilton. Forma racional y de  Jordan. Transformaciones lineales diagonalizables, criterios.
	 \item  Tipos especiales de matrices: Simétricas, antisimétricas, unitaria y ortogonal. Su diagonalización.
   \end{topics}

   \begin{unitgoals}
         \item  Entender los conceptos y características de las Transformaciones lineales
         \item  Resolver problemas
   \end{unitgoals}
\end{unit}

\begin{unit}{Determinantes}{Lages95}{6}
   \begin{topics}
         \item  Función determinante.
	 \item  Propiedades.
         \item  Existencia y Unicidad del determinante
	 \item  Cálculo del determinante y determinante de una transformación lineal.
         \item  Cofactores, menores y adjuntos.
	\item Determinante y rango de una matriz. Aplicaciones.
   \end{topics}

   \begin{unitgoals}
         \item  Entender los conceptos y características de los determinantes
         \item  Resolver problemas
   \end{unitgoals}
\end{unit}

\begin{unit}{Álgebra Multilineal}{Lang90}{8}
   \begin{topics}
         \item  Aplicaciones bilineales.
	 \item  Productos tensoriales.
         \item  Isomorfismos canónicos.
	 \item  Producto tensoriales de aplicaciones lineales.
         \item  Cambio de coordenadas de un tensor.
	 \item  Producto tensorial de espacios vectoriales.
         \item  Álgebra tensorial de un espacio vectorial.
   \end{topics}

   \begin{unitgoals}
         \item  Entender y aplicar los conceptos del Álgebra Multilineal
         \item  Resolver problemas
   \end{unitgoals}
\end{unit}

\begin{unit}{Autovalores y formas canónicas}{Chavez05}{8}
   \begin{topics}
	\item  Valores y vectores propios.
	\item  Triangulación de matrices. El Teorema Cayley-Hamilton
	\item  Criterios de diagonalización.
	\item  Matrices nilpotentes.
	\item Forma canónica de Jordan.
	\item La exponencial de una matriz.
   \end{topics}

   \begin{unitgoals}
         \item  Entender y aplicar los conceptos de Autovalores y formas canónicas.
         \item  Resolver problemas.
   \end{unitgoals}
\end{unit}

\begin{unit}{Operadores sobre espacios con producto interno}{Nomizu66}{6}
   \begin{topics}
	\item  La adjunta de un operador.
	\item  Matrices positivas.
	\item  Isometrías.
	\item  Proyección perpendicular.
	\item  Operadores autoadjuntos. El Teorema Espectral.
	\item  Operadores normales.
	\item Funciones definidas sobre transformaciones lineales.
   \end{topics}

   \begin{unitgoals}
         \item  Entender y aplicar los conceptos de Operadores sobre espacios con producto interno
         \item  Resolver problemas
   \end{unitgoals}
\end{unit}

\begin{unit}{Formas bilineales y cuadráticas}{Kaplansky74}{8}
   \begin{topics}
	\item Formas bilineales.
	\item Suma directa y diagonalización.
	\item El teorema de inercia.
	\item Teorema de cancelación de Witt.
	\item Planos hiperbólicos, formas alternadas.
	\item Witt equivalencia.
	\item Formas hermitianas.
   \end{topics}

   \begin{unitgoals}
         \item Entender y aplicar los conceptos de Formas bilineales y cuadráticas.
         \item Resolver problemas.
   \end{unitgoals}
\end{unit}

\begin{unit}{Geometría afin}{Chavez05}{6}
   \begin{topics}
         \item  Planos afines.
	 \item  Planos proyectivos.
         \item  Transformaciones proyectivas.
	 \item  Razón doble.
         \item  Cónicas.
         \item  Espacios de dimensión superior.
   \end{topics}

   \begin{unitgoals}
         \item  Entender y aplicar los conceptos de Geometría afin
         \item  Resolver problemas
   \end{unitgoals}
\end{unit}

\begin{coursebibliography}
\bibfile{BasicSciences/CM251}
\end{coursebibliography}

\end{syllabus}
