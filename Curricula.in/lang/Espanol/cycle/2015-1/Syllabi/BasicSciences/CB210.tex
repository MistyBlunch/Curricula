\begin{syllabus}

\course{CB210. Investigación Operativa I}{Obligatorio}{CB210}

\begin{justification}
Este curso es importante en la medida que proporciona modelos de optimización útiles para la toma de decisiones en negocios.
\end{justification}

\begin{goals}
\item Reconocer, modelar, resolver, implementar e interpretar modelos de optimización linear en problemas reales.
\end{goals}

\begin{outcomes}
\ExpandOutcome{a}{1}
\ExpandOutcome{b}{1}
\ExpandOutcome{c}{1}
\ExpandOutcome{g}{1}
\ExpandOutcome{j}{1}
\end{outcomes}

\begin{unit}{Introducción a la Programación Lineal}{Taha2004,Hillier2006}{14}{1}
   \begin{topics}
      \item Soluciones gráficas.
      \item Ejemplos de problemas de Programación Lineal.
      \item Casos especiales de Programación Lineal.
   \end{topics}

   \begin{unitgoals}
      \item Que el alumno conozca las técnicas y/o conceptos básicos de la Programación asi como su interpretación.
   \end{unitgoals}
\end{unit}

\begin{unit}{El Algoritmo Simplex}{Taha2004,Hillier2006}{12}{1}
   \begin{topics}
      \item Preliminares del Algoritmo Simplex.
      \item Solución de problemas de minimización con mediante el Algoritmo Simplex.
      \item Soluciones alternativas.
      \item Problemas Lineales no acotados.
      \item Uso de paquetes.
      \item Degeneración y convergencia.
   \end{topics}

   \begin{unitgoals}
      \item Que el alumno sea capaz de comprender, diseñar y aplicar el algoritmo simplex en problemas reales de una empresa.
   \end{unitgoals}
\end{unit}

\begin{unit}{Análisis de sensibilidad}{Taha2004,Hillier2006}{12}{1}
   \begin{topics}
      \item Introducción gráfica del análisis de sensibilidad.
      \item Análisis de sensibilidad cuando cambian los parámetros.
      \item Determinación del dual.
      \item Interpretación económica del problema Dual.
      \item Precios sombra.
      \item Holgura complemantaria.
   \end{topics}

   \begin{unitgoals}
      \item Que el alumno pueda medir la sensibilidad de la solución de un problema frente a la variación de parámetros.
   \end{unitgoals}
\end{unit}

\begin{unit}{Problemas de Transporte y Asignación}{Eppen1999,Izar2010}{12}{1}
   \begin{topics}
      \item ?`Cómo formular problemas de transporte?
      \item Soluciones básicas factibles para problemas de transporte.
      \item El método simplex en el transporte.
      \item Análisis de sensibilidad.
      \item Problemas de asignación.
      \item Problemas de transbordo.
   \end{topics}

   \begin{unitgoals}
      \item Que el alumno sea capaza de resolver problemas de optimización en el área de transporte y optimización.
   \end{unitgoals}
\end{unit}

\begin{unit}{Programación entera}{Eppen1999,Izar2010}{10}{1}
   \begin{topics}
      \item Planteamiento de problemas de programación entera.
      \item Métodos de ramificación y acotación.
      \item Algoritmos de plano cortante.
   \end{topics}

   \begin{unitgoals}
      \item Resolver problemas de optimización lineal para la toma de decisiones booleanas.
   \end{unitgoals}
\end{unit}



\begin{coursebibliography}
\bibfile{BasicSciences/CB210}
\end{coursebibliography}

\end{syllabus}
