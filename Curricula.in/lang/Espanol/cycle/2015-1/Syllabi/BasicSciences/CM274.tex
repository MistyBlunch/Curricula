\begin{syllabus}

\curso{CM274. Introducción a la Estadística y Probabilidades}{Obligatorio }{CM274}

\begin{goals}
\item  Estudiar algunos conceptos básicos de Estadística, tanto Descriptiva como Inferencial, para el mejor desenvolvimiento en el Área de Matemática con respecto a la recopilación de datos y la toma de decisiones a partir de la interpretación de dichos datos.
\item  Usar las diferentes distribuciones de probabilidad junto con las pruebas de hipótesis como entes comparativos de datos muestrales, y como ayuda para pronosticar diferentes situaciones reales a posteriori.
\end{goals}

\begin{outcomes}
\ExpandOutcome{a}
\ExpandOutcome{i}
\ExpandOutcome{j}
\end{outcomes}

\begin{unit}{Nociones de estadística descriptiva}{Garcia98}{4}
   \begin{topics}
         \item  Estadística
	 \item  Población y muestra.
	 \item  Variables estadísticas.
         \item  Organización de los datos: Distribuciones de frecuencias.
   \end{topics}

   \begin{unitgoals}
         \item  Entender los conceptos básicos de la Estadística descriptiva
         \item  Resolver problemas
   \end{unitgoals}
\end{unit}

\begin{unit}{Medidas de posición}{Garcia99}{6}
   \begin{topics}
	\item Introducción.
	\item Media Aritmética.
	\item Mediana.
	\item Moda.
	\item Relación entre media, mediana y moda.
	\item Uso de los promedios.
	\item Cuantiles
	\item Otras medias: Media geométrica, media armónica.
   \end{topics}

   \begin{unitgoals}
         \item  Entender y aplicar los conceptos y características de las medidas de posición
         \item  Resolver problemas
   \end{unitgoals}
\end{unit}

\begin{unit}{Medidas de dispersión}{Cordova97}{6}
   \begin{topics}
         \item  Introducción.
	 \item  Medidas de Dispersión.
         \item  Índices de Asimetría
	 \item  Índice de curtosis o apuntamiento
   \end{topics}

   \begin{unitgoals}
         \item  Entender los conceptos y características de las Medidas de Dispersión.
         \item  Resolver problemas
   \end{unitgoals}
\end{unit}

\begin{unit}{Regresión lineal simple}{Garcia99}{4}
   \begin{topics}
         \item  Introducción.
	 \item  Regresión lineal simple
         \item  Nociones de regresión no lineal.
   \end{topics}

   \begin{unitgoals}
         \item  Entender y aplicar los conceptos de Regresión lineal y no lineal
         \item  Resolver problemas
   \end{unitgoals}
\end{unit}

\begin{unit}{Probabilidad}{Moya92}{8}
   \begin{topics}
         \item  Experimento aleatorio, espacio muestral, eventos.
	 \item  Conteo de puntos muestrales (número de puntos muestrales, variaciones, permutaciones, combinaciones).
         \item  Probabilidad de un evento.
	 \item  Cálculo de Probabilidades.
	\item Probabilidad Condicional.
	\item Eventos independientes.
	\item Reglas de la multiplicación, probabilidad total y de Bayes.
   \end{topics}

   \begin{unitgoals}
         \item  Entender y aplicar los conceptos de probabilidad aleatoria y de Bayes
         \item  Resolver problemas
   \end{unitgoals}
\end{unit}

\begin{unit}{Variables aleatorias y distribución de probabilidad}{Mitacc99}{6}
   \begin{topics}
         \item  Variable aleatoria.
	 \item  Variable aleatoria discreta: Función de probabilidad y función de distribución acumulada.
         \item  Variable aleatoria continua: Función de densidad y función de distribución acumulada.
	 \item  Propiedades de la función de distribución.
         \item  Valor esperado o Esperanza Matemática.
   \end{topics}

   \begin{unitgoals}
         \item  Entender y aplicar los conceptos de variables aleatorias y  distribución de probabilidad
         \item  Resolver problemas
   \end{unitgoals}
\end{unit}

\begin{unit}{Algunas distribuciones importantes}{Larson90}{6}
   \begin{topics}
         \item  Algunas distribuciones importantes de variables aleatorias discretas: Bernoulli, Binomial,  Geométrica, Pascal o Binomial Negativa, Hipergeométrica, Poisson.
	 \item  Algunas distribuciones importantes de variables aleatorias continuas: Uniforme, Normal, Erlang, Gamma, exponencial, Chi-cuadrado, $t$ de Student, F de Fisher.
   \end{topics}

   \begin{unitgoals}
         \item  Entender y aplicar algunas distribuciónes importantes de variables aleatorias discretas y continuas.
         \item  Resolver problemas
   \end{unitgoals}
\end{unit}

\begin{unit}{Distribuciones muestrales}{Moya92}{4}
   \begin{topics}
         \item  Muestreo aleatorio.
	 \item  Distribuciones muestrales: de la media, de una proporción, varianza.
   \end{topics}

   \begin{unitgoals}
         \item  Entender y aplicar algunas distribuciónes muestrales
         \item  Resolver problemas
   \end{unitgoals}
\end{unit}

\begin{unit}{Estimación de parámetros}{Mitacc99}{4}
   \begin{topics}
         \item  Introducción.
	 \item  Estimación puntual de parámetros.
	\item Estimación de parámetros por intervalos.
	\item Intervalos de confianza.
   \end{topics}

   \begin{unitgoals}
         \item  Aprender como aplicar la estimación de parámetros
         \item  Resolver problemas
   \end{unitgoals}
\end{unit}

\begin{unit}{Pruebas de hipótesis}{Larson90}{4}
   \begin{topics}
         \item  Hipótesis estadísticas: Hipótesis nula y alternativa, errores tipo I y tipo II, nivel de significancia.
	 \item  Pruebas de hipótesis.
   \end{topics}

   \begin{unitgoals}
         \item  Aprender como aplicar las pruebas de hipótesis.
         \item  Resolver problemas
   \end{unitgoals}
\end{unit}

\begin{coursebibliography}
\bibfile{BasicSciences/CM274}
\end{coursebibliography}

\end{syllabus}
