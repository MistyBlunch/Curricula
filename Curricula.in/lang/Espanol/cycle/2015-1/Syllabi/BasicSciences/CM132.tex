\begin{syllabus}

\curso{CM132. Cálculo Integral}{Obligatorio}{CM132}

\begin{justification}
El curso consiste en hacer un estudio de la Integral. En un inicio se ve la Antiderivada General o Integral Indefinida planteándose métodos básicos de cómo obtenerla: sustitución y por partes, así como sus aplicaciones en la Geometría y Física. Posteriormente, se hace el estudio formal de la Integral Definida empezando por las sumas de Riemann, sus propiedades, discutiendo la existencia de la integral de una función usando los teoremas respectivos. Se estudian los Teoremas Fundamentales del Cálculo así como las funciones trascendentes: exponencial y logaritmo; las coordenadas polares, la Integración Numérica (Trapecio, Simpson) y se finaliza con las aplicaciones de la integral definida: área, volumen, trabajo, centro de masa; las integrales impropias y el Polinomio de Taylor.
\end{justification}

\begin{goals}
\item Presentar el Cálculo Integral para funciones de una variable.
\item Desarrollar técnicas numéricas y analíticas, para abordar algunos problemas que surgen en las aplicaciones de las matemáticas.
\end{goals}

\begin{outcomes}
\ExpandOutcome{a}
\ExpandOutcome{i}
\ExpandOutcome{j}
\end{outcomes}

\begin{unit}{Antiderivadas}{Hasser97,Apostol70}{4}
\begin{topics}
      \item Antiderivadas. Integral indefinida. Propiedades Básicas de la  Integral Indefinida.
      \item Aplicaciones de la Integral Indefinida.
      \item Integración por partes,  sustitución, trigonométricas
   \end{topics}

   \begin{unitgoals}
      \item Describir matemáticamente el concepto de antiderivada
      \item Conocer y aplicar conceptos de integral definida para resolver problemas
   \end{unitgoals}
\end{unit}

\begin{unit}{La integral}{Hasser97,Apostol70}{8}
\begin{topics}
	\item Inducción Matemática. Principios de Inducción Matemática. Sumatorias.
	\item Áereas de figuras planas
	\item Particiones. Sumas de Riemann. Suma inferior y superior. Propiedades
	\item Integral definida. Áerea e Integral definida
	\item Existencia de funciones Integrables
	\item Cotas para el Error de Aproximación de una Integral Definidas
	\item Integral Definida como límite de Sumas. Propiedades de la Integral Definida
\end{topics}
\begin{unitgoals}
	\item Describir matemáticamente el concepto de integral
	\item Conocer y aplicar conceptos de área de una integral
	\item Resolver problemas
\end{unitgoals}
\end{unit}

\begin{unit}{Teoremas}{Hasser97,Apostol70}{12}
\begin{topics}
      \item Primer Teorema fundamental de Cálculo. Segundo Teorema fundamental de Cálculo
      \item Teorema del valor medio para integrales. Cálculo de integrales definidas
      \item Integración Numérica. Aproximación del trapecio. Cotas para el error.
      \item Teorema del cambio de variable de una Integral Definida
\end{topics}

   \begin{unitgoals}
      \item Describir matemáticamente los teoremas fundamentales del cálculo
      \item Conocer y aplicar conceptos de integración numérica
	\item Resolver problemas
   \end{unitgoals}
\end{unit}

\begin{unit}{Área y volumenes}{Hasser97,Apostol70}{12}
\begin{topics}
      \item Áerea de regiones planas (coordenadas cartesianas)
      \item Volumen de sólidos con secciones Planas. Paralelas
      \item Volumen de sólidos de revolución. Método del disco y de las capas cilíndricas
\end{topics}

\begin{unitgoals}
	\item Conocer y aplicar conceptos de áreas y volúmenes en coordenadas cartesianas
	\item Resolver problemas
\end{unitgoals}
\end{unit}

\begin{unit}{Coordenadas polares. Longitud de arco y áreas de superficie de revolución}{Apostol70,Purcel87}{12}
\begin{topics}
	\item Sistemas de Coordenadas Polares
	\item Fórmulas de transformación. Gráficas en coordenadas polares. Intersección de gráficas en coordenadas polares. Tangentes a curvas polares. Cálculo de áreas
	\item Volumen de sólidos de revolución en coordenadas polares y en ecuaciones paramétricas
	\item Longitud de arco de una curva plana paramétrica, en coordenadas cartesianas, en  coordenadas polares
	\item Áereas de Superficies de Revolución: paramétricas,  generada por una función por una  función f, generada por una Curva  Polar
	\item Centro de masa de un Sistemas de Partículas. Centroides. Teorema de Pappus-Guldin
\end{topics}

\begin{unitgoals}
	\item Conocer y aplicar conceptos de áreas y volúmenes en coordenadas polares
	\item Conocer y aplicar conceptos de centro de masa de sistemas de partículas
	\item Resolver problemas
\end{unitgoals}
\end{unit}

\begin{unit}{Aplicaciones}{Apostol70,Purcel87}{8}
\begin{topics}
      \item Aplicaciones: Fuerza y trabajo. 
      \item Trabajo de un resorte. 
      \item Trabajo realizado contra la gravedad. 
      \item Trabajo realizado al vaciar un tanque. 
      \item Fuerza ejercida por un líquido, etc.
      \item Ecuaciones diferenciales separables. 
      \item Problema de valor inicial. Modelos Matemáticos.
\end{topics}

\begin{unitgoals}
	\item Conocer y aplicar conceptos de Fuerza y Trabajo
	\item Resolver problemas
\end{unitgoals}
\end{unit}

\begin{unit}{El logaritmo y la exponencial}{Purcel87,Edwards96}{8}
\begin{topics}
	\item La Función logaritmo natural. Derivadas e integrales.
	\item La función exponencial. Derivadas e integrales. Función exponencial  generalizada. Logaritmo en otras bases
	\item Crecimiento y decaimiento natural. 
	\item Ecuaciones diferenciales lineales de primer orden Aplicaciones
	\item Funciones hiperbólicas directas e inversas. Derivadas e integrales
\end{topics}

\begin{unitgoals}
	\item Conocer y aplicar conceptos de funciones logarítmicas y exponenciales
	\item Conocer y aplicar conceptos de Ecuaciones Diferenciales lineales de primer orden
	\item Resolver problemas
\end{unitgoals}
\end{unit}

\begin{unit}{Técnicas de integración}{Swokowsky89,Taylor65}{8}
\begin{topics}
	\item Métodos de integración. Sustituciones simples. Integración por partes
	\item Integrales trigonométricas y sustitución trigonométrica.
	\item Métodos de fracciones parciales. Integrales que contienen factores  cuadráticos
	\item Binomio diferencial . Funciones racionales del seno y coseno
\end{topics}

\begin{unitgoals}
	\item Conocer y aplicar diferentes técnicas de integración
	\item Resolver problemas
\end{unitgoals}
\end{unit}

\begin{unit}{Integrales impropias}{Spivak96,Granville74}{8}
\begin{topics}
	\item Integrales impropias de primera y segunda especie
	\item Funciones Gamma y Beta
	\item $\pi$ es irracional y $e$ es trascendente
\end{topics}

\begin{unitgoals}
	\item Resolver problemas de integrales impropias
\end{unitgoals}
\end{unit}

\begin{unit}{Cálculo aproximado de integrales}{Courant71,Lang76,Thomas62}{8}
\begin{topics}
      \item Polinomios de Taylor. Fórmula del resto.
      \item Cálculo aproximado de integrales
  \end{topics}

   \begin{unitgoals}
      \item Resolver problemas de cálculo aproximado de integrales
   \end{unitgoals}
\end{unit}

\begin{coursebibliography}
\bibfile{BasicSciences/CM132}
\end{coursebibliography}
\end{syllabus}
