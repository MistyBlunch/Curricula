\begin{syllabus}

\course{CB307. Matem�tica aplicada a la computaci�n}{Obligatorio}{CB307}

\begin{justification}
Este curso es importante porque desarrolla t�picos del �lgebra Lineal y de Ecuaciones Diferenciales Ordinarias �tiles en todas aquellas �reas de la ciencia de la computaci�n donde se trabaja con sistemas lineales y sistemas din�micos.
\end{justification}

\begin{goals}
\item Que el alumno tenga la base matem�tica para el modelamiento de sistemas lineales y sistemas din�micos necesarios en el �rea de Computaci�n Gr�fica e Inteligencia Artificial.
\end{goals}

\begin{outcomes}
\ShowOutcome{a}{3}
\ShowOutcome{i}{3}
\ShowOutcome{j}{4}
\end{outcomes}

\begin{unit}{Espacios Lineales}{Strang03, Apostol73}{0}{4}
\begin{topics}
      \item Espacios vectoriales.
      \item Independencia, base y dimensi�n.
      \item Dimensiones y ortogonalidad de los cuatro subespacios.
      \item Aproximaciones por m�nimos cuadrados.
      \item Proyecciones
      \item Bases ortogonales y Gram-Schmidt
   \end{topics}

   \begin{learningoutcomes}
      \item Identificar espacios generados por vectores linealmente independientes
      \item Construir conjuntos de vectores ortogonales
      \item Aproximar funciones por polinomios trigonom�tricos
   \end{learningoutcomes}
\end{unit}

\begin{unit}{Transformaciones lineales}{Strang03, Apostol73}{0}{4}
\begin{topics}
      \item Concepto de transformaci�n lineal.
      \item Matriz de una transformaci�n lineal.
      \item Cambio de base.
      \item Diagonalizaci�n y pseudoinversa
   \end{topics}

   \begin{learningoutcomes}
      \item Determinar el n�cleo y la imagen de una transformaci�n
      \item Construir la matriz de una transformaci�n
      \item Determinar la matriz de cambio de base
      \end{learningoutcomes}
\end{unit}

\begin{unit}{Autovalores y autovectores}{Strang03, Apostol73}{0}{3}
\begin{topics}
      \item Diagonalizaci�n de una matriz
      \item Matrices sim�tricas
      \item Matrices definidas positivas
      \item Matrices similares
      \item La descomposici�n de valor singular
  \end{topics}

   \begin{learningoutcomes}
      \item Encontrar la representaci�n diagonal de una matriz
      \item Determinar la similaridad entre matrices
      \item Reducir una forma cuadr�tica real a diagonal
   \end{learningoutcomes}
\end{unit}

\begin{unit}{Sistemas de ecuaciones diferenciales}{Zill02,Apostol73}{0}{3}
\begin{topics}
      \item Exponencial de una matriz
      \item Teoremas de existencia y unicidad para sistemas lineales homog�neos con coeficientes constantes
      \item Sistemas lineales no homog�neas con coeficientes constantes.
    \end{topics}

   \begin{learningoutcomes}
      \item Hallar la soluci�n general de un sistema lineal no  homog�neo
      \item Resolver problemas donde intervengan sistemas de ecuaciones diferenciales
   \end{learningoutcomes}
\end{unit}

\begin{unit}{Teor�a fundamental}{Hirsh74}{0}{3}
\begin{topics}
      \item Sistemas din�micos
      \item El teorema fundamental
      \item Existencia y unicidad
      \item El flujo de una ecuaci�n diferencial
   \end{topics}

   \begin{learningoutcomes}
      \item Discutir la existencia y la unicidad de una ecuaci�n diferencial
      \item Analizar la continuidad de las soluciones
      \item Estudiar la prolongaci�n de una soluci�n

   \end{learningoutcomes}
\end{unit}

\begin{unit}{Estabilidad de equilibrio}{Zill02, Hirsh74}{0}{4}
\begin{topics}
      \item Estabilidad
      \item Funciones de Liapunov
      \item Sistemas gradientes
   \end{topics}

   \begin{learningoutcomes}
      \item Analizar la estabilidad de una soluci�n
      \item Hallar la funci�n de Liapunov para puntos de  equilibrio
      \item Trazar el retrato de fase un flujo gradiente
    \end{learningoutcomes}
\end{unit}



\begin{coursebibliography}
\bibfile{BasicSciences/CB307}
\end{coursebibliography}

\end{syllabus}
