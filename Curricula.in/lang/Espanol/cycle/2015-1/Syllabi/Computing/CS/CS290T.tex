\begin{syllabus}

\course{CS290T. Ingeniería de Software I}{Obligatorio}{CS290T}

\begin{justification}
La taréa de desarrollar software, excepto para aplicaciones sumamente simples, exige la ejecución de un proceso de desarrollo bien definido. 
Los profesionales de esta área requieren un alto grado de conocimiento de los diferentes modelos e proceso de desarrollo, 
para que sean capaces de elegir el más idóneo para cada proyecto de desarrollo. Por otro lado, el desarrollo de sistemas 
de mediana y gran escala requiere del uso de bibliotecas de patrones y componentes y del dominio de técnicas relacionadas al 
diseño basado en componentes.
%Proporciona una introducción intensiva, y orientada a la práctica de las técnicas de desarrollo de software usadas para
%crear aplicaciones interactivas de mediana escala, centrándose en el uso de bibliotecas de orientación a objetos
%grandes para crear interfaces de usuario bien diseñadas.
\end{justification}

\begin{goals}
\item Brindar al alumno un marco teórico y práctico para el desarrollo de software bajo estándares de calidad.
\item Familiarizar al alumno con los procesos de modelamiento y construcción de software a través del uso de herramientas CASE.
\item Los alumnos debe ser capaces de seleccionar Arquitecturas y Plataformas tecnológicas ad-hoc a los escenarios de implementación.
\item Aplicar el modelamiento basado en componentes y fin de asegurar variables como calidad, costo  y {\it time-to-market} en los procesos de desarrollo.
\item Brindar a los alumnos mejores prácticas para la verificación y validación del software.
\end{goals}

\begin{outcomes}
\ShowOutcome{b}{4}
\ShowOutcome{c}{4}
\ShowOutcome{d}{3}
\ShowOutcome{f}{3}
\ShowOutcome{i}{3}
\ShowOutcome{j}{3}
\ShowOutcome{k}{3}
\end{outcomes}

\begin{unit}{\SESoftwareDesignDef}{Pressman2005,Sommerville2008,Larman2008}{12}{4}
    \SESoftwareDesignAllTopics
    \SESoftwareDesignAllObjectives
\end{unit}

\begin{unit}{\SEUsingAPIsDef}{Pressman2005,Sommerville2008}{6}{3}
   \SEUsingAPIsAllTopics
   \SEUsingAPIsAllObjectives
\end{unit}

\begin{unit}{\SEToolsAndEnvironmentsDef}{Pressman2005,Sommerville2008,Long91}{8}{3}
    \SEToolsAndEnvironmentsAllTopics
    \SEToolsAndEnvironmentsAllObjectives
\end{unit}

\begin{unit}{\SESoftwareValidationDef}{Pressman2005,Sommerville2008,Larman2008}{8}{3}
    \SESoftwareValidationAllTopics
    \SESoftwareValidationAllObjectives
\end{unit}

\begin{unit}{\SEComponentBasedComputingDef}{Pressman2005,Sommerville2008,Larman2008}{14}{3}
    \SEComponentBasedComputingAllTopics
    \SEComponentBasedComputingAllObjectives
\end{unit}

\begin{unit}{\SESpecializedSystemsDef}{Pressman2005,Sommerville2008,Larman2008}{4}{3}
    \SESpecializedSystemsAllTopics
    \SESpecializedSystemsAllObjectives
\end{unit}

\begin{unit}{\SERobustAndSecurityDef}{Pressman2005,Sommerville2008,Larman2008}{8}{3}
    \SERobustAndSecurityAllTopics
    \SERobustAndSecurityAllObjectives
\end{unit}



\begin{coursebibliography}
\bibfile{Computing/CS/CS290T}
\end{coursebibliography}

\end{syllabus}
