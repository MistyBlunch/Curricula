\begin{syllabus}

\course{CS370. Big Data}{Obligatorio}{CS370}

\begin{justification}
La gestión de la información (IM) juega un rol principal en casi todas las áreas donde los computadores son usados. Esta área incluye la captura, digitalización, representación, organización, transformación y presentación de información; algorítmos para mejorar la eficiencia y efectividad del acceso y actualización de información almacenada, modelamiento de datos y abstracción, y técnicas de almacenamiento de archivos físicos.

Este también abarca la seguridad de la información, privacidad, integridad y protección en un ambiente compartido. Los estudiantes necesitan ser capaces de desarrollar modelos de datos conceptuales y físicos, determinar que métodos de (IM) y técnicas son apropiados para un problema dado, y ser capaces de seleccionar e implementar una apropiada solución de IM que refleje todas las restricciones aplicables, incluyendo escalabilidad y usabilidad.
\end{justification}

\begin{goals}
\item Llevar al alumno hacia el conocimiento de los nuevos desafíos y complejidades de las bases de datos.
\item Hacer que el alumno cree prototipos de motores de bases de datos para la recuparación de información orientada a datos complejos (imagenes, sonido, hipertexto, etc).
\end{goals}

\begin{outcomes}
\ShowOutcome{b}{4}
\ShowOutcome{d}{3}
\ShowOutcome{e}{3}
\ShowOutcome{g}{3}
\ShowOutcome{h}{4}
\ShowOutcome{i}{3}
\ShowOutcome{j}{3}
\end{outcomes}

\begin{unit}{\IMDataMiningDef}{tan05,witten01,han01,kimball04,inmon04,kimball05}{10}{4}
    \IMDataMiningAllTopics%
    \IMDataMiningAllObjectives%
\end{unit}

\begin{unit}{\IMHypermidiaDef}{brusilovsky98,elmasri04}{10}{4}
    \IMHypermidiaAllTopics%
    \IMHypermidiaAllObjectives%
\end{unit}

\begin{unit}{\IMMultimediaSystemsDef}{elmasri04}{10}{4}
    \IMMultimediaSystemsAllTopics%
    \IMMultimediaSystemsAllObjectives%
\end{unit}

\begin{unit}{\IMDigitalLibrariesDef}{witten02,elmasri04}{10}{4}
    \IMDigitalLibrariesAllTopics%
    \IMDigitalLibrariesAllObjectives%
\end{unit}



\begin{coursebibliography}
\bibfile{Computing/CS/CS270W}
\end{coursebibliography}

\end{syllabus}
