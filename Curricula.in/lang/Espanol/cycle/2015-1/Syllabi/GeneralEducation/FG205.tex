\begin{syllabus}

\course{FG205. Historia de la Cultura}{Obligatorio}{FG205}

\begin{justification}
Asignatura básica de carácter formativo y humanístico. Resulta fundamental encontrar justificados o naturales los actos y los principales hitos de la historia universal desde una perspectiva cultural que procure ser profunda, estructurada y crítica. Este conocimiento permitirá entender mejor el presente para proyectarnos con sabiduría al futuro.
\end{justification}

\begin{goals}
\item Comprender que la formación de un buen profesional no se desliga ni se opone, más bien contribuye al auténtico crecimiento personal. Esto requiere de la asimilación de valores sólidos, horizontes culturales amplios y una visión profunda del entorno cultural.
\end{goals}

\begin{outcomes}
\ShowOutcome{FH}{2}
\ShowOutcome{HU}{3}
\end{outcomes}

\begin{unit}{}{hubenak2006historia}{6}{2}
\begin{topics}
	\item Los motivos del estudio de la historia. 	
	\item Historia como ciencia. 	
	\item ¿Qué es Occidente? 	
	\item Cultura. 	
	\item El cristianismo y la cultura. 
\end{topics}
\begin{learningoutcomes}
	\item Conocer nociones teóricas sobre la concepción, posibilidades y límites de la Historia de la Cultura y obtener nociones básicas de la historia de la cultura universal.
\end{learningoutcomes}
\end{unit}

\begin{unit}{}{hubenak2006historia}{9}{2}
\begin{topics}
	\item El Mundo Helénico. 	
	\item El mundo Romano. 	
	\item Herencia cultural greco-romana. 
\end{topics}
\begin{learningoutcomes}
	\item Conocer las bases greco-latinas de la cultura occidental.
\end{learningoutcomes}
\end{unit}

\begin{unit}{}{hubenak2006historia}{12}{2}
\begin{topics}
	\item La Romanidad y la Iglesia: pilares básicos de la civilización occidental. 	
	\item Surgimiento y desarrollo de la edad Media.
\end{topics}
\begin{learningoutcomes}
	\item Comprender la transformación del mundo romano en cristiano, su preservación y apogeo.
\end{learningoutcomes}
\end{unit}

\begin{unit}{}{hubenak2006historia}{9}{2}
\begin{topics}
	\item El renacimiento y el nacimiento de la imagen moderna del mundo.  	
	\item La época de las revueltas. 	
	\item La Reforma Católica. 	
	\item La ilustración y el endiosamiento de la razón. 
\end{topics}
\begin{learningoutcomes}
	\item Percibir la desintegración de la unidad y el ideal cristiano.
\end{learningoutcomes}
\end{unit}

\begin{unit}{}{hubenak2006historia}{9}{2}
\begin{topics}
	\item Revoluciones burguesas en Europa e independencia de países latinoamericanos en el continente americano. 	
	\item El Mundo Contemporáneo y el modernismo. 	
	\item El comienzo de la crisis del siglo XX: guerra y revolución. 	
	\item La infructuosa búsqueda de una nueva estabilidad: Europa entre las guerras 1919-1939. 	
	\item La profundidad de la crisis europea: la Segunda Guerra Mundial. 	
	\item La Guerra Fría y la nueva Europa. Albores del siglo XXI.
\end{topics}
\begin{learningoutcomes}
	\item Aprehender el desarrollo final de la crisis del mundo occidental con la consecuente crisis de los valores cristianos: el materialismo, el hedonismo, el relativismo en la práctica de la vida de las sociedades. 
\end{learningoutcomes}
\end{unit}

\begin{unit}{La Ilustración}{hubenak2006historia}{5}{2}
\begin{topics}
	\item La ilustración y el endiosamiento de la razón.
\end{topics}
\begin{learningoutcomes}
	\item Valorar las nuevas ideas desarrolladas por este movimiento cultural y sus repercusiones.
\end{learningoutcomes}
\end{unit}



\begin{coursebibliography}
\bibfile{GeneralEducation/FG101}
\end{coursebibliography}

\end{syllabus}
