\begin{syllabus}

\course{FG203. Oratoria y Expresión Personal}{Obligatorio}{FG203}

\begin{justification}
En la sociedad competitiva como la nuestra,  se exige que la persona sea un comunicador eficaz y  sepa utilizar sus potencialidades a fin de resolver problemas y enfrentar los desafíos del mundo moderno dentro de la actividad laboral, intelectual y social. Tener el conocimiento no basta, lo importante es saber comunicarlo y en la medida que la persona sepa emplear sus facultades comunicativas, derivará en éxito o fracaso aquello que tenga que realizar en su desenvolvimiento personal y profesional. Por ello es necesario para lograr un buen decir, recurrir a conocimientos, estrategias y recursos, que debe tener todo orador, para llegar con claridad, precisión y convicción al interlocutor
\end{justification}

\begin{goals}
\item Organizar y asumir la palabra desde la perspectiva del orador, en cualquier situación, en forma más correcta, coherente  y adecuada, mediante el uso de conocimientos y habilidades lingüísticas, buscando en todo momento su realización personal y social  a través de su expresión, teniendo como base  la verdad y la preparación constante.  
\end{goals}

\begin{outcomes}
\ShowOutcome{HU}{3}
\ShowOutcome{FH}{2}
\end{outcomes}

\begin{unit}{La Oratoria}{Monroe,Rodriguez}{15}{2}
\begin{topics}
	\item La Oratoria
	\item La función de la palabra.
	\item El proceso de la comunicación.
	\item Bases racionales y emocionales de la oratoria.
	\item Fuentes de conocimiento para la oratoria: niveles de cultura general.
\end{topics}
\begin{learningoutcomes}
	\item El alumno puede interpretar, ejemplificar y generalizar las bases de la oratoria como fundamento teórico  y  práctico.
\end{learningoutcomes}
\end{unit}

\begin{unit}{El Orador}{Rodriguez,Monroe,Altamirano2008,SaberHablar2008,ComoHablarBienenPublico2004}{15}{3}
\begin{topics}
	\item Cualidades de un buen orador.
	\item Normas para hablar en clase.
	\item Oradores con historia y su ejemplo.
	\item El cuerpo humano como instrumento de comunicación: cuerpo y voz en el discurso.
\end{topics}
\begin{learningoutcomes}
	\item El alumno puede interpretar, ejemplificar y generalizar
conocimientos y habilidades de la comunicación oral mediante la experiencia de grandes oradores y la suya propia.
	\item El alumno puede implementar, usar, elegir y desempeñar los conocimientos adquiridos para  expresarse en público en forma eficiente, inteligente y agradable.
\end{learningoutcomes}
\end{unit}

\begin{unit}{El Discurso}{Rodriguez,Monroe,Altamirano2008,SaberHablar2008,ComoHablarBienenPublico2004}{6}{3}
\begin{topics}
	\item Composiciones - Los primeros discursos en clase.
	\item Clases de discurso.
	\item El propósito del discurso. Discursos informativos. Discursos persuasivos. Discursos sociales, de entretenimiento.
	\item El auditorio.	
	\item Redacción de discurso.
\end{topics}
\begin{learningoutcomes}
	\item El alumno puede crear, elaborar hipótesis, discernir y experimentar al producir sus propios discursos de manera correcta, coherente y oportuna teniendo en cuenta su propósito y hacia quién los dirige.l 
	\item El alumno puede ponderar, juzgar, relacionar y apoyar sus propios discursos y los de sus compañeros.
\end{learningoutcomes}
\end{unit}

\begin{unit}{Material de Apoyo}{Rodriguez}{3}{3}
\begin{topics}
	\item Las fichas, apuntes, citas.
	\item Recursos técnicos.
\end{topics}
\begin{learningoutcomes}
	\item El alumno puede reconocer y utilizar material de apoyo en forma adecuada y correcta para hacer más eficiente su discurso. Nota: esta unidad se desarrollará a lo largo del curso.
\end{learningoutcomes}
\end{unit}



\begin{coursebibliography}
\bibfile{GeneralEducation/FG203}
\end{coursebibliography}
\end{syllabus}
