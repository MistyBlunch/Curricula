\section{An�lisis de Demanda}\label{sec:cs-analisis-de-demanda}
Nuestro egresado puede desempe�arse en el mercado laboral peruano sin ning�n problema ya que en general la exigencia peruana esta mucho m�s orientada al uso de herramientas. Sin embargo, es poco com�n que los propios profesionales de esta carrera se pregunten: ?`que tipo de formaci�n deber�a tener si yo quisiera crear esas herramientas adem�s de saber usarlas?. Ambos perfiles, usuario y creador, son bastante diferentes pues no ser�a posible usar algo que todav�a no fue creado. En otras palabras, los creadores de tecnolog�a son los que \underline{dan origen a nuevos puestos de trabajo} y abren la posibilidad de que otros puedan usar esa tecnolog�a. Ambos tipos de profesional son igualmente importantes en nuestra sociedad.

Debido a la formaci�n basada en la investigaci�n, nuestro profesional debe siempre ser un innovador en su centro de trabajo. Esta misma formaci�n permite que el egresado piense tambi�n en crear su propia empresa de desarrollo de software. Considerando que Per� es un pa�s con un costo de vida mucho menor que Norte Am�rica � Europa, una posibilidad que se muestra interesante es la exportaci�n de software.

Este perfil profesional tambi�n posibilita que nuestros egresados se queden en nuestro pa�s pues, producir en Per� y vender en USA, es m�s rentable que salir al extranjero y comercializarlo all�.

El campo ocupacional de un egresado es amplio y est� en continua expansi�n y cambio. Pr�cticamente toda empresa u organizaci�n hace uso de servicios de computaci�n de alg�n tipo, y la buena formaci�n b�sica de nuestros egresados hace que puedan responder a los requerimientos de las mismas exitosamente. Este egresado no s�lo podr� dar soluciones a los problemas existentes sino que deber� proponer innovaciones tecnol�gicas que impulsen la empresa hacia un progreso constante.

A medida que la informatizaci�n b�sica de las empresas del pa�s se est� completando, la necesidad de personas capacitadas para resolver los problemas de mayor complejidad aumenta y el plan de estudios que hemos desarrollado tiene como objetivo satisfacer esta demanda considerando en mediano y largo plazo. El campo para las tareas de investigaci�n y desarrollo de problemas complejos en computaci�n es tambi�n muy amplio y est� creciendo d�a a d�a en el pa�s. 

%infraestructura, turnos libres en la tarde
%Proyeccion de docentes, al cabo de cuatro a�os
%biblioteca compartida con sistemas
%grado academico, menciones