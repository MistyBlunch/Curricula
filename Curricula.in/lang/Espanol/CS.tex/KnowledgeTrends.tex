\section{Tendencias del Conocimiento}\label{sec:cs-tendencias-del-conocimiento}
En un mundo globalizado, como el que vivimos, es indispensable ajustarse a los estándares internacionales. En esta carrera específica, el estándar internacional está claramente liderado por la \textit{Computing Curricula} que fue propuesta por IEEE-CS/ACM antes mencionada.

El tema de ajuste a estándares internacionales toma especial importancia debido al proceso de acreditación que está entrando cada vez con mayor fuerza en nuestro país. La acreditación exige niveles de calidad que incluyen el ajuste a normas y nomenclatura internacional.

Recientemente, el Colegio de Ingenieros del Perú emitió un informe titulado: ``Denominaciones y perfiles de las carreras de Ingeniería de Sistemas, Computación e Informática" \cite{CIPInforme2006}. En este informe se observa una clara tendencia del CIP hacia los estándares internacionales ya mencionados en este documento. Este informe es, sin duda alguna, un gran paso para nuestro país y nuestra carrera.

En nuestro país, la llegada inminente del Tratado de Libre Comercio nos da cada vez menos tiempo para ajustarnos a un nivel de competitividad internacional. Si no hacemos esto estamos en grave peligro de dejar escapar oportunidades muy valiosas. 

Por otro lado, debemos considerar que el área de la computación ha crecido de forma exponencial especialmente debido a la expansión de Internet. Siendo así, estamos en un área que no conoce fronteras internacionales en ningún aspecto.

En cuanto a nuestro país, la entidad que agrupa a las empresas dedicadas a la producción de software es la \ac{APESOFT}. Esta asociación ha tomado como política principal dedicarse a la producción de software para \underline{exportación}. Siendo así, no tendría sentido preparar a nuestros alumnos sólo para el mercado local o nacional. Nuestros egresados deben estar preparados para desenvolverse en el mundo globalizado que nos ha tocado vivir.

Debido a estas consideraciones, es claro que la tendencia de esta carrera es tener como referencia al mundo y no solamente a nuestro país. En la medida que esto sea posible estaremos creando las condiciones para atraer inversiones extranjeras relacionadas al desarrollo de software a nivel internacional.
