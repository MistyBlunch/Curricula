\documentclass{article}
\usepackage[left=0cm,top=0cm,right=0cm,bottom=0cm,head=0cm]{geometry}
\usepackage[utf8]{inputenc}
\usepackage[spanish]{babel}
\usepackage{pstricks,pst-plot,pst-grad}

\pagestyle{empty}
\begin{document}
    \newgray{lightgray}{1}
    \begin{pspicture}(-3,0.1)(15.5,-6)
%       \psgrid[gridcolor=lightgray,gridlabelcolor=lightgray]
      % Frame, Border
      \psframe[fillstyle=solid,fillcolor=lightgray](-3,0.1)(15.5,-5)

\begin{footnotesize}

\rput[tl](0,0){%
        %\psframebox*[fillstyle=none,fillcolor=lightgray,linecolor=none]{%
\begin{tabular}{l}
\begin{normalsize}Nivel\end{normalsize}\\
1xx = introductorio, 2xx = intermedio, 3xx = avanzado, 4xx = proyecto final de carrera\\
\end{tabular}
}

\rput[tl](0,-0.9){%
        %\psframebox*[fillstyle=none,fillcolor=lightgray,linecolor=none]{%
\begin{tabular}{l@{ }l}
\multicolumn{2}{l}{\normalsize Tema (segundo dígito/letra)}\\
1 = Algoritmos y Complejidad (AL)      		  & B = Desarrollo Basados en Plataforma (PBD)\\
2 = Arquitectura y Organización (AR)   		  & C = Ciencia Computacional (CN) \\
3 = Redes y Comunicaciones (NC)	       		  & D = Estructura Discretas (DS)\\
4 = Lenguajes de Programación (PL)     		  & F = Fundamentos del Desarrollo de Software (SDF)\\
5 = Gráficos y Visualización (GV)		  & H = Interacción Humano Computador (HCI)\\
6 = Sistemas Inteligentes (IS)         		  & I = Aseguramiento y Seguridad de la Información (IAS)\\
7 = Gestión de Información (IM)        		  & P = Computación Paralela y Distribuída (PD)\\
8 = Asuntos Sociales y Práctica Profesional (SP)  & S = Sistemas Operativos (OS)\\ 
9 = Ingeniería de Software (SE)  		  & U = Fundamentos de Sistemas (SF)
\end{tabular}
}

\rput[tl](0,-3){%
        %\psframebox*[fillstyle=none,fillcolor=lightgray,linecolor=none]{%
}

\rput[tC](0,-4.5){%
        %\psframebox*[fillstyle=none,fillcolor=lightgray,linecolor=none]{%
\begin{tabular}{l@{ }l}
\multicolumn{2}{l}{\normalsize Identificador numérico en el área}
\end{tabular}
}
\end{footnotesize}

\rput[r](0,-2.6){\LARGE C S 2 7 1}
\psline{->}(0,-0.2)(-1,-0.2)(-1,-2.3)
\psline{->}(0,-1.1)(-0.6,-1.1)(-0.6,-2.3)
\psline{->}(-0.15,-4.3)(-0.15,-2.85)
%\psline{->}(0,)(-0.6,-5.85)(-0.6,-2.85)

    \end{pspicture}
    
\end{document}
