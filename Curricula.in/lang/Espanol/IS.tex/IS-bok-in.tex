\begin{BKL2}{IT1}{IT1. Arquitectura de Computadores}
Horas:

(begin_topics)
-Representación de datos fundamentales: no numéricos, numéricos (enteros, reales, errores, precisión)
(begin_subtopics)
-Representación básica en máquina de datos numéricos
-Representación básica en máquina de datos no numéricos
-Precisión de la representación de números enteros y de punto flotante
-Errores en aritmética de computador y asuntos relacionados de portabilidad
-Conceptos básicos de arquitectura de computadores
(end_subtopics)
(begin_goals)
(end_goals)
(end_topics)

(begin_topics)
-Representación física de información digital: p.e. datos, texto, imágenes, voz, video.
(begin_subtopics)
(end_subtopics)
(begin_goals)
(end_goals)
(end_topics)

(begin_topics)
-Architecturas de CPU: CPU, memoria, registros, modos de direccionamiento, conjuntos de instrucciones.
(begin_subtopics)
-Organización básica; von Neumann, diagrama de bloques, bus de datos, bus de control.
-Instrucciones y modos de direccionamiento: conjuntos de instrucciones y tipos.
-Instrucciones y modos de direccionamiento: lenguaje de máquina.
-Modos de direccionamiento.
-Unidad de control; {\it fetch} y ejecución de instrucciones, {\it fetch} de operadores.
-CISC, RISC.
-Organización del computador.
-Sistemas de memoria.
(end_subtopics)
(begin_goals)
(end_goals)
(end_topics)

(begin_topics)
-Componentes de sistemas de computadores: bus, controladores, sistemas de almacenamiento, dispositivos periféricos.
(begin_subtopics)
-Periféricos: E/S e interrupciones.
-Periféricos: métodos de control de E/S, interrupciones.
-Periféricos: almacenamiento externo, organización física y {\it drives}.
-Almacenamiento auxiliar, cinta, óptico.
-Sistemas de almacenamiento, jerarquía.
-Organización principal de la memoria, operaciones de bus, tiempos de ciclo.
-Memoria caché, lectura/escritura.
-Memoria virtual.
-Interfaces entre computadores y otros dispositivos (sensores, efectuadores, etc.)
(end_subtopics)
(begin_goals)
(end_goals)
(end_topics)

(begin_topics)
-Arquitectura de multiprocesadores
(begin_subtopics)
-Arquitecturas de sistemas (multi-procesamiento simple y procesamiento distribuido, pilas)
-Tecnologías cliente-servidor.
(end_subtopics)
(begin_goals)
(end_goals)
(end_topics)

(begin_topics)
-Lógica y sistemas discretos.
(begin_subtopics)
-Elementos lógicos y teoría de {\it switching}; conceptos e implementación de minimización.
-Demoras y peligros de propagación.
-Demultiplexers, multiplexers, decodificadores, codificadores, aditores, substractores.
-ROM, PROM, EPROM, EAPROM, RAM.
-Análisis y síntesis de circuitos síncronos, asíncronos vs. síncronos.
-Notación de transferencia de registros, condicional e incondicional.
-Máquinas de estado, redes de conducción, señales de transferencia de carga.
-Tri-estados y estructuras de bus.
-Diagramas de bloque, diagramas de tiempo, lenguaje de transferencia.
(end_subtopics)
(begin_goals)
(end_goals)
(end_topics)
\end{BKL2}

\begin{BKL2}{IT2}{IT2. Algoritmos y Estructuras de Datos}
Horas:

(begin_topics)
-Problemas formales y resolución de problemas.
(begin_subtopics)
-Estrategias de resolución de problemas usando algoritmos voraces.
-Estrategias de resolución de problemas usando algoritmos divide y vencerás.
-Estrategias de resolución de problemas usando algoritmos de {\it back- tracking}.
-Proceso de diseño de software; desde la especificación a la implementación.
-Establecimiento del problema y determinación algorítmica.
-Estrategias de implementación ({\it top-down}, {\it bottom-up}; equipo vs. individual).
-Conceptos de verificación formal.
-Models de computación formales.
(end_subtopics)
(begin_goals)
(end_goals)
(end_topics)

(begin_topics)
-Estructuras de datos básicas: listas, arreglos, cadenas, registros, conjuntos, listas enlazadas, pilas, colas, árboles.
(begin_subtopics)
(end_subtopics)
(begin_goals)
(end_goals)
(end_topics)

(begin_topics)
-Estructuras de datos complejas: p.e., de datos, texto, voz, imagen, video, hipermedia.
(begin_subtopics)
(end_subtopics)
(begin_goals)
(end_goals)
(end_topics)

(begin_topics)
-Tipos abstractos de datos.
(begin_subtopics)
-Propósito e implementación de tipos abstractos de datos.
-Especificaciones informales.
-Especificaciones formales, pre-condiciones y post-condiciones, algebraicas.
-Módulos, cohesión, acoplamiento; diagramas de flujo de datos y conversión a jerarquías.
-Correctitud, verificación y validación: pre- y post-condiciones, invariantes.
-Estructuras de control; selección, iteración, recursión; tipos de datos y sus usos.
(end_subtopics)
(begin_goals)
(end_goals)
(end_topics)

(begin_topics)
-Estructuras de archivos: secuencial, de acceso directo, hashing, indexados.
(begin_subtopics)
-Archivos (estructura, métodos de acceso): distribución de archivos, conceptos de archivos fundamentales.
-Archivos (estructura, métodos de acceso): contenidos y estructuras de directorios, nombramiento.
-Archivos (estructura, métodos de acceso): vista general de seguridad del sistema, métodos de seguridad.
(end_subtopics)
(begin_goals)
(end_goals)
(end_topics)

(begin_topics)
-Estructuras y algoritmos de ordenamiento y búsqueda.
(begin_subtopics)
-Algoritmos de ordenamiento ({\it shell sort}, {\it bucket sort}, {\it radix sort}, {\it quick sort}), edición.
-Algoritmos de búsqueda (búsqueda serial, búsqueda binaria y árboles de búsqueda binaria).
-Búsqueda, hashing, resolución de colisiones.
(end_subtopics)
(begin_goals)
(end_goals)
(end_topics)

(begin_topics)
-Eficiencia de algoritmos, complejidad y métricas.
(begin_subtopics)
-Análisis asintótico.
-Balance entre tiempo y espacio en algoritmos.
-Clases de complejidad P, NP, P-space; problemas tratables e intratables.
-Análisis de límite inferior.
-NP-completitud.
-Algoritmos de ordenamiento $O(n^2)$.
-Algoritmos de ordenamiento $O(n\log n)$.
-{\it Backtracking}, {\it parsing}, simulaciones discretas, etc.
-Fundamentos de análisis de algoritmos.
(end_subtopics)
(begin_goals)
(end_goals)
(end_topics)

(begin_topics)
-Algoritmos recursivos.
(begin_subtopics)
-Conexión de algoritmos recursivos con inducción matemática.
-Comparación de algoritmos iterativos y recursivos.
(end_subtopics)
(begin_goals)
(end_goals)
(end_topics)

(begin_topics)
-Redes neuronales y algoritmos genéticos.
(begin_subtopics)
(end_subtopics)
(begin_goals)
(end_goals)
(end_topics)

(begin_topics)
-Consideraciones avanzadas.
(begin_subtopics)
-Funciones computables: modelos de funciones computables de máquinas de Turing.
-Problemas de decisión: problemas enumerables recursivos; indecibilidad.
-Modelo de arquitecturas paralelas.
-Algoritmos de arquitecturas paralelas.
-Problemas matemáticos: problemas bien acondicionados y mal acondicionados.
-Problemas matemáticos: aproximaciones iterativas a problemas matemáticos.
-Problemas matemáticos: clasificación de error, computacional, representacional.
-Problemas matemáticos: aplicaciones de métodos de aproximación interativa.
-Límites de computación: computabilidad e intratabilidad algorítmica.
(end_subtopics)
(begin_goals)
(end_goals)
(end_topics)
\end{BKL2}

\begin{BKL2}{IT3}{IT3. Lenguajes de Programación}
Horas:
 
(begin_topics)
-Estructuras de lenguajes de programación fundamentales; comparación de lenguajes y aplicaciones.
(begin_subtopics)
(end_subtopics)
(begin_goals)
(end_goals)
(end_topics)

 
(begin_topics)
-Lenguajes de nivel de máquina y ensamblador.
(begin_subtopics)
(end_subtopics)
(begin_goals)
(end_goals)
(end_topics)

 
(begin_topics)
-Lenguajes procedurales.
(begin_subtopics)
-Ventajas y desventajas de la programación procedural.
-Declaraciones básicas de tipos; operaciones aritméticas y asignación; condicionales.
-Procedimientos, funciones y parámetros; arreglos y registros.
(end_subtopics)
(begin_goals)
(end_goals)
(end_topics)

 
(begin_topics)
-Lenguajes no procedurales: lógicos, funcional y basados en eventos.
(begin_subtopics)
(end_subtopics)
(begin_goals)
(end_goals)
(end_topics)

 
(begin_topics)
-Lenguajes de cuarta generación.
(begin_subtopics)
(end_subtopics)
(begin_goals)
(end_goals)
(end_topics)

 

(begin_topics)
-Extensiones orientadas a objetos para lenguajes.
(begin_subtopics)
(end_subtopics)
(begin_goals)
(end_goals)
(end_topics)

 

(begin_topics)

-Lenguajes de programación, diseño, implementación y comparación.
(begin_subtopics)

-Historia de los primeros lenguajes.

-Evolución de los lenguajes procedurales.

-Evolución de los lenguajes no procedurales.

-Computadores virtuales.

-Tipos de datos elementales y estructurados.

-Creación y aplicación de tipos de datos definidos por el usuario.

-Expresiones, orden de evaluación y efectos secundarios.

-Subproigramas y corutinas como abstracciones de expresiones y declaraciones.

-Manejo de excepciones.

-Mecanismos para compartir y restringir el acceso a datos.

-Ámbito estático vs. dinámico, timepo de vida, visibilidad.

-Mecanismos de paso de parámetros; referencia, valor, nombre, resultado, etc.

-Variedades de disciplinas de prueba de tipos y sus mecanismos.

-Aplicación de almacenamiento basado en pilas.

-Aplicación de almacenamiento basado en {\it heaps}.

-Autómatas de estado finito para modelos restringidos de computación.

-Aplicación de expresiones regulares al análisis de lenguajes de programación.

-Uso de gramáticas libres de contexto y de autómatas de pila.

-Equivalencia entre gramáticas libres de contexto y autómatas de pila.

-Uso de autómatas de pila en el {\it parsing} de lenguajes de programación.

-Proceso de traducción de lenguajes, compiladores a interpretadores.

-Semántica de los lenguajes de programación.

-Paradigmas y lenguajes de programación funcional.

-Construcciones programación paralela.

-Lenguajes procedurales: problemas de implementación; mejora del rendimiento.

-Compiladores y traductores.

-Lenguajes de muy alto nivel: SQL, lenguajes de cuarta generación.

-Diseño orientado a objetos, lenguajes y programación.

-Lenguajes de programación lógica: LISP, PROLOG; programación orientada a lógica.

-Generadores de código.

-Shells de sistemas expertos.

-Lenguajes de diseño de software.

(end_subtopics)
(begin_goals)
(end_goals)
(end_topics)

\end{BKL2}
 

\begin{BKL2}{IT4}{IT4. Sistemas Operativos}
Horas:
 
(begin_topics)

-Arquitectura, objetivos y estructura de un sistema operativo; métodos de estructuración, modelos por capas.

(begin_subtopics)
(end_subtopics)
(begin_goals)
(end_goals)
(end_topics)

 

(begin_topics)

-Interacción del sistema operativo con la arquitectura de hardware.

(begin_subtopics)
(end_subtopics)
(begin_goals)
(end_goals)
(end_topics)

 

(begin_topics)

-Administración de procesos: procesos concurrentes, sincronización.

(begin_subtopics)

-Tareas, procesos, interrupciones.

-Estructuras, listas de espera, bloques de control de procesos.

-Ejecución concurrente de procesos.

-Acceso compartido, condiciones de ejecución.

-{\it Deadlock}; causas, condiciones, prevención.

-Modelos y mecanismos (p.e., {\it busy waiting}, {\it spin locks}, algoritmo de Deker).

-{\it Switching} preferente y no preferente.

-{\it Schedulers} y políticas de {\it scheduling}.
(end_subtopics)
(begin_goals)
(end_goals)
(end_topics)

 

(begin_topics)
-Administración de memoria.
(begin_subtopics)

-Memoria física y registros.

-{\it Overlays}, {\it swapping}, particiones.

-Páginas y segmentos.

-Política de posicionamiento y reposicionamiento.

-{\it Thrashing}, {\it working sets}.

-Listas libres, {\it layout}; servidores, interrupciones; recuperación de fallos.

-Protección de memoria, administración de la recuperación.

(end_subtopics)
(begin_goals)
(end_goals)
(end_topics)

 

(begin_topics)

-Asignación y programación de recursos.

(begin_subtopics)

-{\it Suites} de protocolos (comuniación y redes); {\it streams} y datagramas.

-Internet {\it working} y {\it routing}; servidores y servicios.

-Tipos de sistemas operativos: de usuario simple, multi-usuario, de red.

-Sincronización y temporización en sistemas distribuidos y de tiempo real.

-Consideraciones especiales en sistemas de tiempo real; fallas, riesgos y recuperación.

-Utilidades de sistemas operativos.

-Evolución del hardware; fuerzas y restricciones económicas.

-Arquitectura de los sistemas de tiempo real y sistemas empotrados.

-Consideraciones especiales en sistemas de tiempo real empotrados: requerimientos {\it hard-timing}..

(end_subtopics)
(begin_goals)
(end_goals)
(end_topics)

 

(begin_topics)

-Administración de almacenamiento secundario.
(begin_subtopics)
(end_subtopics)
(begin_goals)
(end_goals)
(end_topics)

 

(begin_topics)

-Sistemas de archivos y de directorios.
(begin_subtopics)
(end_subtopics)
(begin_goals)
(end_goals)
(end_topics)

 

(begin_topics)

-Protección y seguridad.
(begin_subtopics)
(end_subtopics)
(begin_goals)
(end_goals)
(end_topics)

 

(begin_topics)

-Sistemas operativos distribuidos.
(begin_subtopics)
(end_subtopics)
(begin_goals)
(end_goals)
(end_topics)

 

(begin_topics)
-Soporte del sistema operativo para interacción humano-computador: p.e., GUI, video interactivo.
(begin_subtopics)
(end_subtopics)
(begin_goals)
(end_goals)
(end_topics)

 

(begin_topics)
-Interoperatividad y compatibilidad de sistemas operativos: p.e., sistemas abiertos.
(begin_subtopics)
(end_subtopics)
(begin_goals)
(end_goals)
(end_topics)

 

(begin_topics)

-Utilidades de los sistemas operativos, herramientas, comandos y programación {\it shell}.
(begin_subtopics)
(end_subtopics)
(begin_goals)
(end_goals)
(end_topics)

 

(begin_topics)

-Administración y gerenciamiento de sistemas.
(begin_subtopics)

-{\it Bootstrapping} del sistema/carga inicial de programa.

-Generación del sistema.

-Configuración del sistema.

-Análisis, evaluación y monitoreo de rendimiento.

-Optimización y {\it tuning} del sistema.

-Funciones de administración del sistema: copias de seguridad, securidad y protección.

(end_subtopics)
(begin_goals)
(end_goals)
(end_topics)

\end{BKL2}
 


\begin{BKL2}{IT5}{IT5. Telecomunicaciones}
Horas:
 
(begin_topics)

-Estándares, modelos y tendencias internacionales en telecomunicaciones.
(begin_subtopics)

-Redes de computadoras y control: topologías, portadores comunes, equipos.

-Diseño y administración de redes: arquitecturas de red (ISO, SNA, DNA).

(end_subtopics)
(begin_goals)
(end_goals)
(end_topics)

 

(begin_topics)

-Transmisión de datos: media, técnicas de señalización, impedimento de transmisión, codificación, error.

(begin_subtopics)

-Tecnologías de sistemas de comunicaciones: medios de transmisión, analógico-digital.
(end_subtopics)
(begin_goals)
(end_goals)
(end_topics)

 

(begin_topics)

-Configuración de línea: control de rror, control de flujo, multiplexado.
(begin_subtopics)
(end_subtopics)
(begin_goals)
(end_goals)
(end_topics)

 

(begin_topics)

-Redes de área local.
(begin_subtopics)

-Topologías, control de acceso al medio, multiplexado.

-Redes de área local y WANs: topología, {\it gateways}, usos (funciones y oficina).

-Determinación de requerimientos, monitoreo y control del rendimiento, aspectos económicos.

-Arquitectura de los sistemas distribuidos.

-Aspectos de hardware de los sitemas distribuidos.

(end_subtopics)
(begin_goals)
(end_goals)
(end_topics)


(begin_topics)
-Redes de área amplia: técnicas de {\it switching}, de {\it broadcast}, {\it routing}.
(begin_subtopics)
(end_subtopics)
(begin_goals)
(end_goals)
(end_topics)

 

(begin_topics)
-Arquitecturas y protocolos de redes.
(begin_subtopics)
(end_subtopics)
(begin_goals)
(end_goals)
(end_topics)

 

(begin_topics)
-{\it Internetworking}
(begin_subtopics)
(end_subtopics)
(begin_goals)
(end_goals)
(end_topics)

 

(begin_topics)
-Configuración de redes, análisis y monitoreo de rendimiento.
(begin_subtopics)
(end_subtopics)
(begin_goals)
(end_goals)
(end_topics)

 

(begin_topics)

-Seguridad de redes: encriptación, firmas digitales, autenticación.

(begin_subtopics)
(end_subtopics)
(begin_goals)
(end_goals)
(end_topics)

 

(begin_topics)

-Redes de alta velocidad: p.e. ISDN, SMDS, ATM, FDDI de banda ancha.

(begin_subtopics)
(end_subtopics)
(begin_goals)
(end_goals)
(end_topics)

 

(begin_topics)

-Tecnologías emergentes: ATM, ISDN, redes de satélites, redes ópticas, etc., voz, datos y videos integrados.

(begin_subtopics)
(end_subtopics)
(begin_goals)
(end_goals)
(end_topics)

 

(begin_topics)

-Aplicación: p.e., cliente-servidor, EDI, EFT, redes de teléfonos, e-mail, multimedia, video.
(begin_subtopics)

-Métodos y transmisión de información gráfica y de video.

(end_subtopics)
(begin_goals)
(end_goals)
(end_topics)

 

(begin_topics)

-Protocolos de sistemas abiertos.
(begin_subtopics)

-Protocolos de transporte.

-Protocolos de soporte de aplicaciones: encriptación; compromiso, concurrencia.
(end_subtopics)
(begin_goals)
(end_goals)
(end_topics)

 

(begin_topics)

-Distribución de información.
(begin_subtopics)

-Structura de redes.

-Tecnología cliente-servidor/cliente-servidor delgada.

-Redes, {\it routing}, análisis de desempeño.

-Sistemas de comunicaciones.
(end_subtopics)
(begin_goals)
(end_goals)
(end_topics)

\end{BKL2}
 

\begin{BKL2}{IT6}{IT6. Bases de Datos}

Horas:
 
(begin_topics)
-DBMS: características, funciones y arquitectura.
(begin_subtopics)
-DBMS (características, funciones, arquitectura); componentes de un sistema de bases de datos.

-DBMS: vista general de álgebra relacional.

-Diseño lógico (diseño independiente de DBMS): ER, orientado a objetos.

(end_subtopics)
(begin_goals)
(end_goals)
(end_topics)

 

(begin_topics)

-Modelos de datos: relacional, jerárquico, de red, de objetos, de objetos semánticos.
(begin_subtopics)

-Teminología de modelado de datos relacional.

-Modelado conceptual (p.e., entidad-relación, orientado a objtos).

-Interconversión entre tipos de modelos (p.e. jerárquico a relacional).
(end_subtopics)
(begin_goals)
(end_goals)
(end_topics)

 

(begin_topics)
-Normalización
(begin_subtopics)
(end_subtopics)
(begin_goals)
(end_goals)
(end_topics)

 

(begin_topics)

-Integridad (referencial, de item de datos, de intra-relaciones): representación de relaciones; de entidad y referencial.
(begin_subtopics)
(end_subtopics)
(begin_goals)
(end_goals)
(end_topics)

 

(begin_topics)

-Lenguajes de definición de datos (lenguajes de definición de esquemas, herramientas de desarrollo gráficas).
(begin_subtopics)
(end_subtopics)
(begin_goals)
(end_goals)
(end_topics)

 

(begin_topics)

-Interfaz de aplicaciones.
(begin_subtopics)

-Función soportada por sistemas de bases de datos típicos; métodos de acceso, seguridad.

-DML, consulta, QBE, SQL, etc.: lenguaje de consultas de bases de datos; definitción de datos, consulta.

-Interfaces de aplicación y de usuario (DML, consulta, QBE, SQL).

-Objetos de pantalla basados en eventos (botones, listas, etc.)

-Procesamiento físico de transacciones; consideraciones cliente-servidor.

-Distribución de consideraciones de procesamiento del cliente y del servidor.
(end_subtopics)
(begin_goals)
(end_goals)
(end_topics)

 

(begin_topics)

-Procesadores inteligentes de consultas y organización de consultas, herramientas OLAP.
(begin_subtopics)
(end_subtopics)
(begin_goals)
(end_goals)
(end_topics)

 

(begin_topics)

-Bases de datos distribuidas, repositorios y {\it warehouses}
(begin_subtopics)
(end_subtopics)
(begin_goals)
(end_goals)
(end_topics)

 

(begin_topics)
-Productos DBMS: desarrollos recientes en sistemas de bases de datos (p.e., hipertexto, hipermedia, ópticos).
(begin_subtopics)
(end_subtopics)
(begin_goals)
(end_goals)
(end_topics)

 

(begin_topics)

-Máquinas y servidores de bases de datos.
(begin_subtopics)
(end_subtopics)
(begin_goals)
(end_goals)
(end_topics)

 

(begin_topics)

-Administración de datos y bases de datos.
(begin_subtopics)

-Administración de datos.

-Administración de bases de datos: impacto social de las bases de datos; seguridad y privacidad.

-Propiedad y controles de acceso para sistema de datos y de aplicaciones.

-Modelos de acceso basados en roles y capacidades.

-Replicación.

-Planeamiento de la capacidad del sistema.

-Planeamiento y administración de redundancia, seguridad y copias de seguridad.

(end_subtopics)
(begin_goals)
(end_goals)
(end_topics)

 

(begin_topics)

-Diccionarios, enciclopedias y repositorios de datos.
(begin_subtopics)
(end_subtopics)
(begin_goals)
(end_goals)
(end_topics)

 

(begin_topics)

-Recuperación de información: p.e. herramientas de Internet, procesamiento de imágenes, hipermedia.
(begin_subtopics)
(end_subtopics)
(begin_goals)
(end_goals)
(end_topics)

\end{BKL2}



\begin{BKL2}{IT7}{IT7. Inteligencia Artificial}
Horas:
 
(begin_topics)

-Representación del conocimiento.
(begin_subtopics)

-Historia, contexto y límites de la inteligencia artificial; el test de Turing.
(end_subtopics)
(begin_goals)
(end_goals)
(end_topics)

 

(begin_topics)

-Ingeniería del conocimiento.
(begin_subtopics)
(end_subtopics)
(begin_goals)
(end_goals)
(end_topics)

 

(begin_topics)

-Proceso de inferencia.
(begin_subtopics)

-Estrategias de control básicas (p.e., por profundidad y por amplitud).

-Razonamiento hacia adelante y hacia atrás.

-Búsqueda heurística.

-Sistemas expertos.
(end_subtopics)
(begin_goals)
(end_goals)
(end_topics)

 

(begin_topics)

-Otras técnicas: lógica difusa, razonamiento basado en casos, lenguaje natural y reconocimiento del habla.
(begin_subtopics)
(end_subtopics)
(begin_goals)
(end_goals)
(end_topics)

 

(begin_topics)

-Sistemas basados en conocimiento.
(begin_subtopics)

-Lenguaje natural, habla y visión.

-Reconocimiento de patrones.

-Aprendizaje de máquina.

-Robótica.

-Redes neuronales.

(end_subtopics)
(begin_goals)
(end_goals)
(end_topics)

\end{BKL2}



\begin{BKL2}{OMC1}{OMC1. Teoría General de Organizaciones}
Horas:
 
(begin_topics)

-Modelos organizacionales jerárquicos y de flujo.
(begin_subtopics)
(end_subtopics)
(begin_goals)
(end_goals)
(end_topics)

 

(begin_topics)

-Grupos de trabajo organizacionales.
(begin_subtopics)
(end_subtopics)
(begin_goals)
(end_goals)
(end_topics)

 

(begin_topics)

-Envergadura organizacional: usuario simple, grupo de trabajo, equipo, empresa, global.

(begin_subtopics)
(end_subtopics)
(begin_goals)
(end_goals)
(end_topics)

 

(begin_topics)

-Rol de Sistemas de Información dentro de la empresa: estratégico, táctico y operativo.
(begin_subtopics)
(end_subtopics)
(begin_goals)
(end_goals)
(end_topics)

 

(begin_topics)

-Efecto de Sistemas de Información en la estructura organizacional; Sistemas de Información y mejora continua.

(begin_subtopics)
(end_subtopics)
(begin_goals)
(end_goals)
(end_topics)

 

(begin_topics)

-Estructura organizacional: centralizada, descentralizada, matriz.

(begin_subtopics)
(end_subtopics)
(begin_goals)
(end_goals)
(end_topics)

 

(begin_topics)

-Aspectos organizacionales para el uso de sistemas de software en organizaciones.
(begin_subtopics)
(end_subtopics)
(begin_goals)
(end_goals)
(end_topics)

(begin_topics)
-Procesos en la organización.
(begin_subtopics)
-Vista estratégica de los procesos organizacionales; conceptos de eficiencia y efectividad organizacional.
-Integración de las áreas funcionales de la organización.
-Procesos relacionados a los objetivos financieros, usuario final y orientados al producto.
-Innovación de procesos: análisis, modelado y simulación. 
(end_subtopics)
(begin_goals)
(end_goals)
(end_topics)

(begin_topics)
-Modelado y simulación de procesos de negocio.
(begin_subtopics)
-Automatización del proceso de negocios.
-Utilización de diagramas de actividad y de la Notación de Modelado de Procesos de Negocio (\emph{Business Process Modeling Notation} - BPMN - ) para el modelado del proceso de negocio.
-Herramientas para el modelado del proceso de negocio.
-Rediseño de tareas; impacto de la automatización en las prácticas de trabajo.
-Alcanzando seguridad y conformidad del proceso.
-Monitoreo y control de procesos.
-Administración de la cadena de abastecimiento (\emph{Supply Chain Management} - SCM). 
-Administración de la relación con los clientes (\emph{Customer Relationship Management} - CRM).
-Sistemas de gerenciamiento empresarial (\emph{Enterprise Management Systems} - ERP).
-El proceso continuo: de procesos estructurados a no estructurados.
(end_subtopics)
(begin_goals)
(end_goals)
(end_topics)

(begin_topics)
-Una vista integral de la firma y su relación con proveedores y clientes.
(begin_subtopics)
(end_subtopics)
(begin_goals)
(end_goals)
(end_topics)

\end{BKL2}



\begin{BKL2}{OMC2}{OMC2. Gerenciamiento de Sistemas de Información}
Horas:
 
(begin_topics)

-Planeamiento de Sistemas de Información.
(begin_subtopics)

-Alineamiento del planeamiento de Sistemas de Información con el planeamiento empresarial.

-Planeamiento estratégico de Sistemas de Información.

-Paneamiento de corto alcance de Sistemas de Información.

-Reingeniería.

-Mejora continua.
(end_subtopics)
(begin_goals)
(end_goals)
(end_topics)

 

(begin_topics)
-Control de la función de Sistemas de Información: p.e., auditoría, {\it outsourcing}.
(begin_subtopics)
(end_subtopics)
(begin_goals)
(end_goals)
(end_topics)

 

(begin_topics)

-Administración y contratación de recursos humanos.
(begin_subtopics)

-Planeamiento de habilidades.

-Administración del rendimiento del personal.

-Educación y entrenamiento.

-Estructuras de competencia, cooperación y premios.

(end_subtopics)
(begin_goals)
(end_goals)
(end_topics)

 

(begin_topics)
-Estructuras funcionales de Sistemas de Información -- internas vs. {\it outsourcing}.
(begin_subtopics)
(end_subtopics)
(begin_goals)
(end_goals)
(end_topics)

 

(begin_topics)

-Determinación de las metas y objetivos de la organización de Sistemas de Información.
(begin_subtopics)
(end_subtopics)
(begin_goals)
(end_goals)
(end_topics)

 

(begin_topics)

-Administración de Sistema de Información como un negocio: p.e., definición del cliente, definición de la misión de Sistemas de Información, aspectos críticos del éxito de Sitemas de Información.
(begin_subtopics)
(end_subtopics)
(begin_goals)
(end_goals)
(end_topics)

 

(begin_topics)
-Oficial de Información en Jefe (\emph{Chief Information Officer} - CIO) y funciones del personal.
(begin_subtopics)
(end_subtopics)
(begin_goals)
(end_goals)
(end_topics)

 

(begin_topics)
-Sistemas de Información como una función de servicio: evaluación del desempeño -- externo e interno, {\it marketing} de servicios.
(begin_subtopics)
(end_subtopics)
(begin_goals)
(end_goals)
(end_topics)

 

(begin_topics)

-Administración financiera de Sistemas de Información.
(begin_subtopics)
(end_subtopics)
(begin_goals)
(end_goals)
(end_topics)

 

(begin_topics)

-Uso estratégico de Sistemas de Información: p.e., ventajas competitivas y Sistemas de Información, proceso de reingeniería, Sistemas de Información y calidad.

(begin_subtopics)
(end_subtopics)
(begin_goals)
(end_goals)
(end_topics)

 

(begin_topics)
-Trabajo del conocimiento, computación de usuario final: soporte, roles, productividad y actividades.
(begin_subtopics)
(end_subtopics)
(begin_goals)
(end_goals)
(end_topics)

 

(begin_topics)

-Política de Sistemas de Información y formulación y comunicación de procesos operativos.
(begin_subtopics)
(end_subtopics)
(begin_goals)
(end_goals)
(end_topics)

 

(begin_topics)

-Copias de seguridad, planeamiento y recuperación de desastres.

(begin_subtopics)
(end_subtopics)
(begin_goals)
(end_goals)
(end_topics)

 

(begin_topics)
-Administración de tecnologías emergentes.
(begin_subtopics)
(end_subtopics)
(begin_goals)
(end_goals)
(end_topics)

 

(begin_topics)
-Administración de sub-funciones.
(begin_subtopics)
-Administración de telecomunicaciones.
-Administración de arquitecturas de computadores.
-Administración de sistemas de soporte a decisión de grupos.
-Administración de datos.
-Sistemas de aplicación y propiedad de datos.
-Optmización del ambiente para la creatividad-
-Administración de la calidad: p.e., ingeniería de calidad; equipos de control de calidad.
-Administración de las relaciones de consultoría, {\it outsourcing}.
-Administración para la contención de recursos.
-Asustos operativos asociados con la instalación, operación, transición y mantenimiento de sistemas.
-Actividades y disciplinas de controlque soportan la evolución del software.
-Actividades de ingeniería de software: desarrollo, control, administración, operaciones.
(end_subtopics)
(begin_goals)
(end_goals)
(end_topics)

 

(begin_topics)

-Seguridad y control, virus e integridad de sistemas.
(begin_subtopics)
-Cómo la información es comprometida incluyendo el acceso sin autorización, modificación de la información, bloqueo de servicios, virus.
-Crimen computacional, terrorismo y guerras cibernéticas.
-Virus computacionales, gusanos, caballos de troya.
-Fraude en Internet, leyendas urbanas.
-Correo no deseado (\emph{spam}), avisos publicitarios (\emph{adware}) y mensajes instantáneos no deseados (\emph{spIM}).
-Suplantación y \emph{phishing}.
-Medidas de seguridad computacionales incluyendo tecnológicas (acceso físico restringido, firewalls, encriptación y controles de auditoría) y métodos humanos (legal, ético y gerencia efectiva.)
-Planeamiento de seguridad computacional, incluyendo evaluación de riesgo, evaluación de políticas, implementación, entrenamiento y auditoría.
(end_subtopics)
(begin_goals)
(end_goals)
(end_topics)

(begin_topics)
-Administración de operaciones del computador: p.e. administración de cinta/DASD, {\it scheduling}.
(begin_subtopics)
(end_subtopics)
(begin_goals)
(end_goals)
(end_topics)

\end{BKL2}



\begin{BKL2}{OMC3}{OMC3. Teoría de Decisiones}
Horas:
 
(begin_topics)

-Medición y modelado.
(begin_subtopics)
(end_subtopics)
(begin_goals)
(end_goals)
(end_topics)

 

(begin_topics)

-Decisiones bajo certeza, incerteza, riesgo.

(begin_subtopics)
(end_subtopics)
(begin_goals)
(end_goals)
(end_topics)

 

(begin_topics)
-Información costo/valor, valor competitivo de Sistemas de Información.
(begin_subtopics)
-Motivación/propiedad del trabajo.
(end_subtopics)
(begin_goals)
(end_goals)
(end_topics)

 

(begin_topics)
-Modelos de decisión y Sistemas de Información: optimización, satisfacción.
(begin_subtopics)
(end_subtopics)
(begin_goals)
(end_goals)
(end_topics)

 

(begin_topics)

-Proceso de decisión de grupo.
(begin_subtopics)
(end_subtopics)
(begin_goals)
(end_goals)
(end_topics)

\end{BKL2}



\begin{BKL2}{OMC4}{OMC4. Comportamiento Organizacional}
Horas:
 
(begin_topics)

-Teoría de diseño del trabajo.
(begin_subtopics)
(end_subtopics)
(begin_goals)
(end_goals)
(end_topics)

 

(begin_topics)

-Diversidad cultural.
(begin_subtopics)
(end_subtopics)
(begin_goals)
(end_goals)
(end_topics)

 

(begin_topics)

-Dinámicas de grupo.
(begin_subtopics)
(end_subtopics)
(begin_goals)
(end_goals)
(end_topics)

 

(begin_topics)

-Trabajo en equipo, liderazgo y motivación.
(begin_subtopics)
(end_subtopics)
(begin_goals)
(end_goals)
(end_topics)

 

(begin_topics)

-Uso de influencias, poder y política.
(begin_subtopics)
(end_subtopics)
(begin_goals)
(end_goals)
(end_topics)

 

(begin_topics)

-Estilos cognitivos.
(begin_subtopics)
(end_subtopics)
(begin_goals)
(end_goals)
(end_topics)

 

(begin_topics)

-Negociación y estilos de negociación.
(begin_subtopics)
(end_subtopics)
(begin_goals)
(end_goals)
(end_topics)

 

(begin_topics)
-Construcción de consenso.
(begin_subtopics)
(end_subtopics)
(begin_goals)
(end_goals)
(end_topics)

(begin_topics)
-La organización virtual.
(begin_subtopics)
(end_subtopics)
(begin_goals)
(end_goals)
(end_topics)

(begin_topics)
-Implicaciones de la globalización.
(begin_subtopics)
(end_subtopics)
(begin_goals)
(end_goals)
(end_topics)

\end{BKL2}



\begin{BKL2}{OMC7}{OMC7. Manejo del Proceso de Cambio}
Horas:

 

(begin_topics)

-Razones para la resistencia al cambio.
(begin_subtopics)
(end_subtopics)
(begin_goals)
(end_goals)
(end_topics)

 

(begin_topics)

-Estrategias para motivar el cambio.
(begin_subtopics)
(end_subtopics)
(begin_goals)
(end_goals)
(end_topics)

 

(begin_topics)

-Planeamiento para el cambio.
(begin_subtopics)
(end_subtopics)
(begin_goals)
(end_goals)
(end_topics)

 

(begin_topics)

-Administración del cambio.
(begin_subtopics)
(end_subtopics)
(begin_goals)
(end_goals)
(end_topics)

 

(begin_topics)

-Modelado de procesos y sistemas.
(begin_subtopics)
(end_subtopics)
(begin_goals)
(end_goals)
(end_topics)

 

(begin_topics)

-Experimentación como un medio para capturar dinámicas.
(begin_subtopics)
(end_subtopics)
(begin_goals)
(end_goals)
(end_topics)

 

(begin_topics)

-Liderazgo en la reingeniería de procesos/software relacionado.
(begin_subtopics)
(end_subtopics)
(begin_goals)
(end_goals)
(end_topics)

 

(begin_topics)

-Estrategia de sobrellevado: {\it shock}, negación, ira, depresión, aceptación.
(begin_subtopics)
(end_subtopics)
(begin_goals)
(end_goals)
(end_topics)

 

(begin_topics)

-Aprendizaje en grupo/equipo.
(begin_subtopics)
(end_subtopics)
(begin_goals)
(end_goals)
(end_topics)


(begin_topics)
-Atributos de un agente de cambio.
(begin_subtopics)
-Escucha y comprensión.
-Mediación y negociación.
-Facilitación.
-Apreciación de la diferencia de tipos: Meyers Briggs, Rohm.
-Miedo y administración del miedo.
-Pensamiento de sistemas.
-Dominio personal.
-Modelos mentales.
-Visión de construcción compartida.
-Expresión de la necesidad e importancia del cambio.
(end_subtopics)
(begin_goals)
(end_goals)
(end_topics)

(begin_topics)
-El rol de los especialistas de Sistemas de Información como agentes de cambio.
(begin_subtopics)
(end_subtopics)
(begin_goals)
(end_goals)
(end_topics)

(begin_topics)
-Visualización del cambio y el proceso de cambio.
(begin_subtopics)
(end_subtopics)
(begin_goals)
(end_goals)
(end_topics)

(begin_topics)
-Diagnóstico y conceptualización del cambio.
(begin_subtopics)
(end_subtopics)
(begin_goals)
(end_goals)
(end_topics)

(begin_topics)
-Lidear con los retos de la implementación y entender y superar la resistencia.
(begin_subtopics)
(end_subtopics)
(begin_goals)
(end_goals)
(end_topics)

(begin_topics)
-Administrar políticas organizacionales.
(begin_subtopics)
(end_subtopics)
(begin_goals)
(end_goals)
(end_topics)

(begin_topics)
-Las limitaciones de los proyectos como iniciativas de cambios organizacionales.
(begin_subtopics)
(end_subtopics)
(begin_goals)
(end_goals)
(end_topics)

(begin_topics)
-Influencias organizacionales en el éxito del proyecto.
(begin_subtopics)
(end_subtopics)
(begin_goals)
(end_goals)
(end_topics)

\end{BKL2}



\begin{BKL2}{OMC8}{OMC8. Aspectos Legales y Éticos de los Sistemas de Información}
Horas:
 
(begin_topics)

-Ventas, licencias y agencias de software.
(begin_subtopics)
(end_subtopics)
(begin_goals)
(end_goals)
(end_topics)

 

(begin_topics)

-Fundamentos contractuales.
(begin_subtopics)

-Ley contractual.
(end_subtopics)
(begin_goals)
(end_goals)
(end_topics)

 

(begin_topics)
-Ley de privacidad.
(begin_subtopics)
(end_subtopics)
(begin_goals)
(end_goals)
(end_topics)

 

(begin_topics)

-Agencias y organismos regulatorios.
(begin_subtopics)
(end_subtopics)
(begin_goals)
(end_goals)
(end_topics)

 

(begin_topics)

-Ética y protección de los derechos de propiedad intelectual.
(begin_subtopics)

-Protección de la propiedad intelectual.

-Formas de propiedad intelectual, medios para protegerla y penas por violación de derechos.

-Ética (plagiarismo, honestidad, privacidad, hackers): uso, maluso y límites de derechos.

(end_subtopics)
(begin_goals)
(end_goals)
(end_topics)

 

(begin_topics)

-Ética: responsabilidad y códigos personales y profesionales; modelos éticos.
(begin_subtopics)

-Responsabilidad personal: principios de honestidad, justicia, autonomía.

-Responsabilidad profesional: expectativas y confianza debido al conocimiento y habilidades.

-Códigos profesionales de conducta ética para profesionales de la computación responsables.

-Motivación hacia e importancia del comportamiento ético.

-Modelos éticos: utilitarismo de Bentham, imperativa moral de Kant.

-Elementos del análisis ético: argumentación por ejemplo, analogía y contra ejemplo.

-Análisis social: influencia contextual social en el desarrollo y la tecnología.
(end_subtopics)
(begin_goals)
(end_goals)
(end_topics)

 

(begin_topics)

-Riesgos, pérdidas y responsabilidad en las aplicaciones de computación.
(begin_subtopics)
(end_subtopics)
(begin_goals)
(end_goals)
(end_topics)

 

(begin_topics)
-Garantías.
(begin_subtopics)
(end_subtopics)
(begin_goals)
(end_goals)
(end_topics)


(begin_topics)
-Crímenes de computador.
(begin_subtopics)
-Virus y otros daños al software.
-Fraude de software, abusos, hackers.
(end_subtopics)
(begin_goals)
(end_goals)
(end_topics)

(begin_topics)
-Ética en los sitemas de información.
(begin_subtopics)
-Aspectos éticos relacionados a la privacidad, accesibilidad, propiedad y precisión de la información.
-Monitoreo de empleados y políticas de uso aceptables.
-Vicios generados por Internet y lo bueno de la sociedad.
-Leyes y regulaciones actuales.
-Lineamientos éticos para profesionales de computación.
-Asuntos éticos relacionados a la recuperación y minería de datos.
-Globalización y desarrollo dentro y fuera de la organización.
-Faltas relacionadas a los derechos de autor y propiedad intelectual, el rol de las redes punto-a-punto (\emph{peer-to-peer}).
-Movilidad, virtualización y privacidad.
-Blogs y los medios.
(end_subtopics)
(begin_goals)
(end_goals)
(end_topics)

\end{BKL2}



\begin{BKL2}{OMC9}{OMC9. Profesionalismo}
Horas:
 
(begin_topics)

-Literatura periódica actual, revistas profesionales y académicas.
(begin_subtopics)
(end_subtopics)
(begin_goals)
(end_goals)
(end_topics)

 

(begin_topics)

-Temas de certificación.
(begin_subtopics)
(end_subtopics)
(begin_goals)
(end_goals)
(end_topics)

 

(begin_topics)

-Organizaciones profesionales: p.e. SPC, ACM, IEEE-CS, TIMS, ASM, DSI, ACE,  ASQC, AIS.
(begin_subtopics)
(end_subtopics)
(begin_goals)
(end_goals)
(end_topics)

 

(begin_topics)
-Conferencias profesionales.
(begin_subtopics)
(end_subtopics)
(begin_goals)
(end_goals)
(end_topics)

 

(begin_topics)
-Pendiente.
(begin_subtopics)
(end_subtopics)
(begin_goals)
(end_goals)
(end_topics)

(begin_topics)

-Industria de Sistemas de Información: fabricantes, OEMs, integradores de sistemas, desarrolladores de software.
(begin_subtopics)
(end_subtopics)
(begin_goals)
(end_goals)
(end_topics)

 

(begin_topics)
-Contexto social e histórico de la Computación.
(begin_subtopics)
(end_subtopics)
(begin_goals)
(end_goals)
(end_topics)

\end{BKL2}



\begin{BKL2}{OMC10}{OMC10. Habilidades Interpersonales}
Horas:
 
(begin_topics)
-Habilidades de comunicación.
(begin_subtopics)
(end_subtopics)
(begin_goals)
(end_goals)
(end_topics)

 

(begin_topics)

-Entrevistas, cuestionarios y escucha.
(begin_subtopics)
(end_subtopics)
(begin_goals)
(end_goals)
(end_topics)

 

(begin_topics)

-Habilidades de presentación.
(begin_subtopics)

-Comunicación oral y escrita.

-Gráficos y el uso de multimedia.

-Entrenamiento: objetivos, metas, basado en computador.
(end_subtopics)
(begin_goals)
(end_goals)
(end_topics)

 

(begin_topics)
-Habilidades de consultoría.
(begin_subtopics)
(end_subtopics)
(begin_goals)
(end_goals)
(end_topics)

 

(begin_topics)

-Habilidades de escritura.
(begin_subtopics)

-Fundamentos de escritura técnica.

-Principios y estándares para documentación.

-Desarrollo de documentación de software.

-Herramientas de documentación.

-Escritura como un medio para al aprendizaje de por vida.

-Escritura de diarios como un método para capturar observaciones.

-Escritura de soluciones a problemas y respuestas a asuntos para explorar el conocimiento.

(end_subtopics)
(begin_goals)
(end_goals)
(end_topics)

 

(begin_topics)

-Actitud y método proactivo.

(begin_subtopics)
(end_subtopics)
(begin_goals)
(end_goals)
(end_topics)

 

(begin_topics)

-Determinación de objetivos personales, toma de decisiones y administración del tiempo.
(begin_subtopics)
(end_subtopics)
(begin_goals)
(end_goals)
(end_topics)

 

(begin_topics)

-Liderazgo centrado en principios.
(begin_subtopics)
(end_subtopics)
(begin_goals)
(end_goals)
(end_topics)

 

(begin_topics)

-Principios de negociación.
(begin_subtopics)
(end_subtopics)
(begin_goals)
(end_goals)
(end_topics)

 

(begin_topics)

-Promoviendo la creatividad y la búsqueda de oportunidades.
(begin_subtopics)
(end_subtopics)
(begin_goals)
(end_goals)
(end_topics)

 

(begin_topics)

-Pensamiento crítico.

(begin_subtopics)

-Abstraer/depurar información.

-Priorizar tareas.

-Pensamiento {\it outside the box}.

-Juicio suspendido.

-Lluvia de ideas para la creación de nuevas ideas.

-Estrategias de prensamiento divergente, lateral, lineal.

-Formulación de preguntas.

-Asumpción de responsabilidades, toma de decisiones, delegación de responsabilidades.

-Trabajo con múltiples y diferentes puntos de vista.

-El trato a otras personas  con respeto y tolerancia.

-Dar y recibir críticas constructivas.

(end_subtopics)
(begin_goals)
(end_goals)
(end_topics)

 
(begin_topics)
-Medición e interpretación de datos.
(begin_subtopics)
-Evaluación de datos en contexto.
-Medición/determinación de los datos.
-Conversión y presentación de datos.
-Transformaciones algebráicas y funcionales de los datos.
-Almacenamiento y organización de datos.
-Usos de los datos en el modelado y simulación de un proceso.
-Conceptos de muestreo en la determinación de los datos.
-Validación de procesos a través del uso de los datos.
(end_subtopics)
(begin_goals)
(end_goals)
(end_topics)

(begin_topics)
-Pendiente.
(begin_subtopics)
(end_subtopics)
(begin_goals)
(end_goals)
(end_topics)

(begin_topics)
-Resolución de problemas.
(begin_subtopics)
-Ámbito y restricciones del problema.
-Expectativas, precisión y limitaciones de tiempo de la solución.
-Balance entre perfección y realismo, recursos y expectativas.
-Criterios de completación del problema.
-Ser creativo, realizar inferencias, deterctar y evitar falacias lógias.
-Determinación de las relaciones entre los componentes.
-Consideración de alternativas y soluciones relacionadas.
-Planeamiento, estimativa y documentación de las fases, resultados y actividades del proceso.
-Tiempo comprometido, presupuestos, multi-tarea, esfuerzo de contabiildad, balanceo de carga de trabajo.
-Verificación de la solución y validación del resultado.
(end_subtopics)
(begin_goals)
(end_goals)
(end_topics)
\end{BKL2}

\begin{BKL2}{OMC11}{OMC11. Funciones Organizacionales Fundamentales}
Horas:
 
(begin_topics)
-Pagos.
(begin_subtopics)
-Conceptos de administración del dinero.
-Dinero digital.
-Flujos de dinero transnacionales.
-Flujos de crédito, transacciones, aprobaciones.
-Conjuntos de transacción EDI.
-Lenguajes de marcado financieros, p.e., IFX.
-Interfaces de sistemas de contabilidad: GL, AR, AP.
-Balance y controles de auditoría.
(end_subtopics)
(begin_goals)
(end_goals)
(end_topics)

 

(begin_topics)
-Relaciones de negocio: C-B, C-C, C-G, B-B, B-G, G-G.
(begin_subtopics)
(end_subtopics)
(begin_goals)
(end_goals)
(end_topics)

 
(begin_topics)
-Modelos de negocio; convencional/e-commerce.
(begin_subtopics)
-Mercados e infraestructura de compras.
-Subastas.
-Bolsa de valores.
-Comunidad y colaboración.
-Proveedor de información.
-Proveedor de servicios.
-Portal.
(end_subtopics)
(begin_goals)
(end_goals)
(end_topics)

 
(begin_topics)
-Conceptos de cadena de valor.
(begin_subtopics)
(end_subtopics)
(begin_goals)
(end_goals)
(end_topics)

 
(begin_topics)
-Conceptos de administración de cadena de recursos.
(begin_subtopics)
(end_subtopics)
(begin_goals)
(end_goals)
(end_topics)

 
(begin_topics)
-Atención.
(begin_subtopics)
(end_subtopics)
(begin_goals)
(end_goals)
(end_topics)

 
(begin_topics)
-{\it Marketing} y propaganda.
(begin_subtopics)
(end_subtopics)
(begin_goals)
(end_goals)
(end_topics)

 
(begin_topics)
-Comercio.
(begin_subtopics)
-Distribución.
-Ventas directas.
-Personalización por cliente.
-Portales y canales.
(end_subtopics)
(begin_goals)
(end_goals)
(end_topics)

 

(begin_topics)
-Manufactura y producción.
(begin_subtopics)
(end_subtopics)
(begin_goals)
(end_goals)
(end_topics)

 
(begin_topics)
-Administración de recursos humanos.
(begin_subtopics)
(end_subtopics)
(begin_goals)
(end_goals)
(end_topics)

 

(begin_topics)
-Administración de inventario.
(begin_subtopics)
(end_subtopics)
(begin_goals)
(end_goals)
(end_topics)

 

(begin_topics)
-Despacho.
(begin_subtopics)
(end_subtopics)
(begin_goals)
(end_goals)
(end_topics)

 

(begin_topics)
-Adquisición.
(begin_subtopics)
(end_subtopics)
(begin_goals)
(end_goals)
(end_topics)

 
(begin_topics)
-Procesamiento de órdenes y servicio al cliente.
(begin_subtopics)
(end_subtopics)
(begin_goals)
(end_goals)
(end_topics)

 

(begin_topics)
-Auditorías y controles.
(begin_subtopics)
(end_subtopics)
(begin_goals)
(end_goals)
(end_topics)

\end{BKL2}



\begin{BKL2}{OMC12}{OMC12. Sistemas y Tecnologías de Información en el Negocio}
Horas:
(begin_topics)
-La función de Sistemas de Información.
(begin_subtopics)
-Procesos de negocio clave basados en Tecnología de Información.
-Estructura organizacional de la Tecnología de Información y alternativas de gobierno.
-Necesidades de talento humano y métodos de gerencia.
-Métodos para medir y demostrar el valor dela Tecnología de Información.
-Métodos de organización para asegurar el cumplimiento de las regulaciones.
-Administración de fuentes de tecnología.
(end_subtopics)
(begin_goals)
(end_goals)
(end_topics)

(begin_topics)
-Las Tecnologías de Información.
(begin_subtopics)
-Evaluación y selección de la plataforma y arquitectura, prioridades y políticas.
-Evaluación del impacto de tecnologías emergentes.
-Evaluación del rol de los estándares.
-Evaluación del efecto de las estrategias de venta.
(end_subtopics)
(begin_goals)
(end_goals)
(end_topics)

\end{BKL2}

\begin{BKL2}{TDS1}{TDS1. Conceptos de Información y de Sistemas}
Horas:
 
(begin_topics)
-Teoría general de sistemas.
(begin_subtopics)
(end_subtopics)
(begin_goals)
(end_goals)
(end_topics)

 
(begin_topics)
-Conceptos de sistemas: p.e. estructura, límites, estados, objetivos.
(begin_subtopics)
-Conceptos fundamentales de teoría de información.
-Razonamiento acerca de sistemas organizacionales, productos y procesos de software.
-Relaciones de usuarios y proveedores al sistema.
(end_subtopics)
(begin_goals)
(end_goals)
(end_topics)

 

(begin_topics)
-Propiedades de sistemas abiertos.
(begin_subtopics)
(end_subtopics)
(begin_goals)
(end_goals)
(end_topics)

 
(begin_topics)
-Componentes y relaciones del sistema.
(begin_subtopics)
(end_subtopics)
(begin_goals)
(end_goals)
(end_topics)

 
(begin_topics)
-Controles del sistema: estándares, teoría de control, retroalimentación, bucles, mediciones, calidad.
(begin_subtopics)
(end_subtopics)
(begin_goals)
(end_goals)
(end_topics)

 
(begin_topics)
-Propiedades de los sistemas de información.
(begin_subtopics)
(end_subtopics)
(begin_goals)
(end_goals)
(end_topics)

\end{BKL2}



\begin{BKL2}{TDS2}{TDS2. Metodologías para el Desarrollo de Sistemas}
Horas:
 
(begin_topics)
-Modelos de desarrollo de sistemas: p.e. SDLC, prototipado.
(begin_subtopics)
-Cilco de vida del desarrollo de sistemas: modelos de ciclos de vida del software.
-Desarrollo con prototipado.
-Desarrollo con paquetes.
-Técnicas de desarrollo orientadas a los datos.
-Técnicas de desarrollo orientadas a los procesos.
-Técnicas de desarrollo orientadas a objetos: diseño {\it bottom-up}; soporte para el reuso.
-Consideraciones de ingeniería de sistemas.
-Software como un componente de un sistema.
-Proceso de software y modelos del ciclo de vida de un producto.
-Métodos y herramientas de generación de software: diseño y codificación desde cero.
-Métodos y herramientas de diseño de sistemas.
(end_subtopics)
(begin_goals)
(end_goals)
(end_topics)

 
(begin_topics)
-Adquisición e implementación de paquetes.
(begin_subtopics)
(end_subtopics)
(begin_goals)
(end_goals)
(end_topics)

 
(begin_topics)
-Integración de los componentes de software.
(begin_subtopics)
(end_subtopics)
(begin_goals)
(end_goals)
(end_topics)

 
(begin_topics)
-Sistemas desarrollados por el usuario final.
(begin_subtopics)
(end_subtopics)
(begin_goals)
(end_goals)
(end_topics)

 
(begin_topics)
-Selección de una metodología de desarrollo de sistemas.
(begin_subtopics)
(end_subtopics)
(begin_goals)
(end_goals)
(end_topics)

\end{BKL2}



\begin{BKL2}{TDS3}{TDS3. Conceptos y Metodologías para el Desarrollo de Sistemas}
Horas:
 
(begin_topics)
-Modelado de procesos organizacionales y de software.
(begin_subtopics)
-Conceptos de modelado.
-Conceptos avanzados de modelado, incluyendo modelos asíncronos y paralelos.
(end_subtopics)
(begin_goals)
(end_goals)
(end_topics)

 
(begin_topics)
-Modelado de datos: p.e. diagramas entidad-relación, normalización.
(begin_subtopics)
(end_subtopics)
(begin_goals)
(end_goals)
(end_topics)

 
(begin_topics)
-Metodologías orientadas a datos.
(begin_subtopics)
(end_subtopics)
(begin_goals)
(end_goals)
(end_topics)

 
(begin_topics)
-Metodologías orientadas a los procesos.
(begin_subtopics)
(end_subtopics)
(begin_goals)
(end_goals)
(end_topics)

 
(begin_topics)
-Metodologías orientadas al comportamiento (modelado de eventos).
(begin_subtopics)
(end_subtopics)
(begin_goals)
(end_goals)
(end_topics)

 

(begin_topics)
-Metodologías orientadas a objetos.
(begin_subtopics)
(end_subtopics)
(begin_goals)
(end_goals)
(end_topics)

 
(begin_topics)
-Procesos y productos de la ingeniería de software.
(begin_subtopics)
(end_subtopics)
(begin_goals)
(end_goals)
(end_topics)

\end{BKL2}



\begin{BKL2}{TDS4}{TDS4. Herramientas y Técnicas para el Desarrollo de Sistemas}
Horas:
 
(begin_topics)
-CASE
(begin_subtopics)
-Metodologías (ingeniería de información, técnicas de Jackson, Yourdon).
-Herramientas: herramientas CASE, generadores de código, GDSS.
-Herramientas: IDEF y otras herramientas de especificación y diseño; diseño e implementación de bases de datos.
(end_subtopics)
(begin_goals)
(end_goals)
(end_topics)

 
(begin_topics)
-Métodos basados en grupos: p.e. JAD, revisiones de diseño y código.
(begin_subtopics)
(end_subtopics)
(begin_goals)
(end_goals)
(end_topics)

 
(begin_topics)
-Conceptos y herramientas de implementación de software: p.e., diccionarios de datos, repositorios, aplicación.
(begin_subtopics)
(end_subtopics)
(begin_goals)
(end_goals)
(end_topics)

\end{BKL2}



\begin{BKL2}{TDS5}{TDS5. Planeamiento de Aplicaciones}
Horas:
 
(begin_topics)
-Planeamiento de infraestructuras: hardware, comunicaciones, bases de datos.
(begin_subtopics)
(end_subtopics)
(begin_goals)
(end_goals)
(end_topics)

 
(begin_topics)
-Planeamiento de la arquitectura de Sistemas de Información.
(begin_subtopics)
(end_subtopics)
(begin_goals)
(end_goals)
(end_topics)

 
(begin_topics)
-Planeamiento para operaciones.
(begin_subtopics)
(end_subtopics)
(begin_goals)
(end_goals)
(end_topics)

 
(begin_topics)
-Métricas para tamaño, puntos de función, control de complejidad.
(begin_subtopics)
(end_subtopics)
(begin_goals)
(end_goals)
(end_topics)

 
(begin_topics)
-Planeamiento para securidad, privacidad y control de Sistemas de Información.
(begin_subtopics)
(end_subtopics)
(begin_goals)
(end_goals)
(end_topics)

\end{BKL2}



\begin{BKL2}{TDS6}{TDS6. Manejo de Riesgos}
Horas:
 
(begin_topics)
-Estimativa de la viabilidad.
(begin_subtopics)
(end_subtopics)
(begin_goals)
(end_goals)
(end_topics)

 
(begin_topics)
-Principios de administración de riesgos.
(begin_subtopics)
(end_subtopics)
(begin_goals)
(end_goals)
(end_topics)

 
(begin_topics)
-Planeamiento de contigencias.
(begin_subtopics)
(end_subtopics)
(begin_goals)
(end_goals)
(end_topics)

\end{BKL2}



\begin{BKL2}{TDS7}{TDS7. Gerenciamiento de Proyectos}
Horas:
 
(begin_topics)

-Planeamiento de proyectos y selección de un modelo de procesos apropiado; cronograma del proyecto y {\it milestones}.
(begin_subtopics)
(end_subtopics)
(begin_goals)
(end_goals)
(end_topics)

 

(begin_topics)

-Organización, administración, principios y conceptos de proyectos.

(begin_subtopics)

-Asuntos organizacionales de la administración de proyectos.

-Principios y conceptos de la administración de proyectos.
(end_subtopics)
(begin_goals)
(end_goals)
(end_topics)

 

(begin_topics)

-Estructuras de división y cronograma de trabajo.
(begin_subtopics)
(end_subtopics)
(begin_goals)
(end_goals)
(end_topics)

 

(begin_topics)

-Consideraciones de personal para proyectos: p.e., administración de matriz, factores humanos, organización de los equipos.

(begin_subtopics)
(end_subtopics)
(begin_goals)
(end_goals)
(end_topics)

 

(begin_topics)

-Control de proyectos: planeamiento, estimativa de costos, asignación de recursos, revisiones técnicas de software.
(begin_subtopics)

-Documentación de la administración del proyecto.

-Representaciones del cronograma del proyecto.

-Aspectos económicos del proyecto: técnias y herramientas de estimación de costos; análisis costo/beneficio.

-Herramientas de cronogramas de proyectos.

(end_subtopics)
(begin_goals)
(end_goals)
(end_topics)

 

(begin_topics)
-Administración de múltiples proyectos.
(begin_subtopics)
(end_subtopics)
(begin_goals)
(end_goals)
(end_topics)

 

(begin_topics)

-Consideraciones de administración; estrés y administración del tiempo.
(begin_subtopics)
(end_subtopics)
(begin_goals)
(end_goals)
(end_topics)

 

(begin_topics)

-Documentación de sistemas.
(begin_subtopics)
(end_subtopics)
(begin_goals)
(end_goals)
(end_topics)

 

(begin_topics)
-Documentación de usuario (p.e., manuales de referencia, procedimientos de operación, documentación {\it on-line}).
(begin_subtopics)
(end_subtopics)
(begin_goals)
(end_goals)
(end_topics)

 

(begin_topics)
-Métricas de sistemas.
(begin_subtopics)
(end_subtopics)
(begin_goals)
(end_goals)
(end_topics)

 

(begin_topics)

-Contextualización y control de alcance.

(begin_subtopics)
(end_subtopics)
(begin_goals)
(end_goals)
(end_topics)

 

(begin_topics)

-Administración de la configuración.
(begin_subtopics)

-Principios y conceptos de la administración de la configuración.

-Rol en el control de la evolución del sistema.

-Rol en la manuntención de la integridad del producto.

-Documentación: controles de cambio, controles de versión, etc.

-Estructuras organizacionales para la administración de la configuración.

-Planes de administración de la configuración.

-Herramientas de administración de la configuración.

(end_subtopics)
(begin_goals)
(end_goals)
(end_topics)

 

(begin_topics)

-Aseguramiento de la calidad del desarrollo de sistemas.
(begin_subtopics)
(end_subtopics)
(begin_goals)
(end_goals)
(end_topics)

 

(begin_topics)

-Seguimiento del proyecto: p.e., PERT, Gantt.
(begin_subtopics)
(end_subtopics)
(begin_goals)
(end_goals)
(end_topics)

 

(begin_topics)

-Finalización y cierre del proyecto.
(begin_subtopics)
(end_subtopics)
(begin_goals)
(end_goals)
(end_topics)

(begin_topics)
-Recursos de administración de proyectos de software y desarrollo personal tales como SMI y PMI.
(begin_subtopics)
(end_subtopics)
(begin_goals)
(end_goals)
(end_topics)

(begin_topics)
-Actividades adicionales requeridas para asegurar el éxito de los proyectos de tecnologías de información (entrenamiento, rediseño del trabajo, comunicación, etc.)
(begin_subtopics)
(end_subtopics)
(begin_goals)
(end_goals)
(end_topics)

(begin_topics)
-Administración de sociedades y definición del contrato y relaciones.
(begin_subtopics)
(end_subtopics)
(begin_goals)
(end_goals)
(end_topics)

(begin_topics)
-Experiencia en campo del uso de software de administración de proyectos.
(begin_subtopics)
(end_subtopics)
(begin_goals)
(end_goals)
(end_topics)



\end{BKL2}



\begin{BKL2}{TDS8}{TDS8. Análisis de Información y de Negocio}
Horas:
 
(begin_topics)

-Identificación de oportunidades: p.e., solicitudes de servicio, a partir de proceso de planeamiento.
(begin_subtopics)
(end_subtopics)
(begin_goals)
(end_goals)
(end_topics)

 

(begin_topics)

-Relación de la aplicación con el modelo empresarial.
(begin_subtopics)
(end_subtopics)
(begin_goals)
(end_goals)
(end_topics)

 

(begin_topics)

-Determinación y especificación de requerimientos.
(begin_subtopics)
(end_subtopics)
(begin_goals)
(end_goals)
(end_topics)

\end{BKL2}



\begin{BKL2}{TDS9}{TDS9. Diseño de Sistemas de Información}
Horas:
 
(begin_topics)

-Diseño: lógico, físico.
(begin_subtopics)

-Métodos y herramientas de diseño de sistemas.

-Rol del diseño de software vs. diseño del sistema.

-Balance de hardware-software para el desempeño y flexibilidad del sistema.

-Diseño de interfaces de alto nivel, hardware a hardware y software a software.

-Predicción del desempeño del sistema.

-Técnicas y representaciones del modelado del sistema.

-Técnicas de diseño de sistemas orientadas a objetos.

-Técnicas de diseño de sistemas: diseño iterativo, modelado, etc.

-Flexibilidad del diseño de sistemas.
(end_subtopics)
(begin_goals)
(end_goals)
(end_topics)

 

(begin_topics)

-Metodologías de sistemas: p.e., en tiempo real, orientadas a objetos, estructuradas, basadas en eventos.
(begin_subtopics)
(end_subtopics)
(begin_goals)
(end_goals)
(end_topics)

 

(begin_topics)

-Objetivos de diseño: p.e., usabilidad, desempeño.
(begin_subtopics)
(end_subtopics)
(begin_goals)
(end_goals)
(end_topics)

 

(begin_topics)

-Técnicas para mejorar el proceso de diseño creativo.
(begin_subtopics)
(end_subtopics)
(begin_goals)
(end_goals)
(end_topics)

 

(begin_topics)

-Alternativas de presentación de información; estilos cognitivos.
(begin_subtopics)
(end_subtopics)
(begin_goals)
(end_goals)
(end_topics)

 

(begin_topics)

-Interacción humano-computador (p.e., ergonomía, interfaces gráficas de usuario, voz, tacto).
(begin_subtopics)

-Interfaces de usuario (voz, tacto...).

-Ergonomía.

-Acceso de usuario común.

-Interfaces de usario; menús del sistema, lenguajes de comandos, manipulación directa.

-Dispositivos de salida gráficos y sus propiedades.

-Primitivas gráficas y sus propiedades.

-Sistemas de software gráficos; estándares gráficos generales.

-Arquitectura de administradores de ventada e interfaces de usuario.

-Arquitectura de cajas de herramientas y ambientes de soporte a la programación.

-Representación de datos gráficos y sonido.

-Técnicas de diseño para problemas de interfaz humano-computador: independencia del dispositivo.

-Factores humanos asociados con interfaces humano-computador.
(end_subtopics)
(begin_goals)
(end_goals)
(end_topics)

 

(begin_topics)

-Desarrollo de software.
(begin_subtopics)

-Requerimientos de software: principios; tipos (funcional, desempeño y otros).

-Especificaciones de software: objetivos; estándares; tipos (funcional, desempeño).

-Diseño de software: principios de diseño (abstracción, ocultamiento de información).

-Aseguramiento de la calidad del software: problemas, definiciones, estándares, aseguramiento de calidad como un proceso continuo).

-Correctitud y confiabilidad del software: principios, conceptos, modelado, métodos.

-Verificación y validación del aseguramiento de la calidad del software: roles y métodos.

-Implementación del software: relación del diseño con la implementación del software.

-Integración de sistemas de software y hardware: métodos, planes, pruebas.

-Pruebas de software: roles, principios y estándares.

(end_subtopics)
(begin_goals)
(end_goals)
(end_topics)

\end{BKL2}



\begin{BKL2}{TDS10}{TDS10. Estrategias de Implementación y Pruebas de Sistemas}
Horas:
 
(begin_topics)

-Construcción de sistemas.
(begin_subtopics)
(end_subtopics)
(begin_goals)
(end_goals)
(end_topics)

 

(begin_topics)

-Construcción de sistemas de software: p.e. programación, pruebas de unidad, empaquetamiento de módulos.
(begin_subtopics)
(end_subtopics)
(begin_goals)
(end_goals)
(end_topics)

 

(begin_topics)
-Integración de software: p.e., paquetes.
(begin_subtopics)
(end_subtopics)
(begin_goals)
(end_goals)
(end_topics)

 

(begin_topics)

-Conversión de sistemas: métodos, planeamiento, implementación.
(begin_subtopics)
(end_subtopics)
(begin_goals)
(end_goals)
(end_topics)

 

(begin_topics)

-Integración y pruebas de sistemas: verificación y validación, generación del plan de pruebas, ejecución de pruebas.
(begin_subtopics)
(end_subtopics)
(begin_goals)
(end_goals)
(end_topics)

 

(begin_topics)

-Entrenamiento de usuarios, gerenciamiento, operación, materiales de entrenamiento.
(begin_subtopics)
(end_subtopics)
(begin_goals)
(end_goals)
(end_topics)

 

(begin_topics)

-Gerenciamiento de proyectos de software: ámbito, planificación, manejo de la configuración, calidad.
(begin_subtopics)
(end_subtopics)
(begin_goals)
(end_goals)
(end_topics)

 

(begin_topics)

-Sistemas de instalación.
(begin_subtopics)
(end_subtopics)
(begin_goals)
(end_goals)
(end_topics)

 

(begin_topics)

-Revisión post-implementación.
(begin_subtopics)
(end_subtopics)
(begin_goals)
(end_goals)
(end_topics)

\end{BKL2}



\begin{BKL2}{TDS11}{TDS11. Operación y Mantenimiento de Sistemas}
Horas:
 
(begin_topics)

-Petición de servicios y control de cambios.
(begin_subtopics)
(end_subtopics)
(begin_goals)
(end_goals)
(end_topics)

 

(begin_topics)
-Ingeniería inversa y reingeniería.
(begin_subtopics)
(end_subtopics)
(begin_goals)
(end_goals)
(end_topics)

 

(begin_topics)

-Balanceo y afinamiento
(begin_subtopics)
(end_subtopics)
(begin_goals)
(end_goals)
(end_topics)

 

(begin_topics)

-Conceptos de mantenimiento de software.
(begin_subtopics)

-Tipos de mantenimiento de software: perceptivo, adaptativo, correctivo.

-Diseño de software para mantenimiento.

-Técnicas de mantenimiento de Software: lectura de programas, ingeniería reversa.

-Modelos de mantenimiento de Software.
(end_subtopics)
(begin_goals)
(end_goals)
(end_topics)

\end{BKL2}



\begin{BKL2}{TDS12}{TDS12. Desarrollo de Sistemas para Tipos Específicos de Sistemas de Información}
Horas:
 
(begin_topics)

-Sistemas para el procesamiento de transacciones.
(begin_subtopics)
(end_subtopics)
(begin_goals)
(end_goals)
(end_topics)

 

(begin_topics)

-Sistemas de Información Gerencial
(begin_subtopics)
(end_subtopics)
(begin_goals)
(end_goals)
(end_topics)

 

(begin_topics)

-Sistemas para el soporte de grupos.
(begin_subtopics)
(end_subtopics)
(begin_goals)
(end_goals)
(end_topics)

 

(begin_topics)

-Sistemas para soporte a la toma de decisiones/Sistemas expertos.
(begin_subtopics)
(end_subtopics)
(begin_goals)
(end_goals)
(end_topics)

 

(begin_topics)

-Sistemas de soporte ejecutivo.
(begin_subtopics)
(end_subtopics)
(begin_goals)
(end_goals)
(end_topics)

 

(begin_topics)
-Sistemas de oficina.
(begin_subtopics)
(end_subtopics)
(begin_goals)
(end_goals)
(end_topics)

 

(begin_topics)
-Sistemas Colaborativos.
(begin_subtopics)
(end_subtopics)
(begin_goals)
(end_goals)
(end_topics)

 

(begin_topics)

-Imagen y Sistemas de Flujos de Información ({\it Work-flow systems}).
(begin_subtopics)
(end_subtopics)
(begin_goals)
(end_goals)
(end_topics)

 

(begin_topics)

-Sistemas de soporte funcional: e.g. control de procesos, marketing.
(begin_subtopics)
(end_subtopics)
(begin_goals)
(end_goals)
(end_topics)

 

(begin_topics)
-Sistemas Interorganizacionales.
(begin_subtopics)
(end_subtopics)
(begin_goals)
(end_goals)
(end_topics)

(begin_topics)
-Sistemas de administración del conocimiento.
(begin_subtopics)
(end_subtopics)
(begin_goals)
(end_goals)
(end_topics)

(begin_topics)
-Procesos que cubren el mundo; mercados virtuales globales.
(begin_subtopics)
(end_subtopics)
(begin_goals)
(end_goals)
(end_topics)

(begin_topics)
-Comercio electrónico.
(begin_subtopics)
(end_subtopics)
(begin_goals)
(end_goals)
(end_topics)

\end{BKL2}


\begin{BKL2}{TDS13}{TDS13. Tecnologías Emergentes}
Horas:
 
(begin_topics)
-Conceptos de tecnologías emergentes.
(begin_subtopics)
-Definición de tecnologías emergentes.
-Impacto económico actual y potencial.
-Proyecciones de impactos económicos futuros.
-Creación y transformación de bienes y servicios a través de tecnologías emergentes.
-Impacto en organizaciones, mercados, industria y sociedad.
(end_subtopics)
(begin_goals)
(end_goals)
(end_topics)

(begin_topics)
-Reacciones de la organización frente a las tecnologías emergentes.
(begin_subtopics)
-El remodelado de las organizaciones por parte de las tecnologías emergentes.
-Reacción de los negocios frente a las tecnologías emergentes.
-Efecto del incremento de la transparencia en el balance entre el poder de los consumidores y los productores.
-Efecto de los mercados ``libres de fricción'' en el precio y posicionamiento de productos y servicios.
-Teorías de innovación tecnológica.
-Examen de las causas y efectos de revoluciones tecnológicas pasadas (p.e. electricidad, ferrovías).
-Relaciones de causa y efecto entre las nuevas tecnologías electrónicas y la globalización económica en aumento.
-Información: costo, valor y precio; efecto de derechos de autor en la industria de tecnologías de información; contraste de la información como bien público y como recurso propietario.
-Modelos de negocio: el esfuerzo de las empresas por generar ganacias usando Internet y tecnologías emergentes 
(end_subtopics)
(begin_goals)
(end_goals)
(end_topics)

(begin_topics)
-Tecnologías emergentes desde el punto de vista tecnológico.
(begin_subtopics)
-Nanotecnologías, RFID y otros.
-Open Source.
-Convergencia de tecnologías.
-Posible evolución de la tecnología en el futuro cercano.
(end_subtopics)
(begin_goals)
(end_goals)
(end_topics)


\end{BKL2}
