% File generated by generate_IS_LU ... do not modify manually !!!
\subsection{LU1. Sistemas y conceptos de tecnología de Informacion}\label{sec:LU1}
\begin{LearningUnit}
\begin{LUGoal}
\item Introducir a los nuevos usuarios a los conceptos de Sistemas y de Tecnología de Información.
\end{LUGoal}

\begin{LUObjective}
\item Describir y explicar en términos de sistemas los componentes de hardware y software de un sistema computacional.
\item Describir, explicar y utilizar un Sistema Operativo e interfaz de usuario para instalar y operar programas, definir y proteger archivos de datos así como utilizar utilitarios del Sistema Operativo.
\item Definir, explicar y utilizar software para el trabajo con conocimiento.
\end{LUObjective}
\end{LearningUnit}

\subsection{LU2. Software de trabajo de conocimiento}\label{sec:LU2}
\begin{LearningUnit}
\begin{LUGoal}
\item Desarrollar las habilidades para utilizar efectivamente software para el manejo de conocimiento tales como sistemas operativos, interfaces de usuario, hojas de cálculo, procesadores de texto, bases de datos, estadísticas y manejo de datos, presentaciones y comunicaciones.
\end{LUGoal}

\begin{LUObjective}
\item Diseñar, desarrollar y utilizar una base de datos simple, importar hojas de cálculo a bases de datos, exportar una tabla de base de datos u hoja de cálculo a un procesador de texto para ser utilizado en un reporte.
\item Implementar una presentación basada en slides utilizando un paquete gráfico de presentaciones para comunicar un problema y su solución así como preparar documentos impresos para una posible audiencia.
\end{LUObjective}
\end{LearningUnit}

\subsection{LU3. Resolución de problemas, Sistemas de Información pequeños}\label{sec:LU3}
\begin{LearningUnit}
\begin{LUGoal}
\item Introducir los conceptos de resolución de problemas dentro del contexto de Sistemas de Información de complejidad limitada usando software de manejo de conocimiento estándar.
\end{LUGoal}

\begin{LUObjective}
\item Describir, explicar y usar una definición de abordaje de sistemas e implementación de soluciones basadas en PCs utilizando software de manejo de conocimiento (sistemas operativos, interfaces de usuario, hojas de cálculo, procesadores de texto, bases de datos, estadísticas y manejo de datos, presentaciones y comunicaciones) para mejorar la productividad del personal e incrementar las capacidades de trabajo con conocimiento.
\item Identificar, definir e implementar una solución que involucre software para el trabajo con conocimiento para organizaciones simples y tareas personales.
\item Seleccionar y configurar macros apropiadas, herramientas y paquetes para implementación de sistemas personales.
\end{LUObjective}
\end{LearningUnit}

\subsection{LU4. Tecnología de Información y la sociedad}\label{sec:LU4}
\begin{LearningUnit}
\begin{LUGoal}
\item Introducir la relevancia y aplicación de la Tecnología de Información en la sociedad.
\end{LUGoal}

\begin{LUObjective}
\item Describir y explicar la relevancia e impacto de la Tecnología de la Información en la sociedad.
\item Explicar el rol de los Sistemas de Información dentro de una empresa versus un entorno global.
\end{LUObjective}
\end{LearningUnit}

\subsection{LU5. Sistemas y calidad}\label{sec:LU5}
\begin{LearningUnit}
\begin{LUGoal}
\item Introducir conceptos de sistemas y de calidad.
\end{LUGoal}

\begin{LUObjective}
\item Explicar conceptos de calidad y teoría de sistemas.
\end{LUObjective}
\end{LearningUnit}

\subsection{LU6. Información y calidad}\label{sec:LU6}
\begin{LearningUnit}
\begin{LUGoal}
\item Proveer una introducción al uso organizacional de la información para mejorar la calidad general.
\end{LUGoal}

\begin{LUObjective}
\item Explicar metodologías para facilitar la medición y alcanzar el ISO9000, Baldridge, {\it National Performance Review} y otros estándares de calidad.
\end{LUObjective}
\end{LearningUnit}

\subsection{LU7. Hardware y software de Tecnología de Información}\label{sec:LU7}
\begin{LearningUnit}
\begin{LUGoal}
\item Presentar conceptos de Tecnología de Información relacionados a hardware y software.
\end{LUGoal}

\begin{LUObjective}
\item Explicar los elementos y su relación funcional de los principales componentes de hardware, software y comunicaciones que forman PCs, LANs y/o WANs.
\end{LUObjective}
\end{LearningUnit}

\subsection{LU8. Especificación de Sistemas de Tecnología de Información}\label{sec:LU8}
\begin{LearningUnit}
\begin{LUGoal}
\item Proveer los conceptos y habilidades para la especificación y diseño o reingeniería de sistemas pequeños relacionados con organizaciones basados en Tecnología de la Información.
\end{LUGoal}

\begin{LUObjective}
\item Explicar los conceptos de implementación de Sistemas de Información acoplados a la reingeniería e mejoramiento continuo.
\end{LUObjective}
\end{LearningUnit}

\subsection{LU9. Tecnología de Información y el consecusión de objetivos}\label{sec:LU9}
\begin{LearningUnit}
\begin{LUGoal}
\item Mostrar como la Tecnología de Información puede ser utilizada para diseñar, facilitar y comunicar los objetivos organizacionales.
\end{LUGoal}

\begin{LUObjective}
\item Explicar la relevancia del manejo de Sistemas de Información alineados con los objetivos organizacionales.
\end{LUObjective}
\end{LearningUnit}

\subsection{LU10. Características de un profesional de Sistemas de Información}\label{sec:LU10}
\begin{LearningUnit}
\begin{LUGoal}
\item Explicar los conceptos de la toma de decisiones personales, objetivos, definición de metas, confiabilidad y motivación.
\end{LUGoal}

\begin{LUObjective}
\item Discutir y explicar los conceptos de definición de metas y toma y alcance de decisiones individuales. Explicar los requerimientos de definición de metas y toma de decisiones personales en la motivación y mejora de las condiciones de trabajo.
\end{LUObjective}
\end{LearningUnit}

\subsection{LU11. Línea de carrera de carrera de Sistemas de Información}\label{sec:LU11}
\begin{LearningUnit}
\begin{LUGoal}
\item Mostrar las áreas de la carrera de Sistemas de Información.
\end{LUGoal}

\begin{LUObjective}
\item Identificar y explicar las carreras de telecomunicaciones y sus áreas.
\end{LUObjective}
\end{LearningUnit}

\subsection{LU12. Ética y el profesional de Sistemas de Información}\label{sec:LU12}
\begin{LearningUnit}
\begin{LUGoal}
\item Presentar y discutir las responsabilidades profesionales y éticas del profesional de Sistemas de Información.
\end{LUGoal}

\begin{LUObjective}
\item Usar códigos de ética profesional para evaluar acciones de Sistemas de Información específicas.
\item Describir asuntos éticos y legales; discutir y explicar consideraciones éticas del uso, distribución, operación y mantenimiento de software.
\end{LUObjective}
\end{LearningUnit}

\subsection{LU13. Sistemas de Información de nivel personal}\label{sec:LU13}
\begin{LearningUnit}
\begin{LUGoal}
\item Identificar, investigar, analizar, diseñar y desarrollar con paquetes (y/o lenguajes de alto nivel) y sistemas de información de nivel personal para mejorar la productividad personal.
\end{LUGoal}

\begin{LUObjective}
\item Analizar, diseñar, desarrollar y usar paquetes (p.e. un paquete de estadística o de administración de datos de alto nivel) y/o bases de datos de alto nivel que requieran lenguajes para implementar soluciones trabajables para resolver problemas de Sistemas de Información asociados con actividades de trabajo del conocimiento.
\item Evaluar el incremento de la productividad realizado a través de la implementación de sistemas personales.
\end{LUObjective}
\end{LearningUnit}

\subsubsection{LU13.01. Conceptos de Trabajo y Actividad}\label{sec:LU13.01}
\begin{LearningUnit}
\begin{LUGoal}
\item Describir el concepto de trabajo del conocimiento y la necesidad de contar con tecnología de información personal que lo soporte.
\end{LUGoal}

\begin{LUObjective}
\item Definir y explicar el concepto de trabajo del conocimiento.
\item Comparar y contrastar datos, información y conocimiento.
\item Describir las actividades del trabajo del conocimiento; identificar y explicar métodos para lograr productividad en el trabajo del conocimiento.
\end{LUObjective}
\end{LearningUnit}

\subsubsection{LU13.02. Soporte: Individuos vs Grupos}\label{sec:LU13.02}
\begin{LearningUnit}
\begin{LUGoal}
\item Relacionar requerimientos de sistemas de información organizacionales vs. personales.
\end{LUGoal}

\begin{LUObjective}
\item Comparar y constrastar el planeamiento, desarrollo y administración de riesgos de las aplicaciones para sistemas de información personales vs. organizacionales.
\item Explicar problemas potenciales de sistemas desarrollados por el usuario.
\end{LUObjective}
\end{LearningUnit}

\subsubsection{LU13.03. Análisis de Información: Individual vs. Grupal}\label{sec:LU13.03}
\begin{LearningUnit}
\begin{LUGoal}
\item Introducir conceptos de trabajo del conocimiento individuales vs. colaborativos y relacionarlos al análisis de las necesidades de información y a la tecnología.
\end{LUGoal}

\begin{LUObjective}
\item Describir y explicar tecnologías individuales vs. grupales; explicar el procesamiento adicional y otros asuntos y necesidades requeridas para el trabajo en grupo.
\item Describir y explicar tecnología de soporte a grupos para requerimientos de conocimiento común.
\item Describir y explicar el proceso de análisis de información y de aplicación de soluciones de tecnología de información.
\end{LUObjective}
\end{LearningUnit}

\subsubsection{LU13.04. Análisis de Información: Encontrando sus requerimientos de Sistemas y Tecnologías de Información}\label{sec:LU13.04}
\begin{LearningUnit}
\begin{LUGoal}
\item Describir y explicar los objetivos y el proceso de análisis y de la documentación del trabajo del conocimiento, tecnología de información y de los requerimientos de información para individuos y grupos de trabajo.
\end{LUGoal}

\begin{LUObjective}
\item Describir y explicar características y atributos del trabajo del conocimiento para individuos y grupos.
\item Discutir y explicar las tareas de construcción y mantenimiento del conocimiento.
\item Usar preguntas para elicitar sistemáticamente e identificar los requerimientos de datos de individuos y grupos.
\item Analizar las tareas individuales y grupales para determinar los requerimientos de información.
\item Identificar requerimientos de tecnología de información relacionada.
\end{LUObjective}
\end{LearningUnit}

\subsubsection{LU13.05. Organizando recursos de datos personales}\label{sec:LU13.05}
\begin{LearningUnit}
\begin{LUGoal}
\item Definir conceptos, principios y métodos prácticos para la administración de software y datos individuales.
\end{LUGoal}

\begin{LUObjective}
\item Dadas tareas y actividades de trabajo del conocimiento, diseñar e implementar un método para la organización de directorios y el etiquetado de archivos que soporte la retención y acceso a los datos.
\item Listar principios que apliquen a la adquisición y actualización de software.
\item Describir métodos para la transferencia de datos entre aplicaciones incluyendo OLE, importación/exportación y métodos alternativos.
\end{LUObjective}
\end{LearningUnit}

\subsubsection{LU13.06. Tecnologías y conceptos de Bases de Datos}\label{sec:LU13.06}
\begin{LearningUnit}
\begin{LUGoal}
\item Explicar conceptos organizacionales, componentes, estructuras, acceso, seguridad y consideraciones de administración de bases de datos.
\end{LUGoal}

\begin{LUObjective}
\item Describir y explicar la terminología y el uso de bases de datos relacionales.
\item Describir y explicar conceptos necesarios para acceder a bases de datos organizacionales.
\item Usar infraestructura de acceso a bases de datos para hacer consultas de datos a partir de un repositorio organizacional.
\end{LUObjective}
\end{LearningUnit}

\subsubsection{LU13.07. Acceso, Recuperación y Almacenamiento de Datos}\label{sec:LU13.07}
\begin{LearningUnit}
\begin{LUGoal}
\item Definir el contenido, disponibilidad y estrategias para acceder información externa a la organización.
\end{LUGoal}

\begin{LUObjective}
\item Definir y discutir recursos de información externa; identificar la fuente, el contenido, los costos y la temporalidad.
\item Localizar y acceder recursos de información externos usando herramientas de Internet disponibles: navegador, búsqueda, ftp.
\item Crear y mantener un directorio individual para los recursos de información externa.
\end{LUObjective}
\end{LearningUnit}

\subsubsection{LU13.08. Ciclo de vida de Sistemas de Información: Desarrollando con Paquetes}\label{sec:LU13.08}
\begin{LearningUnit}
\begin{LUGoal}
\item Presentar y explicar el ciclo de vida de desarrollo de un sistema de información incluyendo los conceptos de adquisición vs. desarrollo de software.
\end{LUGoal}

\begin{LUObjective}
\item Discutir el concepto del ciclo de vida de un sistema de información.
\item Identificar y explicar los criterios para decidir entre la adquisición de paquetes de software vs. el desarrollo de software personalizado.
\end{LUObjective}
\end{LearningUnit}

\subsubsection{LU13.09. Configuración y personalización de un paquete}\label{sec:LU13.09}
\begin{LearningUnit}
\begin{LUGoal}
\item Introducir y explorar el uso de software de aplicación y de propósito general.
\end{LUGoal}

\begin{LUObjective}
\item Instalar y personalizar un paquete de software de propósito general para proveer funcionalidad específica más allá de las opciones por defecto.
\item Adicionar capacidades a un sistema de software por medio de la grabación y almacenamiento de una macro en la librería del paquete de software dado.
\item Acceder a información técnica provista en la forma de facilidades de ``ayuda" del software; observar y usar la infraestructura de ``ayuda".
\end{LUObjective}
\end{LearningUnit}

\subsubsection{LU13.10. Programación Procedural y Orientada a Eventos}\label{sec:LU13.10}
\begin{LearningUnit}
\begin{LUGoal}
\item Introducir y explorar los métodos de desarrollo de software, para luego explicar los objetivos y estrategias de los paradigmas de programación procedural, basado en eventos y orientado a objetos.
\end{LUGoal}

\begin{LUObjective}
\item Discutir y explicar los conceptos de datos y de representación procedural, lenguajes de programación, compiladores, intérpretes, ambientes de desarrollo e interfaces gráficas de usuario basadas en eventos.
\item Comparar, relacionar y explicar conceptos de métodos estructurados, basados en eventos y orientados a objetos para el diseño de programas, con ejemplos para cada método.
\end{LUObjective}
\end{LearningUnit}

\subsubsection{LU13.11. Implementando algoritmos simples}\label{sec:LU13.11}
\begin{LearningUnit}
\begin{LUGoal}
\item Introducir y desarrollar el proceso de desarrollo de algoritmos y código estructurado.
\end{LUGoal}

\begin{LUObjective}
\item Definir un problema sencillo identificando las salidas deseadas para entradas dadas; ofrecer una vista panorámica del problema.
\item Describir tipos de datos fundamentales y sus operaciones.
\item Diseñar lógica de programas usando tanto técnicas gráficas como de pseudocódigo que utilicen estructuras de control estándar: secuencia, iteración y selección.
\item Traducir estructuras de datos y diseño de programas en código de un lenguaje de programación; verificar la traducción y asegurar la correctitud de los resultados; evaluar el código con conjuntos de datos de prueba.
\end{LUObjective}
\end{LearningUnit}

\subsubsection{LU13.12. Implementación de un Diseño simple de Base de Datos}\label{sec:LU13.12}
\begin{LearningUnit}
\begin{LUGoal}
\item Introducir el propósito y desarrollar la habilidad para usar un paquete de software de bases de datos relacionales.
\end{LUGoal}

\begin{LUObjective}
\item Describir y explicar tablas, relaciones, integridad referencial y los conceptos de las formas normales.
\item A partir de un dibujo de flujo de trabajo o de otros documentos de requisitos, derivar un diseño de bases de datos simple con múltiples tablas.
\item Usando un paquete de software de bases de datos relacionales, implementar y poblar las tablas; desarrollar varias consultas simples para explorar los datos.
\end{LUObjective}
\end{LearningUnit}

\subsubsection{LU13.13. Implementación de aplicaciones orientadas a eventos}\label{sec:LU13.13}
\begin{LearningUnit}
\begin{LUGoal}
\item Introducir y desarrollar la habilidad para diseñar e implementar una infraestructura de interfaz gráfica de usuario.
\end{LUGoal}

\begin{LUObjective}
\item Aplicar una solución de GUI basada en eventos en un ambiente de desarrollo.
\item Construir un formulario de aplicación simple con varios objetos (p.e. etiqueta, cajas de edición {\it edit box}, listas, botones de comando).
\end{LUObjective}
\end{LearningUnit}

\subsubsection{LU13.14. Desarrollo de Sistemas de Información con prototipado}\label{sec:LU13.14}
\begin{LearningUnit}
\begin{LUGoal}
\item Presentar el proceso de prototipeo e introducir y aplicar los conceptos de evaluación y refinamiento evolutivo para prototipos de aplicaciones personales.
\end{LUGoal}

\begin{LUObjective}
\item Comparar las capacidades de una aplicación con los requerimientos que debe cubrir.
\item Identificar salidas alternativas del proceso de verificación de aplicaciones.
\item Evaluar y definir los resultados y probabilidades de error en software de aplicación prototipeo.
\item Modificar entradas, salidas y procesamiento para refinar un prototipo.
\end{LUObjective}
\end{LearningUnit}

\subsubsection{LU13.15. Evolución de la Tecnología de Sistemas de Información}\label{sec:LU13.15}
\begin{LearningUnit}
\begin{LUGoal}
\item Presentar tecnologías de fundamento y definir la importancia en el futuro de las capacidades de la tecnología de información.
\end{LUGoal}

\begin{LUObjective}
\item Listar y explicar tecnologías y su relevancia para tecnología de información individual.
\item Dada una tecnología, explicar su importancia para los desarrollos futuros y la productividad futura del trabajador del conocimiento.
\item Identificar los causantes e inhibidores del cambio en la tecnología de la información.
\end{LUObjective}
\end{LearningUnit}

\subsubsection{LU13.16. Implementación de una aplicación de Sistemas de Información personal}\label{sec:LU13.16}
\begin{LearningUnit}
\begin{LUGoal}
\item Identificar, investigar, analizar, diseñar y desarrollar con paquetes (y/o lenguajes de alto nivel) un sistema de información de nivel personal simple para mejorar la productividad individual.
\end{LUGoal}

\begin{LUObjective}
\item Analizar, diseñar, desarrollar y usar paquetes y/o lenguajes de bases de datos de alto nivel para implementar soluciones trabajables que resuelvan un problema de sistemas de información asociado con actividades de trabajo del conocimiento.
\item Evaluar el incremento de productividad realizado implementando sistemas personales.
\end{LUObjective}
\end{LearningUnit}

\subsection{LU14. Resolución de problemas con paquetes}\label{sec:LU14}
\begin{LearningUnit}
\begin{LUGoal}
\item Presentar y aplicar estrategias, metodologías y métodos para usar paquetes de software, así como lenguajes de alto nivel para desarrollar soluciones a problemas formales implementables de ``usuario final'', los cuales se encuentran alineados con los sistemas de información organizacionales.
\end{LUGoal}

\begin{LUObjective}
\item Explicar y usar conceptos de problemas formales e ingeniería de software aplicadas al desarrollo de soluciones efectivas que mejoren la productividad personal que involucre actividades de trabajo del conocimiento, dentro de soluciones que son compatibles con el sistema de información organizacional.
\item Desarrollar, documentar y mantener sistemas pequeños para productividad personal usando bases de datos de alto nivel usando herramientas y ambientes de desarrollo de aplicaciones.
\item Usar los conceptos de definición y resolución de problemas analíticos formales en el uso de paquetes de software; asegurar que dichas soluciones tomen en cuenta los sistemas de información ``reales'' involucrados.
\end{LUObjective}
\end{LearningUnit}

\subsection{LU15. Estrategias de uso de Información}\label{sec:LU15}
\begin{LearningUnit}
\begin{LUGoal}
\item Presentar y aplicar estrategias para acceder y usar recursos de información.
\end{LUGoal}

\begin{LUObjective}
\item Explicar administración de datos y acceso a recursos de información corporativos y alternativos.
\item Discutir inteligentemente las diferencias entre la administración de SI\&T, desarrollo de sistemas, mantenimiento de sistemas, operación de sistemas.
\end{LUObjective}
\end{LearningUnit}

\subsection{LU16. Teoría de Sistemas de Información}\label{sec:LU16}
\begin{LearningUnit}
\begin{LUGoal}
\item Introducir, discutir y describir conceptos fundamentales de teoría de Sistemas de Información y su importancia para los profesionales.
\end{LUGoal}

\begin{LUObjective}
\item Identificar y explicar los conceptos subyacentes de la disciplina de Sistemas de Información.
\end{LUObjective}
\end{LearningUnit}

\subsection{LU17. Sistemas de Información como un componente estratégico}\label{sec:LU17}
\begin{LearningUnit}
\begin{LUGoal}
\item Mostrar como un sistema de información es un componente estratégico e integral de una organización.
\end{LUGoal}

\begin{LUObjective}
\item Describir el desarrollo histórico de la disciplina de Sistemas de Información.
\item Explicar el rol estratégico de los sistemas de información en las organizaciones.
\item Explicar la relación estratégica de las actividades de Sistemas de Información para mejorar la posición competitiva.
\item Explicar las diferencias entre aplicaciones de nivel estratégico, táctico y operativo.
\end{LUObjective}
\end{LearningUnit}

\subsection{LU18. Desarrollo y administración de Sistemas de Información}\label{sec:LU18}
\begin{LearningUnit}
\begin{LUGoal}
\item Discutir cómo se desarrolla un sistema de información y éste es administrado dentro de una organización.
\end{LUGoal}

\begin{LUObjective}
\item Explicar el desarrollo de sistemas de información y el rediseño de los procesos organizacionales; explicar los grupos de individuos y sus responsabilidades en este proceso.
\item Explicar los roles de los profesionales en Sistemas de Información dentro de una organización de Sistemas de Información; explicar las funciones de la administración de Sistemas de Información, administrador de proyectos, analista de información y explicar los caminos de desarrollo profesional posibles.
\end{LUObjective}
\end{LearningUnit}

\subsection{LU19. Proceso cognitivo}\label{sec:LU19}
\begin{LearningUnit}
\begin{LUGoal}
\item Presentar y discutir la relevancia del proceso cognitivo e interacciones humanas en el diseño e implementación de sistemas de información.
\end{LUGoal}

\begin{LUObjective}
\item Explicar el proceso cognitivo y otras consideraciones orientadas al ser humano en el diseño e implementación de sistemas de información.
\end{LUObjective}
\end{LearningUnit}

\subsection{LU20. Objetivos y decisiones}\label{sec:LU20}
\begin{LearningUnit}
\begin{LUGoal}
\item Discutir cómo los individuos toman decisiones y establecen y alcanzan objetivos.
\end{LUGoal}

\begin{LUObjective}
\item Discutir y explicar cómo los individuos toman decisiones, establecen y alcanzan objetivos; explicar lo que significa acción personal dirigida a una misión.
\end{LUObjective}
\end{LearningUnit}

\subsection{LU21. Toma de decisiones: el modelo de Simon}\label{sec:LU21}
\begin{LearningUnit}
\begin{LUGoal}
\item Desarrollar la capacidad para discutir e intercambiar opiniones sobre  el Modelo de Simon para la toma de decisiones organizacionales y su soporte utilizando IS.
\end{LUGoal}

\begin{LUObjective}
\item Discutir y explicar la teoría de decisiones y el proceso de toma de decisiones.
\item Explicar el soporte de IS para la toma de decisiones; explicar el uso de sistemas expertos en el soporte en la toma de decisiones heurísticas.
\item Explicar y dar una ilustración del modelo de decisión organizacional de Simon.
\end{LUObjective}
\end{LearningUnit}

\subsection{LU22. Sistemas y calidad}\label{sec:LU22}
\begin{LearningUnit}
\begin{LUGoal}
\item Introducir a la teoría de Sistemas, calidad  y modelado organizacional y demostrar su relevancia en los sistemas de información.
\end{LUGoal}

\begin{LUObjective}
\item Discutir y explicar los objetivos de los sistemas, expectativas de los clientes y conceptos de calidad.
\item Discutir y explicar los componentes y relaciones de los sistemas.
\item Aplicar conceptos de sistemas para definir y explicar el rol de los sistemas de información.
\item Explicar el uso de la información y sistemas de información en actividades de documentación toma de decisiones y control organizacional.
\end{LUObjective}
\end{LearningUnit}

\subsection{LU23. Rol de la administración, usuarios, diseñadores de sistemas}\label{sec:LU23}
\begin{LearningUnit}
\begin{LUGoal}
\item Discutir un sistema basado en reglas para la administradores, usuarios y diseñadores.
\end{LUGoal}

\begin{LUObjective}
\item Identificar la responsabilidad de los usuarios, diseñadores y administradores en términos descritos en la trinidad Churchman;  discutir en términos de sistemas detallando obligaciones de cada uno, relatar esas observaciones para mejorar los modelos de calidad para el desarrollo organizacional; identificar la función de los  IS en esos términos.
\end{LUObjective}
\end{LearningUnit}

\subsection{LU24. Flujo de trabajo de Sistemas Organizacionales}\label{sec:LU24}
\begin{LearningUnit}
\begin{LUGoal}
\item Explicar los sistemas físicos y el flujo de trabajo y como los sistemas de información están relacionados a los sistemas organizacionales.
\end{LUGoal}

\begin{LUObjective}
\item Explicar la relación entre el modelo de base de datos   y la actividad física organizacional.
\end{LUObjective}
\end{LearningUnit}

\subsection{LU25. Modelos y relaciones organizacionales con Sistemas de Información}\label{sec:LU25}
\begin{LearningUnit}
\begin{LUGoal}
\item Presentar otros modelos organizacionales  y su relevancia para los IS.
\end{LUGoal}

\begin{LUObjective}
\item Describir el rol de la tecnología de información (IT) y las reglas de las personas usando, diseñando y manteniendo IT en las organizaciones.
\item Discutir como la teoría general de sistemas es aplicada al análisis y desarrollo de los sistemas de información.
\end{LUObjective}
\end{LearningUnit}

\subsection{LU26. Planeamiento de Sistemas de Información}\label{sec:LU26}
\begin{LearningUnit}
\begin{LUGoal}
\item Discutir la relación  entre el planeamiento de los IS  con el planeamiento organizacional.
\end{LUGoal}

\begin{LUObjective}
\item Explicar metas y procesos de planeamiento.
\item Explicar la importancia del planeamiento estratégico  y cooperativo  así como el alineamiento del plan proyecto de los sistemas de información.
\end{LUObjective}
\end{LearningUnit}

\subsection{LU27. Tipos de Sistemas de Información}\label{sec:LU27}
\begin{LearningUnit}
\begin{LUGoal}
\item Demostrar clases específicas de sistemas de aplicación incluyendo TPS y DSS.
\end{LUGoal}

\begin{LUObjective}
\item Describir la clasificación de los sistemas de información, por  ejemplo, TPS, DSS, ESS, WFS.
\item Explicar la relevancia organizacional de los IS: TPS, DSS, EIS, ES, {\it Work Flow System}.
\end{LUObjective}
\end{LearningUnit}

\subsection{LU28. Estándares de desarrollo de Sistemas de Información}\label{sec:LU28}
\begin{LearningUnit}
\begin{LUGoal}
\item Discutir  y examinar los procesos, estándares y políticas para el desarrollo de sistemas de información. Desarrollo de metodologías, ciclo de vida, workflow, OOA, prototipeo, espiral, usuario final entre otros.
\end{LUGoal}

\begin{LUObjective}
\item Discutir y explicar el concepto de una metodología de desarrollo de IS, explicar el ciclo de vida, workflow, OOA, prototipeo, modelos basado en riesgos, modelo en espiral, entre otros; mostrar como esto puede ser usado en la práctica.   
\end{LUObjective}
\end{LearningUnit}

\subsection{LU29. Implementación de Sistemas de Información: \textit{outsourcing}}\label{sec:LU29}
\begin{LearningUnit}
\begin{LUGoal}
\item Discutir {\it outsourcing} e implementaciones alternativas de IS.
\end{LUGoal}

\begin{LUObjective}
\item Explicar las ventajas  y desventajas del desarrollo {\it outsourcing} en algunas o todas las funciones de IS; establecer  los requerimientos del personal con o sin {\it outsourcing}
\end{LUObjective}
\end{LearningUnit}

\subsection{LU30. Evaluación de desempeño del personal}\label{sec:LU30}
\begin{LearningUnit}
\begin{LUGoal}
\item Discutir la evaluación del rendimiento  la cual consiste e con la administración de la calidad  y la mejora continua.
\end{LUGoal}

\begin{LUObjective}
\item Describir, explicar y aplicar las responsabilidades del líder del proyecto, administrar el desarrollo de pequeños sistemas.
\item Discutir, explicar e implementar una metodología para hacer seguimiento a los clientes dentro de todo las fases del ciclo de vida
\item Explicar metodologías para facilitar el uso de estándares como el ISO 9000, {\it National Performance Review} y otros estándares de calidad.
\end{LUObjective}
\end{LearningUnit}

\subsection{LU31. Sociedad de Sistemas de Información y ética}\label{sec:LU31}
\begin{LearningUnit}
\begin{LUGoal}
\item Introducir las implicaciones sociales y éticas de los Sistemas de Información para introducir a la exploración de los conceptos éticos y asuntos relacionados al comportamiento profesional.
\item Comparar y contrastar los modelos e abordajes éticos.
\end{LUGoal}

\begin{LUObjective}
\item Discutir y explicar ética y comportamiento basado en principios así como el concepto de práctica ética en el área de Sistemas de Información.
\item Discutir modelos éticos importantes y discutir las razones por las cuales hay que ser ético.
\item Explicar el uso del código de ética profesional.
\item Explicar la carga responsabilidad y de profesionalismo resultante de la confianza asociada con el conocimiento y habilidades de computación.
\item Discutir y explicar las bases y naturaleza de los abordajes éticos cuestionables.
\item Discutir y explicar el análisis ético y social del desarrollo de Sistemas de Información.
\item Discutir y explicar los asuntos de poder y su impacto social en el ciclo de vida del desarrollo.
\end{LUObjective}
\end{LearningUnit}

\subsection{LU32. Dispositivos, medios, sistemas de Telecomunicaciones}\label{sec:LU32}
\begin{LearningUnit}
\begin{LUGoal}
\item Desarrollar la preocupación y la terminología asociada de los diferentes y dispositivos necesarios para telecomunicaciones, incluyendo redes LAN y WAN.
\end{LUGoal}

\begin{LUObjective}
\item Identificar las características de la transmisión de datos en telecomunicaciones a nivel de LANs, WANs y MANs.
\item Accesar información remota para transferencia de archivos en entornos LAN y WAN.
\item Discutir y explicar la industria de las telecomunicaciones así como sus estándares y regulaciones.
\end{LUObjective}
\end{LearningUnit}

\subsection{LU33. Soporte organizacional basado en Telecomunicaciones}\label{sec:LU33}
\begin{LearningUnit}
\begin{LUGoal}
\item Desarrollar una preocupación por la forma en la que los sistemas de telecomunicaciones son utilizados para soportar la infraestructura de comunicaciones de la organización incluyendo a los Sistemas de Información, teleconferencias, etc.
\end{LUGoal}

\begin{LUObjective}
\item Explicar el uso de los Sistemas de Información para soportar el flujo de trabajo;
\item Discutir los conceptos de teleconferencias y conferencias por telecomputadoras en el rol de las comunicaciones y en la toma de decisiones.
\item Discutir y explicar la infraestructura involucrada en los sistemas de telecomunicaciones.
\end{LUObjective}
\end{LearningUnit}

\subsection{LU34. Economía y problemas de diseño de sistemas de Telecomunicaciones}\label{sec:LU34}
\begin{LearningUnit}
\begin{LUGoal}
\item Explorar los asuntos relacionados al diseño y manejo económico de las redes de computadores.
\end{LUGoal}

\begin{LUObjective}
\item Explicar los pasos en el análisis y configuración de un sistema de telecomunicaciones, incluyendo hardware específico y componentes de software.
\item Explicar el propósito de modems, bridges, gateways, hubs y ruteadores en la interconexión de sistemas.
\end{LUObjective}
\end{LearningUnit}

\subsection{LU35. Estándares de Telecomunicaciones}\label{sec:LU35}
\begin{LearningUnit}
\begin{LUGoal}
\item Familiarizar al estudiante con los estándares de telecomunicaciones, con las organizaciones que las regulan y con sus estándares.
\end{LUGoal}

\begin{LUObjective}
\item Identificar el rol de los estándares y de las organizaciones regulatorias y sus estándares como facilitadores para lograr desde telecomunicaciones locales hasta aquellas globales.
\item Explicar la codificación digital de datos relevantes a las telecomunicaciones.
\end{LUObjective}
\end{LearningUnit}

\subsection{LU36. Sistemas centralizados vs distribuidos}\label{sec:LU36}
\begin{LearningUnit}
\begin{LUGoal}
\item Discutir y explicar los principios fundamentales y temas relacionados a comparar la computación centralizada versus computación distribuida.
\end{LUGoal}

\begin{LUObjective}
\item Explicar, diagramar y discutir las estructuras y principios involucrados en la computación distribuida en cuanto a recursos y datos.
\item Identificar requerimientos de hardware y de software y una aproximación de costos para sistemas centralizados y distribuidos.
\item Discutir y explicar riesgos, seguridad y privacidad en configuraciones alternativas de sistemas.
\end{LUObjective}
\end{LearningUnit}

\subsection{LU37. Arquitecturas, topologías y protocolos de Telecomunicaciones}\label{sec:LU37}
\begin{LearningUnit}
\begin{LUGoal}
\item Presentar arquitecturas, topologías y protocolos de telecomunicaciones.
\end{LUGoal}

\begin{LUObjective}
\item Identificar y explicar las funciones de cada una de las capas del modelo ISO.
\item Explicar el concepto de comunicaciones virtuales entre dos computadores a cada nivel del modelo ISO.
\item Identificar y explicar topologías comunes y métodos de implementación de sistemas de telecomunicaciones.
\item Identificar y describir la organización y operación de los protocolos de bits y bytes.
\item Discutir los servicios de telecomunicaciones y analizar una implementación específica del modelo ISO.
\end{LUObjective}
\end{LearningUnit}

\subsection{LU38. Hardware y software de Telecomunicaciones}\label{sec:LU38}
\begin{LearningUnit}
\begin{LUGoal}
\item Presentar los componentes de hardware y software involucrados en un sistema de telecomunicaciones y como ellos están organizados para proveer los servicios requeridos.
\end{LUGoal}

\begin{LUObjective}
\item Describir, diagramar, discutir y explicar los componentes de hardware y software de los sistemas de telecomunicaciones. Describir la integración de teléfono, fax, redes LAN y WAN. Diagramar y discutir  varias organizaciones de hardware de cada tipo de dispositivo requerido.
\item Explicar el uso de ruteadores y hubs en el diseño de sistemas interconectados.
\item Explicar los requerimientos de telecomunicaciones para voz, audio, datos, imágenes, vídeo y multimedia en general.
\item Explicar tecnologías y aplicaciones de paquetes rápidos.
\item Explicar asuntos relacionados al diseño de redes de telecomunicaciones.
\item Dar ejemplos de aplicaciones de telecomunicaciones en negocios y explicar su utilización en el sistema descrito.
\end{LUObjective}
\end{LearningUnit}

\subsection{LU39. Servicios, confiabilidad y seguridad de los sistemas de telecomunicaciones}\label{sec:LU39}
\begin{LearningUnit}
\begin{LUGoal}
\item Concientizar al estudiante sobre la preocupación inherente cuando uno provee servicios de telecomunicaciones incluyendo seguridad, privacidad, confiabilidad y desempeño.
\end{LUGoal}

\begin{LUObjective}
\item Explicar medidas del desempeño de sistemas de telecomunicaciones y asegurar un adecuado desempeño y confiabilidad.
\end{LUObjective}
\end{LearningUnit}

\subsection{LU40. Instalación e implantación de sistemas de Telecomunicaciones}\label{sec:LU40}
\begin{LearningUnit}
\begin{LUGoal}
\item Explicar como instalar el equipo necesario para implementar un sistema de telecomunicaciones. Esto es: cable, modems, Ethernet, conexiones, gateways, ruteadores.
\end{LUGoal}

\begin{LUObjective}
\item Explicar, instalar y probar modems, multiplexores y componentes Ethernet.
\item Explicar, instalar y probar {\it bridges} y ruteadores en el hardware apropiado.
\item Instalar y operar software de emulación de un terminal en una PC.
\item Explicar y construir planes organizacionales para el uso de EDI.
\end{LUObjective}
\end{LearningUnit}

\subsection{LU41. Instalación y configuración de redes LAN}\label{sec:LU41}
\begin{LearningUnit}
\begin{LUGoal}
\item Explicar cómo diseñar, instalar, configurar y administrar una LAN
\end{LUGoal}

\begin{LUObjective}
\item Diseñar, instalar y administrar una LAN
\item Explicar e implementar seguridad apropiada para un ambiente de usuario final involucrando acceso a un sistema de información de nivel empresarial.
\end{LUObjective}
\end{LearningUnit}

\subsection{LU42. Medidas de información/datos/eventos}\label{sec:LU42}
\begin{LearningUnit}
\begin{LUGoal}
\item Presentar que el concepto de datos es una representación y medición de eventos del mundo real.
\end{LUGoal}

\begin{LUObjective}
\item Explicar el concepto de medición e información, representación de información, organización, almacenamiento y procesamiento.
\item Describir que el concepto de datos es una representación y medición de eventos del mundo real y el proceso de capturarlos en una máquina de forma legible.
\end{LUObjective}
\end{LearningUnit}

\subsection{LU43. Datos: caracteres, registros, archivos, multimedia}\label{sec:LU43}
\begin{LearningUnit}
\begin{LUGoal}
\item Mostrar y explicar la lógica y la estructura física de los datos para representar caracteres, registros, archivos y objetos multimedia.
\end{LUGoal}

\begin{LUObjective}
\item Identificar, explicar y discutir la jerarquía de datos e identificar todas las operaciones primarias asociadas con cada nivel de la jerarquía.
\end{LUObjective}
\end{LearningUnit}

\subsection{LU44. Tipos abstractos de datos, clases, objetos}\label{sec:LU44}
\begin{LearningUnit}
\begin{LUGoal}
\item Explicar los conceptos de clases, tipos abstractos de datos y objetos.
\end{LUGoal}

\begin{LUObjective}
\item Discutir clases que involucran elementos de la jerarquía de datos (bit, byte, campos, registros, archivos, bases de datos), y usar estas definiciones como base para la resolución de problemas; describir las estructuras de un programa y su uso relacionado a cada estructura.
\end{LUObjective}
\end{LearningUnit}

\subsection{LU45. Resolución de problemas formales y Sistemas de Información}\label{sec:LU45}
\begin{LearningUnit}
\begin{LUGoal}
\item Explicar e ilustrar con ejemplos de sistemas de información la resolución formal y analítica de problemas.
\end{LUGoal}

\begin{LUObjective}
\item Explicar y dar ejemplos del concepto de escribir programas de computador y usar lenguajes de desarrollo de software e infraestructuras para el desarrollo de aplicaciones para resolver problemas.
\end{LUObjective}
\end{LearningUnit}

\subsection{LU46. Representación de objetos en un Sistemas de Información}\label{sec:LU46}
\begin{LearningUnit}
\begin{LUGoal}
\item Presentar un sistema de vista de representaciones de objetos y compararlo con modelos de flujo de datos.
\end{LUGoal}

\begin{LUObjective}
\item Discutir y explicar un sistema desde el punto de vista de una representación de objetos; explicar la similitud de una representación de objetos para una notación de flujo de datos convencional.
\end{LUObjective}
\end{LearningUnit}

\subsection{LU47. Diseño de algoritmos}\label{sec:LU47}
\begin{LearningUnit}
\begin{LUGoal}
\item Desarrollar capacidades en el desarrollo de una solución algorítmica a un problema para ser capaces de representarla con un programa y objetos de datos apropiados.
\end{LUGoal}

\begin{LUObjective}
\item Diseñar algoritmos y traducirlos en soluciones operativas en un lenguaje de programación para muchos componentes del problema involucrados en aplicaciones de sistemas de información completas.
\end{LUObjective}
\end{LearningUnit}

\subsection{LU48. Implementación \textit{Top-Down}}\label{sec:LU48}
\begin{LearningUnit}
\begin{LUGoal}
\item Presentar estrategias de implementación {\it top-down}.
\end{LUGoal}

\begin{LUObjective}
\item Diseñar e implementar programas en una forma {\it top-down}, construyendo inicialmente los niveles superiores desarrollando un esqueleto para los niveles inferiores; completar sucesivamente los niveles inferiores de la misma manera; identificar el concepto de éxito continuado en este método.
\end{LUObjective}
\end{LearningUnit}

\subsection{LU49. Implementación de objetos}\label{sec:LU49}
\begin{LearningUnit}
\begin{LUGoal}
\item Presentar los conceptos implementación de objetos.
\end{LUGoal}

\begin{LUObjective}
\item Explicar e implementar estructuras modulares; mostrar la relación de flujo de datos y de representaciones de objetos con el código producido.
\end{LUObjective}
\end{LearningUnit}

\subsection{LU50. Módulos/cohesión/acoplamiento}\label{sec:LU50}
\begin{LearningUnit}
\begin{LUGoal}
\item Presentar conceptos de diseño modular, cohesión y acoplamiento.
\end{LUGoal}

\begin{LUObjective}
\item Desarrollar y traducir una representación de flujo de datos de una solución a un problema para una representación jerárquica y/o de objetos.
\item Usar diseño algorítmico y modular en la solución de un problema e implementar la solución con un lenguaje procedural.
\item Usar paso de parámetros en la implementación de una solución modular a un problema; explicar la importancia de una alta cohesión y un bajo acoplamiento.
\item Aplicar conceptos de diseño modular para definir módulos cohesivos de tamaño apropiado.
\item Aplicar estructuras de control de programación y verificar correctitud.
\item Demostrar habilidad para probar y validar la solución.
\end{LUObjective}
\end{LearningUnit}

\subsection{LU51. Verificación y validación, una visión de Sistemas}\label{sec:LU51}
\begin{LearningUnit}
\begin{LUGoal}
\item Presentar una vista sistemas de verificación y validación.
\end{LUGoal}

\begin{LUObjective}
\item Explicar el proceso de verificación y validación; verificar código a través de reingeniería manual tanto para representaciones procedurales y/o de objetos.
\item Desarrollar diseños de flujos de datos y traducir esos diseños a pseudocódigo de lenguajes de cuarta generación.
\end{LUObjective}
\end{LearningUnit}

\subsection{LU52. Resolución de problemas, ambientes y herramientas}\label{sec:LU52}
\begin{LearningUnit}
\begin{LUGoal}
\item Presentar y exponer estudiantes a una variedad de ambientes de programación, desarrollar herramientas y ambientes de desarrollo gráficos.
\end{LUGoal}

\begin{LUObjective}
\item Demostrar habilidad para evaluar y usar componentes de GUI existentes en la construcción de una interfaz de usuario efectiva para una aplicación.
\end{LUObjective}
\end{LearningUnit}

\subsection{LU53. Tipos abstractos de datos: estructuras de datos y archivos}\label{sec:LU53}
\begin{LearningUnit}
\begin{LUGoal}
\item Introducir los conceptos y técnicas usadas para representar y operar en estructuras de datos y de archivos, con ejemplos simples.
\end{LUGoal}

\begin{LUObjective}
\item Explicar los tipos abstractos de datos necesarios para acceder a registros en un archivo de datos indexados; mostrar ejemplos de cada tipo de operación requerida.
\end{LUObjective}
\end{LearningUnit}

\subsection{LU54. Tipos abstractos de datos: arreglos, listas, árboles, registros}\label{sec:LU54}
\begin{LearningUnit}
\begin{LUGoal}
\item Explicar cómo desarrollar estructuras usando tipos de datos abstractos representando arreglos, listas árboles, registros y archivos, y demostrar cómo son aplicadas como componentes de programas y aplicaciones.
\end{LUGoal}

\begin{LUObjective}
\item Usar representaciones de arreglos para simular el acceso a un archivo indexado, y usar la representación en el diseño de un tipo abstracto de datos para insertar, eliminar, encontrar e iterar sobre elementos.
\end{LUObjective}
\end{LearningUnit}

\subsection{LU55. Tipos abstractos de datos: archivos indexados, llaves}\label{sec:LU55}
\begin{LearningUnit}
\begin{LUGoal}
\item Presentar y usar estructuras de indexación de archivos, incluyendo organizaciones de llaves.
\end{LUGoal}

\begin{LUObjective}
\item Discutir y explicar el concepto de archivos indexados; describir construcciones de llaves y comparar requerimientos de administración de datos involucrados en elegir las llaves óptimas; explicar las funciones que son necesarias para implementar y acceder a registros indexados; explicar la similitud de arreglos y archivos indexados en términos de similitudes de funciones en tipos abstractos de datos.
\end{LUObjective}
\end{LearningUnit}

\subsection{LU56. Resolución de problemas, aplicaciones de Sistemas de Información}\label{sec:LU56}
\begin{LearningUnit}
\begin{LUGoal}
\item Explicar una variedad de estructuras fundamentales que son los bloques de construcción para el desarrollo de programas y de aplicaciones de sistemas de información.
\end{LUGoal}

\begin{LUObjective}
\item Aplicar software de aplicación para resolver problemas de pequeña escala.
\item Desarrollar documentación de usuario y del sistema para una solución programática para un problema de complejidad moderada.
\end{LUObjective}
\end{LearningUnit}

\subsection{LU57. Resolución de problemas, aplicaciones de datos y archivos}\label{sec:LU57}
\begin{LearningUnit}
\begin{LUGoal}
\item Proveer los fundamentos para aplicaciones de estructuras de datos y técnicas de procesamiento de archivos.
\end{LUGoal}

\begin{LUObjective}
\item Usar tipos abstractos de datos involucrados en aplicaciones de sistemas de información comunes para implementar soluciones a problemas involucrando técnicas de procesamiento de archivos indexados.
\end{LUObjective}
\end{LearningUnit}

\subsection{LU58. Resolución de problemas con archivos y bases de datos}\label{sec:LU58}
\begin{LearningUnit}
\begin{LUGoal}
\item Presentar y asegurar la resolución de problemas involucrando archivos y representaciones de bases de datos.
\end{LUGoal}

\begin{LUObjective}
\item Usar archivos indexados e tipos de datos abstractos para resolver problemas simples involucrando archivos usados como elementos de una solución de bases de datos.
\end{LUObjective}
\end{LearningUnit}

\subsection{LU59. Resolución de problemas, archivos/editores de bases de datos/reportes}\label{sec:LU59}
\begin{LearningUnit}
\begin{LUGoal}
\item Presentar y desarrollar editores útiles de archivos estructurados (bases de datos), mecanismos de posteo, y reportes.
\end{LUGoal}

\begin{LUObjective}
\item Construir y documentar varias aplicaciones usando archivos indexados, editores de pantalla y reportes.
\end{LUObjective}
\end{LearningUnit}

\subsection{LU60. Resolución de problemas, diseño, pruebas, depuración}\label{sec:LU60}
\begin{LearningUnit}
\begin{LUGoal}
\item Continuar el desarrollo de técnicas de programación, particularmente en el diseño, pruebas y depuración de programas relacionados a sistemas de información de cierta complejidad.
\end{LUGoal}

\begin{LUObjective}
\item Definir, explicar y presentar el proceso de definir y resolver problemas analíticos formales.
\end{LUObjective}
\end{LearningUnit}

\subsection{LU61. Programación: comparación de lenguajes}\label{sec:LU61}
\begin{LearningUnit}
\begin{LUGoal}
\item Desarrollar una conciencia de las capacidades y limitaciones relativas de los lenguajes de programación más comunes.
\end{LUGoal}

\begin{LUObjective}
\item Explicar las capacidades y diferencias para ambientes y lenguajes de programación.
\end{LUObjective}
\end{LearningUnit}

\subsection{LU62. Telecomunicaciones, visión de sistemas de hardware y software}\label{sec:LU62}
\begin{LearningUnit}
\begin{LUGoal}
\item Explicar en términos de sistemas las características fundamentales y componentes de computadores y hardware de telecomunicaciones, y software de sistemas, y demostrar cómo estos componentes interactúan.
\end{LUGoal}

\begin{LUObjective}
\item Usar la metodología de sistemas para explicar los componentes de hardware y software de un sistema de telecomunicaciones así como diagramar y discutir la naturaleza de las interacciones de los componentes.
\item Explicar en términos de sistemas el propósito, resultados esperados y calidad de un sistema de telecomunicaciones así mostrar cómo los componentes trabajan juntos.
\end{LUObjective}
\end{LearningUnit}

\subsection{LU63. Dispositivos periféricos}\label{sec:LU63}
\begin{LearningUnit}
\begin{LUGoal}
\item Proveer una vista general de dispositivos periféricos y sus funciones.
\end{LUGoal}

\begin{LUObjective}
\item Identificar clases importantes de dispositivos periféricos y explicar los principios de operación de los requerimientos de software y funciones provistas por cada tipo de dispositivo; dar ejemplos específicos de cada dispositivo identificado y discutir los requerimientos de instalación para el hardware y software requerido.
\end{LUObjective}
\end{LearningUnit}

\subsection{LU64. Arquitectura de computadores}\label{sec:LU64}
\begin{LearningUnit}
\begin{LUGoal}
\item Introducir los conceptos de arquitecturas de computadores.
\end{LUGoal}

\begin{LUObjective}
\item Definir requerimientos de datos y comunicación para acceder datos locales (discos duros o servidores) y remotos (p.e., vía Internet) para resolver problemas individuales.
\item Describir y explicar los componentes más importantes de hardware y software de un sistema de computación y cómo interactúan.
\end{LUObjective}
\end{LearningUnit}

\subsection{LU65. Componentes de software e interacciones}\label{sec:LU65}
\begin{LearningUnit}
\begin{LUGoal}
\item Introducir los conceptos de componentes de software e interacciones.
\end{LUGoal}

\begin{LUObjective}
\item Describir y explicar los componentes más importantes de un sistema operativo y como interactúan.
\item Explicar el control de funciones de entrada/salida; instalar y configurar controladores.
\end{LUObjective}
\end{LearningUnit}

\subsection{LU66. Unidad no definida}\label{sec:LU66}
\subsection{LU67. Funciones de sistemas operativos}\label{sec:LU67}
\begin{LearningUnit}
\begin{LUGoal}
\item Introducir los conceptos más importantes en sistemas operativos, incluyendo la definición de procesos, procesamiento concurrentes, administración de memoria, {\it scheduling}, procesamiento de interrupciones, seguridad y sistemas de archivos.
\end{LUGoal}

\begin{LUObjective}
\item Explicar el concepto de tareas y procesos.
\item Explicar el concepto de concurrencia y multi tareas ({\it Multitasking}).
\item Explicar el comportamiento rutinario de {\it schedulers} de tareas, colas de prioridad, procesamiento de interrupciones, administración de memoria y sistemas de archivos.
\end{LUObjective}
\end{LearningUnit}

\subsection{LU68. Ambientes y recursos de sistemas operativos}\label{sec:LU68}
\begin{LearningUnit}
\begin{LUGoal}
\item Introducir una variedad de ambientes de operación (tradicional, GUI, multimedia) y requerimientos de recursos.
\end{LUGoal}

\begin{LUObjective}
\item Describir y discutir varios ambientes operativos de sistemas de computadoras incluyendo tradicionales, de interfaz de usuario gráfica y multimedia;
\item Estimar los items de hardware y software y aproximar el costo para cada ambiente; discutir ventajas relativas para cada ambiente.
\end{LUObjective}
\end{LearningUnit}

\subsection{LU69. Instalación y configuración de sistemas operativos para multimedia}\label{sec:LU69}
\begin{LearningUnit}
\begin{LUGoal}
\item Discutir, explicar e instalar infraestructuras multimedia.
\end{LUGoal}

\begin{LUObjective}
\item Discutir y explicar los requerimientos de hardware y software necesarios para soportar multimedia.
\item Explicar herramientas de desarrollo que soporten ambientes multimedia; discutir las ventajas y desventajas de diferentes herramientas y ambientes de desarrollo.
\item Instalar componentes de hardware y software para sonido y video; instalar ambientes de desarrollo y demostrar el uso de los sistemas de software instalados.
\end{LUObjective}
\end{LearningUnit}

\subsection{LU70. Interoperatividad e integración de Sistemas Operativos}\label{sec:LU70}
\begin{LearningUnit}
\begin{LUGoal}
\item Introducir los requerimientos para interoperatividad e integración de sistemas.
\end{LUGoal}

\begin{LUObjective}
\item Explicar conceptos de interoperatividad e integración de sistemas en relación a políticas y prácticas.
\item Explicar componentes de hardware y software para conectar e implementar soluciones de red para redes de computadores y ambientes LAN y WAN más avanzados.
\item Explicar la instalación y configuración de un sistema distribuido.
\item Explicar consideraciones de sistemas operativos para habilitar un ambiente cliente-servidor.
\end{LUObjective}
\end{LearningUnit}

\subsection{LU71. Instalación y configuración de Sistemas Multi-usuario}\label{sec:LU71}
\begin{LearningUnit}
\begin{LUGoal}
\item Instalar, configurar y operar un sistema operativo multi-usuario.
\end{LUGoal}

\begin{LUObjective}
\item Construir estructuras de comandos de software de sistemas (p.e. JCL) tanto para sistemas de {\it mainframe} como de microcomputadores involucrando las infraestructuras macro de los sistemas operativos.
\item Instalar, configurar y operar un sistema operativo multi-usuario.
\end{LUObjective}
\end{LearningUnit}

\subsection{LU72. Tareas de análisis y diseño de sistemas}\label{sec:LU72}
\begin{LearningUnit}
\begin{LUGoal}
\item Presentar los conceptos necesarios para proveer las habilidades necesarias para realizar el análisis, modelado y definición de problemas de sistemas de información.
\end{LUGoal}

\begin{LUObjective}
\item Explicar las fases del ciclo de vida de un sistema de información, así como conceptos y alternativas.
\item Detectar problemas a resolver, realizar reingeniería del flujo físico.
\end{LUObjective}
\end{LearningUnit}

\subsection{LU73. Implementaciones comerciales de Sistemas de Información}\label{sec:LU73}
\begin{LearningUnit}
\begin{LUGoal}
\item Dar a los estudiantes una exposición al uso de productos de programas comerciales para implementar sistemas de información.
\end{LUGoal}

\begin{LUObjective}
\item Demostrar habilidades para analizar métodos alternativos para aplicaciones incluyendo paquetes, personalización de paquetes, adición de módulos a paquetes y construcción de aplicaciones únicas.
\item Explicar los conceptos de adquisición de hardware y software de computadores.
\item Explicar el proceso de escritura de propuestas y contratos.
\item Explicar las fases contractuales y escribir ejemplos realistas para relaciones de consultoría, adquisición de software y hardware, u otros ejemplos relevantes.
\end{LUObjective}
\end{LearningUnit}

\subsection{LU74. Requerimientos y especificaciones de Sistemas de Información}\label{sec:LU74}
\begin{LearningUnit}
\begin{LUGoal}
\item Mostrar como recolectar y estructurar información en el desarrollo de requerimientos y especificaciones.
\end{LUGoal}

\begin{LUObjective}
\item Conducir una entrevista de adquisición de información con individuos y con un grupo.
\item Conducir una sesión JAD usando una herramienta GDS.
\item Usar CASE, I-CASE u otras herramientas automatizadas o no automatizadas.
\item Ser capaz de usar una herramienta CASE comercial para generar documentación {\it ``upper case??}.
\end{LUObjective}
\end{LearningUnit}

\subsection{LU75. Diseño e implementación de Sistemas de Información}\label{sec:LU75}
\subsection{LU76. Prototipado rápido de Sistemas de Información}\label{sec:LU76}
\begin{LearningUnit}
\begin{LUGoal}
\item Desarrollar un entendimiento funcional del prototipado rápido y otros mecanismos alternativos similares para el desarrollo rápido de sistemas de información.
\end{LUGoal}

\begin{LUObjective}
\item Usar prototipeo rápido y otros mecanismos alternativos similares para el desarrollo rápido de sistemas de información.
\end{LUObjective}
\end{LearningUnit}

\subsection{LU77. Riesgos y viabilidad en el desarrollo de Sistemas de Información}\label{sec:LU77}
\begin{LearningUnit}
\begin{LUGoal}
\item Mostrar cómo estimar riesgos y viabilidad.
\end{LUGoal}

\begin{LUObjective}
\item Identificar requerimientos y especificaciones de Sistemas de Información y alternativas lógicas de diseño tentativas; evaluar ventajas competitivas, viabilidad y riesgos propuestos.
\end{LUObjective}
\end{LearningUnit}

\subsection{LU78. Mejora continua de Sistemas de Información}\label{sec:LU78}
\begin{LearningUnit}
\begin{LUGoal}
\item Mostrar a los estudiantes cómo analizar sistemas organizacionales para determinar cómo los sistemas pueden ser mejorados.
\end{LUGoal}

\begin{LUObjective}
\item Comparar varias soluciones de sistemas propuestos basados en criterios para el éxito.
\item Identificar, explicar y usar metodologías de desarrollo compatibles con el concepto del proceso de mejora continua.
\item Aplicar teoría de sistemas, de decisión y de calidad y técnicas y metodologías de desarrollo de sistemas de información para iniciar, especificar e implementar un sistema de información multi-usuario relativamente complejo originado en una organización consciente por la calidad involucrada en la mejora continua de sus procesos.
\item En un nivel empresaria o multi-departamental, desarrollar flujos físicos así como un diseño de flujo de trabajo completo.
\end{LUObjective}
\end{LearningUnit}

\subsection{LU79. Desarrollo consensual}\label{sec:LU79}
\begin{LearningUnit}
\begin{LUGoal}
\item Desarrollar habilidades para la comunicación interpersonal efectiva para desarrollar un consenso usando técnicas clásicas así como {\it groupware} facilitado por computador.
\end{LUGoal}

\begin{LUObjective}
\item Explicar el concepto de visión compartida en el desarrollo efectivo de soluciones a proceso organizacionales.
\item Explicar formas comunes de comportamiento que puedan llevar a un carencia de comunicación.
\end{LUObjective}
\end{LearningUnit}

\subsection{LU80. Dinámicas de grupo}\label{sec:LU80}
\begin{LearningUnit}
\begin{LUGoal}
\item Demostrar y analizar dinámicas de grupos pequeños y cómo se relaciona al trabajo con usuarios.
\end{LUGoal}

\begin{LUObjective}
\item Explicar el comportamiento de grupo y de equipo en un contexto de Sistemas de Información.
\item Explicar cómo los grupos y equipos deben trabajar juntos, motivar colegas y aplicar métodos de equipos; medir y probar la motivación y efectividad; participar eficientemente en trabajo de grupo colaborativo; evaluar el éxito del trabajo.
\end{LUObjective}
\end{LearningUnit}

\subsection{LU81. Aplicaciones de bases de datos}\label{sec:LU81}
\begin{LearningUnit}
\begin{LUGoal}
\item Desarrollar habilidades de aplicación para implementar bases de datos y aplicaciones operando y probando estas bases de datos.
\end{LUGoal}

\begin{LUObjective}
\item Diseñar e implementar un sistema de información dentro de un ambiente de bases de datos.
\item Desarrollar flujo de datos y/o modelos basados en eventos de los componentes de un sistema de información y diseñar la implementación del concepto.
\item Desarrollar la base de datos correspondiente e implementar el esquema con un paquete DBMS.
\item Desarrollar pantallas basadas en eventos correspondientes con el diseño de la base de datos; desarrollar diseños de reportes para la documentación y notificación necesaria; resolver la indexación de la base de datos y construir una aplicación adecuada.
\end{LUObjective}
\end{LearningUnit}

\subsection{LU82. Resolución de problemas, métricas de complejidad}\label{sec:LU82}
\begin{LearningUnit}
\begin{LUGoal}
\item Presentar y usar métricas de complejidad para evaluar las soluciones desarrolladas.
\end{LUGoal}

\begin{LUObjective}
\item Aplicar funciones de software de sistemas para analizar el uso de recursos y el rendimiento característico para una aplicación.
\end{LUObjective}
\end{LearningUnit}

\subsection{LU83. Calidad de software}\label{sec:LU83}
\begin{LearningUnit}
\begin{LUGoal}
\item Desarrollar métricas de calidad para la evaluación del desarrollo de software y control de proyectos del desarrollo de software.
\end{LUGoal}

\begin{LUObjective}
\item Explicar cómo los estándares escritos describiendo cada fase del ciclo de vida puede evolucionar; explicar la relevancia de los estándares escritos, y la conveniencia de desarrollar procedimientos de aseguramiento de la calidad.
\item Describir y explicar el uso de métricas de calidad en la evaluación del desarrollo de software y en facilitar el control de proyecto de las actividades de desarrollo.
\end{LUObjective}
\end{LearningUnit}

\subsection{LU84. Métricas de calidad}\label{sec:LU84}
\begin{LearningUnit}
\begin{LUGoal}
\item Desarrollar métricas de calidad para evaluar la satisfacción del cliente en todas las fases del ciclo de vida.
\end{LUGoal}

\begin{LUObjective}
\item Usar métricas de calidad y {\it benchmarks} de rendimiento para asegurar la satisfacción del cliente para cada fase del ciclo de vida. Probar las métricas durante las actividades de desarrollo del sistema.
\end{LUObjective}
\end{LearningUnit}

\subsection{LU85. Código de ética profesional}\label{sec:LU85}
\begin{LearningUnit}
\begin{LUGoal}
\item Explicar el uso de un código de ética profesional para evaluar acciones de Sistemas de Información específicas.
\end{LUGoal}

\begin{LUObjective}
\item Identificar y describir organizaciones profesionales.
\item Explicar el establecimiento de un estándar ético.
\item Explicar y examinar asuntos éticos y argumentos y métodos fallidos como una función de contexto social.
\item Identificar a los accionistas en un contexto de desarrollo de Sistemas de Información y el efecto del desarrollo en estos individuos.
\item Describir el uso de los códigos de ética y asegurar que las acciones del proyecto son consistentes con estas prescripciones.
\end{LUObjective}
\end{LearningUnit}

\subsection{LU86. Soluciones sinérgicas}\label{sec:LU86}
\begin{LearningUnit}
\begin{LUGoal}
\item Discutir la importancia de encontrar soluciones sinérgicas con equipos y clientes.
\end{LUGoal}

\begin{LUObjective}
\item Describir y explicar los hábitos de interdependencia de escucha simpatética ({\it empathetic listening}), sinergía y construcción de consensos.
\item Explicar actividades de negociación e interdependencia.
\end{LUObjective}
\end{LearningUnit}

\subsection{LU87. Compromiso y acuerdos}\label{sec:LU87}
\begin{LearningUnit}
\begin{LUGoal}
\item Mostrar cómo desarrollar acuerdos describiendo el trabajo a ser realizado, y comprometer, completar rigurosamente y auto-evaluar el trabajo acordado.
\end{LUGoal}

\begin{LUObjective}
\item Desarrollar estimativas de trabajo, comprometerse al trabajo y completar rigurosamente, evaluar comparando con estándares y responder por el trabajo.
\end{LUObjective}
\end{LearningUnit}

\subsection{LU88. Modelado de datos}\label{sec:LU88}
\begin{LearningUnit}
\begin{LUGoal}
\item Desarrollar habilidades para el modelado de datos que describen bases de datos.
\end{LUGoal}

\begin{LUObjective}
\item Usar DBMS, modelado de datos y lenguajes de manipulación de datos.
\item Usar modelos de datos de conocimiento para diferenciar tipos de modelos; explicar los diferentes modelos para bases de datos, p.e. relacional, jerárquico, de red y bases de datos orientadas a objetos; explicar cómo son implementadas en sistemas de administración de bases de datos.
\end{LUObjective}
\end{LearningUnit}

\subsection{LU89. Tipos abstractos de datos: modelos y funciones de Bases de Datos}\label{sec:LU89}
\begin{LearningUnit}
\begin{LUGoal}
\item Desarrollar conciencia de las diferencias sintácticas y teóricas entre los modelos de bases de datos.
\end{LUGoal}

\begin{LUObjective}
\item Identificar los componentes de modelos de bases de datos jerárquicos, de red y relacionales; discutir las definiciones de los datos requeridos para cada modelo; explicar las razones para especificar comandos dentro de las infraestructuras de manipulación de datos; discutir conversión lógica entre los modelos.
\end{LUObjective}
\end{LearningUnit}

\subsection{LU90. Implementación de Bases de Datos y Sistemas de Información}\label{sec:LU90}
\begin{LearningUnit}
\begin{LUGoal}
\item Desarrollar habilidades en la aplicación del desarrollo de sistemas de bases de datos y recuperación de la infraestructura necesaria para facilitar la creación de aplicaciones de sistemas de información.
\end{LUGoal}

\begin{LUObjective}
\item Aplicar implementación de ciclo de vida.
\item Explicar la administración y mantenimiento de bases de datos.
\end{LUObjective}
\end{LearningUnit}

\subsection{LU91. Estructuración de aplicaciones de Bases de Datos}\label{sec:LU91}
\begin{LearningUnit}
\begin{LUGoal}
\item Desarrollar habilidades con aplicación y estructuración de sistemas de administración de  bases de datos.
\end{LUGoal}

\begin{LUObjective}
\item Desarrollar editores para facilitar la entrada de datos en la base de datos.
\item Demostrar el diseño e implementación de habilidades tanto con una interfaz gráfica de usuario como una interfaz basada en caracteres para implementar listas, diálogos, botones y estructuras de menú.
\item Diseñar e implementar reportes simples para validad el rendimiento de sistemas de aplicación.
\item Aplicar principios de desarrollo de software, métodos y herramientas para la implementación de una aplicación de Sistemas de Información.
\end{LUObjective}
\end{LearningUnit}

\subsection{LU92. Implementación de aplicaciones de Bases de Datos}\label{sec:LU92}
\begin{LearningUnit}
\begin{LUGoal}
\item Desarrollar habilidades para la aplicación e implementación física de sistemas de bases de datos, usando un ambiente de programación.
\end{LUGoal}

\begin{LUObjective}
\item Aplicar técnicas de diseño de bases de datos para implementar una solución con llamadas desde programas al DBMS.
\item Explicar y aplicar consideraciones de redes en la implementación de sistemas distribuidos.
\item Desarrollar aplicaciones cliente-servidor e instalarlos y operarlos en un ambiente multi-usuario.
\end{LUObjective}
\end{LearningUnit}

\subsection{LU93. Desarrollo de aplicaciones/generación de código}\label{sec:LU93}
\begin{LearningUnit}
\begin{LUGoal}
\item Desarrollar habilidades para el uso de una combinación de generadores de código e infraestructuras de lenguajes para implementar sistemas de nivel departamentales multi-usuario o de nivel empresarial simple.
\end{LUGoal}

\begin{LUObjective}
\item Usar generadores de código para implementar una aplicación de Sistemas de Información y comparar los resultados con versiones desarrolladas manualmente.
\end{LUObjective}
\end{LearningUnit}

\subsection{LU94. Desarrollo y administración de proyectos}\label{sec:LU94}
\begin{LearningUnit}
\begin{LUGoal}
\item Proveer una oportunidad para desarrollar y usar administración de proyectos, estándares de proyectos y un plan de implementación de sistemas, e implementar un plan de documentación.
\end{LUGoal}

\begin{LUObjective}
\item Crear y presentar documentación técnica y de usuario final de sistemas de telecomunicaciones.
\item Identificar consideraciones de seguridad y de privacidad y cómo pueden ser resueltas dentro del contexto de un sistema de telecomunicaciones.
\item Explicar los controles de configuración.
\item Desarrollar consistentemente con buenas prácticas un proyecto DBMS de nivel departamental y desarrollar documentación de desarrollo y de usuario.
\item Trabajar en equipos realizando seguimiento de resultados individuales y de equipo; desarrollar medidas de calificación para tareas asignadas y rendimiento para evaluar y asegurar la calidad de un proceso de desarrollo.
\item Desarrollar documentación a nivel de programa, de sistema y de usuario.
\item Aplicar conceptis de desarrollo en un proyecto de complejidad razonable en un ambiente de equipo.
\end{LUObjective}
\end{LearningUnit}

\subsection{LU95. Modelos lógicos y conceptuales de Bases de Datos}\label{sec:LU95}
\begin{LearningUnit}
\begin{LUGoal}
\item Mostrar cómo diseñar un modelo de base de datos relacional conceptual y lógico, convertir los diseños de bases de datos lógicos en diseños físicos.
\end{LUGoal}

\begin{LUObjective}
\item Desarrollar la base de datos física y generar datos de prueba.
\item Explicar un {\it framework} para evaluar una función de sistema de información y valorizar las aplicaciones individuales.
\item Explicar el uso de factores de éxito crítico para traducir un diseño de sistemas lógico en un diseño físico en un ambiente objetivo e implementar esta especificación en un sistema operacional usando tecnología DBMS.
\end{LUObjective}
\end{LearningUnit}

\subsection{LU96. Especificaciones funcionales de Sistemas de Información}\label{sec:LU96}
\begin{LearningUnit}
\begin{LUGoal}
\item Proveer oportunidades para desarrollar especificaciones funcionales para un sistema de información, desarrollar un diseño de sistema de información detallado y desarrollar controles de aplicación para sistemas de información.
\end{LUGoal}

\begin{LUObjective}
\item Usar una metodología para especificar y desarrollar un sistema de información significativo para un nivel departamental; asegurar que la recolección de datos, verificación y control son realizados; asegurar que las auditorías externas establecerán objetivos y logros consistentes.
\end{LUObjective}
\end{LearningUnit}

\subsection{LU97. Planeamiento de conversión de Sistemas de Información}\label{sec:LU97}
\begin{LearningUnit}
\begin{LUGoal}
\item Mostrar cómo desarrollar un plan de conversión e instalación, desarrollar un sistema de hardware y un plan ambiental.
\end{LUGoal}

\begin{LUObjective}
\item Desarrollar un plan de entrenamiento, conversión e instalación detallado para hardware y software involucrando una aplicación de sistema de información recién desarrollada.
\item Diseñar soluciones de red e instalar el DBMS en el servidor junto con un sistema operativo apropiado y hardware y software de telecomunicaciones.
\end{LUObjective}
\end{LearningUnit}

\subsection{LU98. Desarrollo y conversión de Sistemas de Información}\label{sec:LU98}
\begin{LearningUnit}
\begin{LUGoal}
\item Mostrar cómo desarrollar especificaciones de programa detalladas, desarrollar programas, configurar parámetros de prueba del sistema, instalar y probar el nuevo sistemas, implementar el plan de conversión, emplear administración de configuración.
\end{LUGoal}

\begin{LUObjective}
\item Desarrollar, probar, instalar y operar un programa de aplicación de sistema de información significativo.
\item Desarrollar, probar, instalar y operar tanto aplicaciones de cliente como de servidor; asegurar que todos los aspectos multi-usuario de la aplicación funcionan como planeado.
\item Desarrollar, probar, instalar y operar sistemas de aplicación acoplados que no posean mecanismos de acoplamiento patológico; describir y explicar cómo otros mecanismos pueden involucrar un mecanismos de acoplamiento inapropiados e ilustrar las consecuencias de tales errores de diseño; discutir y explicar mecanismos de acoplamiento {\it batch} y {\it on-line}.
\end{LUObjective}
\end{LearningUnit}

\subsection{LU99. Requerimientos de Sistemas de Información/planeamiento de flujo de trabajo}\label{sec:LU99}
\begin{LearningUnit}
\begin{LUGoal}
\item Mostrar cómo desarrollar un plan de flujo de trabajo físico con un cliente.
\end{LUGoal}

\begin{LUObjective}
\item Participar de forma no confrontacional en un ambiente de equipo y demostrar habilidades de escucha simpatética para facilitar la determinación de mecanismos alternos para un grupo de trabajo integrado horizontal en la mejora de sus funciones a través del rediseño de procesos, incluyendo la incorporación de sistemas de información para asegurar la documentación y la calidad.
\item Diseñar un flujo de trabajo usando herramientas gráficas o software de desarrollo de sistemas imagen en la presencia de un cliente.
\item Convertir el flujo de trabajo a un diseño de tipo tanto IDEF0 como IDEF3; convertir el diseño IDEF3 en un modelo basado en eventos satisfactorio para una interfaz gráfica de usuario.
\end{LUObjective}
\end{LearningUnit}

\subsection{LU100. Aplicaciones de Sistemas de Información con lenguajes de programación}\label{sec:LU100}
\begin{LearningUnit}
\begin{LUGoal}
\item Desarrollar habilidades de análisis, diseño e implementación de software de aplicación usando un ambiente de programación.
\end{LUGoal}

\begin{LUObjective}
\item Diseñar e implementar software de aplicación de sistemas de información usando un ambiente de programación que utilice programación de bases de datos (el diseño debe incluir editores de pantalla, mecanismos de actualización de datos, controles de auditoria y operación, y debe contender reportes impresos apropiados.)
\item Usar herramientas de productividad para desarrollar modelos conceptuales de datos y funcionales.
\end{LUObjective}
\end{LearningUnit}

\subsection{LU101. Implementación de Sistemas de Información con objetos, dirigido a eventos}\label{sec:LU101}
\begin{LearningUnit}
\begin{LUGoal}
\item Identificar diferencias entre un diseño de aplicación estructurado, basado en eventos y orientado a objetos y explicar las implicaciones de estos métodos para el proceso de diseño y desarrollo.
\end{LUGoal}

\begin{LUObjective}
\item Emplear un ambiente de programación para desarrollar una aplicación simple basado en eventos con una interfaz GUI.
\end{LUObjective}
\end{LearningUnit}

\subsection{LU102. Unidad no definida}\label{sec:LU102}
\subsection{LU103. Pruebas de desarrollo de Sistemas de Información}\label{sec:LU103}
\begin{LearningUnit}
\begin{LUGoal}
\item Ser capaz de desarrollar pruebas de software y de sistemas
\end{LUGoal}

\begin{LUObjective}
\item Construir consultas efectivas usando herramientas de búsqueda tanto estructuradas como no estructuradas.
\item Efectuar ingeniería inversa de aplicaciones de flujos de datos de lenguajes de cuarta generación para asegurar la verificación.
\end{LUObjective}
\end{LearningUnit}

\subsection{LU104. Ambientes de programación para aplicaciones de Sistemas de Información}\label{sec:LU104}
\begin{LearningUnit}
\begin{LUGoal}
\item Entender los diferentes ambientes de programación disponibles para el desarrollo de aplicaciones de negocio.
\end{LUGoal}

\begin{LUObjective}
\item Explicar las características, requerimientos y uso de varios ambientes de programación incluyendo ambientes gráficos y convencionales; explicar los conceptos de portabilidad de software y los conceptos de interoperatibilidad.
\end{LUObjective}
\end{LearningUnit}

\subsection{LU105. Planeamiento de proyectos  de Sistemas de Información}\label{sec:LU105}
\begin{LearningUnit}
\begin{LUGoal}
\item Asegurar las habilidades requeridas para diseñar un plan de desarrollo e implementación de proyectos.
\end{LUGoal}

\begin{LUObjective}
\item Explicar funciones de comité y la racionalidad de equipos horizontales en desarrollo organizacional y reingeniería de sistemas de información.
\end{LUObjective}
\end{LearningUnit}

\subsection{LU106. Administración de proyectos  de Sistemas de Información I}\label{sec:LU106}
\begin{LearningUnit}
\begin{LUGoal}
\item Desarrollar y practicar habilidades esenciales de administración de proyectos.
\end{LUGoal}

\begin{LUObjective}
\item Aplicar conceptos de diseño de reuniones para organizar y conducir un equipo efectivo y reuniones de clientes que aseguren una visión compartida y acciones efectivas.
\end{LUObjective}
\end{LearningUnit}

\subsection{LU107. Administración de proyectos  de Sistemas de Información II}\label{sec:LU107}
\begin{LearningUnit}
\begin{LUGoal}
\item Desarrollar habilidades en el uso de herramientas y métodos de administración de proyectos dentro del contexto de un proyecto de sistemas de información.
\end{LUGoal}

\begin{LUObjective}
\item Usar y aplicar herramientas, técnicas y software de administración de proyectos en la definición, implementación y modificación de objetivos de proyecto.
\item Producir reportes de progreso con información de administración de plazos, de individuos y de equipo para asegurar un desarrollo de software, una implementación de sistema de flujo de trabajo físico y una instalación de sistemas de computadores de calidad.
\end{LUObjective}
\end{LearningUnit}

\subsection{LU108. Herramientas de administración de proyectos}\label{sec:LU108}
\begin{LearningUnit}
\begin{LUGoal}
\item Seleccionar las herramientas de administración de proyectos apropiadas y demostrar su uso.
\end{LUGoal}

\begin{LUObjective}
\item Usar conceptos y herramientas de administración de proyectos  (PERT, GANTT)
\item Usar técnicas de administración de proyectos como PERT, GANTT.
\item Usar CASE y otras herramientas.
\end{LUObjective}
\end{LearningUnit}

\subsection{LU109. Cierre del proyecto}\label{sec:LU109}
\begin{LearningUnit}
\begin{LUGoal}
\item Iniciar, diseñar, implementar y discutir la implantación y término de un proyecto.
\end{LUGoal}

\begin{LUObjective}
\item Discutir y explicar los conceptos involucrados en el término de un proyecto; explicar y listar los requerimientos para finalizar un proyecto.
\end{LUObjective}
\end{LearningUnit}

\subsection{LU110. Sistemas de producción}\label{sec:LU110}
\begin{LearningUnit}
\begin{LUGoal}
\item Determinar y analizar un problema significativo usando el método de sistemas para resolver problemas.
\end{LUGoal}

\begin{LUObjective}
\item Desarrollar y usar especificaciones detalladas para definir y resolver un problema de aplicación de sistemas de información incluyendo flujos físicos, diseño de bases de datos, funciones de sistema, requerimientos y diseño de programación, así como  implementación de software y de bases de datos.
\item Diseñar e implementar un plan de integración de sistemas para un sistema de nivel empresarial involucrando técnicas LAN y WAN; implementar conexiones de sistemas, instalar y configurar sistemas e instalar, probar y operar soluciones diseñadas.
\item Integrar soluciones y metodologías de usuario final en un modelo empresarial; desarrollar e implementar planes de conversión y entrenamiento.
\item Desarrollar y evolucionar estándares escritos para todas las actividades de un ciclo de desarrollo; presentar y defender soluciones: conformar administración de tiempo y de contabilidad para los estándares desarrollados.
\end{LUObjective}
\end{LearningUnit}

\subsection{LU111. Requerimientos y sistemas de bases de datos}\label{sec:LU111}
\begin{LearningUnit}
\begin{LUGoal}
\item Desarrollar requerimientos y especificaciones para una base de datos requiriendo un sistema de información multi-usuario.
\end{LUGoal}

\begin{LUObjective}
\item Identificar flujos físicos e integración horizontal de los procesos organizacionales y relacionar estos flujos a las bases de datos relevantes que los describan
\item Desarrollar modelos funcionales basados en eventos para un proceso organizacional involucrado.
\item Identificar y especificar los procesos que resolverán el problema organizacional y definir la aplicación de bases de datos relacionada.
\end{LUObjective}
\end{LearningUnit}

\subsection{LU112. Acción personal, proactiva, basada en principios}\label{sec:LU112}
\begin{LearningUnit}
\begin{LUGoal}
\item Desarrollar una comprensión funcional de comportamiento proactivo y administración de tiempo.
\end{LUGoal}

\begin{LUObjective}
\item Describir y explicar los hábitos característicos del liderazgo proactivo y de la administración de tiempo.
\end{LUObjective}
\end{LearningUnit}

\subsection{LU113. Escucha simpatética}\label{sec:LU113}
\begin{LearningUnit}
\begin{LUGoal}
\item Asegurar actitudes necesarias para el comportamiento exitoso de equipo incluyendo la escucha simpatética, negociación de consenso, resolución de conflicto y hallazgo de una solución sinérgica, y aplicar el concepto de compromiso y de completitud rigurosa.
\end{LUGoal}

\begin{LUObjective}
\item Usar y aplicar trabajo en equipo, métodos de motivación, aplicar conceptos y métodos de reuniones, usar técnicas de grupo, usar habilidades de escucha simpatética, emplear desarrollo de soluciones sinérgicas.
\item Asegurar que la escucha simpatética es practicada, asegurar que los individuos escuchan, se comprometen y completan rigurosamente las actividades asignadas; explicar la relevancia de dichas acciones para asegurar la efectividad del equipo.
\end{LUObjective}
\end{LearningUnit}

\subsection{LU114. Alineamiento de objetivos, misión}\label{sec:LU114}
\begin{LearningUnit}
\begin{LUGoal}
\item Asegurar el establecimiento de objetivos y el alineamiento de las actividades del equipo con las obligaciones del proyecto.
\end{LUGoal}

\begin{LUObjective}
\item Discutir y explicar los conceptos actividad dirigida por una visión y misión compartida en el desarrollo de sistemas de información.
\item Discutir y aplicar trabajo dirigido a la misión alineando la misión del equipo a la misión del proyecto por medio del seguimiento para asegurar resultados.
\end{LUObjective}
\end{LearningUnit}

\subsection{LU115. Responsabilidad en la venta de diseños a la administración}\label{sec:LU115}
\begin{LearningUnit}
\begin{LUGoal}
\item Describir la interacción con niveles superiores de administración en la explicación de los objetivos de proyecto y en efectuar las tareas de administración del proyecto.
\end{LUGoal}

\begin{LUObjective}
\item Explicar y demostrar la relación de las actividades de Sistemas de Información para mejorar la posición competitiva.
\item Explicar funciones de administración de Sistemas de Información, administrador de proyecto.
\end{LUObjective}
\end{LearningUnit}

\subsection{LU116. Ciclo de vida de sistemas de información y proyectos}\label{sec:LU116}
\begin{LearningUnit}
\begin{LUGoal}
\item Describir y explicar los conceptos ciclo de vida y aplicarlos al proyecto del curso.
\end{LUGoal}

\begin{LUObjective}
\item Explicar los diferentes conceptos de ciclo de vida en la participación y la culminación de un proyecto de tamaño y rango considerables, involucrando equipos; mostrar cómo asegurar la aceptación y la incorporación de estándares compatibles con un ciclo de vida exitoso.
\item Explicar las diferentes responsabilidades de Sistemas de Información, Ciencia de la Computación e Ingeniería de Software, de la forma en que éstas pertenecen a las actividades de desarrollo de software y de sistemas; aplicar las lecciones aprendidas al proyecto del curso.
\item Explicar cómo técnicas formales de ingeniería de software pueden contribuir al éxito de los esfuerzos de desarrollo de software y de sistemas; aplicar estas técnicas al proyecto del curso (calidad, verificación y validación, correctitud y confiabilidad, pruebas, etc.)
\end{LUObjective}
\end{LearningUnit}

\subsection{LU117. Presentación}\label{sec:LU117}
\begin{LearningUnit}
\begin{LUGoal}
\item Mostrar cómo presentar un diseño de sistema, plan de pruebas, plan de implementación y evaluación en forma oral y escrita.
\end{LUGoal}

\begin{LUObjective}
\item Presentar y explicar soluciones a un grupo de pares para recibir críticas y mejorar.
\item Aplicar habilidades de comunicación oral y escritas para presentar soluciones propuestas y logros.
\end{LUObjective}
\end{LearningUnit}

\subsection{LU118. Aprendizaje continuo}\label{sec:LU118}
\begin{LearningUnit}
\begin{LUGoal}
\item Discutir y aplicar el concepto de aprendizaje continuo.
\end{LUGoal}

\begin{LUObjective}
\item Discutir y aplicar el concepto de aprendizaje continuo.
\end{LUObjective}
\end{LearningUnit}

\subsection{LU119. Ética y asuntos legales}\label{sec:LU119}
\begin{LearningUnit}
\begin{LUGoal}
\item Discutir y explicar principios y asuntos éticos y legales; discutir y explicar consideraciones éticas del desarrollo, planeamiento, implementación, uso, venta, distribución, operación y mantenimiento de sistemas de información.
\end{LUGoal}

\begin{LUObjective}
\item Listar y explicar los asuntos éticos y legales en el desarrollo, posesión, venta, adquisición, uso y mantenimiento de sistemas y software de computador.
\item Explicar el uso de modelos éticos, p.e. liderazgo centrado en principios para las fases del ciclo de vida de un sistema de información.
\item Dar ejemplos de los efectos de contexto social en el desarrollo de tecnología.
\end{LUObjective}
\end{LearningUnit}

\subsection{LU120. Administración de SI y organización del departamento de SI}\label{sec:LU120}
\begin{LearningUnit}
\begin{LUGoal}
\item Presentar y explicar el rol de liderazgo evolutivo de la administración de información en las organizaciones.
\end{LUGoal}

\begin{LUObjective}
\item Describir y explicar la composición del personal requerido para formar un equipo para un proyecto determinado y usar estrategias de administración de personal.
\item Explicar a un trabajador, no relacionado al área de sistemas de información, lo que debe hacer para administrar sus recursos y requerimientos de información.
\end{LUObjective}
\end{LearningUnit}

\subsection{LU121. Liderazgo y Sistemas de Información}\label{sec:LU121}
\begin{LearningUnit}
\begin{LUGoal}
\item Presentar y explicar el rol de liderazgo evolutivo de la administración de información en las organizaciones.
\end{LUGoal}

\begin{LUObjective}
\item Explicar el establecimiento un estándar ético.
\item Explicar la relevancia y uso de un código de ética profesional.
\item Explicar y demostrar la aplicación exitosa de un argumento ético en la identificación y evaluación de alternativas basados en un análisis contextual social en un ambiente de desarrollo de sistemas de información centrados en el cliente.
\item Explicar el alineamiento de Sistemas de Información con la misión organizacional; explicar la relación de los procesos departamentales con el éxito estratégico de la organización.
\item Explicar planeamiento y administración de presupuesto.
\item Explicar e ilustrar la aplicación de modelos éticos, p.e. liderazgo centrado en principios, en los estándares y la práctica de administración de proyectos.
\end{LUObjective}
\end{LearningUnit}

\subsection{LU122. Políticas y estándares de Sistemas de Información}\label{sec:LU122}
\begin{LearningUnit}
\begin{LUGoal}
\item Examinar el proceso para el desarrollo de políticas, procedimientos y estándares de los sistemas de información en la organización.
\end{LUGoal}

\begin{LUObjective}
\item Explicar la relevancia de la administración de Sistemas de Información alineándose con el proceso de negocio.
\item Explicar y desarrollar estándares y políticas involucradas en el desarrollo de sistemas de información de ámbito organizacional.
\item Explicar los beneficios de equipos multi-funcionales en el desarrollo de políticas y procedimientos.
\item Explicar los beneficios del desarrollo de la definición de la misión del equipo y del alineamiento de la misión del equipo con las misiones organizacionales.
\end{LUObjective}
\end{LearningUnit}

\subsection{LU123. Administración de la función de los Sistemas de Información}\label{sec:LU123}
\begin{LearningUnit}
\begin{LUGoal}
\item Investigar asuntos relacionados a la administración del funcionamiento de los sistemas de información.
\end{LUGoal}

\begin{LUObjective}
\item Explicar asuntos de seguridad y privacidad.
\item Explicar la base para un contrato legal para desarrollar sistemas.
\end{LUObjective}
\end{LearningUnit}

\subsection{LU124. Administración de tecnologías emergentes}\label{sec:LU124}
\begin{LearningUnit}
\begin{LUGoal}
\item Discutir asuntos pertinentes a la administración y transferencia de las tecnologías emergentes.
\end{LUGoal}

\begin{LUObjective}
\item Explicar y detallar métodos para reconocimiento de ambiente y selección efectiva de hardware y software.
\item Explicar la administración de tecnologías emergentes.
\end{LUObjective}
\end{LearningUnit}

\subsection{LU125. Implementación de Sistemas de Información y {\it outsourcing}}\label{sec:LU125}
\begin{LearningUnit}
\begin{LUGoal}
\item Discutir {\it outsourcing} e implementaciones alternativas de la funcionalidad de un Sistema de Información.
\end{LUGoal}

\begin{LUObjective}
\item Explicar {\it outsourcing} como una alternativa a la funcionalidad de un Sistema de Información interno.
\item Definir, explicar y comparar desde una perspectiva costo-beneficio varios arreglos de {\it outsourcing}.
\item Administrar la funcionalidad de un Sistema de Información en una organización pequeña.
\item Explicar {\it outsourcing}.
\end{LUObjective}
\end{LearningUnit}

\subsection{LU126. Administración de tiempo y relaciones}\label{sec:LU126}
\begin{LearningUnit}
\begin{LUGoal}
\item Discutir la administración del tiempo y las relaciones interpersonales.
\end{LUGoal}

\begin{LUObjective}
\item Explicar cuatro generaciones de conceptos de administración del tiempo, así como razones personales e interpersonales para el éxito de cada etapa; usar los mecanismos dentro de un ambiente de proyectos.
\end{LUObjective}
\end{LearningUnit}

\subsection{LU127. Administración de calidad y desempeño}\label{sec:LU127}
\begin{LearningUnit}
\begin{LUGoal}
\item Discutir la evaluación de rendimiento consistente con la administración de la calidad y mejora continua.
\end{LUGoal}

\begin{LUObjective}
\item Desarrollar medidas de rendimiento consistentes con los conceptos de empleados de valor que faciliten la cooperación del equipo y disminuya la competencia entre miembros del equipo; discutir las razones para tales medidas y explicar las consecuencias negativas de entender erróneamente estos asuntos.
\end{LUObjective}
\end{LearningUnit}

\subsection{LU201. Conceptos Fundamentales de Modelos Empresariales}\label{sec:LU201}
\begin{LearningUnit}
\begin{LUGoal}
\item Discutir, conocer y modelar los procesos dentro de organizaciones desde un punto de vista estratégico.
\end{LUGoal}

\end{LearningUnit}

\subsection{LU202. Automatización y Modelado de Procesos}\label{sec:LU202}
\begin{LearningUnit}
\begin{LUGoal}
\item Estudiar herramientas para el modelado, análisis y monitoreo de procesos empresariales.
\end{LUGoal}

\end{LearningUnit}

\subsection{LU203. Sistemas de Información Estratégicos}\label{sec:LU203}
\begin{LearningUnit}
\begin{LUGoal}
\item Conocer el impacto de la automatización de las prácticas de trabajo, procesos de administración del conocimiento y procesos colaborativos no estructurados.
\end{LUGoal}

\end{LearningUnit}

\subsection{LU204. Unidad no definida}\label{sec:LU204}
\begin{LearningUnit}
\begin{LUGoal}
\item Pendiente.
\end{LUGoal}

\end{LearningUnit}

\subsection{LU205. Introducción a Tecnologías Emergentes}\label{sec:LU205}
\begin{LearningUnit}
\begin{LUGoal}
\item Conocer tecnologías emergentes relevantes para empresas, así como su posible impacto en la economía y las organizaciones y en la transformación de bienes y servicios.
\end{LUGoal}

\end{LearningUnit}

\subsection{LU206. Impacto de las Tecnologías Emergentes en las Empresas}\label{sec:LU206}
\begin{LearningUnit}
\begin{LUGoal}
\item Entender de manera más profunda asuntos técnicos y organizacionales de las tecnologías emergentes, así como las implicanciones estratégicas de éstas.
\end{LUGoal}

\end{LearningUnit}

\subsection{LU207. Aspectos Técnicos de las Tecnologías Emergentes}\label{sec:LU207}
\begin{LearningUnit}
\begin{LUGoal}
\item Tener un mayor conocimiento de los aspectos técnicos de las tecnologías emergentes y los posibles rumbos que puedan tomar.
\end{LUGoal}

\end{LearningUnit}

\subsection{LU208. Administración de Proyectos de Software/Tecnología}\label{sec:LU208}
\begin{LearningUnit}
\begin{LUGoal}
\item Aprender las herramientas y técnicas de la administración y planeamiento de proyectos, incluyendo el uso de software de administración de proyectos.
\end{LUGoal}

\end{LearningUnit}

\subsection{LU209. Administración del Proceso de Cambio}\label{sec:LU209}
\begin{LearningUnit}
\begin{LUGoal}
\item Desarrollar habilidades en las implicaciones humanas y organizacionales del cambio incluyendo el proceso de cambio organizacional.
\end{LUGoal}

\end{LearningUnit}

\subsection{LU210. Planeamiento de Sistemas de Información para el Negocio}\label{sec:LU210}
\begin{LearningUnit}
\begin{LUGoal}
\item Desarrollar un entendimiento del uso estratégico de la tecnología de información desde una perspectiva de negocios al nivel empresarial.
\end{LUGoal}

\end{LearningUnit}

\subsection{LU211. Aspectos Gerenciales de Proyectos Estratégicos de Tecnología de Información}\label{sec:LU211}
\begin{LearningUnit}
\begin{LUGoal}
\item Entender el gerenciamiento interno de los servicios de sistemas de información desde un punto de vista gerencial y examinar estrategias y tácticas alternativas disponibles a la administración para alcanzar los objetivos.
\end{LUGoal}

\end{LearningUnit}

\subsection{LU212. El Sistema Empresarial}\label{sec:LU212}
\begin{LearningUnit}
\begin{LUGoal}
\item Estudiar asuntos administrativos y organizacionales a nivel de empresa como un todo.
\end{LUGoal}

\end{LearningUnit}

\subsection{LU213. La Función de Sistemas de Información}\label{sec:LU213}
\begin{LearningUnit}
\begin{LUGoal}
\item Estudiar la administración de la función de Sistemas de Información para avanzar la política y estrategias de la empresa.
\end{LUGoal}

\end{LearningUnit}

\subsection{LU214. Las Tecnologías en la Organización}\label{sec:LU214}
\begin{LearningUnit}
\begin{LUGoal}
\item Describir el desarrollo de una arquitectura empresarial integrada consonante con las políticas y estrategias de la organización.
\end{LUGoal}

\end{LearningUnit}

\subsection{LU215. Rol de los Sistemas de Información en las Empresas}\label{sec:LU215}
\begin{LearningUnit}
\begin{LUGoal}
\item Proveer una visión general del rol de los sistemas de información en las empresas.
\end{LUGoal}

\end{LearningUnit}

\subsection{LU216. Ética de Sistemas de Información}\label{sec:LU216}
\begin{LearningUnit}
\begin{LUGoal}
\item Discutir aspectos éticos relacionados a los avances desarrollados debido a la digitalización de la información.
\end{LUGoal}

\end{LearningUnit}

\subsection{LU217. Seguridad de Sistemas de Información}\label{sec:LU217}
\begin{LearningUnit}
\begin{LUGoal}
\item Examinar los elementos constitutivos de un ambiente digital seguro.
\end{LUGoal}

\end{LearningUnit}

