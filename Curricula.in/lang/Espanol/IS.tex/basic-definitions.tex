\section{Definiciones básicas}\label{sec:cs-definiciones-basicas}
La referencia más sólida a nivel mundial en cuanto a la propuesta de carreras de computación 
para nivel de pregrado es la que fue propuesta en conjunto por la \ac{ACM}, 
\ac{IEEE-CS} y la \ac{AIS}. Estas tres organizaciones propusieron la 
{\it Computing Curricula} en el documento denominado: 
{\t Joint Task Force for Computing Curricula 2005, Computing Curricula 2005. Overview Report}\cite{ComputingCurricula2005}.

A nivel internacional, la computación presenta 5 perfiles claramente definidos: 
\begin{itemize}
\item \acl{CS} (\acs{CS}) \cite{ComputerScience2001},
\item \acl{CE} (\acs{CE}) \cite{ComputerEngineering2004},
\item \acl{SE} (\acs{SE}) \cite{SoftwareEngineering2004},
\item \acl{IS} (\acs{IS}) \cite{InformationSystems2002Journal} y 
\item \acl{TI} (\acs{TI}) \cite{InformationTechnology2005}
\end{itemize}


La Figura \ref{fig.is} es tomada de la definición propuesta en la {\it Computing Curricula}
\cite{ComputerScience2001,ComputingCurricula2005} en el área de \ac{IS}. 
%\ac{IS} cubre la mayor parte entre el
%extremo superior y el extremo inferior, porque el profesional en \ac{CC} no trata ``solamente con el hardware'' que
%utiliza un software o de ``solamente la organización'' que hace uso de la información que la computación le puede
%proveer. 

\begin{figure}[ht]
   \centering
   \includegraphics[width=13cm]{\OutputFigDir/\currentarea}
   \caption{Campo acción de Sistemas de Información.}
   \label{fig.is}
\end{figure}

Sistemas de Información como un campo de estudio académico empezó en 1960, unos años después del primer uso de las
computadoras para el procesamiento de transacciones y generación de reportes para organizaciones. A medida que las
organizaciones extendieron el uso del procesamiento de información y de tecnologías de comunicación a procesos
operativos, soporte de decisiones y estrategias competitivas, el campo académico también creció en ámbito y profundidad.
La función de Sistemas de Información en las organizaciones emergió para administrar tecnologías computacionales y de
comunicaciones y los recursos de información dentro de una organización. En la misma forma que las universidades tienen
programas profesionales que reflejan las funciones organizacionales importantes, tales como administración de recursos
funancieros, administración de recursos de marketing y administración de recursos humanos, un programa profesional fue
creado para la administración de tecnología de información y recursos de información. Durante este periodo de 40 años de
crecimiento y cambio diferentes nombres han sido usados y la definición del campo ha sido ampliada. El térmido Sistemas
de Información se ha tornado el de mayor aceptación y generalidad para describir la disciplina. Entre otros términos
usados para denominar el área también se han encontrado en el pasado los siguientes.

\begin{itemize}
\item Sistema de Administración de Información.
\item Sistemas de Información Computacional.
\item Administración de la Información.
\item Sistemas de Información de Negocios.
\item Informática (únicamente en Estados Unidos de América). Informática en Europa es sinónimo de Ciencia de la
Computación.
\item Administración de Recursos de Información.
\item Tecnología de Información.
\item Sistemas de Tecnología de Información.
\item Administración de Recursos de Tecnología de Información.
\item Sistemas de Información Contables.
\item Ciencia de la Información.
\end{itemize}
 
Sin embargo, es importante recalcar que con la evolución del área, estos términos han ido desapareciendo y el término
Sistemas de Información es el que se acepta de manera general como aquel que describe todos los aspectos actuales de ésta. 
Tómese como ejemplo el térmido Sistemas de Información Contable, el cual restringe enórmemente el ámbito del área.
 
Sistemas de Información como un campo de estudio académico incluye los conceptos, principios y procesos de dos amplias 
áreas  de actividad dentro de las organizaciones: 
\begin{enumerate}
\item Adquisición, entrega y administración de los recursos y servicios de tecnología de información 
      (la función de sistemas de información)
\item desarrollo, operación y evolución de la infraestructura y los sistemas para su uso en procesos 
      organizacionales (desarrollo de sistemas, operación de sistemas y mantenimiento de sistemas)
\end{enumerate}

Los sistemas y servicios de información que proveen de información a la organización combinan tanto componentes 
técnicos como operadores y usuarios humanos. Estos capturan, almacenan, procesan y comunican datos, 
información y conocimiento.

La función de Sistemas de Información dentro de una organización tiene la amplia responsabilidad de planear, desarrollar
o adquirir, implementar y administrar una infraestructura de tecnologías de información (computadores y comunicaciones),
datos (tanto internos como externos) y sistemas de procesamiento de información a lo largo y ancho de la empresa. Tiene
la responsabilidad de realizar seguimiento de nuevas tecnologías de información para asisitir en su incorporación en la
estrategia, planes y prácticas de la organización. La función de Sistemas de Información también soporta sistemas de
tecnología de información departamentales e individuales. La tecnología empleada puede variar desde sistemas grandes
centralizados a móviles distribuidos. El desarrollo y administración de la infraestructura de tecnología de información 
y sistemas de procesamiento pueden involucran empleados organizacionales, consultores y servicios de \emph{outsourcing}.

La actividad de desarrollar o adquirir aplicaciones de tecnología de información para procesos organizacionales e
inter-organizacionales involucra proyectos que defininen el uso creativo y productivo de la tecnología de información
para el proceso de transaccionaes, adquisición de datos, comunicación, coordinación, análisis y soporte a las
decisiones. Técnicas de diseño, desarrollo o adquisición e implementación, tecnologías y metodologías son empleadas.
Procesos para crear e implementar sistemas de información en organizaciones incorporan conceptos de diseño,
innovación, sistemas humano-computador, interfaces humano-computador, diseño de e-business, sistemas socio-tecnológicos,
administración del camio y calidad de procesos de negocio.







