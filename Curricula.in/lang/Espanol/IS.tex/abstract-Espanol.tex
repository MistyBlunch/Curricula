\chapter*{Resumen ejecutivo}
\AbstractIntro

Todo el contenido del documento está basado en el estandar internacional denominado
\textit{Computing Curricula} en el área específica de Sistemas de Información.
Este documento es el resultado de un trabajo conjunto de la 
\textit{Association for Computing Machinery} (ACM) y la 
Sociedad de Computación de IEEE (IEEE-CS) y puede ser accesado a través de la 
dirección \textit{http://www.acm.org/education/is2002.pdf} en internet.

Considerando que existen peculiaridades menores al aplicar este modelo internacional a Perú,
el modelo de \textit{Computing Curricula} fue utilizado para proponer el documento base de 
la malla curricular peruana de la \acl{SPC}. Dicho documento, por estar en español, 
es la base de esta malla.

La computación hoy en día presenta 5 perfiles claramente definidos: 
\begin{itemize}
\item Ciencia de la Computación (\textit{Computer Science} -- CS),
\item Ingeniería de Computación (\textit{Computer Engineering} -- CE),
\item Ingeniería de Software (\textit{Software Engineering} -- SE),
\item Sistemas de Información (\textit{Information Systems} --IS) y 
\item Tecnología de la Información (\textit{Information Technology} -- IT).
\end{itemize}

Los pilares fundamentales que consideramos en esta propuesta curricular son:
\begin{itemize}
\item Una sólida formación profesional en el área de \SchoolShortName,
\item Preparación orientada al trabajo en organizaciones,
\OnlyUCSP{\item Una sólida formación ética y proyección a la sociedad.}
\end{itemize}

Estos pilares redundarán en la formación de profesionales que se puedan 
desempeñar en cualquier parte del mundo y que ayuden de forma clara al 
desarrollo de la Industria de Software de nuestro país.

\OtherKeyStones

El resto de este documento está organizado de la siguiente forma: 
El Capítulo \ref{chap:intro}, define y explica el campo de acción 
de la \SchoolShortName, además se hace una muy breve explicación de las 
distintas carreras del área de computación, reconocidas en la ACM Curricula.

% \OnlySPC{
% En el Capítulo \ref{chap:cs-market} se presenta el perfil profesional, 
% un análisis de mercado que incluye un anàlisis de la oferta, demanda y 
% tendencias del conocimiento.
% 
% Un anàlisis màs detallado del mercado junto con una encuesta a empresarios
% es presentada en el Capítulo \ref{chap:cs-estudio-de-mercado}. Este estudio 
% arrojó datos interesantes con relaciòn a la percepciòn y la necesidad 
% empresarial de este perfil profesional.
% }

% \OnlyUNSA{
% En el Capítulo \ref{chap:cs-market} se presenta el perfil profesional, 
% un análisis de mercado que incluye un anàlisis de la oferta, demanda 
% y tendencias del conocimiento.
% 
% Un anàlisis màs detallado del mercado junto con una encuesta a empresarios 
% es presentada en el Capítulo \ref{chap:cs-estudio-de-mercado}. Este estudio 
% arrojó datos interesantes con relaciòn a la percepciòn y la necesidad 
% empresarial de este perfil profesional.
% 
% En el Capítulo \ref{chap:cs-resources} se presentan los recursos de 
% lana docente, infraestructura de laboratorios y financieros necesarios 
% para poder crear esta carrera profesional.
% }

El Capítulo \ref{chap:IS-BOK}, es un listado de las unidades de aprendizaje y 
del cuerpo de conocimiento del área de Sistemas de Información indicando los 
tópicos y objetivos cubiertos por cada unos de los temas.

En el Capítulo \ref{chap:IS-malla} se presentan diversas estadísticas con 
relación a la malla tales como: la malla en forma tabular, distribución de 
cursos por áreas, créditos por semestres y por niveles, relación de tópicos 
con cada uno de los cursos y relación de las habilidades esperadas distribuidas 
por curso.

El Capítulo \ref{IS-sumillas} se detalla el contenido y objetivos de los cursos 
de esta propuesta; sus dependencias; número de horas dedicadas a teoría, practica, 
laboratorio, contenido y objetivos por unidad así como la bibliografía recomendada.
