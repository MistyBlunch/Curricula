\begin{LU1}{Sistemas y conceptos de tecnología de Informacion}{OBrien2008IntroductionIS,Turban2005ITforManagement}{}
\begin{goal}
-Introducir a los nuevos usuarios a los conceptos de Sistemas y de Tecnología de Información.
\end{goal}
\begin{objectives}
-Describir y explicar en términos de sistemas los componentes de hardware y software de un sistema computacional.
-Describir, explicar y utilizar un Sistema Operativo e interfaz de usuario para instalar y operar programas, definir y proteger archivos de datos así como utilizar utilitarios del Sistema Operativo.
-Definir, explicar y utilizar software para el trabajo con conocimiento.
\end{objectives}
\end{LU}

\begin{LU2}{Software de trabajo de conocimiento}{Excel2007Walkenbach,Bain2007OpenOffice,PPT2007Wempen,Muir2007PPT,Sun2008OO}{}
\begin{goal}
-Desarrollar las habilidades para utilizar efectivamente software para el manejo de conocimiento tales como sistemas operativos, interfaces de usuario, hojas de cálculo, procesadores de texto, bases de datos, estadísticas y manejo de datos, presentaciones y comunicaciones.
\end{goal}
\begin{objectives}
-Diseñar, desarrollar y utilizar una base de datos simple, importar hojas de cálculo a bases de datos, exportar una tabla de base de datos u hoja de cálculo a un procesador de texto para ser utilizado en un reporte.
-Implementar una presentación basada en slides utilizando un paquete gráfico de presentaciones para comunicar un problema y su solución así como preparar documentos impresos para una posible audiencia.
\end{objectives}
\end{LU}

\begin{LU3}{Resolución de problemas, Sistemas de Información pequeños}{Excel2007Walkenbach,Robbins2006WebDesign,Zakas2007Ajax,Tayntor2005SWImplementation}{}
\begin{goal}
-Introducir los conceptos de resolución de problemas dentro del contexto de Sistemas de Información de complejidad limitada usando software de manejo de conocimiento estándar.
\end{goal}
\begin{objectives}
-Describir, explicar y usar una definición de abordaje de sistemas e implementación de soluciones basadas en PCs utilizando software de manejo de conocimiento (sistemas operativos, interfaces de usuario, hojas de cálculo, procesadores de texto, bases de datos, estadísticas y manejo de datos, presentaciones y comunicaciones) para mejorar la productividad del personal e incrementar las capacidades de trabajo con conocimiento.
-Identificar, definir e implementar una solución que involucre software para el trabajo con conocimiento para organizaciones simples y tareas personales.
-Seleccionar y configurar macros apropiadas, herramientas y paquetes para implementación de sistemas personales.
\end{objectives}
\end{LU}

\begin{LU4}{Tecnología de Información y la sociedad}{Turban2005ITforManagement}{}
\begin{goal}
-Introducir la relevancia y aplicación de la Tecnología de Información en la sociedad.
\end{goal}
\begin{objectives}
-Describir y explicar la relevancia e impacto de la Tecnología de la Información en la sociedad.
-Explicar el rol de los Sistemas de Información dentro de una empresa versus un entorno global.
\end{objectives}
\end{LU}

\begin{LU5}{Sistemas y calidad}{Oz2002ManagementIS,OBrien2008IntroductionIS,OBrien2008IntroductionIS,Laudon2007ManagementIS}{}
\begin{goal}
-Introducir conceptos de sistemas y de calidad.
\end{goal}
\begin{objectives}
-Explicar conceptos de calidad y teoría de sistemas.
\end{objectives}
\end{LU}

\begin{LU6}{Información y calidad}{Daft2000Organization,Hodge1996OrganizationalTheory,Oz2002ManagementIS,OBrien2008IntroductionIS}{}
\begin{goal}
-Proveer una introducción al uso organizacional de la información para mejorar la calidad general.
\end{goal}
\begin{objectives}
-Explicar metodologías para facilitar la medición y alcanzar el ISO9000, Baldridge, {\it National Performance Review} y otros estándares de calidad.
\end{objectives}
\end{LU}

\begin{LU7}{Hardware y software de Tecnología de Información}{ITHWSW2003Englander}{}
\begin{goal}
-Presentar conceptos de Tecnología de Información relacionados a hardware y software.
\end{goal}
\begin{objectives}
-Explicar los elementos y su relación funcional de los principales componentes de hardware, software y comunicaciones que forman PCs, LANs y/o WANs.
\end{objectives}
\end{LU}

\begin{LU8}{Especificación de Sistemas de Tecnología de Información}{OBrien2008IntroductionIS,Stair2007FundamentalsIS,Tayntor2005SWImplementation}{}
\begin{goal}
-Proveer los conceptos y habilidades para la especificación y diseño o reingeniería de sistemas pequeños relacionados con organizaciones basados en Tecnología de la Información.
\end{goal}
\begin{objectives}
-Explicar los conceptos de implementación de Sistemas de Información acoplados a la reingeniería e mejoramiento continuo.
\end{objectives}
\end{LU}

\begin{LU9}{Tecnología de Información y el consecusión de objetivos}{Gray2005MakingDecisions,Turban2005ITforManagement,McNurlin2005ISManagement,Applegate2002Corporate,Kaplan1996Scorecard,Tayntor2005SWImplementation}{}
\begin{goal}
-Mostrar como la Tecnología de Información puede ser utilizada para diseñar, facilitar y comunicar los objetivos organizacionales.
\end{goal}
\begin{objectives}
-Explicar la relevancia del manejo de Sistemas de Información alineados con los objetivos organizacionales.
\end{objectives}
\end{LU}

\begin{LU10}{Características de un profesional de Sistemas de Información}{OBrien2008IntroductionIS}{}
\begin{goal}
-Explicar los conceptos de la toma de decisiones personales, objetivos, definición de metas, confiabilidad y motivación.
\end{goal}
\begin{objectives}
-Discutir y explicar los conceptos de definición de metas y toma y alcance de decisiones individuales. Explicar los requerimientos de definición de metas y toma de decisiones personales en la motivación y mejora de las condiciones de trabajo.
\end{objectives}
\end{LU}


\begin{LU11}{Línea de carrera de carrera de Sistemas de Información}{OBrien2008IntroductionIS,InformationSystemsCurricula2002Book}{}
\begin{goal}
-Mostrar las áreas de la carrera de Sistemas de Información.
\end{goal}
\begin{objectives}
-Identificar y explicar las carreras de telecomunicaciones y sus áreas.
\end{objectives}
\end{LU}
 
\begin{LU12}{Ética y el profesional de Sistemas de Información}{Reynolds2006ITEthics}{}
\begin{goal}
-Presentar y discutir las responsabilidades profesionales y éticas del profesional de Sistemas de Información.
\end{goal}
\begin{objectives}
-Usar códigos de ética profesional para evaluar acciones de Sistemas de Información específicas.
-Describir asuntos éticos y legales; discutir y explicar consideraciones éticas del uso, distribución, operación y mantenimiento de software.
\end{objectives}
\end{LU}
 
\begin{LU13}{Sistemas de Información de nivel personal}{Tayntor2005SWImplementation}{}
\begin{goal}
-Identificar, investigar, analizar, diseñar y desarrollar con paquetes (y/o lenguajes de alto nivel) y sistemas de información de nivel personal para mejorar la productividad personal.
\end{goal}
\begin{objectives}
-Analizar, diseñar, desarrollar y usar paquetes (p.e. un paquete de estadística o de administración de datos de alto nivel) y/o bases de datos de alto nivel que requieran lenguajes para implementar soluciones trabajables para resolver problemas de Sistemas de Información asociados con actividades de trabajo del conocimiento.
-Evaluar el incremento de la productividad realizado a través de la implementación de sistemas personales.
\end{objectives}
\end{LU}
 
\begin{LU13.01}[LU13]{Conceptos de Trabajo y Actividad}{Halpern2002Thought,Tiwana1999Knowledge}{}
\begin{goal}
-Describir el concepto de trabajo del conocimiento y la necesidad de contar con tecnología de información personal que lo soporte.
\end{goal}
\begin{objectives}
-Definir y explicar el concepto de trabajo del conocimiento.
-Comparar y contrastar datos, información y conocimiento.
-Describir las actividades del trabajo del conocimiento; identificar y explicar métodos para lograr productividad en el trabajo del conocimiento.
\end{objectives}
\end{LU}
 
\begin{LU13.02}[LU13]{Soporte: Individuos vs Grupos}{Oz2002ManagementIS,Laudon2007ManagementIS,OBrien2005ManagementIS}{}
\begin{goal}
-Relacionar requerimientos de sistemas de información organizacionales vs. personales.
\end{goal}
\begin{objectives}
-Comparar y constrastar el planeamiento, desarrollo y administración de riesgos de las aplicaciones para sistemas de información personales vs. organizacionales.
-Explicar problemas potenciales de sistemas desarrollados por el usuario.
\end{objectives}
\end{LU}
 
\begin{LU13.03}[LU13]{Análisis de Información: Individual vs. Grupal}{Oz2002ManagementIS,Laudon2007ManagementIS,OBrien2005ManagementIS}{}
\begin{goal}
-Introducir conceptos de trabajo del conocimiento individuales vs. colaborativos y relacionarlos al análisis de las necesidades de información y a la tecnología.
\end{goal}
\begin{objectives}
-Describir y explicar tecnologías individuales vs. grupales; explicar el procesamiento adicional y otros asuntos y necesidades requeridas para el trabajo en grupo.
-Describir y explicar tecnología de soporte a grupos para requerimientos de conocimiento común.
-Describir y explicar el proceso de análisis de información y de aplicación de soluciones de tecnología de información.
\end{objectives}
\end{LU}
 
\begin{LU13.04}[LU13]{Análisis de Información: Encontrando sus requerimientos de Sistemas y Tecnologías de Información}{Stair2007FundamentalsIS,OBrien2008IntroductionIS}{}
\begin{goal}
-Describir y explicar los objetivos y el proceso de análisis y de la documentación del trabajo del conocimiento, tecnología de información y de los requerimientos de información para individuos y grupos de trabajo.
\end{goal}
\begin{objectives}
-Describir y explicar características y atributos del trabajo del conocimiento para individuos y grupos.
-Discutir y explicar las tareas de construcción y mantenimiento del conocimiento.
-Usar preguntas para elicitar sistemáticamente e identificar los requerimientos de datos de individuos y grupos.
-Analizar las tareas individuales y grupales para determinar los requerimientos de información.
-Identificar requerimientos de tecnología de información relacionada.
\end{objectives}
\end{LU}
 
\begin{LU13.05}[LU13]{Organizando recursos de datos personales}{Gray2005MakingDecisions,Tayntor2005SWImplementation}{}
\begin{goal}
-Definir conceptos, principios y métodos prácticos para la administración de software y datos individuales.
\end{goal}
\begin{objectives}
-Dadas tareas y actividades de trabajo del conocimiento, diseñar e implementar un método para la organización de directorios y el etiquetado de archivos que soporte la retención y acceso a los datos.
-Listar principios que apliquen a la adquisición y actualización de software.
-Describir métodos para la transferencia de datos entre aplicaciones incluyendo OLE, importación/exportación y métodos alternativos.
\end{objectives}
\end{LU}
 
\begin{LU13.06}[LU13]{Tecnologías y conceptos de Bases de Datos}{elmasri04}{}
\begin{goal}
-Explicar conceptos organizacionales, componentes, estructuras, acceso, seguridad y consideraciones de administración de bases de datos.
\end{goal}
\begin{objectives}
-Describir y explicar la terminología y el uso de bases de datos relacionales.
-Describir y explicar conceptos necesarios para acceder a bases de datos organizacionales.
-Usar infraestructura de acceso a bases de datos para hacer consultas de datos a partir de un repositorio organizacional.
\end{objectives}
\end{LU}
 
\begin{LU13.07}[LU13]{Acceso, Recuperación y Almacenamiento de Datos}{elmasri04}{}
\begin{goal}
-Definir el contenido, disponibilidad y estrategias para acceder información externa a la organización.
\end{goal}
\begin{objectives}
-Definir y discutir recursos de información externa; identificar la fuente, el contenido, los costos y la temporalidad.
-Localizar y acceder recursos de información externos usando herramientas de Internet disponibles: navegador, búsqueda, ftp.
-Crear y mantener un directorio individual para los recursos de información externa.
\end{objectives}
\end{LU}
 
\begin{LU13.08}[LU13]{Ciclo de vida de Sistemas de Información: Desarrollando con Paquetes}{OBrien2008IntroductionIS,Stair2007FundamentalsIS,Tayntor2005SWImplementation}{}
\begin{goal}
-Presentar y explicar el ciclo de vida de desarrollo de un sistema de información incluyendo los conceptos de adquisición vs. desarrollo de software.
\end{goal}
\begin{objectives}
-Discutir el concepto del ciclo de vida de un sistema de información.
-Identificar y explicar los criterios para decidir entre la adquisición de paquetes de software vs. el desarrollo de software personalizado.
\end{objectives}
\end{LU}
 
\begin{LU13.09}[LU13]{Configuración y personalización de un paquete}{Tayntor2005SWImplementation}{}
\begin{goal}
-Introducir y explorar el uso de software de aplicación y de propósito general.
\end{goal}
\begin{objectives}
-Instalar y personalizar un paquete de software de propósito general para proveer funcionalidad específica más allá de las opciones por defecto.
-Adicionar capacidades a un sistema de software por medio de la grabación y almacenamiento de una macro en la librería del paquete de software dado.
-Acceder a información técnica provista en la forma de facilidades de ``ayuda" del software; observar y usar la infraestructura de ``ayuda".
\end{objectives}
\end{LU}
 
\begin{LU13.10}[LU13]{Programación Procedural y Orientada a Eventos}{Cormen91,Stroustrup97,Meyer98}{}
\begin{goal}
-Introducir y explorar los métodos de desarrollo de software, para luego explicar los objetivos y estrategias de los paradigmas de programación procedural, basado en eventos y orientado a objetos.
\end{goal}
\begin{objectives}
-Discutir y explicar los conceptos de datos y de representación procedural, lenguajes de programación, compiladores, intérpretes, ambientes de desarrollo e interfaces gráficas de usuario basadas en eventos.
-Comparar, relacionar y explicar conceptos de métodos estructurados, basados en eventos y orientados a objetos para el diseño de programas, con ejemplos para cada método.
\end{objectives}
\end{LU}
 
\begin{LU13.11}[LU13]{Implementando algoritmos simples}{Cormen91,Stroustrup97,Meyer98}{}
\begin{goal}
-Introducir y desarrollar el proceso de desarrollo de algoritmos y código estructurado.
\end{goal}
\begin{objectives}
-Definir un problema sencillo identificando las salidas deseadas para entradas dadas; ofrecer una vista panorámica del problema.
-Describir tipos de datos fundamentales y sus operaciones.
-Diseñar lógica de programas usando tanto técnicas gráficas como de pseudocódigo que utilicen estructuras de control estándar: secuencia, iteración y selección.
-Traducir estructuras de datos y diseño de programas en código de un lenguaje de programación; verificar la traducción y asegurar la correctitud de los resultados; evaluar el código con conjuntos de datos de prueba.
\end{objectives}
\end{LU}
 
\begin{LU13.12}[LU13]{Implementación de un Diseño simple de Base de Datos}{elmasri04}{}
\begin{goal}
-Introducir el propósito y desarrollar la habilidad para usar un paquete de software de bases de datos relacionales.
\end{goal}
\begin{objectives}
-Describir y explicar tablas, relaciones, integridad referencial y los conceptos de las formas normales.
-A partir de un dibujo de flujo de trabajo o de otros documentos de requisitos, derivar un diseño de bases de datos simple con múltiples tablas.
-Usando un paquete de software de bases de datos relacionales, implementar y poblar las tablas; desarrollar varias consultas simples para explorar los datos.
\end{objectives}
\end{LU}
 
\begin{LU13.13}[LU13]{Implementación de aplicaciones orientadas a eventos}{Cormen91,Stroustrup97,Meyer98}{}
\begin{goal}
-Introducir y desarrollar la habilidad para diseñar e implementar una infraestructura de interfaz gráfica de usuario.
\end{goal}
\begin{objectives}
-Aplicar una solución de GUI basada en eventos en un ambiente de desarrollo.
-Construir un formulario de aplicación simple con varios objetos (p.e. etiqueta, cajas de edición {\it edit box}, listas, botones de comando).
\end{objectives}
\end{LU}
 
\begin{LU13.14}[LU13]{Desarrollo de Sistemas de Información con prototipado}{Avison98ISD,Kirikova03ISD}{}
\begin{goal}
-Presentar el proceso de prototipeo e introducir y aplicar los conceptos de evaluación y refinamiento evolutivo para prototipos de aplicaciones personales.
\end{goal}
\begin{objectives}
-Comparar las capacidades de una aplicación con los requerimientos que debe cubrir.
-Identificar salidas alternativas del proceso de verificación de aplicaciones.
-Evaluar y definir los resultados y probabilidades de error en software de aplicación prototipeo.
-Modificar entradas, salidas y procesamiento para refinar un prototipo.
\end{objectives}
\end{LU}
 
\begin{LU13.15}[LU13]{Evolución de la Tecnología de Sistemas de Información}{OBrien2008IntroductionIS,Stair2007FundamentalsIS}{}
\begin{goal}
-Presentar tecnologías de fundamento y definir la importancia en el futuro de las capacidades de la tecnología de información.
\end{goal}
\begin{objectives}
-Listar y explicar tecnologías y su relevancia para tecnología de información individual.
-Dada una tecnología, explicar su importancia para los desarrollos futuros y la productividad futura del trabajador del conocimiento.
-Identificar los causantes e inhibidores del cambio en la tecnología de la información.
\end{objectives}
\end{LU}
 
\begin{LU13.16}[LU13]{Implementación de una aplicación de Sistemas de Información personal}{Tayntor2005SWImplementation}{}
\begin{goal}
-Identificar, investigar, analizar, diseñar y desarrollar con paquetes (y/o lenguajes de alto nivel) un sistema de información de nivel personal simple para mejorar la productividad individual.
\end{goal}
\begin{objectives}
-Analizar, diseñar, desarrollar y usar paquetes y/o lenguajes de bases de datos de alto nivel para implementar soluciones trabajables que resuelvan un problema de sistemas de información asociado con actividades de trabajo del conocimiento.
-Evaluar el incremento de productividad realizado implementando sistemas personales.
\end{objectives}
\end{LU}
 
\begin{LU14}{Resolución de problemas con paquetes}{Tayntor2005SWImplementation}{}
\begin{goal}
-Presentar y aplicar estrategias, metodologías y métodos para usar paquetes de software, así como lenguajes de alto nivel para desarrollar soluciones a problemas formales implementables de ``usuario final'', los cuales se encuentran alineados con los sistemas de información organizacionales.
\end{goal}
\begin{objectives}
-Explicar y usar conceptos de problemas formales e ingeniería de software aplicadas al desarrollo de soluciones efectivas que mejoren la productividad personal que involucre actividades de trabajo del conocimiento, dentro de soluciones que son compatibles con el sistema de información organizacional.
-Desarrollar, documentar y mantener sistemas pequeños para productividad personal usando bases de datos de alto nivel usando herramientas y ambientes de desarrollo de aplicaciones.
-Usar los conceptos de definición y resolución de problemas analíticos formales en el uso de paquetes de software; asegurar que dichas soluciones tomen en cuenta los sistemas de información ``reales'' involucrados.
\end{objectives}
\end{LU}

\begin{LU15}{Estrategias de uso de Información}{Halpern2002Thought,Tiwana1999Knowledge}{}
\begin{goal}
-Presentar y aplicar estrategias para acceder y usar recursos de información.
\end{goal}
\begin{objectives}
-Explicar administración de datos y acceso a recursos de información corporativos y alternativos.
-Discutir inteligentemente las diferencias entre la administración de SI\&T, desarrollo de sistemas, mantenimiento de sistemas, operación de sistemas.
\end{objectives}
\end{LU}

\begin{LU16}{Teoría de Sistemas de Información}{OBrien2008IntroductionIS,Stair2007FundamentalsIS}{}
\begin{goal}
-Introducir, discutir y describir conceptos fundamentales de teoría de Sistemas de Información y su importancia para los profesionales.
\end{goal}
\begin{objectives}
-Identificar y explicar los conceptos subyacentes de la disciplina de Sistemas de Información.
\end{objectives}
\end{LU}

\begin{LU17}{Sistemas de Información como un componente estratégico}{Oz2002ManagementIS,OBrien2005ManagementIS,Laudon2007ManagementIS}{}
\begin{goal}
-Mostrar como un sistema de información es un componente estratégico e integral de una organización.
\end{goal}
\begin{objectives}
-Describir el desarrollo histórico de la disciplina de Sistemas de Información.
-Explicar el rol estratégico de los sistemas de información en las organizaciones.
-Explicar la relación estratégica de las actividades de Sistemas de Información para mejorar la posición competitiva.
-Explicar las diferencias entre aplicaciones de nivel estratégico, táctico y operativo.
\end{objectives}
\end{LU}

\begin{LU18}{Desarrollo y administración de Sistemas de Información}{OBrien2008IntroductionIS}{}
\begin{goal}
-Discutir cómo se desarrolla un sistema de información y éste es administrado dentro de una organización.
\end{goal}
 \begin{objectives}
-Explicar el desarrollo de sistemas de información y el rediseño de los procesos organizacionales; explicar los grupos de individuos y sus responsabilidades en este proceso.
-Explicar los roles de los profesionales en Sistemas de Información dentro de una organización de Sistemas de Información; explicar las funciones de la administración de Sistemas de Información, administrador de proyectos, analista de información y explicar los caminos de desarrollo profesional posibles.
\end{objectives}
 \end{LU}

\begin{LU19}{Proceso cognitivo}{Halpern2002Thought,Tiwana1999Knowledge}{}
 \begin{goal}
-Presentar y discutir la relevancia del proceso cognitivo e interacciones humanas en el diseño e implementación de sistemas de información.
\end{goal}
 \begin{objectives}
-Explicar el proceso cognitivo y otras consideraciones orientadas al ser humano en el diseño e implementación de sistemas de información.
\end{objectives}
 \end{LU}

\begin{LU20}{Objetivos y decisiones}{Robbins2002Behavior,Ivancevich2008Behavior,Gray2005MakingDecisions}{}
 \begin{goal}
-Discutir cómo los individuos toman decisiones y establecen y alcanzan objetivos.
\end{goal}
\begin{objectives}
-Discutir y explicar cómo los individuos toman decisiones, establecen y alcanzan objetivos; explicar lo que significa acción personal dirigida a una misión.
\end{objectives}
 \end{LU}

\begin{LU21}{Toma de decisiones: el modelo de Simon}{Gray2005MakingDecisions}{}
\begin{goal}
-Desarrollar la capacidad para discutir e intercambiar opiniones sobre  el Modelo de Simon para la toma de decisiones organizacionales y su soporte utilizando IS.
\end{goal}
\begin{objectives}
-Discutir y explicar la teoría de decisiones y el proceso de toma de decisiones.
-Explicar el soporte de IS para la toma de decisiones; explicar el uso de sistemas expertos en el soporte en la toma de decisiones heurísticas.
-Explicar y dar una ilustración del modelo de decisión organizacional de Simon.
\end{objectives}
\end{LU}

\begin{LU22}{Sistemas y calidad}{Jalote99,Peach02,Kulpa03}{}
\begin{goal}
-Introducir a la teoría de Sistemas, calidad  y modelado organizacional y demostrar su relevancia en los sistemas de información.
\end{goal}
\begin{objectives}
-Discutir y explicar los objetivos de los sistemas, expectativas de los clientes y conceptos de calidad.
-Discutir y explicar los componentes y relaciones de los sistemas.
-Aplicar conceptos de sistemas para definir y explicar el rol de los sistemas de información.
-Explicar el uso de la información y sistemas de información en actividades de documentación toma de decisiones y control organizacional.
\end{objectives}
\end{LU}

\begin{LU23}{Rol de la administración, usuarios, diseñadores de sistemas}{Oz2002ManagementIS,OBrien2005ManagementIS}{}
\begin{goal}
-Discutir un sistema basado en reglas para la administradores, usuarios y diseñadores.
\end{goal}
\begin{objectives}
-Identificar la responsabilidad de los usuarios, diseñadores y administradores en términos descritos en la trinidad Churchman;  discutir en términos de sistemas detallando obligaciones de cada uno, relatar esas observaciones para mejorar los modelos de calidad para el desarrollo organizacional; identificar la función de los  IS en esos términos.
\end{objectives}
\end{LU}

\begin{LU24}{Flujo de trabajo de Sistemas Organizacionales}{Oz2002ManagementIS,OBrien2005ManagementIS}{}
\begin{goal}
-Explicar los sistemas físicos y el flujo de trabajo y como los sistemas de información están relacionados a los sistemas organizacionales.
\end{goal}
\begin{objectives}
-Explicar la relación entre el modelo de base de datos   y la actividad física organizacional.
\end{objectives}
\end{LU}

\begin{LU25}{Modelos y relaciones organizacionales con Sistemas de Información}{Oz2002ManagementIS,OBrien2005ManagementIS}{}
\begin{goal}
-Presentar otros modelos organizacionales  y su relevancia para los IS.
\end{goal}
\begin{objectives}
-Describir el rol de la tecnología de información (IT) y las reglas de las personas usando, diseñando y manteniendo IT en las organizaciones.
-Discutir como la teoría general de sistemas es aplicada al análisis y desarrollo de los sistemas de información.
\end{objectives}
\end{LU}

\begin{LU26}{Planeamiento de Sistemas de Información}{Stair2007FundamentalsIS,OBrien2008IntroductionIS}{}
\begin{goal}
-Discutir la relación  entre el planeamiento de los IS  con el planeamiento organizacional.
\end{goal}
\begin{objectives}
-Explicar metas y procesos de planeamiento.
-Explicar la importancia del planeamiento estratégico  y cooperativo  así como el alineamiento del plan proyecto de los sistemas de información.
\end{objectives}
\end{LU}

\begin{LU27}{Tipos de Sistemas de Información}{OBrien2008IntroductionIS,Stair2007FundamentalsIS}{}
\begin{goal}
-Demostrar clases específicas de sistemas de aplicación incluyendo TPS y DSS.
\end{goal}
\begin{objectives}
-Describir la clasificación de los sistemas de información, por  ejemplo, TPS, DSS, ESS, WFS.
-Explicar la relevancia organizacional de los IS: TPS, DSS, EIS, ES, {\it Work Flow System}.
\end{objectives}
\end{LU}

\begin{LU28}{Estándares de desarrollo de Sistemas de Información}{OBrien2008IntroductionIS}{}
\begin{goal}
-Discutir  y examinar los procesos, estándares y políticas para el desarrollo de sistemas de información. Desarrollo de metodologías, ciclo de vida, workflow, OOA, prototipeo, espiral, usuario final entre otros.
\end{goal}
\begin{objectives}
-Discutir y explicar el concepto de una metodología de desarrollo de IS, explicar el ciclo de vida, workflow, OOA, prototipeo, modelos basado en riesgos, modelo en espiral, entre otros; mostrar como esto puede ser usado en la práctica.   
\end{objectives}
\end{LU}

\begin{LU29}{Implementación de Sistemas de Información: \textit{outsourcing}}{OBrien2008IntroductionIS,Stair2007FundamentalsIS}{}
\begin{goal}
-Discutir {\it outsourcing} e implementaciones alternativas de IS.
\end{goal}
\begin{objectives}
-Explicar las ventajas  y desventajas del desarrollo {\it outsourcing} en algunas o todas las funciones de IS; establecer  los requerimientos del personal con o sin {\it outsourcing}
\end{objectives}
\end{LU}

\begin{LU30}{Evaluación de desempeño del personal}{Stoner1995Management,Daft2007Management}{}
\begin{goal}
-Discutir la evaluación del rendimiento  la cual consiste e con la administración de la calidad  y la mejora continua.
\end{goal}
\begin{objectives}
-Describir, explicar y aplicar las responsabilidades del líder del proyecto, administrar el desarrollo de pequeños sistemas.
-Discutir, explicar e implementar una metodología para hacer seguimiento a los clientes dentro de todo las fases del ciclo de vida
-Explicar metodologías para facilitar el uso de estándares como el ISO 9000, {\it National Performance Review} y otros estándares de calidad.
\end{objectives}
\end{LU}

\begin{LU31}{Sociedad de Sistemas de Información y ética}{Reynolds2006ITEthics}{}
\begin{goal}
-Introducir las implicaciones sociales y éticas de los Sistemas de Información para introducir a la exploración de los conceptos éticos y asuntos relacionados al comportamiento profesional.
-Comparar y contrastar los modelos e abordajes éticos.
\end{goal}
\begin{objectives}
-Discutir y explicar ética y comportamiento basado en principios así como el concepto de práctica ética en el área de Sistemas de Información.
-Discutir modelos éticos importantes y discutir las razones por las cuales hay que ser ético.
-Explicar el uso del código de ética profesional.
-Explicar la carga responsabilidad y de profesionalismo resultante de la confianza asociada con el conocimiento y habilidades de computación.
-Discutir y explicar las bases y naturaleza de los abordajes éticos cuestionables.
-Discutir y explicar el análisis ético y social del desarrollo de Sistemas de Información.
-Discutir y explicar los asuntos de poder y su impacto social en el ciclo de vida del desarrollo.
\end{objectives}
\end{LU}

\begin{LU32}{Dispositivos, medios, sistemas de Telecomunicaciones}{sheldon1994,tanenbaum2003,Deitel04}{}
\begin{goal}
-Desarrollar la preocupación y la terminología asociada de los diferentes y dispositivos necesarios para telecomunicaciones, incluyendo redes LAN y WAN.
\end{goal}
\begin{objectives}
-Identificar las características de la transmisión de datos en telecomunicaciones a nivel de LANs, WANs y MANs.
-Accesar información remota para transferencia de archivos en entornos LAN y WAN.
-Discutir y explicar la industria de las telecomunicaciones así como sus estándares y regulaciones.
\end{objectives}
\end{LU}

\begin{LU33}{Soporte organizacional basado en Telecomunicaciones}{Cisco04,sheldon1994,tanenbaum2003}{}
\begin{goal}
-Desarrollar una preocupación por la forma en la que los sistemas de telecomunicaciones son utilizados para soportar la infraestructura de comunicaciones de la organización incluyendo a los Sistemas de Información, teleconferencias, etc.
\end{goal}
\begin{objectives}
-Explicar el uso de los Sistemas de Información para soportar el flujo de trabajo;
-Discutir los conceptos de teleconferencias y conferencias por telecomputadoras en el rol de las comunicaciones y en la toma de decisiones.
-Discutir y explicar la infraestructura involucrada en los sistemas de telecomunicaciones.
\end{objectives}
\end{LU}

\begin{LU34}{Economía y problemas de diseño de sistemas de Telecomunicaciones}{Cisco04,sheldon1994,tanenbaum2003}{}
\begin{goal}
-Explorar los asuntos relacionados al diseño y manejo económico de las redes de computadores.
\end{goal}
\begin{objectives}
-Explicar los pasos en el análisis y configuración de un sistema de telecomunicaciones, incluyendo hardware específico y componentes de software.
-Explicar el propósito de modems, bridges, gateways, hubs y ruteadores en la interconexión de sistemas.
\end{objectives}
\end{LU}

\begin{LU35}{Estándares de Telecomunicaciones}{Cisco04,sheldon1994,tanenbaum2003}{}
\begin{goal}
-Familiarizar al estudiante con los estándares de telecomunicaciones, con las organizaciones que las regulan y con sus estándares.
\end{goal}
\begin{objectives}
-Identificar el rol de los estándares y de las organizaciones regulatorias y sus estándares como facilitadores para lograr desde telecomunicaciones locales hasta aquellas globales.
-Explicar la codificación digital de datos relevantes a las telecomunicaciones.
\end{objectives}
\end{LU}

\begin{LU36}{Sistemas centralizados vs distribuidos}{ozsu99,coulouris2005}{}
\begin{goal}
-Discutir y explicar los principios fundamentales y temas relacionados a comparar la computación centralizada versus computación distribuida.
\end{goal}
\begin{objectives}
-Explicar, diagramar y discutir las estructuras y principios involucrados en la computación distribuida en cuanto a recursos y datos.
-Identificar requerimientos de hardware y de software y una aproximación de costos para sistemas centralizados y distribuidos.
-Discutir y explicar riesgos, seguridad y privacidad en configuraciones alternativas de sistemas.
\end{objectives}
\end{LU}

\begin{LU37}{Arquitecturas, topologías y protocolos de Telecomunicaciones}{Cisco04,sheldon1994,tanenbaum2003}{}
\begin{goal}
-Presentar arquitecturas, topologías y protocolos de telecomunicaciones.
\end{goal}
\begin{objectives}
-Identificar y explicar las funciones de cada una de las capas del modelo ISO.
-Explicar el concepto de comunicaciones virtuales entre dos computadores a cada nivel del modelo ISO.
-Identificar y explicar topologías comunes y métodos de implementación de sistemas de telecomunicaciones.
-Identificar y describir la organización y operación de los protocolos de bits y bytes.
-Discutir los servicios de telecomunicaciones y analizar una implementación específica del modelo ISO.
\end{objectives}
\end{LU}

\begin{LU38}{Hardware y software de Telecomunicaciones}{Cisco04,sheldon1994,tanenbaum2003}{}
\begin{goal}
-Presentar los componentes de hardware y software involucrados en un sistema de telecomunicaciones y como ellos están organizados para proveer los servicios requeridos.
\end{goal}
\begin{objectives}
-Describir, diagramar, discutir y explicar los componentes de hardware y software de los sistemas de telecomunicaciones. Describir la integración de teléfono, fax, redes LAN y WAN. Diagramar y discutir  varias organizaciones de hardware de cada tipo de dispositivo requerido.
-Explicar el uso de ruteadores y hubs en el diseño de sistemas interconectados.
-Explicar los requerimientos de telecomunicaciones para voz, audio, datos, imágenes, vídeo y multimedia en general.
-Explicar tecnologías y aplicaciones de paquetes rápidos.
-Explicar asuntos relacionados al diseño de redes de telecomunicaciones.
-Dar ejemplos de aplicaciones de telecomunicaciones en negocios y explicar su utilización en el sistema descrito.
\end{objectives}
\end{LU}

\begin{LU39}{Servicios, confiabilidad y seguridad de los sistemas de telecomunicaciones}{Cisco04,sheldon1994,tanenbaum2003}{}
\begin{goal}
-Concientizar al estudiante sobre la preocupación inherente cuando uno provee servicios de telecomunicaciones incluyendo seguridad, privacidad, confiabilidad y desempeño.
\end{goal}
\begin{objectives}
-Explicar medidas del desempeño de sistemas de telecomunicaciones y asegurar un adecuado desempeño y confiabilidad.
\end{objectives}
\end{LU}

\begin{LU40}{Instalación e implantación de sistemas de Telecomunicaciones}{Cisco04,sheldon1994,tanenbaum2003}{}
\begin{goal}
-Explicar como instalar el equipo necesario para implementar un sistema de telecomunicaciones. Esto es: cable, modems, Ethernet, conexiones, gateways, ruteadores.
\end{goal}
\begin{objectives}
-Explicar, instalar y probar modems, multiplexores y componentes Ethernet.
-Explicar, instalar y probar {\it bridges} y ruteadores en el hardware apropiado.
-Instalar y operar software de emulación de un terminal en una PC.
-Explicar y construir planes organizacionales para el uso de EDI.
\end{objectives}
\end{LU}

\begin{LU41}{Instalación y configuración de redes LAN}{Cisco04,sheldon1994,tanenbaum2003}{}
\begin{goal}
-Explicar cómo diseñar, instalar, configurar y administrar una LAN
\end{goal}
\begin{objectives}
-Diseñar, instalar y administrar una LAN
-Explicar e implementar seguridad apropiada para un ambiente de usuario final involucrando acceso a un sistema de información de nivel empresarial.
\end{objectives}
\end{LU}

\begin{LU42}{Medidas de información/datos/eventos}{Ivancevich2008Behavior,Daft2000Organization,Hodge1996OrganizationalTheory}{}
\begin{goal}
-Presentar que el concepto de datos es una representación y medición de eventos del mundo real.
\end{goal}
\begin{objectives}
-Explicar el concepto de medición e información, representación de información, organización, almacenamiento y procesamiento.
-Describir que el concepto de datos es una representación y medición de eventos del mundo real y el proceso de capturarlos en una máquina de forma legible.
\end{objectives}
\end{LU}

\begin{LU43}{Datos: caracteres, registros, archivos, multimedia}{Folk97FileStructures}{}
\begin{goal}
-Mostrar y explicar la lógica y la estructura física de los datos para representar caracteres, registros, archivos y objetos multimedia.
\end{goal}
\begin{objectives}
-Identificar, explicar y discutir la jerarquía de datos e identificar todas las operaciones primarias asociadas con cada nivel de la jerarquía.
\end{objectives}
\end{LU}

\begin{LU44}{Tipos abstractos de datos, clases, objetos}{Meyer98,Stroustrup97}{}
\begin{goal}
-Explicar los conceptos de clases, tipos abstractos de datos y objetos.
\end{goal}
\begin{objectives}
-Discutir clases que involucran elementos de la jerarquía de datos (bit, byte, campos, registros, archivos, bases de datos), y usar estas definiciones como base para la resolución de problemas; describir las estructuras de un programa y su uso relacionado a cada estructura.
\end{objectives}
\end{LU}

\begin{LU45}{Resolución de problemas formales y Sistemas de Información}{OBrien2008IntroductionIS,Stair2007FundamentalsIS}{}
\begin{goal}
-Explicar e ilustrar con ejemplos de sistemas de información la resolución formal y analítica de problemas.
\end{goal}
\begin{objectives}
-Explicar y dar ejemplos del concepto de escribir programas de computador y usar lenguajes de desarrollo de software e infraestructuras para el desarrollo de aplicaciones para resolver problemas.
\end{objectives}
\end{LU}

\begin{LU46}{Representación de objetos en un Sistemas de Información}{Schach04}{}
\begin{goal}
-Presentar un sistema de vista de representaciones de objetos y compararlo con modelos de flujo de datos.
\end{goal}
\begin{objectives}
-Discutir y explicar un sistema desde el punto de vista de una representación de objetos; explicar la similitud de una representación de objetos para una notación de flujo de datos convencional.
\end{objectives}
\end{LU}

\begin{LU47}{Diseño de algoritmos}{Cormen91}{}
\begin{goal}
-Desarrollar capacidades en el desarrollo de una solución algorítmica a un problema para ser capaces de representarla con un programa y objetos de datos apropiados.
\end{goal}
\begin{objectives}
-Diseñar algoritmos y traducirlos en soluciones operativas en un lenguaje de programación para muchos componentes del problema involucrados en aplicaciones de sistemas de información completas.
\end{objectives}
\end{LU}

\begin{LU48}{Implementación \textit{Top-Down}}{Cormen91}{}
\begin{goal}
-Presentar estrategias de implementación {\it top-down}.
\end{goal}
\begin{objectives}
-Diseñar e implementar programas en una forma {\it top-down}, construyendo inicialmente los niveles superiores desarrollando un esqueleto para los niveles inferiores; completar sucesivamente los niveles inferiores de la misma manera; identificar el concepto de éxito continuado en este método.
\end{objectives}
\end{LU}

\begin{LU49}{Implementación de objetos}{Meyer98,Stroustrup97}{}
\begin{goal}
-Presentar los conceptos implementación de objetos.
\end{goal}
\begin{objectives}
-Explicar e implementar estructuras modulares; mostrar la relación de flujo de datos y de representaciones de objetos con el código producido.
\end{objectives}
\end{LU}

\begin{LU50}{Módulos/cohesión/acoplamiento}{Wang00,Pressman04,Blum92}{}
\begin{goal}
-Presentar conceptos de diseño modular, cohesión y acoplamiento.
\end{goal}
\begin{objectives}
-Desarrollar y traducir una representación de flujo de datos de una solución a un problema para una representación jerárquica y/o de objetos.
-Usar diseño algorítmico y modular en la solución de un problema e implementar la solución con un lenguaje procedural.
-Usar paso de parámetros en la implementación de una solución modular a un problema; explicar la importancia de una alta cohesión y un bajo acoplamiento.
-Aplicar conceptos de diseño modular para definir módulos cohesivos de tamaño apropiado.
-Aplicar estructuras de control de programación y verificar correctitud.
-Demostrar habilidad para probar y validar la solución.
\end{objectives}
\end{LU}

\begin{LU51}{Verificación y validación, una visión de Sistemas}{Wang00,Pressman04,Blum92}{}
\begin{goal}
-Presentar una vista sistemas de verificación y validación.
\end{goal}
\begin{objectives}
-Explicar el proceso de verificación y validación; verificar código a través de reingeniería manual tanto para representaciones procedurales y/o de objetos.
-Desarrollar diseños de flujos de datos y traducir esos diseños a pseudocódigo de lenguajes de cuarta generación.
\end{objectives}
\end{LU}

\begin{LU52}{Resolución de problemas, ambientes y herramientas}{Kirikova03ISD,Avison98ISD}{}
\begin{goal}
-Presentar y exponer estudiantes a una variedad de ambientes de programación, desarrollar herramientas y ambientes de desarrollo gráficos.
\end{goal}
\begin{objectives}
-Demostrar habilidad para evaluar y usar componentes de GUI existentes en la construcción de una interfaz de usuario efectiva para una aplicación.
\end{objectives}
\end{LU}

\begin{LU53}{Tipos abstractos de datos: estructuras de datos y archivos}{Folk97FileStructures}{}
\begin{goal}
-Introducir los conceptos y técnicas usadas para representar y operar en estructuras de datos y de archivos, con ejemplos simples.
\end{goal}
\begin{objectives}
-Explicar los tipos abstractos de datos necesarios para acceder a registros en un archivo de datos indexados; mostrar ejemplos de cada tipo de operación requerida.
\end{objectives}
\end{LU}

\begin{LU54}{Tipos abstractos de datos: arreglos, listas, árboles, registros}{Folk97FileStructures}{}
\begin{goal}
-Explicar cómo desarrollar estructuras usando tipos de datos abstractos representando arreglos, listas árboles, registros y archivos, y demostrar cómo son aplicadas como componentes de programas y aplicaciones.
\end{goal}
\begin{objectives}
-Usar representaciones de arreglos para simular el acceso a un archivo indexado, y usar la representación en el diseño de un tipo abstracto de datos para insertar, eliminar, encontrar e iterar sobre elementos.
\end{objectives}
\end{LU}

\begin{LU55}{Tipos abstractos de datos: archivos indexados, llaves}{Folk97FileStructures}{}
\begin{goal}
-Presentar y usar estructuras de indexación de archivos, incluyendo organizaciones de llaves.
\end{goal}
\begin{objectives}
-Discutir y explicar el concepto de archivos indexados; describir construcciones de llaves y comparar requerimientos de administración de datos involucrados en elegir las llaves óptimas; explicar las funciones que son necesarias para implementar y acceder a registros indexados; explicar la similitud de arreglos y archivos indexados en términos de similitudes de funciones en tipos abstractos de datos.
\end{objectives}
\end{LU}

\begin{LU56}{Resolución de problemas, aplicaciones de Sistemas de Información}{OBrien2008IntroductionIS,Stair2007FundamentalsIS}{}
\begin{goal}
-Explicar una variedad de estructuras fundamentales que son los bloques de construcción para el desarrollo de programas y de aplicaciones de sistemas de información.
\end{goal}
\begin{objectives}
-Aplicar software de aplicación para resolver problemas de pequeña escala.
-Desarrollar documentación de usuario y del sistema para una solución programática para un problema de complejidad moderada.
\end{objectives}
\end{LU}

\begin{LU57}{Resolución de problemas, aplicaciones de datos y archivos}{Folk97FileStructures}{}
\begin{goal}
-Proveer los fundamentos para aplicaciones de estructuras de datos y técnicas de procesamiento de archivos.
\end{goal}
\begin{objectives}
-Usar tipos abstractos de datos involucrados en aplicaciones de sistemas de información comunes para implementar soluciones a problemas involucrando técnicas de procesamiento de archivos indexados.
\end{objectives}
\end{LU}

\begin{LU58}{Resolución de problemas con archivos y bases de datos}{elmasri04}{}
\begin{goal}
-Presentar y asegurar la resolución de problemas involucrando archivos y representaciones de bases de datos.
\end{goal}
\begin{objectives}
-Usar archivos indexados e tipos de datos abstractos para resolver problemas simples involucrando archivos usados como elementos de una solución de bases de datos.
\end{objectives}
\end{LU}

\begin{LU59}{Resolución de problemas, archivos/editores de bases de datos/reportes}{elmasri04,rob04}{}
\begin{goal}
-Presentar y desarrollar editores útiles de archivos estructurados (bases de datos), mecanismos de posteo, y reportes.
\end{goal}
\begin{objectives}
-Construir y documentar varias aplicaciones usando archivos indexados, editores de pantalla y reportes.
\end{objectives}
\end{LU}

\begin{LU60}{Resolución de problemas, diseño, pruebas, depuración}{Cormen91}{}
\begin{goal}
-Continuar el desarrollo de técnicas de programación, particularmente en el diseño, pruebas y depuración de programas relacionados a sistemas de información de cierta complejidad.
\end{goal}
\begin{objectives}
-Definir, explicar y presentar el proceso de definir y resolver problemas analíticos formales.
\end{objectives}
\end{LU}

\begin{LU61}{Programación: comparación de lenguajes}{hen96,rob05,rav96}{}
\begin{goal}
-Desarrollar una conciencia de las capacidades y limitaciones relativas de los lenguajes de programación más comunes.
\end{goal}
\begin{objectives}
-Explicar las capacidades y diferencias para ambientes y lenguajes de programación.
\end{objectives}
\end{LU}

\begin{LU62}{Telecomunicaciones, visión de sistemas de hardware y software}{Cisco04,sheldon1994,tanenbaum2003}{}
\begin{goal}
-Explicar en términos de sistemas las características fundamentales y componentes de computadores y hardware de telecomunicaciones, y software de sistemas, y demostrar cómo estos componentes interactúan.
\end{goal}
\begin{objectives}
-Usar la metodología de sistemas para explicar los componentes de hardware y software de un sistema de telecomunicaciones así como diagramar y discutir la naturaleza de las interacciones de los componentes.
-Explicar en términos de sistemas el propósito, resultados esperados y calidad de un sistema de telecomunicaciones así mostrar cómo los componentes trabajan juntos.
\end{objectives}
\end{LU}

\begin{LU63}{Dispositivos periféricos}{Flores74Peripheral}{}
\begin{goal}
-Proveer una vista general de dispositivos periféricos y sus funciones.
\end{goal}
\begin{objectives}
-Identificar clases importantes de dispositivos periféricos y explicar los principios de operación de los requerimientos de software y funciones provistas por cada tipo de dispositivo; dar ejemplos específicos de cada dispositivo identificado y discutir los requerimientos de instalación para el hardware y software requerido.
\end{objectives}
\end{LU}

\begin{LU64}{Arquitectura de computadores}{Morris92,BarryBrey05,Norton88}{}
\begin{goal}
-Introducir los conceptos de arquitecturas de computadores.
\end{goal}
\begin{objectives}
-Definir requerimientos de datos y comunicación para acceder datos locales (discos duros o servidores) y remotos (p.e., vía Internet) para resolver problemas individuales.
-Describir y explicar los componentes más importantes de hardware y software de un sistema de computación y cómo interactúan.
\end{objectives}
\end{LU}

\begin{LU65}{Componentes de software e interacciones}{Schach04,Pressman04}{}
\begin{goal}
-Introducir los conceptos de componentes de software e interacciones.
\end{goal}
\begin{objectives}
-Describir y explicar los componentes más importantes de un sistema operativo y como interactúan.
-Explicar el control de funciones de entrada/salida; instalar y configurar controladores.
\end{objectives}
\end{LU}

\begin{LU66}{}{}{}
\begin{goal}
\end{goal}
\begin{objectives}
\end{objectives}
\end{LU}

\begin{LU67}{Funciones de sistemas operativos}{tanenbaum1996,orfali1999}{}
\begin{goal}
-Introducir los conceptos más importantes en sistemas operativos, incluyendo la definición de procesos, procesamiento concurrentes, administración de memoria, {\it scheduling}, procesamiento de interrupciones, seguridad y sistemas de archivos.
\end{goal}
\begin{objectives}
-Explicar el concepto de tareas y procesos.
-Explicar el concepto de concurrencia y multi tareas ({\it Multitasking}).
-Explicar el comportamiento rutinario de {\it schedulers} de tareas, colas de prioridad, procesamiento de interrupciones, administración de memoria y sistemas de archivos.
\end{objectives}
\end{LU}

\begin{LU68}{Ambientes y recursos de sistemas operativos}{tanenbaum1996,orfali1999}{}
\begin{goal}
-Introducir una variedad de ambientes de operación (tradicional, GUI, multimedia) y requerimientos de recursos.
\end{goal}
\begin{objectives}
-Describir y discutir varios ambientes operativos de sistemas de computadoras incluyendo tradicionales, de interfaz de usuario gráfica y multimedia;
-Estimar los items de hardware y software y aproximar el costo para cada ambiente; discutir ventajas relativas para cada ambiente.
\end{objectives}
\end{LU}

\begin{LU69}{Instalación y configuración de sistemas operativos para multimedia}{tanenbaum1996,orfali1999}{}
\begin{goal}
-Discutir, explicar e instalar infraestructuras multimedia.
\end{goal}
\begin{objectives}
-Discutir y explicar los requerimientos de hardware y software necesarios para soportar multimedia.
-Explicar herramientas de desarrollo que soporten ambientes multimedia; discutir las ventajas y desventajas de diferentes herramientas y ambientes de desarrollo.
-Instalar componentes de hardware y software para sonido y video; instalar ambientes de desarrollo y demostrar el uso de los sistemas de software instalados.
\end{objectives}
\end{LU}

\begin{LU70}{Interoperatividad e integración de Sistemas Operativos}{tanenbaum1996,orfali1999}{}
\begin{goal}
-Introducir los requerimientos para interoperatividad e integración de sistemas.
\end{goal}
\begin{objectives}
-Explicar conceptos de interoperatividad e integración de sistemas en relación a políticas y prácticas.
-Explicar componentes de hardware y software para conectar e implementar soluciones de red para redes de computadores y ambientes LAN y WAN más avanzados.
-Explicar la instalación y configuración de un sistema distribuido.
-Explicar consideraciones de sistemas operativos para habilitar un ambiente cliente-servidor.
\end{objectives}
\end{LU}

\begin{LU71}{Instalación y configuración de Sistemas Multi-usuario}{tanenbaum1996,orfali1999}{}
\begin{goal}
-Instalar, configurar y operar un sistema operativo multi-usuario.
\end{goal}
\begin{objectives}
-Construir estructuras de comandos de software de sistemas (p.e. JCL) tanto para sistemas de {\it mainframe} como de microcomputadores involucrando las infraestructuras macro de los sistemas operativos.
-Instalar, configurar y operar un sistema operativo multi-usuario.
\end{objectives}
\end{LU}

\begin{LU72}{Tareas de análisis y diseño de sistemas}{Kirikova03ISD,Avison98ISD}{}
\begin{goal}
-Presentar los conceptos necesarios para proveer las habilidades necesarias para realizar el análisis, modelado y definición de problemas de sistemas de información.
\end{goal}
\begin{objectives}
-Explicar las fases del ciclo de vida de un sistema de información, así como conceptos y alternativas.
-Detectar problemas a resolver, realizar reingeniería del flujo físico.
\end{objectives}
\end{LU}

\begin{LU73}{Implementaciones comerciales de Sistemas de Información}{Kirikova03ISD,Avison98ISD}{}
\begin{goal}
-Dar a los estudiantes una exposición al uso de productos de programas comerciales para implementar sistemas de información.
\end{goal}
\begin{objectives}
-Demostrar habilidades para analizar métodos alternativos para aplicaciones incluyendo paquetes, personalización de paquetes, adición de módulos a paquetes y construcción de aplicaciones únicas.
-Explicar los conceptos de adquisición de hardware y software de computadores.
-Explicar el proceso de escritura de propuestas y contratos.
-Explicar las fases contractuales y escribir ejemplos realistas para relaciones de consultoría, adquisición de software y hardware, u otros ejemplos relevantes.
\end{objectives}
\end{LU}

\begin{LU74}{Requerimientos y especificaciones de Sistemas de Información}{Kirikova03ISD,Avison98ISD}{}
\begin{goal}
-Mostrar como recolectar y estructurar información en el desarrollo de requerimientos y especificaciones.
\end{goal}
\begin{objectives}
-Conducir una entrevista de adquisición de información con individuos y con un grupo.
-Conducir una sesión JAD usando una herramienta GDS.
-Usar CASE, I-CASE u otras herramientas automatizadas o no automatizadas.
-Ser capaz de usar una herramienta CASE comercial para generar documentación {\it ``upper case??}.
\end{objectives}
\end{LU}

\begin{LU75}{Diseño e implementación de Sistemas de Información}{Kirikova03ISD,Avison98ISD}{}
\begin{goal}
\end{goal}
\begin{objectives}
\end{objectives}
\end{LU}

\begin{LU76}{Prototipado rápido de Sistemas de Información}{Kirikova03ISD,Avison98ISD}{}
\begin{goal}
-Desarrollar un entendimiento funcional del prototipado rápido y otros mecanismos alternativos similares para el desarrollo rápido de sistemas de información.
\end{goal}
\begin{objectives}
-Usar prototipeo rápido y otros mecanismos alternativos similares para el desarrollo rápido de sistemas de información.
\end{objectives}
\end{LU}

\begin{LU77}{Riesgos y viabilidad en el desarrollo de Sistemas de Información}{Kirikova03ISD,Avison98ISD}{}
\begin{goal}
-Mostrar cómo estimar riesgos y viabilidad.
\end{goal}
\begin{objectives}
-Identificar requerimientos y especificaciones de Sistemas de Información y alternativas lógicas de diseño tentativas; evaluar ventajas competitivas, viabilidad y riesgos propuestos.
\end{objectives}
\end{LU}

\begin{LU78}{Mejora continua de Sistemas de Información}{Kirikova03ISD,Avison98ISD}{}
\begin{goal}
-Mostrar a los estudiantes cómo analizar sistemas organizacionales para determinar cómo los sistemas pueden ser mejorados.
\end{goal}
\begin{objectives}
-Comparar varias soluciones de sistemas propuestos basados en criterios para el éxito.
-Identificar, explicar y usar metodologías de desarrollo compatibles con el concepto del proceso de mejora continua.
-Aplicar teoría de sistemas, de decisión y de calidad y técnicas y metodologías de desarrollo de sistemas de información para iniciar, especificar e implementar un sistema de información multi-usuario relativamente complejo originado en una organización consciente por la calidad involucrada en la mejora continua de sus procesos.
-En un nivel empresaria o multi-departamental, desarrollar flujos físicos así como un diseño de flujo de trabajo completo.
\end{objectives}
\end{LU}

\begin{LU79}{Desarrollo consensual}{Kirikova03ISD,Avison98ISD}{}
\begin{goal}
-Desarrollar habilidades para la comunicación interpersonal efectiva para desarrollar un consenso usando técnicas clásicas así como {\it groupware} facilitado por computador.
\end{goal}
\begin{objectives}
-Explicar el concepto de visión compartida en el desarrollo efectivo de soluciones a proceso organizacionales.
-Explicar formas comunes de comportamiento que puedan llevar a un carencia de comunicación.
\end{objectives}
\end{LU}

\begin{LU80}{Dinámicas de grupo}{Kirikova03ISD,Avison98ISD}{}
\begin{goal}
-Demostrar y analizar dinámicas de grupos pequeños y cómo se relaciona al trabajo con usuarios.
\end{goal}
\begin{objectives}
-Explicar el comportamiento de grupo y de equipo en un contexto de Sistemas de Información.
-Explicar cómo los grupos y equipos deben trabajar juntos, motivar colegas y aplicar métodos de equipos; medir y probar la motivación y efectividad; participar eficientemente en trabajo de grupo colaborativo; evaluar el éxito del trabajo.
\end{objectives}
\end{LU}

\begin{LU81}{Aplicaciones de bases de datos}{elmasri04,rob04}{}
\begin{goal}
-Desarrollar habilidades de aplicación para implementar bases de datos y aplicaciones operando y probando estas bases de datos.
\end{goal}
\begin{objectives}
-Diseñar e implementar un sistema de información dentro de un ambiente de bases de datos.
-Desarrollar flujo de datos y/o modelos basados en eventos de los componentes de un sistema de información y diseñar la implementación del concepto.
-Desarrollar la base de datos correspondiente e implementar el esquema con un paquete DBMS.
-Desarrollar pantallas basadas en eventos correspondientes con el diseño de la base de datos; desarrollar diseños de reportes para la documentación y notificación necesaria; resolver la indexación de la base de datos y construir una aplicación adecuada.
\end{objectives}
\end{LU}

\begin{LU82}{Resolución de problemas, métricas de complejidad}{Leffingwell03,Wang00,Pressman04,Schach04}{}
\begin{goal}
-Presentar y usar métricas de complejidad para evaluar las soluciones desarrolladas.
\end{goal}
\begin{objectives}
-Aplicar funciones de software de sistemas para analizar el uso de recursos y el rendimiento característico para una aplicación.
\end{objectives}
\end{LU}

\begin{LU83}{Calidad de software}{Wang00,Pressman04,Schach04}{}
\begin{goal}
-Desarrollar métricas de calidad para la evaluación del desarrollo de software y control de proyectos del desarrollo de software.
\end{goal}
\begin{objectives}
-Explicar cómo los estándares escritos describiendo cada fase del ciclo de vida puede evolucionar; explicar la relevancia de los estándares escritos, y la conveniencia de desarrollar procedimientos de aseguramiento de la calidad.
-Describir y explicar el uso de métricas de calidad en la evaluación del desarrollo de software y en facilitar el control de proyecto de las actividades de desarrollo.
\end{objectives}
\end{LU}

\begin{LU84}{Métricas de calidad}{Wang00,Pressman04,Schach04}{}
\begin{goal}
-Desarrollar métricas de calidad para evaluar la satisfacción del cliente en todas las fases del ciclo de vida.
\end{goal}
\begin{objectives}
-Usar métricas de calidad y {\it benchmarks} de rendimiento para asegurar la satisfacción del cliente para cada fase del ciclo de vida. Probar las métricas durante las actividades de desarrollo del sistema.
\end{objectives}
\end{LU}

\begin{LU85}{Código de ética profesional}{Reynolds2006ITEthics,Mendell2005ComputerCrime,Schmidt2000Etica,Gallegos1999Etica,Escola2000Etica}{}
\begin{goal}
-Explicar el uso de un código de ética profesional para evaluar acciones de Sistemas de Información específicas.
\end{goal}
\begin{objectives}
-Identificar y describir organizaciones profesionales.
-Explicar el establecimiento de un estándar ético.
-Explicar y examinar asuntos éticos y argumentos y métodos fallidos como una función de contexto social.
-Identificar a los accionistas en un contexto de desarrollo de Sistemas de Información y el efecto del desarrollo en estos individuos.
-Describir el uso de los códigos de ética y asegurar que las acciones del proyecto son consistentes con estas prescripciones.
\end{objectives}
\end{LU}

\begin{LU86}{Soluciones sinérgicas}{Hesselbein1999,Covey2004SevenHabits}{}
\begin{goal}
-Discutir la importancia de encontrar soluciones sinérgicas con equipos y clientes.
\end{goal}
\begin{objectives}
-Describir y explicar los hábitos de interdependencia de escucha simpatética ({\it empathetic listening}), sinergía y construcción de consensos.
-Explicar actividades de negociación e interdependencia.
\end{objectives}
\end{LU}

\begin{LU87}{Compromiso y acuerdos}{Covey2004SevenHabits}{}
\begin{goal}
-Mostrar cómo desarrollar acuerdos describiendo el trabajo a ser realizado, y comprometer, completar rigurosamente y auto-evaluar el trabajo acordado.
\end{goal}
\begin{objectives}
-Desarrollar estimativas de trabajo, comprometerse al trabajo y completar rigurosamente, evaluar comparando con estándares y responder por el trabajo.
\end{objectives}
\end{LU}

\begin{LU88}{Modelado de datos}{veryard94,elmasri04,oppel04}{}
\begin{goal}
-Desarrollar habilidades para el modelado de datos que describen bases de datos.
\end{goal}
\begin{objectives}
-Usar DBMS, modelado de datos y lenguajes de manipulación de datos.
-Usar modelos de datos de conocimiento para diferenciar tipos de modelos; explicar los diferentes modelos para bases de datos, p.e. relacional, jerárquico, de red y bases de datos orientadas a objetos; explicar cómo son implementadas en sistemas de administración de bases de datos.
\end{objectives}
\end{LU}

\begin{LU89}{Tipos abstractos de datos: modelos y funciones de Bases de Datos}{simsion04,elmasri04}{}
\begin{goal}
-Desarrollar conciencia de las diferencias sintácticas y teóricas entre los modelos de bases de datos.
\end{goal}
\begin{objectives}
-Identificar los componentes de modelos de bases de datos jerárquicos, de red y relacionales; discutir las definiciones de los datos requeridos para cada modelo; explicar las razones para especificar comandos dentro de las infraestructuras de manipulación de datos; discutir conversión lógica entre los modelos.
\end{objectives}
\end{LU}

\begin{LU90}{Implementación de Bases de Datos y Sistemas de Información}{veryard94,simsion04,elmasri04}{}
\begin{goal}
-Desarrollar habilidades en la aplicación del desarrollo de sistemas de bases de datos y recuperación de la infraestructura necesaria para facilitar la creación de aplicaciones de sistemas de información.
\end{goal}
\begin{objectives}
-Aplicar implementación de ciclo de vida.
-Explicar la administración y mantenimiento de bases de datos.
\end{objectives}
\end{LU}

\begin{LU91}{Estructuración de aplicaciones de Bases de Datos}{kimball04,veryard94}{}
\begin{goal}
-Desarrollar habilidades con aplicación y estructuración de sistemas de administración de  bases de datos.
\end{goal}
\begin{objectives}
-Desarrollar editores para facilitar la entrada de datos en la base de datos.
-Demostrar el diseño e implementación de habilidades tanto con una interfaz gráfica de usuario como una interfaz basada en caracteres para implementar listas, diálogos, botones y estructuras de menú.
-Diseñar e implementar reportes simples para validad el rendimiento de sistemas de aplicación.
-Aplicar principios de desarrollo de software, métodos y herramientas para la implementación de una aplicación de Sistemas de Información.
\end{objectives}
\end{LU}

\begin{LU92}{Implementación de aplicaciones de Bases de Datos}{rob04,simsion04}{}
\begin{goal}
-Desarrollar habilidades para la aplicación e implementación física de sistemas de bases de datos, usando un ambiente de programación.
\end{goal}
\begin{objectives}
-Aplicar técnicas de diseño de bases de datos para implementar una solución con llamadas desde programas al DBMS.
-Explicar y aplicar consideraciones de redes en la implementación de sistemas distribuidos.
-Desarrollar aplicaciones cliente-servidor e instalarlos y operarlos en un ambiente multi-usuario.
\end{objectives}
\end{LU}

\begin{LU93}{Desarrollo de aplicaciones/generación de código}{dietrich01}{}
\begin{goal}
-Desarrollar habilidades para el uso de una combinación de generadores de código e infraestructuras de lenguajes para implementar sistemas de nivel departamentales multi-usuario o de nivel empresarial simple.
\end{goal}
\begin{objectives}
-Usar generadores de código para implementar una aplicación de Sistemas de Información y comparar los resultados con versiones desarrolladas manualmente.
\end{objectives}
\end{LU}

\begin{LU94}{Desarrollo y administración de proyectos}{PMI2005}{}
\begin{goal}
-Proveer una oportunidad para desarrollar y usar administración de proyectos, estándares de proyectos y un plan de implementación de sistemas, e implementar un plan de documentación.
\end{goal}
\begin{objectives}
-Crear y presentar documentación técnica y de usuario final de sistemas de telecomunicaciones.
-Identificar consideraciones de seguridad y de privacidad y cómo pueden ser resueltas dentro del contexto de un sistema de telecomunicaciones.
-Explicar los controles de configuración.
-Desarrollar consistentemente con buenas prácticas un proyecto DBMS de nivel departamental y desarrollar documentación de desarrollo y de usuario.
-Trabajar en equipos realizando seguimiento de resultados individuales y de equipo; desarrollar medidas de calificación para tareas asignadas y rendimiento para evaluar y asegurar la calidad de un proceso de desarrollo.
-Desarrollar documentación a nivel de programa, de sistema y de usuario.
-Aplicar conceptis de desarrollo en un proyecto de complejidad razonable en un ambiente de equipo.
\end{objectives}
\end{LU}

\begin{LU95}{Modelos lógicos y conceptuales de Bases de Datos}{dietrich01,whitehorn01}{}
\begin{goal}
-Mostrar cómo diseñar un modelo de base de datos relacional conceptual y lógico, convertir los diseños de bases de datos lógicos en diseños físicos.
\end{goal}
\begin{objectives}
-Desarrollar la base de datos física y generar datos de prueba.
-Explicar un {\it framework} para evaluar una función de sistema de información y valorizar las aplicaciones individuales.
-Explicar el uso de factores de éxito crítico para traducir un diseño de sistemas lógico en un diseño físico en un ambiente objetivo e implementar esta especificación en un sistema operacional usando tecnología DBMS.
\end{objectives}
\end{LU}

\begin{LU96}{Especificaciones funcionales de Sistemas de Información}{OBrien2008IntroductionIS}{}
\begin{goal}
-Proveer oportunidades para desarrollar especificaciones funcionales para un sistema de información, desarrollar un diseño de sistema de información detallado y desarrollar controles de aplicación para sistemas de información.
\end{goal}
\begin{objectives}
-Usar una metodología para especificar y desarrollar un sistema de información significativo para un nivel departamental; asegurar que la recolección de datos, verificación y control son realizados; asegurar que las auditorías externas establecerán objetivos y logros consistentes.
\end{objectives}
\end{LU}

\begin{LU97}{Planeamiento de conversión de Sistemas de Información}{OBrien2008IntroductionIS}{}
\begin{goal}
-Mostrar cómo desarrollar un plan de conversión e instalación, desarrollar un sistema de hardware y un plan ambiental.
\end{goal}
\begin{objectives}
-Desarrollar un plan de entrenamiento, conversión e instalación detallado para hardware y software involucrando una aplicación de sistema de información recién desarrollada.
-Diseñar soluciones de red e instalar el DBMS en el servidor junto con un sistema operativo apropiado y hardware y software de telecomunicaciones.
\end{objectives}
\end{LU}

\begin{LU98}{Desarrollo y conversión de Sistemas de Información}{OBrien2008IntroductionIS}{}
\begin{goal}
-Mostrar cómo desarrollar especificaciones de programa detalladas, desarrollar programas, configurar parámetros de prueba del sistema, instalar y probar el nuevo sistemas, implementar el plan de conversión, emplear administración de configuración.
\end{goal}
\begin{objectives}
-Desarrollar, probar, instalar y operar un programa de aplicación de sistema de información significativo.
-Desarrollar, probar, instalar y operar tanto aplicaciones de cliente como de servidor; asegurar que todos los aspectos multi-usuario de la aplicación funcionan como planeado.
-Desarrollar, probar, instalar y operar sistemas de aplicación acoplados que no posean mecanismos de acoplamiento patológico; describir y explicar cómo otros mecanismos pueden involucrar un mecanismos de acoplamiento inapropiados e ilustrar las consecuencias de tales errores de diseño; discutir y explicar mecanismos de acoplamiento {\it batch} y {\it on-line}.
\end{objectives}
\end{LU}

\begin{LU99}{Requerimientos de Sistemas de Información/planeamiento de flujo de trabajo}{OBrien2008IntroductionIS}{}
\begin{goal}
-Mostrar cómo desarrollar un plan de flujo de trabajo físico con un cliente.
\end{goal}
\begin{objectives}
-Participar de forma no confrontacional en un ambiente de equipo y demostrar habilidades de escucha simpatética para facilitar la determinación de mecanismos alternos para un grupo de trabajo integrado horizontal en la mejora de sus funciones a través del rediseño de procesos, incluyendo la incorporación de sistemas de información para asegurar la documentación y la calidad.
-Diseñar un flujo de trabajo usando herramientas gráficas o software de desarrollo de sistemas imagen en la presencia de un cliente.
-Convertir el flujo de trabajo a un diseño de tipo tanto IDEF0 como IDEF3; convertir el diseño IDEF3 en un modelo basado en eventos satisfactorio para una interfaz gráfica de usuario.
\end{objectives}
\end{LU}

\begin{LU100}{Aplicaciones de Sistemas de Información con lenguajes de programación}{OBrien2008IntroductionIS}{}
\begin{goal}
-Desarrollar habilidades de análisis, diseño e implementación de software de aplicación usando un ambiente de programación.
\end{goal}
\begin{objectives}
-Diseñar e implementar software de aplicación de sistemas de información usando un ambiente de programación que utilice programación de bases de datos (el diseño debe incluir editores de pantalla, mecanismos de actualización de datos, controles de auditoria y operación, y debe contender reportes impresos apropiados.)
-Usar herramientas de productividad para desarrollar modelos conceptuales de datos y funcionales.
\end{objectives}
\end{LU}

\begin{LU101}{Implementación de Sistemas de Información con objetos, dirigido a eventos}{OBrien2008IntroductionIS}{}
\begin{goal}
-Identificar diferencias entre un diseño de aplicación estructurado, basado en eventos y orientado a objetos y explicar las implicaciones de estos métodos para el proceso de diseño y desarrollo.
\end{goal}
\begin{objectives}
-Emplear un ambiente de programación para desarrollar una aplicación simple basado en eventos con una interfaz GUI.
\end{objectives}
\end{LU}

\begin{LU102}{}{}{}
\begin{goal}
\end{goal}
\begin{objectives}
\end{objectives}
\end{LU}

\begin{LU103}{Pruebas de desarrollo de Sistemas de Información}{OBrien2008IntroductionIS}{}
\begin{goal}
-Ser capaz de desarrollar pruebas de software y de sistemas
\end{goal}
\begin{objectives}
-Construir consultas efectivas usando herramientas de búsqueda tanto estructuradas como no estructuradas.
-Efectuar ingeniería inversa de aplicaciones de flujos de datos de lenguajes de cuarta generación para asegurar la verificación.
\end{objectives}
\end{LU}

\begin{LU104}{Ambientes de programación para aplicaciones de Sistemas de Información}{OBrien2008IntroductionIS}{}
\begin{goal}
-Entender los diferentes ambientes de programación disponibles para el desarrollo de aplicaciones de negocio.
\end{goal}
\begin{objectives}
-Explicar las características, requerimientos y uso de varios ambientes de programación incluyendo ambientes gráficos y convencionales; explicar los conceptos de portabilidad de software y los conceptos de interoperatibilidad.
\end{objectives}
\end{LU}

\begin{LU105}{Planeamiento de proyectos  de Sistemas de Información}{OBrien2008IntroductionIS,PMI2005}{}
\begin{goal}
-Asegurar las habilidades requeridas para diseñar un plan de desarrollo e implementación de proyectos.
\end{goal}
\begin{objectives}
-Explicar funciones de comité y la racionalidad de equipos horizontales en desarrollo organizacional y reingeniería de sistemas de información.
\end{objectives}
\end{LU}

\begin{LU106}{Administración de proyectos  de Sistemas de Información I}{PMI2005}{}
\begin{goal}
-Desarrollar y practicar habilidades esenciales de administración de proyectos.
\end{goal}
\begin{objectives}
-Aplicar conceptos de diseño de reuniones para organizar y conducir un equipo efectivo y reuniones de clientes que aseguren una visión compartida y acciones efectivas.
\end{objectives}
\end{LU}

\begin{LU107}{Administración de proyectos  de Sistemas de Información II}{PMI2005}{}
\begin{goal}
-Desarrollar habilidades en el uso de herramientas y métodos de administración de proyectos dentro del contexto de un proyecto de sistemas de información.
\end{goal}
\begin{objectives}
-Usar y aplicar herramientas, técnicas y software de administración de proyectos en la definición, implementación y modificación de objetivos de proyecto.
-Producir reportes de progreso con información de administración de plazos, de individuos y de equipo para asegurar un desarrollo de software, una implementación de sistema de flujo de trabajo físico y una instalación de sistemas de computadores de calidad.
\end{objectives}
\end{LU}

\begin{LU108}{Herramientas de administración de proyectos}{PMI2005}{}
\begin{goal}
-Seleccionar las herramientas de administración de proyectos apropiadas y demostrar su uso.
\end{goal}
\begin{objectives}
-Usar conceptos y herramientas de administración de proyectos  (PERT, GANTT)
-Usar técnicas de administración de proyectos como PERT, GANTT.
-Usar CASE y otras herramientas.
\end{objectives}
\end{LU}

\begin{LU109}{Cierre del proyecto}{OBrien2008IntroductionIS}{}
\begin{goal}
-Iniciar, diseñar, implementar y discutir la implantación y término de un proyecto.
\end{goal}
\begin{objectives}
-Discutir y explicar los conceptos involucrados en el término de un proyecto; explicar y listar los requerimientos para finalizar un proyecto.
\end{objectives}
\end{LU}

\begin{LU110}{Sistemas de producción}{Oz2002ManagementIS,Stair2007FundamentalsIS,Gray2005MakingDecisions}{}
\begin{goal}
-Determinar y analizar un problema significativo usando el método de sistemas para resolver problemas.
\end{goal}
\begin{objectives}
-Desarrollar y usar especificaciones detalladas para definir y resolver un problema de aplicación de sistemas de información incluyendo flujos físicos, diseño de bases de datos, funciones de sistema, requerimientos y diseño de programación, así como  implementación de software y de bases de datos.
-Diseñar e implementar un plan de integración de sistemas para un sistema de nivel empresarial involucrando técnicas LAN y WAN; implementar conexiones de sistemas, instalar y configurar sistemas e instalar, probar y operar soluciones diseñadas.
-Integrar soluciones y metodologías de usuario final en un modelo empresarial; desarrollar e implementar planes de conversión y entrenamiento.
-Desarrollar y evolucionar estándares escritos para todas las actividades de un ciclo de desarrollo; presentar y defender soluciones: conformar administración de tiempo y de contabilidad para los estándares desarrollados.
\end{objectives}
\end{LU}

\begin{LU111}{Requerimientos y sistemas de bases de datos}{whitehorn01}{}
\begin{goal}
-Desarrollar requerimientos y especificaciones para una base de datos requiriendo un sistema de información multi-usuario.
\end{goal}
\begin{objectives}
-Identificar flujos físicos e integración horizontal de los procesos organizacionales y relacionar estos flujos a las bases de datos relevantes que los describan
-Desarrollar modelos funcionales basados en eventos para un proceso organizacional involucrado.
-Identificar y especificar los procesos que resolverán el problema organizacional y definir la aplicación de bases de datos relacionada.
\end{objectives}
\end{LU}

\begin{LU112}{Acción personal, proactiva, basada en principios}{Ivancevich2008Behavior,Daft2000Organization,Stoner1995Management}{}
\begin{goal}
-Desarrollar una comprensión funcional de comportamiento proactivo y administración de tiempo.
\end{goal}
\begin{objectives}
-Describir y explicar los hábitos característicos del liderazgo proactivo y de la administración de tiempo.
\end{objectives}
\end{LU}

\begin{LU113}{Escucha simpatética}{Gelles1995Sociology,Fedman2006Psychology,Daft2007Management}{}
\begin{goal}
-Asegurar actitudes necesarias para el comportamiento exitoso de equipo incluyendo la escucha simpatética, negociación de consenso, resolución de conflicto y hallazgo de una solución sinérgica, y aplicar el concepto de compromiso y de completitud rigurosa.
\end{goal}
\begin{objectives}
-Usar y aplicar trabajo en equipo, métodos de motivación, aplicar conceptos y métodos de reuniones, usar técnicas de grupo, usar habilidades de escucha simpatética, emplear desarrollo de soluciones sinérgicas.
-Asegurar que la escucha simpatética es practicada, asegurar que los individuos escuchan, se comprometen y completan rigurosamente las actividades asignadas; explicar la relevancia de dichas acciones para asegurar la efectividad del equipo.
\end{objectives}
\end{LU}

\begin{LU114}{Alineamiento de objetivos, misión}{Daft2007Management,OBrien2005ManagementIS}{}
\begin{goal}
-Asegurar el establecimiento de objetivos y el alineamiento de las actividades del equipo con las obligaciones del proyecto.
\end{goal}
\begin{objectives}
-Discutir y explicar los conceptos actividad dirigida por una visión y misión compartida en el desarrollo de sistemas de información.
-Discutir y aplicar trabajo dirigido a la misión alineando la misión del equipo a la misión del proyecto por medio del seguimiento para asegurar resultados.
\end{objectives}
\end{LU}

\begin{LU115}{Responsabilidad en la venta de diseños a la administración}{OBrien2005ManagementIS,Oz2002ManagementIS,OBrien2008IntroductionIS,Laudon2007ManagementIS}{}
\begin{goal}
-Describir la interacción con niveles superiores de administración en la explicación de los objetivos de proyecto y en efectuar las tareas de administración del proyecto.
\end{goal}
\begin{objectives}
-Explicar y demostrar la relación de las actividades de Sistemas de Información para mejorar la posición competitiva.
-Explicar funciones de administración de Sistemas de Información, administrador de proyecto.
\end{objectives}
\end{LU}

\begin{LU116}{Ciclo de vida de sistemas de información y proyectos}{OBrien2008IntroductionIS}{}
\begin{goal}
-Describir y explicar los conceptos ciclo de vida y aplicarlos al proyecto del curso.
\end{goal}
\begin{objectives}
-Explicar los diferentes conceptos de ciclo de vida en la participación y la culminación de un proyecto de tamaño y rango considerables, involucrando equipos; mostrar cómo asegurar la aceptación y la incorporación de estándares compatibles con un ciclo de vida exitoso.
-Explicar las diferentes responsabilidades de Sistemas de Información, Ciencia de la Computación e Ingeniería de Software, de la forma en que éstas pertenecen a las actividades de desarrollo de software y de sistemas; aplicar las lecciones aprendidas al proyecto del curso.
-Explicar cómo técnicas formales de ingeniería de software pueden contribuir al éxito de los esfuerzos de desarrollo de software y de sistemas; aplicar estas técnicas al proyecto del curso (calidad, verificación y validación, correctitud y confiabilidad, pruebas, etc.)
\end{objectives}
\end{LU}

\begin{LU117}{Presentación}{OBrien2008IntroductionIS}{}
\begin{goal}
-Mostrar cómo presentar un diseño de sistema, plan de pruebas, plan de implementación y evaluación en forma oral y escrita.
\end{goal}
\begin{objectives}
-Presentar y explicar soluciones a un grupo de pares para recibir críticas y mejorar.
-Aplicar habilidades de comunicación oral y escritas para presentar soluciones propuestas y logros.
\end{objectives}
\end{LU}

\begin{LU118}{Aprendizaje continuo}{Ivancevich2008Behavior,Robbins2002Behavior}{}
\begin{goal}
-Discutir y aplicar el concepto de aprendizaje continuo.
\end{goal}
\begin{objectives}
-Discutir y aplicar el concepto de aprendizaje continuo.
\end{objectives}
\end{LU}

\begin{LU119}{Ética y asuntos legales}{Reynolds2006ITEthics,Mendell2005ComputerCrime,Erdozain2002PropiedadIntelectual,Casanovas2003Juridico,Carranza2004SoftwareLubreJuridica}{}
\begin{goal}
-Discutir y explicar principios y asuntos éticos y legales; discutir y explicar consideraciones éticas del desarrollo, planeamiento, implementación, uso, venta, distribución, operación y mantenimiento de sistemas de información.
\end{goal}
\begin{objectives}
-Listar y explicar los asuntos éticos y legales en el desarrollo, posesión, venta, adquisición, uso y mantenimiento de sistemas y software de computador.
-Explicar el uso de modelos éticos, p.e. liderazgo centrado en principios para las fases del ciclo de vida de un sistema de información.
-Dar ejemplos de los efectos de contexto social en el desarrollo de tecnología.
\end{objectives}
\end{LU}

\begin{LU120}{Administración de SI y organización del departamento de SI}{OBrien2008IntroductionIS}{}
\begin{goal}
-Presentar y explicar el rol de liderazgo evolutivo de la administración de información en las organizaciones.
\end{goal}
\begin{objectives}
-Describir y explicar la composición del personal requerido para formar un equipo para un proyecto determinado y usar estrategias de administración de personal.
-Explicar a un trabajador, no relacionado al área de sistemas de información, lo que debe hacer para administrar sus recursos y requerimientos de información.
\end{objectives}
\end{LU}

\begin{LU121}{Liderazgo y Sistemas de Información}{OBrien2008IntroductionIS,Hesselbein1999}{}
\begin{goal}
-Presentar y explicar el rol de liderazgo evolutivo de la administración de información en las organizaciones.
\end{goal}
\begin{objectives}
-Explicar el establecimiento un estándar ético.
-Explicar la relevancia y uso de un código de ética profesional.
-Explicar y demostrar la aplicación exitosa de un argumento ético en la identificación y evaluación de alternativas basados en un análisis contextual social en un ambiente de desarrollo de sistemas de información centrados en el cliente.
 -Explicar el alineamiento de Sistemas de Información con la misión organizacional; explicar la relación de los procesos departamentales con el éxito estratégico de la organización.
-Explicar planeamiento y administración de presupuesto.
-Explicar e ilustrar la aplicación de modelos éticos, p.e. liderazgo centrado en principios, en los estándares y la práctica de administración de proyectos.
\end{objectives}
\end{LU}

\begin{LU122}{Políticas y estándares de Sistemas de Información}{OBrien2008IntroductionIS}{}
\begin{goal}
-Examinar el proceso para el desarrollo de políticas, procedimientos y estándares de los sistemas de información en la organización.
\end{goal}
\begin{objectives}
-Explicar la relevancia de la administración de Sistemas de Información alineándose con el proceso de negocio.
-Explicar y desarrollar estándares y políticas involucradas en el desarrollo de sistemas de información de ámbito organizacional.
-Explicar los beneficios de equipos multi-funcionales en el desarrollo de políticas y procedimientos.
-Explicar los beneficios del desarrollo de la definición de la misión del equipo y del alineamiento de la misión del equipo con las misiones organizacionales.
\end{objectives}
\end{LU}

\begin{LU123}{Administración de la función de los Sistemas de Información}{OBrien2008IntroductionIS}{}
\begin{goal}
-Investigar asuntos relacionados a la administración del funcionamiento de los sistemas de información.
\end{goal}
\begin{objectives}
-Explicar asuntos de seguridad y privacidad.
-Explicar la base para un contrato legal para desarrollar sistemas.
\end{objectives}
\end{LU}

\begin{LU124}{Administración de tecnologías emergentes}{Gunther00Emerging}{}
\begin{goal}
-Discutir asuntos pertinentes a la administración y transferencia de las tecnologías emergentes.
\end{goal}
\begin{objectives}
-Explicar y detallar métodos para reconocimiento de ambiente y selección efectiva de hardware y software.
-Explicar la administración de tecnologías emergentes.
\end{objectives}
\end{LU}

\begin{LU125}{Implementación de Sistemas de Información y {\it outsourcing}}{OBrien2008IntroductionIS}{}
\begin{goal}
-Discutir {\it outsourcing} e implementaciones alternativas de la funcionalidad de un Sistema de Información.
\end{goal}
\begin{objectives}
-Explicar {\it outsourcing} como una alternativa a la funcionalidad de un Sistema de Información interno.
-Definir, explicar y comparar desde una perspectiva costo-beneficio varios arreglos de {\it outsourcing}.
-Administrar la funcionalidad de un Sistema de Información en una organización pequeña.
-Explicar {\it outsourcing}.
\end{objectives}
\end{LU}

\begin{LU126}{Administración de tiempo y relaciones}{PMI2005}{}
\begin{goal}
-Discutir la administración del tiempo y las relaciones interpersonales.
\end{goal}
\begin{objectives}
-Explicar cuatro generaciones de conceptos de administración del tiempo, así como razones personales e interpersonales para el éxito de cada etapa; usar los mecanismos dentro de un ambiente de proyectos.
\end{objectives}
\end{LU}

\begin{LU127}{Administración de calidad y desempeño}{stoner96,hezier04:PAO}{}
\begin{goal}
-Discutir la evaluación de rendimiento consistente con la administración de la calidad y mejora continua.
\end{goal}
\begin{objectives}
-Desarrollar medidas de rendimiento consistentes con los conceptos de empleados de valor que faciliten la cooperación del equipo y disminuya la competencia entre miembros del equipo; discutir las razones para tales medidas y explicar las consecuencias negativas de entender erróneamente estos asuntos.
\end{objectives}
\end{LU}

\begin{LU201}{Conceptos Fundamentales de Modelos Empresariales}{Daft2000Organization,Longenecker2007SmallBusiness,Franklin1998Empresas}{}
\begin{goal}
-Discutir, conocer y modelar los procesos dentro de organizaciones desde un punto de vista estratégico.
\end{goal}
\begin{objectives}
\end{objectives}
\end{LU}

\begin{LU202}{Automatización y Modelado de Procesos}{Daft2000Organization,Longenecker2007SmallBusiness,Franklin1998Empresas}{}
\begin{goal}
-Estudiar herramientas para el modelado, análisis y monitoreo de procesos empresariales.
\end{goal}
\begin{objectives}
\end{objectives}
\end{LU}

\begin{LU203}{Sistemas de Información Estratégicos}{OBrien2008IntroductionIS}{}
\begin{goal}
-Conocer el impacto de la automatización de las prácticas de trabajo, procesos de administración del conocimiento y procesos colaborativos no estructurados.
\end{goal}
\begin{objectives}
\end{objectives}
\end{LU}

\begin{LU204}{}{}{}
\begin{goal}
-Pendiente.
\end{goal}
\begin{objectives}
\end{objectives}
\end{LU}

\begin{LU205}{Introducción a Tecnologías Emergentes}{Gunther00Emerging}{}
\begin{goal}
-Conocer tecnologías emergentes relevantes para empresas, así como su posible impacto en la economía y las organizaciones y en la transformación de bienes y servicios.
\end{goal}
\begin{objectives}
\end{objectives}
\end{LU}

\begin{LU206}{Impacto de las Tecnologías Emergentes en las Empresas}{Gunther00Emerging}{}
\begin{goal}
-Entender de manera más profunda asuntos técnicos y organizacionales de las tecnologías emergentes, así como las implicanciones estratégicas de éstas.
\end{goal}
\begin{objectives}
\end{objectives}
\end{LU}

\begin{LU207}{Aspectos Técnicos de las Tecnologías Emergentes}{Gunther00Emerging}{}
\begin{goal}
-Tener un mayor conocimiento de los aspectos técnicos de las tecnologías emergentes y los posibles rumbos que puedan tomar.
\end{goal}
\begin{objectives}
\end{objectives}
\end{LU}

\begin{LU208}{Administración de Proyectos de Software/Tecnología}{OBrien2008IntroductionIS,PMI2005}{}
\begin{goal}
-Aprender las herramientas y técnicas de la administración y planeamiento de proyectos, incluyendo el uso de software de administración de proyectos.
\end{goal}
\begin{objectives}
\end{objectives}
\end{LU}

\begin{LU209}{Administración del Proceso de Cambio}{OBrien2008IntroductionIS}{}
\begin{goal}
-Desarrollar habilidades en las implicaciones humanas y organizacionales del cambio incluyendo el proceso de cambio organizacional.
\end{goal}
\begin{objectives}
\end{objectives}
\end{LU}

\begin{LU210}{Planeamiento de Sistemas de Información para el Negocio}{OBrien2008IntroductionIS}{}
\begin{goal}
-Desarrollar un entendimiento del uso estratégico de la tecnología de información desde una perspectiva de negocios al nivel empresarial.
\end{goal}
\begin{objectives}
\end{objectives}
\end{LU}

\begin{LU211}{Aspectos Gerenciales de Proyectos Estratégicos de Tecnología de Información}{Gray2005MakingDecisions,Turban2005ITforManagement,Applegate2002Corporate}{}
\begin{goal}
-Entender el gerenciamiento interno de los servicios de sistemas de información desde un punto de vista gerencial y examinar estrategias y tácticas alternativas disponibles a la administración para alcanzar los objetivos.
\end{goal}
\begin{objectives}
\end{objectives}
\end{LU}

\begin{LU212}{El Sistema Empresarial}{Daft2000Organization,Hodge1996OrganizationalTheory}{}
\begin{goal}
-Estudiar asuntos administrativos y organizacionales a nivel de empresa como un todo.
\end{goal}
\begin{objectives}
\end{objectives}
\end{LU}

\begin{LU213}{La Función de Sistemas de Información}{OBrien2008IntroductionIS}{}
\begin{goal}
-Estudiar la administración de la función de Sistemas de Información para avanzar la política y estrategias de la empresa.
\end{goal}
\begin{objectives}
\end{objectives}
\end{LU}

\begin{LU214}{Las Tecnologías en la Organización}{Gray2005MakingDecisions,Turban2005ITforManagement,Applegate2002Corporate}{}
\begin{goal}
-Describir el desarrollo de una arquitectura empresarial integrada consonante con las políticas y estrategias de la organización.
\end{goal}
\begin{objectives}
\end{objectives}
\end{LU}

\begin{LU215}{Rol de los Sistemas de Información en las Empresas}{OBrien2008IntroductionIS}{}
\begin{goal}
-Proveer una visión general del rol de los sistemas de información en las empresas.
\end{goal}
\begin{objectives}
\end{objectives}
\end{LU}

\begin{LU216}{Ética de Sistemas de Información}{OBrien2008IntroductionIS}{}
\begin{goal}
-Discutir aspectos éticos relacionados a los avances desarrollados debido a la digitalización de la información.
\end{goal}
\begin{objectives}
\end{objectives}
\end{LU}

\begin{LU217}{Seguridad de Sistemas de Información}{OBrien2008IntroductionIS}{}
\begin{goal}
-Examinar los elementos constitutivos de un ambiente digital seguro.
\end{goal}
\begin{objectives}
\end{objectives}
\end{LU}