\begin{LU1}{Sistemas y conceptos de tecnología de Informacion}{OBrien2008IntroductionIS,Turban2005ITforManagement}{}
\begin{goal}
-Introducir a los nuevos usuarios a los conceptos de Sistemas y de Tecnología de Información.
\end{goal}
\begin{objectives}
-Describir y explicar en términos de sistemas los componentes de hardware y software de un sistema computacional.
-Describir, explicar y utilizar un Sistema Operativo e interfaz de usuario para instalar y operar programas, definir y proteger archivos de datos así como utilizar utilitarios del Sistema Operativo.
-Definir, explicar y utilizar software para el trabajo con conocimiento.
\end{objectives}
\end{LU}

\begin{LU2}{Software de trabajo de conocimiento}{Excel2007Walkenbach,Bain2007OpenOffice,PPT2007Wempen,Muir2007PPT,Sun2008OO}{}
\begin{goal}
-Desarrollar las habilidades para utilizar efectivamente software para el manejo de conocimiento tales como sistemas operativos, interfaces de usuario, hojas de cálculo, procesadores de texto, bases de datos, estadísticas y manejo de datos, presentaciones y comunicaciones.
\end{goal}
\begin{objectives}
-Diseñar, desarrollar y utilizar una base de datos simple, importar hojas de cálculo a bases de datos, exportar una tabla de base de datos u hoja de cálculo a un procesador de texto para ser utilizado en un reporte.
-Implementar una presentación basada en slides utilizando un paquete gráfico de presentaciones para comunicar un problema y su solución así como preparar documentos impresos para una posible audiencia.
\end{objectives}
\end{LU}

\begin{LU3}{Resolución de problemas, Sistemas de Información pequeños}{Excel2007Walkenbach,Robbins2006WebDesign,Zakas2007Ajax,Tayntor2005SWImplementation}{}
\begin{goal}
-Introducir los conceptos de resolución de problemas dentro del contexto de Sistemas de Información de complejidad limitada usando software de manejo de conocimiento estándar.
\end{goal}
\begin{objectives}
-Describir, explicar y usar una definición de abordaje de sistemas e implementación de soluciones basadas en PCs utilizando software de manejo de conocimiento (sistemas operativos, interfaces de usuario, hojas de cálculo, procesadores de texto, bases de datos, estadísticas y manejo de datos, presentaciones y comunicaciones) para mejorar la productividad del personal e incrementar las capacidades de trabajo con conocimiento.
-Identificar, definir e implementar una solución que involucre software para el trabajo con conocimiento para organizaciones simples y tareas personales.
-Seleccionar y configurar macros apropiadas, herramientas y paquetes para implementación de sistemas personales.
\end{objectives}
\end{LU}

\begin{LU4}{Tecnología de Información y la sociedad}{Turban2005ITforManagement}{}
\begin{goal}
-Introducir la relevancia y aplicación de la Tecnología de Información en la sociedad.
\end{goal}
\begin{objectives}
-Describir y explicar la relevancia e impacto de la Tecnología de la Información en la sociedad.
-Explicar el rol de los Sistemas de Información dentro de una empresa versus un entorno global.
\end{objectives}
\end{LU}

\begin{LU5}{Sistemas y calidad}{Oz2002ManagementIS,OBrien2008IntroductionIS,OBrien2008IntroductionIS,Laudon2007ManagementIS}{}
\begin{goal}
-Introducir conceptos de sistemas y de calidad.
\end{goal}
\begin{objectives}
-Explicar conceptos de calidad y teoría de sistemas.
\end{objectives}
\end{LU}

\begin{LU6}{Información y calidad}{Daft2000Organization,Hodge1996OrganizationalTheory,Oz2002ManagementIS,OBrien2008IntroductionIS}{}
\begin{goal}
-Proveer una introducción al uso organizacional de la información para mejorar la calidad general.
\end{goal}
\begin{objectives}
-Explicar metodologías para facilitar la medición y alcanzar el ISO9000, Baldridge, {\it National Performance Review} y otros estándares de calidad.
\end{objectives}
\end{LU}

\begin{LU7}{Hardware y software de Tecnología de Información}{ITHWSW2003Englander}{}
\begin{goal}
-Presentar conceptos de Tecnología de Información relacionados a hardware y software.
\end{goal}
\begin{objectives}
-Explicar los elementos y su relación funcional de los principales componentes de hardware, software y comunicaciones que forman PCs, LANs y/o WANs.
\end{objectives}
\end{LU}

\begin{LU8}{Especificación de Sistemas de Tecnología de Información}{OBrien2008IntroductionIS,Stair2007FundamentalsIS,Tayntor2005SWImplementation}{}
\begin{goal}
-Proveer los conceptos y habilidades para la especificación y diseño o reingeniería de sistemas pequeños relacionados con organizaciones basados en Tecnología de la Información.
\end{goal}
\begin{objectives}
-Explicar los conceptos de implementación de Sistemas de Información acoplados a la reingeniería e mejoramiento continuo.
\end{objectives}
\end{LU}

\begin{LU9}{Tecnología de Información y el consecusión de objetivos}{Gray2005MakingDecisions,Turban2005ITforManagement,McNurlin2005ISManagement,Applegate2002Corporate,Kaplan1996Scorecard,Tayntor2005SWImplementation}{}
\begin{goal}
-Mostrar como la Tecnología de Información puede ser utilizada para diseñar, facilitar y comunicar los objetivos organizacionales.
\end{goal}
\begin{objectives}
-Explicar la relevancia del manejo de Sistemas de Información alineados con los objetivos organizacionales.
\end{objectives}
\end{LU}

\begin{LU10}{Características de un profesional de Sistemas de Información}{OBrien2008IntroductionIS}{}
\begin{goal}
-Explicar los conceptos de la toma de decisiones personales, objetivos, definición de metas, confiabilidad y motivación.
\end{goal}
\begin{objectives}
-Discutir y explicar los conceptos de definición de metas y toma y alcance de decisiones individuales. Explicar los requerimientos de definición de metas y toma de decisiones personales en la motivación y mejora de las condiciones de trabajo.
\end{objectives}
\end{LU}


\begin{LU11}{Línea de carrera de carrera de Sistemas de Información}{OBrien2008IntroductionIS,InformationSystemsCurricula2002Book}{}
\begin{goal}
-Mostrar las áreas de la carrera de Sistemas de Información.
\end{goal}
\begin{objectives}
-Identificar y explicar las carreras de telecomunicaciones y sus áreas.
\end{objectives}
\end{LU}
 
\begin{LU12}{Ética y el profesional de Sistemas de Información}{Reynolds2006ITEthics}{}
\begin{goal}
-Presentar y discutir las responsabilidades profesionales y éticas del profesional de Sistemas de Información.
\end{goal}
\begin{objectives}
-Usar códigos de ética profesional para evaluar acciones de Sistemas de Información específicas.
-Describir asuntos éticos y legales; discutir y explicar consideraciones éticas del uso, distribución, operación y mantenimiento de software.
\end{objectives}
\end{LU}
 
\begin{LU13}{Sistemas de Información de nivel personal}{Tayntor2005SWImplementation}{}
\begin{goal}
-Identificar, investigar, analizar, diseñar y desarrollar con paquetes (y/o lenguajes de alto nivel) y sistemas de información de nivel personal para mejorar la productividad personal.
\end{goal}
\begin{objectives}
-Analizar, diseñar, desarrollar y usar paquetes (p.e. un paquete de estadística o de administración de datos de alto nivel) y/o bases de datos de alto nivel que requieran lenguajes para implementar soluciones trabajables para resolver problemas de Sistemas de Información asociados con actividades de trabajo del conocimiento.
-Evaluar el incremento de la productividad realizado a través de la implementación de sistemas personales.
\end{objectives}
\end{LU}
 
\begin{LU13.01}[LU13]{Conceptos de Trabajo y Actividad}{Halpern2002Thought,Tiwana1999Knowledge}{}
\begin{goal}
-Describir el concepto de trabajo del conocimiento y la necesidad de contar con tecnología de información personal que lo soporte.
\end{goal}
\begin{objectives}
-Definir y explicar el concepto de trabajo del conocimiento.
-Comparar y contrastar datos, información y conocimiento.
-Describir las actividades del trabajo del conocimiento; identificar y explicar métodos para lograr productividad en el trabajo del conocimiento.
\end{objectives}
\end{LU}
 
\begin{LU13.02}[LU13]{Soporte: Individuos vs Grupos}{Oz2002ManagementIS,Laudon2007ManagementIS,OBrien2005ManagementIS}{}
\begin{goal}
-Relacionar requerimientos de sistemas de información organizacionales vs. personales.
\end{goal}
\begin{objectives}
-Comparar y constrastar el planeamiento, desarrollo y administración de riesgos de las aplicaciones para sistemas de información personales vs. organizacionales.
-Explicar problemas potenciales de sistemas desarrollados por el usuario.
\end{objectives}
\end{LU}
 
\begin{LU13.03}[LU13]{Análisis de Información: Individual vs. Grupal}{Oz2002ManagementIS,Laudon2007ManagementIS,OBrien2005ManagementIS}{}
\begin{goal}
-Introducir conceptos de trabajo del conocimiento individuales vs. colaborativos y relacionarlos al análisis de las necesidades de información y a la tecnología.
\end{goal}
\begin{objectives}
-Describir y explicar tecnologías individuales vs. grupales; explicar el procesamiento adicional y otros asuntos y necesidades requeridas para el trabajo en grupo.
-Describir y explicar tecnología de soporte a grupos para requerimientos de conocimiento común.
-Describir y explicar el proceso de análisis de información y de aplicación de soluciones de tecnología de información.
\end{objectives}
\end{LU}
 
\begin{LU13.04}[LU13]{Análisis de Información: Encontrando sus requerimientos de Sistemas y Tecnologías de Información}{Stair2007FundamentalsIS,OBrien2008IntroductionIS}{}
\begin{goal}
-Describir y explicar los objetivos y el proceso de análisis y de la documentación del trabajo del conocimiento, tecnología de información y de los requerimientos de información para individuos y grupos de trabajo.
\end{goal}
\begin{objectives}
-Describir y explicar características y atributos del trabajo del conocimiento para individuos y grupos.
-Discutir y explicar las tareas de construcción y mantenimiento del conocimiento.
-Usar preguntas para elicitar sistemáticamente e identificar los requerimientos de datos de individuos y grupos.
-Analizar las tareas individuales y grupales para determinar los requerimientos de información.
-Identificar requerimientos de tecnología de información relacionada.
\end{objectives}
\end{LU}
 
\begin{LU13.05}[LU13]{Organizando recursos de datos personales}{Gray2005MakingDecisions,Tayntor2005SWImplementation}{}
\begin{goal}
-Definir conceptos, principios y métodos prácticos para la administración de software y datos individuales.
\end{goal}
\begin{objectives}
-Dadas tareas y actividades de trabajo del conocimiento, diseñar e implementar un método para la organización de directorios y el etiquetado de archivos que soporte la retención y acceso a los datos.
-Listar principios que apliquen a la adquisición y actualización de software.
-Describir métodos para la transferencia de datos entre aplicaciones incluyendo OLE, importación/exportación y métodos alternativos.
\end{objectives}
\end{LU}
 
\begin{LU13.06}[LU13]{Tecnologías y conceptos de Bases de Datos}{elmasri04}{}
\begin{goal}
-Explicar conceptos organizacionales, componentes, estructuras, acceso, seguridad y consideraciones de administración de bases de datos.
\end{goal}
\begin{objectives}
-Describir y explicar la terminología y el uso de bases de datos relacionales.
-Describir y explicar conceptos necesarios para acceder a bases de datos organizacionales.
-Usar infraestructura de acceso a bases de datos para hacer consultas de datos a partir de un repositorio organizacional.
\end{objectives}
\end{LU}
 
\begin{LU13.07}[LU13]{Acceso, Recuperación y Almacenamiento de Datos}{elmasri04}{}
\begin{goal}
-Definir el contenido, disponibilidad y estrategias para acceder información externa a la organización.
\end{goal}
\begin{objectives}
-Definir y discutir recursos de información externa; identificar la fuente, el contenido, los costos y la temporalidad.
-Localizar y acceder recursos de información externos usando herramientas de Internet disponibles: navegador, búsqueda, ftp.
-Crear y mantener un directorio individual para los recursos de información externa.
\end{objectives}
\end{LU}
 
\begin{LU13.08}[LU13]{Ciclo de vida de Sistemas de Información: Desarrollando con Paquetes}{OBrien2008IntroductionIS,Stair2007FundamentalsIS,Tayntor2005SWImplementation}{}
\begin{goal}
-Presentar y explicar el ciclo de vida de desarrollo de un sistema de información incluyendo los conceptos de adquisición vs. desarrollo de software.
\end{goal}
\begin{objectives}
-Discutir el concepto del ciclo de vida de un sistema de información.
-Identificar y explicar los criterios para decidir entre la adquisición de paquetes de software vs. el desarrollo de software personalizado.
\end{objectives}
\end{LU}
 
\begin{LU13.09}[LU13]{Configuración y personalización de un paquete}{Tayntor2005SWImplementation}{}
\begin{goal}
-Introducir y explorar el uso de software de aplicación y de propósito general.
\end{goal}
\begin{objectives}
-Instalar y personalizar un paquete de software de propósito general para proveer funcionalidad específica más allá de las opciones por defecto.
-Adicionar capacidades a un sistema de software por medio de la grabación y almacenamiento de una macro en la librería del paquete de software dado.
-Acceder a información técnica provista en la forma de facilidades de ``ayuda" del software; observar y usar la infraestructura de ``ayuda".
\end{objectives}
\end{LU}
 
\begin{LU13.10}[LU13]{Programación Procedural y Orientada a Eventos}{Cormen91,Stroustrup97,Meyer98}{}
\begin{goal}
-Introducir y explorar los métodos de desarrollo de software, para luego explicar los objetivos y estrategias de los paradigmas de programación procedural, basado en eventos y orientado a objetos.
\end{goal}
\begin{objectives}
-Discutir y explicar los conceptos de datos y de representación procedural, lenguajes de programación, compiladores, intérpretes, ambientes de desarrollo e interfaces gráficas de usuario basadas en eventos.
-Comparar, relacionar y explicar conceptos de métodos estructurados, basados en eventos y orientados a objetos para el diseño de programas, con ejemplos para cada método.
\end{objectives}
\end{LU}
 
\begin{LU13.11}[LU13]{Implementando algoritmos simples}{Cormen91,Stroustrup97,Meyer98}{}
\begin{goal}
-Introducir y desarrollar el proceso de desarrollo de algoritmos y código estructurado.
\end{goal}
\begin{objectives}
-Definir un problema sencillo identificando las salidas deseadas para entradas dadas; ofrecer una vista panorámica del problema.
-Describir tipos de datos fundamentales y sus operaciones.
-Diseñar lógica de programas usando tanto técnicas gráficas como de pseudocódigo que utilicen estructuras de control estándar: secuencia, iteración y selección.
-Traducir estructuras de datos y diseño de programas en código de un lenguaje de programación; verificar la traducción y asegurar la correctitud de los resultados; evaluar el código con conjuntos de datos de prueba.
\end{objectives}
\end{LU}
 
\begin{LU13.12}[LU13]{Implementación de un Diseño simple de Base de Datos}{elmasri04}{}
\begin{goal}
-Introducir el propósito y desarrollar la habilidad para usar un paquete de software de bases de datos relacionales.
\end{goal}
\begin{objectives}
-Describir y explicar tablas, relaciones, integridad referencial y los conceptos de las formas normales.
-A partir de un dibujo de flujo de trabajo o de otros documentos de requisitos, derivar un diseño de bases de datos simple con múltiples tablas.
-Usando un paquete de software de bases de datos relacionales, implementar y poblar las tablas; desarrollar varias consultas simples para explorar los datos.
\end{objectives}
\end{LU}
 
\begin{LU13.13}[LU13]{Implementación de aplicaciones orientadas a eventos}{Cormen91,Stroustrup97,Meyer98}{}
\begin{goal}
-Introducir y desarrollar la habilidad para diseñar e implementar una infraestructura de interfaz gráfica de usuario.
\end{goal}
\begin{objectives}
-Aplicar una solución de GUI basada en eventos en un ambiente de desarrollo.
-Construir un formulario de aplicación simple con varios objetos (p.e. etiqueta, cajas de edición {\it edit box}, listas, botones de comando).
\end{objectives}
\end{LU}
 
\begin{LU13.14}[LU13]{Desarrollo de Sistemas de Información con prototipado}{Avison98ISD,Kirikova03ISD}{}
\begin{goal}
-Presentar el proceso de prototipeo e introducir y aplicar los conceptos de evaluación y refinamiento evolutivo para prototipos de aplicaciones personales.
\end{goal}
\begin{objectives}
-Comparar las capacidades de una aplicación con los requerimientos que debe cubrir.
-Identificar salidas alternativas del proceso de verificación de aplicaciones.
-Evaluar y definir los resultados y probabilidades de error en software de aplicación prototipeo.
-Modificar entradas, salidas y procesamiento para refinar un prototipo.
\end{objectives}
\end{LU}
 
\begin{LU13.15}[LU13]{Evolución de la Tecnología de Sistemas de Información}{OBrien2008IntroductionIS,Stair2007FundamentalsIS}{}
\begin{goal}
-Presentar tecnologías de fundamento y definir la importancia en el futuro de las capacidades de la tecnología de información.
\end{goal}
\begin{objectives}
-Listar y explicar tecnologías y su relevancia para tecnología de información individual.
-Dada una tecnología, explicar su importancia para los desarrollos futuros y la productividad futura del trabajador del conocimiento.
-Identificar los causantes e inhibidores del cambio en la tecnología de la información.
\end{objectives}
\end{LU}
 
\begin{LU13.16}[LU13]{Implementación de una aplicación de Sistemas de Información personal}{Tayntor2005SWImplementation}{}
\begin{goal}
-Identificar, investigar, analizar, diseñar y desarrollar con paquetes (y/o lenguajes de alto nivel) un sistema de información de nivel personal simple para mejorar la productividad individual.
\end{goal}
\begin{objectives}
-Analizar, diseñar, desarrollar y usar paquetes y/o lenguajes de bases de datos de alto nivel para implementar soluciones trabajables que resuelvan un problema de sistemas de información asociado con actividades de trabajo del conocimiento.
-Evaluar el incremento de productividad realizado implementando sistemas personales.
\end{objectives}
\end{LU}
 
\begin{LU14}{Resolución de problemas con paquetes}{Tayntor2005SWImplementation}{}
\begin{goal}
-Presentar y aplicar estrategias, metodologías y métodos para usar paquetes de software, así como lenguajes de alto nivel para desarrollar soluciones a problemas formales implementables de ``usuario final'', los cuales se encuentran alineados con los sistemas de información organizacionales.
\end{goal}
\begin{objectives}
-Explicar y usar conceptos de problemas formales e ingeniería de software aplicadas al desarrollo de soluciones efectivas que mejoren la productividad personal que involucre actividades de trabajo del conocimiento, dentro de soluciones que son compatibles con el sistema de información organizacional.
-Desarrollar, documentar y mantener sistemas pequeños para productividad personal usando bases de datos de alto nivel usando herramientas y ambientes de desarrollo de aplicaciones.
-Usar los conceptos de definición y resolución de problemas analíticos formales en el uso de paquetes de software; asegurar que dichas soluciones tomen en cuenta los sistemas de información ``reales'' involucrados.
\end{objectives}
\end{LU}

\begin{LU15}{Estrategias de uso de Información}{Halpern2002Thought,Tiwana1999Knowledge}{}
\begin{goal}
-Presentar y aplicar estrategias para acceder y usar recursos de información.
\end{goal}
\begin{objectives}
-Explicar administración de datos y acceso a recursos de información corporativos y alternativos.
-Discutir inteligentemente las diferencias entre la administración de SI\&T, desarrollo de sistemas, mantenimiento de sistemas, operación de sistemas.
\end{objectives}
\end{LU}

\begin{LU16}{Teoría de Sistemas de Información}{OBrien2008IntroductionIS,Stair2007FundamentalsIS}{}
\begin{goal}
-Introducir, discutir y describir conceptos fundamentales de teoría de Sistemas de Información y su importancia para los profesionales.
\end{goal}
\begin{objectives}
-Identificar y explicar los conceptos subyacentes de la disciplina de Sistemas de Información.
\end{objectives}
\end{LU}

\begin{LU17}{Sistemas de Información como un componente estratégico}{Oz2002ManagementIS,OBrien2005ManagementIS,Laudon2007ManagementIS}{}
\begin{goal}
-Mostrar como un sistema de información es un componente estratégico e integral de una organización.
\end{goal}
\begin{objectives}
-Describir el desarrollo histórico de la disciplina de Sistemas de Información.
-Explicar el rol estratégico de los sistemas de información en las organizaciones.
-Explicar la relación estratégica de las actividades de Sistemas de Información para mejorar la posición competitiva.
-Explicar las diferencias entre aplicaciones de nivel estratégico, táctico y operativo.
\end{objectives}
\end{LU}

\begin{LU18}{Desarrollo y administración de Sistemas de Información}{OBrien2008IntroductionIS}{}
\begin{goal}
-Discutir cómo se desarrolla un sistema de información y éste es administrado dentro de una organización.
\end{goal}
 \begin{objectives}
-Explicar el desarrollo de sistemas de información y el rediseño de los procesos organizacionales; explicar los grupos de individuos y sus responsabilidades en este proceso.
-Explicar los roles de los profesionales en Sistemas de Información dentro de una organización de Sistemas de Información; explicar las funciones de la administración de Sistemas de Información, administrador de proyectos, analista de información y explicar los caminos de desarrollo profesional posibles.
\end{objectives}
 \end{LU}

\begin{LU19}{Proceso cognitivo}{Halpern2002Thought,Tiwana1999Knowledge}{}
 \begin{goal}
-Presentar y discutir la relevancia del proceso cognitivo e interacciones humanas en el diseño e implementación de sistemas de información.
\end{goal}
 \begin{objectives}
-Explicar el proceso cognitivo y otras consideraciones orientadas al ser humano en el diseño e implementación de sistemas de información.
\end{objectives}
 \end{LU}

\begin{LU20}{Objetivos y decisiones}{Robbins2002Behavior,Ivancevich2008Behavior,Gray2005MakingDecisions}{}
 \begin{goal}
-Discutir cómo los individuos toman decisiones y establecen y alcanzan objetivos.
\end{goal}
\begin{objectives}
-Discutir y explicar cómo los individuos toman decisiones, establecen y alcanzan objetivos; explicar lo que significa acción personal dirigida a una misión.
\end{objectives}
 \end{LU}

\begin{LU21}{Toma de decisiones: el modelo de Simon}{Gray2005MakingDecisions}{}
\begin{goal}
-Desarrollar la capacidad para discutir e intercambiar opiniones sobre  el Modelo de Simon para la toma de decisiones organizacionales y su soporte utilizando IS.
\end{goal}
\begin{objectives}
-Discutir y explicar la teoría de decisiones y el proceso de toma de decisiones.
-Explicar el soporte de IS para la toma de decisiones; explicar el uso de sistemas expertos en el soporte en la toma de decisiones heurísticas.
-Explicar y dar una ilustración del modelo de decisión organizacional de Simon.
\end{objectives}
\end{LU}

\begin{LU22}{Sistemas y calidad}{Jalote99,Peach02,Kulpa03}{}
\begin{goal}
-Introducir a la teoría de Sistemas, calidad  y modelado organizacional y demostrar su relevancia en los sistemas de información.
\end{goal}
\begin{objectives}
-Discutir y explicar los objetivos de los sistemas, expectativas de los clientes y conceptos de calidad.
-Discutir y explicar los componentes y relaciones de los sistemas.
-Aplicar conceptos de sistemas para definir y explicar el rol de los sistemas de información.
-Explicar el uso de la información y sistemas de información en actividades de documentación toma de decisiones y control organizacional.
\end{objectives}
\end{LU}

\begin{LU23}{Rol de la administración, usuarios, diseñadores de sistemas}{Oz2002ManagementIS,OBrien2005ManagementIS}{}
\begin{goal}
-Discutir un sistema basado en reglas para la administradores, usuarios y diseñadores.
\end{goal}
\begin{objectives}
-Identificar la responsabilidad de los usuarios, diseñadores y administradores en términos descritos en la trinidad Churchman;  discutir en términos de sistemas detallando obligaciones de cada uno, relatar esas observaciones para mejorar los modelos de calidad para el desarrollo organizacional; identificar la función de los  IS en esos términos.
\end{objectives}
\end{LU}

\begin{LU24}{Flujo de trabajo de Sistemas Organizacionales}{Oz2002ManagementIS,OBrien2005ManagementIS}{}
\begin{goal}
-Explicar los sistemas físicos y el flujo de trabajo y como los sistemas de información están relacionados a los sistemas organizacionales.
\end{goal}
\begin{objectives}
-Explicar la relación entre el modelo de base de datos   y la actividad física organizacional.
\end{objectives}
\end{LU}

\begin{LU25}{Modelos y relaciones organizacionales con Sistemas de Información}{Oz2002ManagementIS,OBrien2005ManagementIS}{}
\begin{goal}
-Presentar otros modelos organizacionales  y su relevancia para los IS.
\end{goal}
\begin{objectives}
-Describir el rol de la tecnología de información (IT) y las reglas de las personas usando, diseñando y manteniendo IT en las organizaciones.
-Discutir como la teoría general de sistemas es aplicada al análisis y desarrollo de los sistemas de información.
\end{objectives}
\end{LU}

\begin{LU26}{Planeamiento de Sistemas de Información}{Stair2007FundamentalsIS,OBrien2008IntroductionIS}{}
\begin{goal}
-Discutir la relación  entre el planeamiento de los IS  con el planeamiento organizacional.
\end{goal}
\begin{objectives}
-Explicar metas y procesos de planeamiento.
-Explicar la importancia del planeamiento estratégico  y cooperativo  así como el alineamiento del plan proyecto de los sistemas de información.
\end{objectives}
\end{LU}

\begin{LU27}{Tipos de Sistemas de Información}{OBrien2008IntroductionIS,Stair2007FundamentalsIS}{}
\begin{goal}
-Demostrar clases específicas de sistemas de aplicación incluyendo TPS y DSS.
\end{goal}
\begin{objectives}
-Describir la clasificación de los sistemas de información, por  ejemplo, TPS, DSS, ESS, WFS.
-Explicar la relevancia organizacional de los IS: TPS, DSS, EIS, ES, {\it Work Flow System}.
\end{objectives}
\end{LU}

\begin{LU28}{Estándares de desarrollo de Sistemas de Información}{OBrien2008IntroductionIS}{}
\begin{goal}
-Discutir  y examinar los procesos, estándares y políticas para el desarrollo de sistemas de información. Desarrollo de metodologías, ciclo de vida, workflow, OOA, prototipeo, espiral, usuario final entre otros.
\end{goal}
\begin{objectives}
-Discutir y explicar el concepto de una metodología de desarrollo de IS, explicar el ciclo de vida, workflow, OOA, prototipeo, modelos basado en riesgos, modelo en espiral, entre otros; mostrar como esto puede ser usado en la práctica.   
\end{objectives}
\end{LU}

\begin{LU29}{Implementación de Sistemas de Información: \textit{outsourcing}}{OBrien2008IntroductionIS,Stair2007FundamentalsIS}{}
\begin{goal}
-Discutir {\it outsourcing} e implementaciones alternativas de IS.
\end{goal}
\begin{objectives}
-Explicar las ventajas  y desventajas del desarrollo {\it outsourcing} en algunas o todas las funciones de IS; establecer  los requerimientos del personal con o sin {\it outsourcing}
\end{objectives}
\end{LU}

\begin{LU30}{Evaluación de desempeño del personal}{Stoner1995Management,Daft2007Management}{}
\begin{goal}
-Discutir la evaluación del rendimiento  la cual consiste e con la administración de la calidad  y la mejora continua.
\end{goal}
\begin{objectives}
-Describir, explicar y aplicar las responsabilidades del líder del proyecto, administrar el desarrollo de pequeños sistemas.
-Discutir, explicar e implementar una metodología para hacer seguimiento a los clientes dentro de todo las fases del ciclo de vida
-Explicar metodologías para facilitar el uso de estándares como el ISO 9000, {\it National Performance Review} y otros estándares de calidad.
\end{objectives}
\end{LU}

\begin{LU31}{Sociedad de Sistemas de Información y ética}{Reynolds2006ITEthics}{}
\begin{goal}
-Introducir las implicaciones sociales y éticas de los Sistemas de Información para introducir a la exploración de los conceptos éticos y asuntos relacionados al comportamiento profesional.
-Comparar y contrastar los modelos e abordajes éticos.
\end{goal}
\begin{objectives}
-Discutir y explicar ética y comportamiento basado en principios así como el concepto de práctica ética en el área de Sistemas de Información.
-Discutir modelos éticos importantes y discutir las razones por las cuales hay que ser ético.
-Explicar el uso del código de ética profesional.
-Explicar la carga responsabilidad y de profesionalismo resultante de la confianza asociada con el conocimiento y habilidades de computación.
-Discutir y explicar las bases y naturaleza de los abordajes éticos cuestionables.
-Discutir y explicar el análisis ético y social del desarrollo de Sistemas de Información.
-Discutir y explicar los asuntos de poder y su impacto social en el ciclo de vida del desarrollo.
\end{objectives}
\end{LU}

\begin{LU32}{Dispositivos, medios, sistemas de Telecomunicaciones}{sheldon1994,tanenbaum2003,Deitel04}{}
\begin{goal}
-Desarrollar la preocupación y la terminología asociada de los diferentes y dispositivos necesarios para telecomunicaciones, incluyendo redes LAN y WAN.
\end{goal}
\begin{objectives}
-Identificar las características de la transmisión de datos en telecomunicaciones a nivel de LANs, WANs y MANs.
-Accesar información remota para transferencia de archivos en entornos LAN y WAN.
-Discutir y explicar la industria de las telecomunicaciones así como sus estándares y regulaciones.
\end{objectives}
\end{LU}

\begin{LU33}{Soporte organizacional basado en Telecomunicaciones}{Cisco04,sheldon1994,tanenbaum2003}{}
\begin{goal}
-Desarrollar una preocupación por la forma en la que los sistemas de telecomunicaciones son utilizados para soportar la infraestructura de comunicaciones de la organización incluyendo a los Sistemas de Información, teleconferencias, etc.
\end{goal}
\begin{objectives}
-Explicar el uso de los Sistemas de Información para soportar el flujo de trabajo;
-Discutir los conceptos de teleconferencias y conferencias por telecomputadoras en el rol de las comunicaciones y en la toma de decisiones.
-Discutir y explicar la infraestructura involucrada en los sistemas de telecomunicaciones.
\end{objectives}
\end{LU}

\begin{LU34}{Economía y problemas de diseño de sistemas de Telecomunicaciones}{Cisco04,sheldon1994,tanenbaum2003}{}
\begin{goal}
-Explorar los asuntos relacionados al diseño y manejo económico de las redes de computadores.
\end{goal}
\begin{objectives}
-Explicar los pasos en el análisis y configuración de un sistema de telecomunicaciones, incluyendo hardware específico y componentes de software.
-Explicar el propósito de modems, bridges, gateways, hubs y ruteadores en la interconexión de sistemas.
\end{objectives}
\end{LU}

\begin{LU35}{Estándares de Telecomunicaciones}{Cisco04,sheldon1994,tanenbaum2003}{}
\begin{goal}
-Familiarizar al estudiante con los estándares de telecomunicaciones, con las organizaciones que las regulan y con sus estándares.
\end{goal}
\begin{objectives}
-Identificar el rol de los estándares y de las organizaciones regulatorias y sus estándares como facilitadores para lograr desde telecomunicaciones locales hasta aquellas globales.
-Explicar la codificación digital de datos relevantes a las telecomunicaciones.
\end{objectives}
\end{LU}

\begin{LU36}{Sistemas centralizados/distribuidos}{ozsu99,coulouris2005}{}
\begin{goal}
-Discutir y explicar los principios fundamentales y temas relacionados a comparar la computación centralizada versus computación distribuida.
\end{goal}
\begin{objectives}
-Explicar, diagramar y discutir las estructuras y principios involucrados en la computación distribuida en cuanto a recursos y datos