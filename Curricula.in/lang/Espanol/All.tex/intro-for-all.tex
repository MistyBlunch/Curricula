La computación ha sufrido un desarrollo impresionante en las últimas décadas, convirtiéndose en 
el motor del desarrollo científico, tecnológico, industrial, social, económico y cultural, 
transformando de manera significativa nuestro diario accionar.

El surgimiento del computador ha marcado una nueva era en la historia de la humanidad que era 
imposible de imaginar varias décadas atrás. La gran cantidad de aplicaciones que se han desarrollado 
en los últimos años están transformando el desarrollo de todas las disciplinas del saber, 
la comercialización en el ámbito globalizado en que vivimos, la manera en que nos comunicamos, 
los procesos de enseñanza-aprendizaje y hasta en la manera como nos entretenemos.

Para darnos una idea de la relevancia e importancia, que en nuestro país ha alcanzado esta 
disciplina, basta mencionar que actualmente se ofrecen aproximadamente más de 110 carreras de Computación a nivel nacional. 
Esto sin considerar los programas de nivel Técnico Superior No Universitario que se ofertan.

Todas estas carreras existentes tienen como centro de su estudio a la computación pero lo 
hacen con 28 nombres distintos como: Ingeniería de Sistemas, Ingeniería de Computación, 
Ingeniería de Computación y Sistemas, entre otros. A pesar de que todas ellas apuntan al mismo 
mercado de trabajo, resulta por lo menos sorprendente que no sea posible encontrar por lo menos dos 
que compartan la misma curricula.

Muchos países consideran a la computación como estratégica para su desarrollo. En Perú, 
el \ac{CONCYTEC} ha recomendado al gobierno que considere a la Computación como una de las 
áreas prioritarias de vinculación entre la academia e industria para fomentar la 
competitividad y la innovación.

Comúnmente, durante la década de los setenta, la Computación se desarrolló dentro 
de las Facultades de Ciencias en la mayoría de las universidades estadounidenses, 
británicas y de otros países. Durante la década de los ochenta, los grupos de 
computación en las universidades se esforzaron por lograr una legitimidad académica 
en su ámbito local. Frecuentemente, se transformaron en departamentos de Matemáticas y 
Computación, hasta finalmente dividirse en dos departamentos de Matemáticas y de 
Computación, en la década de los noventa. Es en esta década en que un número 
creciente de instituciones reconocieron la influencia penetrante de la Computación, 
creando unidades independientes como departamentos, escuelas o institutos dedicados a 
tal área de estudio, un cambio que ha demostrado tanto perspicacia como previsión. 

En Perú, un número cada vez mayor de instituciones de educación superior han 
tratado de seguir el desarrollo de las universidades extranjeras (aunque no siempre en 
forma muy seria o exitosa), reconociendo a la Computación como un área de estudio 
en sí misma, así como su importancia estratégica en la educación, y creando 
departamentos, escuelas o institutos dedicados a su estudio. La Facultad de \FacultadName 
no puede ser la excepción a este cambio, en el que ya se tiene un retraso relativo con 
muchas de las instituciones educativas dentro y fuera de Perú.
