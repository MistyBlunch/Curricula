% Responsable : Luis Díaz Basurco
% Sumilla de  : Análisis Matemático III
% Versión     : 1

\begin{syllabus}

\course{CB201. Análisis Matemático III}{Obligatorio}{CB201}

\begin{justification}
Es una extensión de los cursos de Análisis Matemático I y Análisis Matemático II, tomando en cuenta dos o más variables, indispensables para aquellas materias que requieren trabajar con geometría en curvas y superficies, así como en procesos de búsqueda de puntos extremos.
\end{justification}

\begin{goals}
\item Diferenciar e integrar funciones vectoriales de variable real, entender y manejar el concepto de parametrización. Describir una curva en forma paramétrica.
\item Describir, analizar, diseñar y formular modelos continuos que dependen de más de una variable.
\item Establecer relaciones entre diferenciación e integración y aplicar el cálculo diferencial e integral ala resolución de problemas geométricos y de optimización.
\end{goals}

\begin{outcomes}
\ExpandOutcome{a}{3}
\ExpandOutcome{i}{3}
\ExpandOutcome{j}{4}
\end{outcomes}

\begin{unit}{Geometría en el espacio}{Apostol73,Simmons95}{8}{3}
   \begin{topics}
      \item $R^3$ como espacio euclídeo y álgebra.
      \item Superficies básicas en el espacio.
   \end{topics}

   \begin{unitgoals}
      \item Manejar el álgebra vectorial en $R^3$
      \item Identificar tipos de superficies en el espacio
      \item Graficar superficies básicas
      \end{unitgoals}
\end{unit}

\begin{unit}{Curvas y parametrizaciones}{Apostol73,Simmons95}{20}{3}
   \begin{topics}
      \item Funciones vectoriales de variable real. Reparametrizaciones
      \item Diferenciación e integración
      \item Velocidad, aceleración, curvatura, torsión
      \end{topics}

   \begin{unitgoals}
      \item Describir las diferentes características de una curva
      \end{unitgoals}
\end{unit}

\begin{unit}{Campos escalares}{Apostol73,Bartle76,Simmons95}{20}{3}
   \begin{topics}
      \item Curvas de nivel
      \item Límites y continuidad
      \item Diferenciación
      \end{topics}

   \begin{unitgoals}
      \item Graficar campos escalares
      \item Discutir la existencia de un límite y la continuidad de un campo escalar
      \item Calcular derivadas parciales y totales.
      \end{unitgoals}
\end{unit}

\begin{unit}{Aplicaciones}{Apostol73,Simmons95,Bartle76}{12}{3}
   \begin{topics}
      \item Máximos y mínimos
      \item Multiplicadores de Lagrange
      \end{topics}

   \begin{unitgoals}
      \item Interpretar la noción de gradiente en curvas de nivel y en superficies de nivel
      \item Usar técnicas para hallar extremos
      \end{unitgoals}
\end{unit}

\begin{unit}{Integración Múltiple}{Apostol73}{12}{4}
   \begin{topics}
      \item Integración de Riemann
      \item Integración sobre regiones
      \item Cambio de coordenadas
      \item Aplicaciones
      \end{topics}

   \begin{unitgoals}
      \item Reconocer regiones de integración adecuadas
      \item Realizar cambios de coordenadas adecuados
      \item Aplicar la integración múltiple a problemas
      \end{unitgoals}
\end{unit}

\begin{unit}{Campos vectoriales}{Apostol73}{18}{3}
   \begin{topics}
      \item Integrales de linea
      \item campos conservativos
      \item Integrales de superficie
   \end{topics}

   \begin{unitgoals}
      \item Calcular la integral de linea de campos vectoriales
      \item Reconocer campos conservativos
      \item Hallar funciones potenciales de campos conservativos
      \item Hallar integrales de superficies y aplicarlas
      \end{unitgoals}
\end{unit}



\begin{coursebibliography}
\bibfile{BasicSciences/CB201}
\end{coursebibliography}

\end{syllabus}
