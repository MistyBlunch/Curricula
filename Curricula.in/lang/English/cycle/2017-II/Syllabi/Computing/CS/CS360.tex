\begin{syllabus}

\course{CS360. Computación Bioinspirada}{Electivo}{CS360}

\begin{justification}
La computación bioinspirada es el área de investigación que estudia
las diferentes técnicas computacionales que tienen inspiración
biológica, las cuales permiten desarrollar nuevas herramientas para
la solución de problemas y pueden estar basadas en patrones
naturales, en comportamiento de los seres vivos, en la estructura
misma de los organismos, etc. 
\end{justification}

\begin{goals}
\item Elaborar modelos teóricos inspirados biológicamente, que puedan ser implementados en las computadoras, a fin de reproducir su funcionamiento tanto cualitativa como cuantitativamente.
\item Estudiar los fenómenos naturales, los procesos, modelos teóricos, para construir algoritmos capaces de resolver problemas complejos.
\end{goals}

\begin{outcomes}
\ExpandOutcome{a}{3} 
\ExpandOutcome{h}{3} 
\ExpandOutcome{i}{3}
\ExpandOutcome{j}{4}
\end{outcomes}

\begin{unit}{Introducción a la Computacion Bioinspirada}{Castro06,Baldi01}{2}{2}
\begin{topics}
        \item Introducción
        \item Motivación
        \item La filosofía de la computación natural
        \item Computación inspirada por la naturaleza
        \item Simulación y emulación de la naturaleza en las computadoras
        \item Computación con materiales naturales
    \end{topics}
    \begin{unitgoals}
        \item Conocer el fundamento de la computación bioinspirada.
        \item Diferenciar las diferentes ramas de la computación naturalmente inspirada.
    \end{unitgoals}
\end{unit}

\begin{unit}{Conceptualización}{Castro06}{4}{2}
\begin{topics}
        \item Entidades Individuales y Agentes.
        \item Procesamiento paralelo y distribuido.
        \item Interactividad.
        \item Adaptación.
        \item Auto Organización.
        \item Complejidad, emergencia y reduccionismo.
        \item Determinismo.
        \item Teoria del Caos.
        \item Fractales.
    \end{topics}

    \begin{unitgoals}
        \item Conocer los conceptos básicos en los que se fundamentan la computación bioinspirada
        \item Caracterizar los sistemas bioinspirados
        \item Identificar los comportamientos complejos
    \end{unitgoals}
\end{unit}

\begin{unit}{\ISAdvancedSearchDef}{Goldberg89,Mitchell98,Castro06}{8}{3}
     \ISAdvancedSearchAllTopics
     \ISAdvancedSearchAllObjectives
\end{unit}

% Nueural Networks
\begin{unit}{\ISMachineLearningDef}{Haykin99,Castro06}{10}{3}
    \ISMachineLearningAllTopics
    \ISMachineLearningAllObjectives
\end{unit}

\begin{unit}{Inteligencia de enjambre}{Dorigo04,James01,Castro06}{6}{3}
\begin{topics}
        \item Introducción
        \item Colonias de hormigas: inspiración biológica.
        \item Colonias de hormigas: algoritmo básico.
        \item Optimización de enjambre de partículas: inspiración biológica.
        \item Optimización de enjambre de partículas: algoritmo básico.
        \item Aplicación de la inteligencia de enjambre.
        \item Tendencias y problemas abiertos.
    \end{topics}
    \begin{unitgoals}
        \item Conocer la inteligencia de enjambre.
        \item Implementar la colonia de hormigas.
        \item Estudiar la optimización de enjambre de partículas.
    \end{unitgoals}
\end{unit}

\begin{unit}{Sistema inmunológico artificial}{Castro06}{6}{3}
\begin{topics}
        \item Motivación biológica.
        \item Sistemas inmunológicos.
        \item Sistemas inmunológicos artificiales.
        \item Redes de sistemas inmunológicos.
        \item Principios de diseño.
        \item Ambito de aplicación de los sistemas inmunológicos.
        \item Tendencias y problemas abiertos.
    \end{topics}
    \begin{unitgoals}
        \item Conocer la motivación de los sistemas inmunológicos.
    \end{unitgoals}
\end{unit}

\begin{unit}{Geometria fractal}{Castro06}{6}{3}
\begin{topics}
        \item Introducción.
        \item Dimensión fractal.
        \item Naturaleza de la geometría fractal.
        \item Automatas celulares.
        \item Automatas celulares y sistemas dinámicos.
        \item sistema de Lindenmayer.
        \item Tendencias y problemas abiertos.
    \end{topics}
    \begin{unitgoals}
        \item Estudiar la geometría fractal.
        \item Estudiar los autómatas celulares.
        \item Implementar autómatas celulares.
    \end{unitgoals}
\end{unit}

\begin{unit}{Vida artificial}{Castro06}{6}{3}
\begin{topics}
        \item Introducción.
        \item La esencia de la vida.
        \item Proyectos basados en vida artificial.
        \item Autómatas Celulares para la creación de vida artificial.
        \item Ámbito de aplicación de la vida artificial.
        \item Tendencias y problemas abiertos.
    \end{topics}
    \begin{unitgoals}
        \item Estudiar como generar vida artificial.
        \item Implementar autómatas celulares para generar vida artificial.
    \end{unitgoals}
\end{unit}

\begin{unit}{Computación basada en ADN}{Castro06}{6}{3}
\begin{topics}
        \item Introducción.
        \item Motivación biológica.
        \item Filtrando modelos.
        \item Modelos Formales.
        \item Computadores de ADN universales.
        \item Ámbito de aplicación de la vida artificial.
        \item Tendencias y problemas abiertos.
    \end{topics}
    \begin{unitgoals}
        \item Estudiar la computación basada en ADN.
        \item Estudiar de la potencia computacional de las variantes consideradas, comparada con la potencia de las máquinas de Turing.
    \end{unitgoals}
\end{unit}

\begin{unit}{Computación cuántica}{Castro06}{6}{2}
\begin{topics}
        \item Introducción.
        \item conceptos básicos de la teoría cuántica.
        \item Principales mecanismos de la teoría cuántica.
        \item Algoritmos cuánticos.
        \item Computadores cuánticos.
        \item Ámbito de aplicación de la vida artificial.
        \item Tendencias y problemas abiertos.
    \end{topics}
    \begin{unitgoals}
        \item Estudiar la computación cuántica.
        \item Codificar algoritmos cuánticos.
        \item Simular y calcular la eficiencia de algoritmos cuánticos.
    \end{unitgoals}
\end{unit}



\begin{coursebibliography}
\bibfile{Computing/CS/CS261T}
\end{coursebibliography}

\end{syllabus}
