\begin{syllabus}

\curso{CS325. Sistemas Distribuídos}{Obligatorio}{CS325}

\begin{justification}
Los Sistemas Distribuídos, son aquellos cuyos componentes hardware y software, que están en ordenadores conectados en red, se comunican y coordinan sus acciones mediante el paso de mensajes, para el logro de un objetivo. Se establece la comunicación mediante un protocolo prefijado por un esquema cliente-servidor

El propósito de un Sistema Distribuído es proveer un ambiente en que el usuario puede ejecutar sus aplicaciones distribuída de forma transparente.

Además el curso contempla actividades prácticas en donde se resolverán problemas de concurrencia y se modificará el funcionamiento de un pseudo Sistema Distribuído.
\end{justification}

\begin{goals}
\item Conocer los elementos básicos del diseño de Sistemas Distribuídos
\item aprender a instalar y usar aplicaciones en Sistemas Distribuídos
\end{goals}

\begin{outcomes}
\ExpandOutcome{a}
\ExpandOutcome{b}
\ExpandOutcome{c}
\ExpandOutcome{i}
\ExpandOutcome{j}
\end{outcomes}

\begin{unit}{Introducción a los Sistemas Distribuídos}{tanenbaum1996,coulouris2005}{3}
   \begin{topics}
      \item Objetivos
      \item Conceptos de Hardware y Software
      \item Características
      \item Aspectos de Diseño
   \end{topics}

   \begin{unitgoals}
      \item Entender los conceptos básicos de los Sistemas Distribuídos
   \end{unitgoals}
\end{unit}

\begin{unit}{Comunicación de los Sistemas Distribuídos}{tanenbaum1996,coulouris2005}{3}
   \begin{topics}
      \item Stacks de Comunicaciones (Protocolos con capas)
      \item El modelo Cliente/Servidor
   \end{topics}

   \begin{unitgoals}
      \item Entender los conceptos básicos de comunicación en Sistemas Distribuídos
   \end{unitgoals}
\end{unit}

\begin{unit}{El modelo Cliente/Servidor}{orfali1999,tanenbaum2003}{3}
   \begin{topics}
      \item Arquitectura
      \item Clientes y Servidores
      \item Plataformas
      \item Modelos 2-Tier, 3-Tier, Multi-Tier
   \end{topics}

   \begin{unitgoals}
      \item Entender el modelo Cliente/Servidor
   \end{unitgoals}
\end{unit}

\begin{unit}{Soporte del Sistema Operativo}{orfali1999,tanenbaum2003}{3}
   \begin{topics}
      \item Procesos e Hilos
      \item Modelos
   \end{topics}

   \begin{unitgoals}
      \item Conocer el soporte del Sistema Operativo a los Sistemas Distribuídos
   \end{unitgoals}
\end{unit}

\begin{unit}{Middleware}{sheldon1995,tanenbaum2003,sheldon1994}{3}
   \begin{topics}
      \item Llamada a un procedimiento remoto (RPC)
      \item Middleware Orientado a Mensajes (MOM)
      \item Peer-to-Peer
      \item Servicio de directorio
      \item Seguridad
   \end{topics}

   \begin{unitgoals}
      \item Entender los principios del middleware
   \end{unitgoals}
\end{unit}

\begin{unit}{Sistemas Distribuídos de Archivos}{tanenbaum1996,coulouris2005}{3}
   \begin{topics}
      \item Diseño
      \item Implementación
      \item Tendencias
   \end{topics}

   \begin{unitgoals}
      \item Entender el funcionamiento de los Sistemas Distribuídos de Archivos
   \end{unitgoals}
\end{unit}

\begin{unit}{Transacciones Distribuidas y Control de Concurrencia}{tanenbaum1996,coulouris2005}{3}
   \begin{topics}
      \item Sincronización
      \item Exclusión Mutua
      \item Transacciones Atómicas
      \item Bloqueos en Sistemas Distribuídos
   \end{topics}

   \begin{unitgoals}
      \item Conocer diferentes casos de transacciones distribuidas
      \item Entender los conceptos básicos de control de concurrencia
      \item 
   \end{unitgoals}
\end{unit}

\begin{unit}{Objetos Distribuídos ~ Modelos de Componentes}{tanenbaum1996,coulouris2005}{3}
   \begin{topics}
      \item Objetos y Componentes
      \item Beneficios
      \item Modelos de Componentes
   \end{topics}

   \begin{unitgoals}
      \item Conocer diferentes casos de objetos distribuidas
      \item Entender los conceptos básicos del modelo de componentes
   \end{unitgoals}
\end{unit}

\begin{unit}{CORBA}{hoque1998}{3}
   \begin{topics}
      \item Arquitectura
      \item Metadata y Servicios
      \item ORB e IDL
      \item CORBA IIOP
      \item Implementaciones
   \end{topics}

   \begin{unitgoals}
      \item Entender el estandar CORBA para sistemas computacionales heterogeneos
   \end{unitgoals}
\end{unit}

\begin{unit}{COM}{hoque1998}{3}
   \begin{topics}
      \item Arquitectura
      \item Servicios
      \item Documentos Compuestos y OCX/ActiveX
      \item Integración COM-CORBA
      \item Implementaciones
   \end{topics}

   \begin{unitgoals}
      \item  Entender el estandar COM para sistemas computacionales heterogeneos y su compatibilidad con CORBA
   \end{unitgoals}
\end{unit}

\begin{unit}{Enterprise Java Beans (EJB)}{hoque1998,sheldon1995}{3}
   \begin{topics}
      \item Arquitectura
      \item Servicios
      \item Componentes
      \item EJB y RMI
      \item Integración EJB-CORBA
      \item Integración EJB-COM
      \item Implementaciones
   \end{topics}

   \begin{unitgoals}
      \item Aprender a usar el EJB para el desarrollo de aplicaciones distribuidas, transaccionales, portables y seguras, basadas en Java
   \end{unitgoals}
\end{unit}

\begin{unit}{Web Services}{sheldon1995}{3}
   \begin{topics}
      \item Arquitectura
      \item Servicios
      \item XML, UDDI, SOAP
      \item Implementaciones .NET y J2EE
   \end{topics}

   \begin{unitgoals}
      \item Conocer y saber utilizar diferentes tipos de servivios web
   \end{unitgoals}
\end{unit}

\begin{unit}{Comparación entre Modelos de Componentes}{hoque1998}{3}
   \begin{topics}
      \item Comparación CORBA, COM, EJB y Web Services
      \item Integración de las arquitecturas con Web Services (Microsoft .NET, Sun One, IBM)
   \end{topics}

   \begin{unitgoals}
      \item Entender las similitudes y diferencias, ventajas y desventajas de los diferentes modelos de componentes
   \end{unitgoals}
\end{unit}

\begin{coursebibliography}
\bibfile{Computing/CS/CS325}
\end{coursebibliography}
\end{syllabus}
