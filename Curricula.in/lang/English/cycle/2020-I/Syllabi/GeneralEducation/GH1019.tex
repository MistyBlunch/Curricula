\begin{syllabus}

\course{GH0019. Entrepreneurs in Action}{Electivo}{GH0019}
% Source file: ../Curricula.in/lang/English/cycle/2019-I/Syllabi/GeneralEducation/GH1019.tex

\begin{justification}
The purpose of this course is for students to acquire the specific tools and knowledge to perform a market analysis that will be reflected in: (i) a business plan; And (ii) the development of leadership skills, teamwork and effective communication.
This will be achieved by working together with an entrepreneur, bringing the student to the daily problems that arise in the enterprises.
This course is mostly practical, where what is learned in the classroom will be used to analyze the market together with the entrepreneur, following the structure of a business plan. Thus, the student will apply this knowledge and acquired during his career, always guided by the teacher and helpers.
On the one hand, the student will be connected with a real case of entrepreneurship, so that he / she learns by means of the technique "learning by doing". On the other hand, an attempt will be made to reduce the failure rate of entrepreneurs (according to Small Business Administration [http://www.sba.gov]), 95 percent of entrepreneurs fail before the fifth year, mainly due to lack of differentiation With competition and lack of an effective marketing strategy).
The entrepreneurs who will be advised in the Applied Entrepreneurship course belong to Fundación IndependTecnologíazate (www.fundacionindependizate.cl), and they are people with a technical or professional level who know a lot about their product but who have failures in market analysis and strategies Sales and marketing.

\end{justification}

\begin{goals}
\item Analyze the edges that make up a business plan, such as segmentation, marketing strategies and cash flows.
\item Analyze the market and the opportunities that exist to open a new business, where the emphasis will be on identifying these opportunities and the value proposition.
\item Understand the current operation of the business, its weaknesses and strengths, and then make a proposal with value for the entrepreneur.
\item Understand how to get a project forward, freeing it from the "death valley", where entrepreneurs often stay stuck.
\item Apply knowledge already acquired by the student throughout his career through practical work with entrepreneurs, which represents the main axis of this course.
\item Develop leadership in research and development of small business evaluation methodologies.
\end{goals}

\begin{outcomes}{V1}
    \item \ShowOutcome{n}{2}
    \item \ShowOutcome{ñ}{2}
\end{outcomes}

\begin{competences}{V1}
    \item \ShowCompetence{C24}{n,ñ}
\end{competences}

\begin{unit}{Business Models}{}{Kotler08}{12}{4}
   \begin{topics}
      \item .
   \end{topics}

   \begin{learningoutcomes}
      \item Make the student understand what the different ways a business can generate income. Many times entrepreneurs are sure that it is only through a single path, without realizing that they have multiple opportunities.
   \end{learningoutcomes}

\end{unit}

\begin{unit}{Segmenting the Market}{}{Kotler08}{24}{3}
   \begin{topics}
      \item .
   \end{topics}

   \begin{learningoutcomes}
      \item It seeks to deliver tools to the students in order to lead the entrepreneurs to achieve a good market segmentation. Practical tools will be delivered to carry out a market study, and different forms of segmentation will be analyzed.
   \end{learningoutcomes}

\end{unit}

\begin{unit}{Studying the Competition}{}{Kotler08}{24}{3}
   \begin{topics}
      \item .
   \end{topics}

   \begin{learningoutcomes}
      \item That the student can transmit to the entrepreneur the benefits of knowing the competition in depth, and the importance of achieving differentiation.
   \end{learningoutcomes}

\end{unit}

\begin{unit}{Marketing strategies}{}{Wiley07}{30}{3}
   \begin{topics}
      \item . 
   \end{topics}

   \begin{learningoutcomes}
      \item That the student dominates the efficient marketing tactics for entrepreneurs with low budget.
   \end{learningoutcomes}
\end{unit}

\begin{unit}{Sales Strategies}{}{Wiley07}{30}{3}
   \begin{topics}
      \item .
   \end{topics}

   \begin{learningoutcomes}
      \item That the student master of the tools to carry out a sale, deepening in the introduction of products in the points of sale, as well as in the sale of services to third parties.
   \end{learningoutcomes}
\end{unit}

\begin{unit}{Implementation / Operations}{}{Kotler08}{30}{3}
   \begin{topics}
      \item . 
   \end{topics}

   \begin{learningoutcomes}
      \item That the student masters the subjects related to the organization, planning and control management in small companies.
   \end{learningoutcomes}

\end{unit}

\begin{unit}{Financial projections}{}{Wiley07}{30}{3}
   \begin{topics}
      \item That the student can make financial projections, deepening the cash flow.
   \end{topics}

   \begin{learningoutcomes}
      \item .
   \end{learningoutcomes}
\end{unit}

\begin{coursebibliography}
\bibfile{GeneralEducation/GH1019}
\end{coursebibliography}

\end{syllabus}
