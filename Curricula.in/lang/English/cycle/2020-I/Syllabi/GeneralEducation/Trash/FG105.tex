\begin{syllabus}

\course{FG105. Apreciación de la Música}{Electivos}{FG105}

\begin{justification}
El egresado de la Universidad San Pablo, no sólo deberá ser un excelente profesional, conocedor de la más avanzada tecnología, sino también, un ser humano sensible y de amplia cultura. En esta perspectiva, el curso proporciona los instrumentos conceptuales básicos para una óptima comprensión de las obras musicales como producto cultural y artístico creado por el hombre.
La asignatura de Apreciación Musical pertenece al área de Humanidades, es  un curso teórico-práctico y contribuye a la formación integral del estudiante a través del  conocimiento del hecho musical como manifestación cultural e histórica y al afianzamiento de una postura abierta, reflexiva y crítica ante la creación y difusión de la música en nuestra sociedad, mediante la apreciación musical permitiendo así­  complementar  su formación académica. El curso pretende realizar una revisión de los conocimientos teóricos básicos de la música, explorar las diferentes épocas de la historia de la música en el desarrollo de la humanidad y revisar los antecedentes y carácteristicas de la música peruana.
\end{justification}

\begin{goals}
\item Analizar de manera crítica las diferentes manifestaciones artísticas a través de la historia identificando su naturaleza expresiva, compositiva y carácteristicas estéticas así­ como las nuevas tendencias artísticas identificando su relación directa con los actuales indicadores socioculturales. Demostrar conducta sensible, crítica, creativa y asertiva, y conductas valorativas como indicadores de un elevado desarrollo personal.[\Familiarity]
\end{goals}

\begin{outcomes}
    \item \ShowOutcome{ñ}{2}
\end{outcomes}
\begin{competences}
    \item \ShowCompetence{C24}{ñ}
\end{competences}

\begin{unit}{}{Primera Unidad}{Aopland2010,Fubini2004,Hamel1984}{9}{C24}
\begin{topics}
	\item La música en la vida del hombre. 
		\subitem Concepto. 
		\subitem El Sonido: cualidades.
	\item Los elementos de la música. 
\end{topics}
\begin{learningoutcomes}
	\item Dotar al alumno de un lenguaje musical básico, que le permita apreciar y emitir un juicio con propiedad [\Usage].
\end{learningoutcomes}
\end{unit}

\begin{unit}{}{Segunda Unidad}{Fubini2004a,Hamel1984}{15}{C24}
\begin{topics}
	\item La voz, el canto y sus intérpretes. 
	\item Los instrumentos musicales. 
		\subitem El conjunto instrumental.
	\item El estilo, género y las formas musicales. Actividades y audiciones.
\end{topics}
\begin{learningoutcomes}
	\item Reconocer la producción y carácteristicas de la voz humana. [\Usage]
	\item Identificar la producción del sonido y las carácteristicas de los diversos instrumentos musicales.[\Usage]
\end{learningoutcomes}
\end{unit}

\begin{unit}{}{Tercera Unidad}{Fubini2004,Grout2010,Tranchefort2010}{15}{C24}
\begin{topics}
	\item El origen de la música - fuentes. 
		\subitem La música en la antigüedad.
	\item La música medieval: Música religiosa.  
		\subitem Canto Gregoriano. 
		\subitem Música profana.
	\item El Renacimiento.
		\subitem Música instrumental.
		\subitem Música vocal.
	\item El Barroco y sus representantes. 
		\subitem Nuevos instrumentos.
		\subitem Nuevas formas.
	\item El Clasicismo.
		\subitem Las formas clásicas.
		\subitem Más destacados representantes.
	\item El Romanticismo y el Nacionalismo
		\subitem Características generales instrumentos y formas. 
		\subitem Las escuelas nacionalistas europeas.
		\subitem Instrumentos y formas. 
		\subitem Las escuelas nacionalistas europeas.
\end{topics}
\begin{learningoutcomes}
	\item Reconocer  las diferentes épocas de la evolución de la música a través de la historia.[\Usage]
	\item Conocer y valor la obra de los más importantes compositores en las diferentes épocas de la historia de la música.[\Usage]
\end{learningoutcomes}
\end{unit}

\begin{unit}{}{Cuarta Unidad}{Fubini2004,Grout2010,Bellenger2007}{12}{C24}
\begin{topics}
	\item La música contemporánea.
		\subitem Impresionismo.
		\subitem Postromanticismo.
		\subitem Expresionismo.
		\subitem Las nuevas corrientes de vanguardia.
	\item Principales corrientes musicales del Siglo XX.
	\item La música peruana.
		\subitem Autóctona.
		\subitem Mestiza.
		\subitem Manifestaciones musicales actuales.
	\item Música arequipeña.
		\subitem Principales expresiones.
	\item Música latinoamericana y sus principales manifestaciones
\end{topics}
\begin{learningoutcomes}
	\item Identificar las principales carácteristicas de la música peruana como parte importante de la identidad cultural.[\Usage]
	\item Que el alumno conozca e identifique las diferentes manifestaciones populares actuales y se identifique con sus raíces musicales.[\Usage]
\end{learningoutcomes}
\end{unit}

\begin{coursebibliography}
\bibfile{GeneralEducation/FG105}
\end{coursebibliography}

\end{syllabus}
