\begin{syllabus}

\course{FG1020. Bioethics and nature}{Electivo}{FG1020}
% Source file: ../Curricula.in/lang/English/cycle/2021-I/Syllabi/GeneralEducation/FG1020.tex

\begin{justification}
The course will expose students to the main bioethical debates when they facing their professional activity. Special emphasis will be placed on issues related to the transformations of nature, with a view to their sustainability, as well as the limits on experimentation with humans and animals in branches related to chemical research, medicine, bioengineering.
\end{justification}

\begin{goals}
\item Capacidad de interpretar información.
\end{goals}

\begin{outcomes}{V1}
    \item \ShowOutcome{d}{2}
    \item \ShowOutcome{e}{2}
    \item \ShowOutcome{n}{2}
    
\end{outcomes}

\begin{competences}{V1}
    \item \ShowCompetence{C10}{d,n}
    \item \ShowCompetence{C17}{d}
    \item \ShowCompetence{C18}{n}
    \item \ShowCompetence{C21}{e}
\end{competences}

\begin{unit}{Culturas de Gobernanza y Distribución de Poder}{}{Lessig15}{12}{4}
   \begin{topics}
      \item ?`Cómo se relaciona la economía con la política?.
      \item El rol de las Instituciones.
      \item Análisis de casos.
   \end{topics}
   \begin{learningoutcomes}
      \item Desarrollo del innterés por conocer sobre temas actuales en la sociedad peruana y el mundo.
   \end{learningoutcomes}
\end{unit}

\begin{coursebibliography}
\bibfile{GeneralEducation/FG1020}
\end{coursebibliography}

\end{syllabus}
