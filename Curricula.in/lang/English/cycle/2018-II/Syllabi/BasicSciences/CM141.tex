\begin{syllabus}

\curso{CM141. Cálculo Vectorial I}{Obligatorio}{CM141}

\begin{justification}
Comprender los fundamentos del Cálculo Vectorial en el plano y del espacio, y adquirir habilidades que le permitan usar los conceptos estudiados, en el desarrollo de otras asignaturas, así como también en la solución de problemas vinculados a su especialidad.
\end{justification}

\begin{goals}
\item Comprender los principios Cálculo Vectorial.
\item Utilizar los conceptos de la Geometría vectorial espacial.
\item Entender y aplicar el desarrollo de cónicas.
\end{goals}

\begin{outcomes}
\ExpandOutcome{a}
\ExpandOutcome{i}
\ExpandOutcome{j}
\end{outcomes}

\begin{unit}{Vectores en el Plano}{Hasser97}{4}
   \begin{topics}
      \item Sistemas de coordenadas cartesianas; producto cartesiano $RxR$. Sus elementos. Espacio Vectorial Bidimensional. Definición
      \item Representación geométrica de vectores. Paralelismo de vectores. Longitud de un  vector. Paralelismo de Vectores. Producto interno en $R^2$ Propiedades. Ortogonalidad de vectores
      \item Producto Escalar. Propiedades. Ángulo entre vectores Proyección ortogonal. Componentes
      \item Combinación Lineal de Vectores. Independencia lineal de vectores. Bases
   \end{topics}

   \begin{unitgoals}
      \item Describir matemáticamente los vectores en el plano
      \item Conocer las propiedades de los vectores en el plano y aplicarlos en la solución de problemas
   \end{unitgoals}
\end{unit}

\begin{unit}{Geometría Vectorial en el Plano}{Hasser97}{8}
\begin{topics}
	\item El Plano euclideano. Definición. Punto. Recta. Distancia entre dos puntos
	\item La recta. Sus ecuaciones. Posiciones relativas de las rectas. Paralelismo de rectas. Ortogonalidad de rectas. Distancia de un punto a una recta
      \item Intersección de rectas. Ecuaciones Lineales simultáneas. Pendiente de una recta. Ángulo entre rectas. Área del triángulo. Área del polígono
\end{topics}
   \begin{unitgoals}
      \item Describir matemáticamente la Geometría Vectorial en el Plano
      \item Conocer y aplicar conceptos de rectas para resolver problemas
   \end{unitgoals}
\end{unit}

\begin{unit}{Vectores en el Espacio}{Burgos94}{12}
\begin{topics}
      \item Espacio Vectorial Tridimensional. Definición: Igualdad de vectores. Adición de Vectores. Multiplicación de un vector por un número real. Representación Geométrica de los Vectores. Paralelismo de vectores
      \item Longitud de un vector. Propiedades. Vectores unitarios. Producto Escalar. Propiedades. Ortogonalidad de vectores. Ángulo entre vectores. Proyección ortogonal
      \item Combinación lineal de vectores. Independencia lineal de vectores. Bases
      \item Producto Vectorial. Definición. Significado geométrico. Triple producto escalar. Significado geométrico. Caracterización de la independencia lineal de tres vectores con el triple producto escalar
\end{topics}

   \begin{unitgoals}
      \item Describir matemáticamente los vectores en el espacio
      \item Conocer las propiedades de los vectores en el espacio y aplicarlos en la solución de problemas
   \end{unitgoals}
\end{unit}

\begin{unit}{Geometría Vectorial Espacial}{Venero94, Edwards96}{12}
\begin{topics}
      \item Espacio Euclideano tridimensional. Definición. Punto, recta, plano. Distancia entre dos puntos
      \item La Recta. Sus Ecuaciones. Posición relativa de rectas; paralelismo de rectas y Ángulo entre rectas. Distancia de un punto a una recta. Distancia entre rectas. Casos
      \item El Plano. Sus ecuaciones. Posiciones relativas de planos. Paralelismo y Ángulo entre planos. Intersección de rectas y planos. Distancia de un punto a un plano. Caracterización de la independencia lineal de  vectores con la intersección de planos. Área del paralelogramo. Volumen del paralelepípedo y del tetraedro, etc.
	\end{topics}

   \begin{unitgoals}
      \item Describir matemáticamente la Geometría Vectorial Espacial
      \item Conocer y aplicar conceptos de planos y rectas para resolver problemas
   \end{unitgoals}
\end{unit}

\begin{unit}{Cónicas: (en forma vectorial)}{Granero, Edwards96}{12}
\begin{topics}
      	\item Coordenadas Homogéneas o Absolutas en el plano
	\item Ecuación de la Recta en coordenada homogéneas
	\item Definición de Cónica y su Ecuación general interpretación geométrica
	\item Polar de un punto y polo de una  rectas
	\item Intersección de una cónica con una recta
	\item Puntos singulares de una cónica: Cónicas degeneradas
	\item Composición de las cónicas degeneradas. (|A| = o)
	\item Clasificación de las cónicas mediante sus intersección con la recta $X_3 = 0$
	\item Cónicas Imaginarias
	\item Clasificación General de las cónicas
	\item Rectas Tangentes a una cónica: Asíntotas
	\item Elementos principales de las cónicas no degeneradas
	\item Focos y Directrices
	\item Reducción de la Ecuación General de las cónicas no degeneradas a formas cónicas
	\item Obtención de los Coeficientes de la forma canónica para la Elipse, hipérbolas y parábolas
	\item Determinación analítica de las cónicas
	\item Haces lineales de cónicas
   \end{topics}

   \begin{unitgoals}
      \item Describir matemáticamente las cónicas
      \item Conocer y aplicar conceptos de cónicas en la solución de problemas
   \end{unitgoals}
\end{unit}

\begin{coursebibliography}
\bibfile{BasicSciences/CM141}
\end{coursebibliography}
\end{syllabus}
