\begin{syllabus}

\course{CB102. Análisis Matemático I}{Obligatorio}{CB102}

\begin{justification}
Un aspecto muy importante en el nivel universitario lo constituye el cálculo diferencial,  aspecto que constituye la piedra angular de las posteriores asignaturas de matemáticas así como de la utilidad de la matemática en la solución de problemas aplicados a la ciencia y la tecnología. Cualquier profesional con rango universitario debe por lo tanto tener conocimiento amplio de esta asignatura, pues se convertirá en su punto de partida para los intereses de su desarrollo profesional; así también será soporte para no tener dificultades en las asignaturas de matemática y física de toda la carrera.
\end{justification}

\begin{goals}
\item Asimilar y manejar los conceptos de función, sucesión y relacionarlos con los de límites y continuidad.
\item Describir, analizar, diseñar y formular modelos continuos que dependan de una variable.
\item Conocer y manejar la propiedades del cálculo diferencial y aplicarlas a la resolución de problemas.
\end{goals}

\begin{outcomes}
\ExpandOutcome{a}{3}
\ExpandOutcome{i}{3}
\ExpandOutcome{j}{4}
\end{outcomes}

\begin{unit}{Números reales y funciones}{Simmons95,Bartle99}{20}{3}
   \begin{topics}
      \item Números reales
      \item Funciones de variable real
   \end{topics}

   \begin{unitgoals}
      \item Comprender la importancia del sistema de los números reales (construcción), manipular los axiomas algebraicos y de orden.
      \item Comprender el concepto de función. Manejar dominios, operaciones, gráficas, inversas.
      \end{unitgoals}
\end{unit}

\begin{unit}{Sucesiones numéricas de números reales}{Avila93,Bartle99}{18}{3}
   \begin{topics}
      \item Sucesiones
      \item Covergencia
      \item Límites. Operaciones con sucesiones
   \end{topics}

   \begin{unitgoals}
      \item Entender el concepto de sucesión y su importancia.
      \item Conecer los principales tipos de sucesiones, manejar sus propiedades
      \item Manejar y calcular límites de sucesiones
      \end{unitgoals}
\end{unit}

\begin{unit}{Límites de funciones y continuidad}{Apostol97,Avila93,Simmons95}{14}{4}
   \begin{topics}
      \item Límites
      \item Continuidad
      \item Aplicaciones de funciones continuas. Teorema del valor intermedio
   \end{topics}

   \begin{unitgoals}
      \item Comprender el concepto de límite. calcular límites
      \item Analizar la continuidad de una función
      \item Aplicar el teorema del valor intermedio
      \end{unitgoals}
\end{unit}

\begin{unit}{Diferenciación}{Apostol97,Bartle99,Simmons95}{18}{4}
   \begin{topics}
      \item Definición. reglas de derivación
      \item Incrementos y diferenciales
      \item Regla de la cadena. Derivación implícita
   \end{topics}

   \begin{unitgoals}
      \item Comprender el concepto de derivada e interpretarlo.
      \item Manipular las reglas de derivación
      \end{unitgoals}
\end{unit}

\begin{unit}{Aplicaciones}{Simmons95,Apostol97}{20}{4}
   \begin{topics}
      \item Funciones crecientes, decrecientes
      \item Extremos de funciones
      \item Razón de cambio
      \item Límites infinitos
      \item Teorema de Taylor
   \end{topics}

   \begin{unitgoals}
      \item Utilizar la derivada para hallar extremos de funciones
      \item Resolver problemas aplicativos
      \item Utilizar el Teorema de Taylor
      \end{unitgoals}
\end{unit}



\begin{coursebibliography}
\bibfile{BasicSciences/CB102}
\end{coursebibliography}

\end{syllabus}
