\begin{syllabus}

\curso{CF141. Física General I}{Obligatorio}{CF141}

\begin{justification}
This course is useful in this career so that the student learns to show a high degree of mastery of the laws of the movement of General Physics.
\end{justification}

\begin{goals}
\item Train and present to the student the basic principles of Physics as a natural science encompassing its most important topics and their relation  with everyday problems.
\end{goals}

\begin{outcomes}
  \item \ShowOutcome{a}{2}
  \item \ShowOutcome{i}{2}
  \item \ShowOutcome{j}{2}
\end{outcomes}

\begin{competences}
    \item \ShowCompetence{C1}{a}
    \item \ShowCompetence{C20}{i,j}
\end{competences}

\begin{unit}{FI1. Introducción}{}{Serway2002,Alonso95}{4}{C1,C20}
\begin{topics}
      \item Scientific research. The cientific method.
      \item Concept of Chemistry. Chemistry today
      \item Matter.Classification and physical, chemical, intensive and extensive properties. 
      \item Idealized model.
      \item Physical magnitudes.
      \item Properties of vectors
      \item Components of a vector and unit vectors.
      \item Vector product.
      \item Exercises and problems.
   \end{topics}

   \begin{learningoutcomes}
      \item Understand and work with the physical magnitudes of the SI.
      \item Abstracting the rigorous physical concepts of nature and represent them in vector models.
      \item Understand and apply vector concepts to real physical problems.
   \end{learningoutcomes}
\end{unit}

\begin{unit}{FI2. Movimiento de partículas en una dimensión}{}{Serway2002,Alonso95}{2}{C1,C20}
\begin{topics}
      \item Displacement,Velocity,Speed.
      \item Instant velocity
      \item Medium and Instant Acceleration.
      \item Movement with constant acceleration
      \item Free fall of bodies
      \item Exercises and problems.
    \end{topics}
   \begin{learningoutcomes}
      \item Describe mathematically the mechanical motion of a one-dimensional particle as a body of negligible dimensions
      \item Know and apply concepts of kinematic magnitudes.
      \item Describe the particle motion behavior, theoretically and graphically
      \item Knowing one-dimensional vector representations of these movements.
      \item Solve problems.
   \end{learningoutcomes}
\end{unit}

\begin{unit}{FI3. Movimiento de partículas en dos y tres dimensiones}{}{Serway2002,Alonso95}{4}{C1,C20}
\begin{topics}
      \item Displacement and Velocity.
      \item The vector acceleration
      \item Parabolic movement.
      \item Circular movement
      \item Tangential and radial acceleration components.
      \item Exercises and problems.
\end{topics}

   \begin{learningoutcomes}
      \item Describe mathematically the mechanical motion of a particle in two and three dimensions as a body of negligible dimensions.
      \item Know and apply concepts of vector kinematic quantities in two and three dimensions.
      \item Describe the behavior of particle motion theoretically and graphically in two and three dimensions
      \item Know and apply circular movement concepts.
      \item Solve problems.
   \end{learningoutcomes}
\end{unit}

\begin{unit}{FI4. Leyes del movimiento}{}{Serway2002,Alonso95}{6}{C1,C20}
\begin{topics}
      \item Force and interactions.
      \item Newton's First Law
      \item Inertial mass.
      \item Newton's Second Law
      \item Weight.
      \item Free Body Diagrams.
      \item Newton's Third Law
      \item Friction forces.
      \item Dynamics of circular motion
      \item Exercises and problems.
   \end{topics}

   \begin{learningoutcomes}
      \item Know the concepts of force.
      \item know the most important interactions of nature and to represent them in a free-body diagram
      \item Know the concepts of static equilibrium.
      \item know and apply the laws of motion and to characterize them vectorially.
      \item Know and apply Newton's laws.
      \item Solve problems.
   \end{learningoutcomes}
\end{unit}

\begin{unit}{FI5. Trabajo y Energía}{}{Serway2002,Alonso95}{4}{C1,C20}
\begin{topics}
	\item Trabajo realizado por una fuerza constante.
	\item Trabajo realizado por fuerzas variables.
	\item Work and kinetic energy.
	\item Potency
	\item Gravitational potential energy
	\item Elastic potential energy
	\item Conservative and non-conservative forces
	\item Principles of energy conservation
	\item Exercises and problems
\end{topics}

   \begin{learningoutcomes}
      \item Establish the concepts of physical energy. (Classical Physics)
      \item Know some forms of energy.
      \item Establish the relation between work and energy
      \item Know and apply the concepts of energy conservation
      \item Solve problems.
   \end{learningoutcomes}
\end{unit}

\begin{unit}{FI6. Momento lineal}{}{Serway2002,Alonso95}{3}{C1,C20}
\begin{topics}
      \item Linear momentum.
      \item Conservation of linear momentum
      \item Mass and gravity center
      \item Movement of a particle system
      \item Exercises and problems.
  \end{topics}

   \begin{learningoutcomes}
      \item Establish the concepts of linear momentum.
      \item Know the concepts of conservation of linear momentum
      \item Know the movement of a system of particles
      \item Solve problems.
   \end{learningoutcomes}
\end{unit}

\begin{unit}{FI7. Rotación de cuerpos rígidos}{}{Serway2002,Alonso95}{4}{C1,C20}
\begin{topics}
      \item Velocity and angular accelerations.
      \item Rotation with constant angular acceleration.
      \item Relation between linear and angular kinematics
      \item Energía en el movimiento de rotación.
      \item Torsional moment.
      \item Relationship between torsional moment and angular acceleration.
      \item Exercises and problems.
   \end{topics}

   \begin{learningoutcomes}
      \item Know the basic concepts of rigid body.
      \item Know and apply concepts of rigid body rotation.
      \item Know torsion concepts.
      \item Apply energy concepts to the rotating motion.
      \item Solve problems.
   \end{learningoutcomes}
\end{unit}

\begin{unit}{FI8. Dinámica del movimiento de rotación}{}{Serway2002,Alonso95}{3}{C1,C20}
\begin{topics}
      \item Moment of torsion and angular acceleration of a rigid body
      \item Rotation of a rigid body on a movable axis.
      \item Work and potency in the rotation movement
      \item Angular momentum.
      \item Conservation of angular momentum
      \item Exercises and problems.
    \end{topics}

   \begin{learningoutcomes}
      \item Understand basic concepts of rotation dynamics.
      \item Know and apply torsion concepts.
      \item Understanding angular momentum and its conservation
      \item Solve problems.
   \end{learningoutcomes}
\end{unit}

\begin{coursebibliography}
\bibfile{BasicSciences/CF141}
\end{coursebibliography}
\end{syllabus}
