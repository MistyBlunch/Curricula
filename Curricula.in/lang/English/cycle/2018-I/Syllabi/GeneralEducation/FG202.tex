\begin{syllabus}

\course{FG202. Apreciación Literaria}{Electivos}{FG202}

\begin{justification}
Siendo la literatura una actividad artTecnologíastica que tiene por objeto la expresión de ideas y sentimientos por medio de la palabra, esta constituye, la reconstrucción de experiencias de la realidad en diversos órdenes, gracias a la expresión personal y emotiva del escritor y el momento social que le tocó vivir.
Partiendo de este enunciado las grandes obras literarias, son poderosos agentes de cultura. De ahTecnología que, la literatura cumple un rol integrador en la formación cultural del ser humano; pero,  para lograr este objetivo en su verdadera dimensión, hay que saber apreciar la belleza de la expresión literaria con un sentido analTecnologíatico, crTecnologíatico y valorativo
El curso de Apreciación Literaria corresponde a los cursos del Área de Formación General y es considerado como un curso electivo que tiene el valor de dos créditos. Es de carácter teórico-práctico, ya que los alumnos reciben información teórica sobre el análisis de textos literarios y sobre los diferentes movimientos literarios que se han dado a través del tiempo; dicha información, los alumnos la ponen en práctica al analizar fragmentos y obras literarias. La metodologTecnologíaa consiste en trabajos individuales y grupales de análisis de textos, los cuales serán expuestos por los alumnos a sus compañeros, al mismo tiempo, responden a una serie de interrogantes referentes a los temas tratados. El propósito fundamental es, sensibilizar a los estudiantes en la percepción de la belleza escrita que se expresa a través de las distintas obras literarias; además, busca desarrollar en los alumnos la capacidad crTecnologíatica y valorativa que le ayudará en su formación personal y cultural. AsTecnología mismo, este curso permite que los estudiantes desarrollen destrezas comunicativas a nivel verbal y escrito.
La temática abarca los siguientes aspectos: análisis de textos, teorTecnologíaa de los géneros literarios, lenguaje literario y figurado, los movimientos literarios como: la antigüedad clásica, edad media, humanismo y renacimiento, neoclasicismo, romanticismo, realismo, naturalismo y la literatura contemporánea.
\end{justification}

\begin{goals}
\item Desarrollar su capacidad crTecnologíatica, creativa y valorativa, a través de la lectura, análisis e interpretación de textos literarios con el fin de estimular y formar su sensibilidad estética y reforzar el hábito lector.
\item Adquirir destreza en la técnica del Comentario de Textos y en la utilización de los mismos.
\item Promover el desarrollo de destrezas comunicativas a nivel verbal ( escrito y hablado).
\end{goals}

\begin{outcomes}
    \item \ShowOutcome{f}{2}
    \item \ShowOutcome{ñ}{2}
\end{outcomes}
\begin{competences}
    \item \ShowCompetence{C24}{ñ, f}
\end{competences}

\begin{unit}{}{Primera Unidad}{Cáceres, Bello}{9}{C24}
\begin{topics}
	\item Textos literarios y no literarios.- Conceptos y caracterTecnologíasticas.
	\item El comentario de textos. Ficha de análisis literario.
	\item Corrientes o movimientos literarios de la literatura universal y peruana a través de la historia.- Panorama general. CaracterTecnologíasticas y diferencias.
	3.1.Vigencia de la teorTecnologíaa de los géneros: lTecnologíarico, épico, dramático, narrativo y didáctico.
	\item El lenguaje figurado: figuras literarias.-Análisis y reconocimiento.
\end{topics}
\begin{learningoutcomes}
	\item Distinguir textos literarios de otros tipos de textos. [\Usage].
	\item Comentar textos literarios y desarrollar adecuadamente la ficha de análisis literario. [\Usage].
	\item Diferenciar los distintos tipos de expresión literaria a través de la evolución histórica de la misma, valorándolas en su verdadera dimensión. [\Usage].
	\item Aplicar las figuras literarias en textos tanto en prosa como en verso. [\Usage].
\end{learningoutcomes}
\end{unit}

\begin{unit}{}{Segunda Unidad}{Torres, Homero, Sanzos, Alighieri}{15}{C24}
\begin{topics}
	\item Homero ``La Iliada''
	\item Sófocles ``Edipo Rey''
	\item Virgilio ``La Eneida''
	\item Literatura Cristiana ``La Biblia''
\end{topics}
\begin{learningoutcomes}
	\item Perspectiva CrTecnologíatica- Literaria (Con lecturas de fragmentos de obras representativas)
	\item Literatura de la Antigüedad Clásica: Perspectiva crTecnologíatica: CaracterTecnologíasticas, Representantes, Análisis del fragmento: Proverbios, La Iliada [\Usage].
	\item Literatura de la Edad Media: Perspectiva crTecnologíatica: CaracterTecnologíasticas, Representantes, Análisis de texto: El Quijote de la Mancha [\Usage].
	\item Literatura del Humanismo y Renacimiento: Perspectiva crTecnologíatica: CaracterTecnologíasticas, Representantes, Análisis del fragmento: La Divina Comedia.[\Usage].
\end{learningoutcomes}
\end{unit}

\begin{unit}{}{Tercera Unidad}{Hugo, Hemingway, Goethe}{18}{C24}
\begin{topics}
	\item Literatura del Neoclasicismo y Romanticismo: Perspectiva crTecnologíatica: CaracterTecnologíasticas, Representantes, Análisis del fragmento.
	\item Literatura del Realismo y Naturalismo: Perspectiva crTecnologíatica: CaracterTecnologíasticas, Representantes, Análisis del fragmento.
	\item Literatura Contemporánea: Perspectiva crTecnologíatica: CaracterTecnologíasticas, Representantes, Análisis del fragmento.
	\item Análisis, interpretación, valoración y comentario de una completa de la literatura: Exposición individual y/o grupal de una obra completa de la literatura universal, Exposición individual y/o grupal de una obra  completa de la literatura peruana.
	\item Redacción de ensayo sobre obras y/o fragmentos leTecnologíados o expuestos.
\end{topics}
\begin{learningoutcomes}
	\item Descubrir en las obras literarias los valores humanos más importantes reconociéndolos en su verdadera dimensión. [\Usage].
	\item Comprender y textos literarios y reflexionar sobre el contenido. [\Usage].
	\item Valorar sus propias cualidades en relación a la literatura [\Usage].
	\item Exponer y comentar adecuadamente  obras clásicas del canon literario- universal y peruano. [\Usage].
	\item Redactar textos argumentativos (ensayo) sobre una obra de la literatura universal o peruana leTecnologíada. [\Usage].
\end{learningoutcomes}
\end{unit}



\begin{coursebibliography}
\bibfile{GeneralEducation/FG101}
\end{coursebibliography}
\end{syllabus}
