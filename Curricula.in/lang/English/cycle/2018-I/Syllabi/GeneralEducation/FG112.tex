\begin{syllabus}

\course{FG112. Persona, Matrimonio y Familia}{Electivos}{FG112}

\begin{justification}
Los tiempos actuales muestran la necesidad - cada vez más apremiante-  de una adecuada visión antropológica sobre el matrimonio y la familia.

La referencia de la familia como institución natural fundada en el matrimonio, viene en diversas organizaciones internacionales promovida como una construcción social y cultural que tiende a desconocer la complementariedad del varón y la mujer.

Este curso intentará mostrar los presupuestos de una perspectiva de familia que destaque la riqueza de la familia como auténtico eje de desarrollo humano.
\end{justification}

\begin{goals}
	\item Comprender que la familia es una comunión de vida y amor, fundado en el matrimonio entre un hombre y una mujer, para toda la vida en orden al perfeccionamiento mutuo y a la procreación y educación de los hijos.
	\item Que el alumno entienda los criterios fundamentales sobre los que descansa una recta comprensión de la persona, el matrimonio y la familia
	\item Que el alumno tenga elementos para comprender la vida afectiva como un llamado a la vida matrimonial y familiar
	\item Comprender la importancia de la familia para la persona y para la sociedad entera.
\end{goals}

\begin{outcomes}
    \item \ShowOutcome{g}{3}
    \item \ShowOutcome{ñ}{2}
    \item \ShowOutcome{o}{2}
\end{outcomes}

\begin{competences}
    \item \ShowCompetence{C10}{g, ñ, o}
    \item \ShowCompetence{C20}{g}
\end{competences}

\begin{unit}{}{Primera Unidad: Consideraciones sobre la persona humana}{JuanPabloII}{12}{C10, C20}
\begin{topics}
	\item Persona y ser humano (el problema del reduccionismo antropológico)
	\item La persona humana: Unidad. Niveles de acción. Las emociones. Integración.
	\item La dignidad humana
\end{topics}

\begin{learningoutcomes}
	\item Comprender los fundamentos que permitan conocer a la persona valorando su dignidad.
\end{learningoutcomes}
\end{unit}

\begin{unit}{}{Segunda Unidad: Matrimonio: Aportes para una reflexión actual}{Concilio,Pontificio,SantaSede}{18}{C20}
\begin{topics}
	\item El enamoramiento.
    \item El matrimonio	
\end{topics}
\begin{learningoutcomes}
	\item Comprender que el ser humano ha sido creado por amor y para el amor, que lo direcciona hacia una unión de las naturalezas (complementariedad) y como vocación al matrimonio.
\end{learningoutcomes}
\end{unit}

\begin{unit}{}{Tercera Unidad}{Biblia}{21}{C10, C20}
\begin{topics}
	\item Ideología de Género.
	\item Divorcio.
	\item Convivencia y relaciones libres.
	\item Homosexualidad.
	\item Anticoncepción y mito poblacional.
	\item El Futuro de la humanidad.
\end{topics}
\begin{learningoutcomes}
	\item Comprender la importancia de la familia como célula fundamental de la sociedad y corazón de la civilización.
\end{learningoutcomes}
\end{unit}

\begin{unit}{}{Cuarta Unidad: Protección jurídica y políticas familiares}{Biblia}{3}{C20}
\begin{topics}
	\item Ideología de Género.
	\item Divorcio.
	\item Convivencia y relaciones libres.
	\item Homosexualidad.
	\item Anticoncepción y mito poblacional.
	\item El Futuro de la humanidad.
\end{topics}
\begin{learningoutcomes}
	\item Comprender la importancia de la familia como célula fundamental de la sociedad y corazón de la civilización.
    \item Identificar los organismos e instituciones que velan, protegen y promueven la familia como institución de derecho natural.
\end{learningoutcomes}
\end{unit}



\begin{coursebibliography}
\bibfile{GeneralEducation/FG112}
\end{coursebibliography}

\end{syllabus}
