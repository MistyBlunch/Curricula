\begin{syllabus}

\course{CS1D1. Estructuras Discretas I}{Obligatorio}{CS1D1}

\begin{justification}


Discrete structures provide the theoretical foundations necessary for computation. These fundamentals are not only useful to develop computation from a theoretical point of view as it happens 
in the course of computational theory, but also is useful for the practice of computing; In particular in applications such as verification,
cryptography, formal methods, etc.

\end{justification}

\begin{goals}
\item Apply Properly concepts of finite mathematics (sets, relations, functions) to represent data of real problems.
\item Model real situations described in natural language, using propositional logic and predicate logic.
\item Determine the abstract properties of binary relations.
\item Choose the most appropriate demonstration method to determine the veracity of a proposal and construct correct mathematical arguments.
\item Interpret mathematical solutions to a problem and determine their reliability, advantages and disadvantages.
\item Express the operation of a simple electronic circuit using Boolean algebra.
\end{goals}

\begin{outcomes}
    \item \ShowOutcome{a}{2}
    \item \ShowOutcome{i}{3}
    \item \ShowOutcome{j}{2}
\end{outcomes}

\begin{competences}
    \item \ShowCompetence{C1}{a}
    \item \ShowCompetence{C20}{i,j}
\end{competences}

 \begin{unit}{\DSSetsRelationsandFunctions}{}{Grimaldi03,Rosen2007}{13}{C1,C20}
   \begin{topics}
        \item \DSSetsRelationsandFunctionsTopicSets
        \item \DSSetsRelationsandFunctionsTopicRelations
        \item \DSSetsRelationsandFunctionsTopicFunctions
   \end{topics}
   \begin{learningoutcomes}
	\item \DSSetsRelationsandFunctionsLOExplainWith [\Assessment]
	\item \DSSetsRelationsandFunctionsLOPerformThe [\Assessment]
	\item \DSSetsRelationsandFunctionsLORelate [\Assessment]
   \end{learningoutcomes}
 \end{unit}

 \begin{unit}{\DSBasicLogic}{}{Rosen2007,Grimaldi03}{14}{C1,C20}
   \begin{topics}
        \item \DSBasicLogicTopicPropositional%
        \item \DSBasicLogicTopicLogical%
        \item \DSBasicLogicTopicTruth%
        \item \DSBasicLogicTopicNormal%
        \item \DSBasicLogicTopicValidity%
        \item \DSBasicLogicTopicPropositionalInference%
        \item \DSBasicLogicTopicPredicate%
        \item \DSBasicLogicTopicLimitations%
   \end{topics}
   \begin{learningoutcomes}
	\item \DSBasicLogicLOConvertLogical [\Usage ]
	\item \DSBasicLogicLOApplyFormal [\Usage ]
	\item \DSBasicLogicLOUseThe [\Usage]
	\item \DSBasicLogicLODescribeHowCan [\Familiarity]
	\item \DSBasicLogicLOApplyFormalAnd [\Usage ]
	\item \DSBasicLogicLODescribeTheLimitationsAnd [\Usage]
   \end{learningoutcomes}
 \end{unit}

\begin{unit}{\DSProofTechniques}{}{Rosen2007, Epp10, Scheinerman12}{14}{C1,C20}
\begin{topics}
        \item \DSProofTechniquesTopicNotions%
        \item \DSProofTechniquesTopicThe%
        \item \DSProofTechniquesTopicDirect%
        \item \DSProofTechniquesTopicDisproving%
        \item \DSProofTechniquesTopicProof%s
        \item \DSProofTechniquesTopicInduction%
        \item \DSProofTechniquesTopicStructural%
        \item \DSProofTechniquesTopicWeak%
        \item \DSProofTechniquesTopicRecursive%
        \item \DSProofTechniquesTopicWell%
\end{topics}

\begin{learningoutcomes}
    %% itemizar cada learning outcomes [nivel segun el curso]
	\item \DSProofTechniquesLOIdentifyTheUsed [\Assessment]
	\item \DSProofTechniquesLOOutline [\Usage ]
	\item \DSProofTechniquesLOApplyEach [\Usage ]
	\item \DSProofTechniquesLODetermineWhich [\Assessment]
	\item \DSProofTechniquesLOExplainTheIdeas [\Familiarity ]
	\item \DSProofTechniquesLOExplainTheWeak [\Assessment]
	\item \DSProofTechniquesLOStateThe [\Familiarity]
\end{learningoutcomes}
\end{unit}

\begin{unit}{Digital Logic and Data Representation}{}{Rosen2007,Grimaldi03}{19}{C1,C20}
   \begin{topics}
	\item Partial orders and Partially ordered sets.   
 	\item Extreme elements of a partially ordered set.
	\item Reticles: Types and properties.
	\item Boolean algebras.
	\item Boolean Functions and Expressions.
	\item Representation of Boolean Functions: Normal Disjunctive and Conjunctive Form.
	\item Logical Doors.
	\item Circuit Minimization.
   \end{topics}

   \begin{learningoutcomes}
	\item Explain the importance of Boolean algebra as a unification of set theory and propositional logic [\Assessment].
	\item Know the algebraic structures of reticulum and its types [\Assessment].
	\item Explain the relationship between the reticulum and the ordinate set and the wise use to show that a set is a reticulum [\Assessment].
	\item Know the properties that satisfies a Boolean algebra [\Assessment].
	\item Demonstrate if a terna formed by a set and two internal operations is or not Boolean algebra [\Assessment].
	\item Find the canonical forms of a Boolean function  [\Assessment].
	\item Represent a Boolean function as a Boolean circuit using logic gates  [\Assessment].
	\item Minimize a Boolean function. [\Assessment].
    \end{learningoutcomes}
 \end{unit}



\begin{coursebibliography}
\bibfile{Computing/CS/CS1D1}
\end{coursebibliography}

\end{syllabus}

%\end{document}
