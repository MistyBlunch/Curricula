\begin{syllabus}

\course{CS314. Algoritmos Paralelos}{Obligatorio}{CS314}

\begin{justification}
Las arquitecturas de computadores están tendiendo a incluir cada vez más núcleos 
y/o procesadores por máquina como método de incrementar la capacidad computacional
de cada unidad. La posibilidad de realizar múltiples tareas simultaneamente mediante hardware 
no es inmediatamente traducida al software, pues las aplicaciones deben 
ser diseñadas para aprovechar estas nuevas capacidades, mediante el uso de hebras y/o procesos.
\end{justification}

\begin{goals}
\item Que el alumno sea capaz de crear aplicaciones paralelas de mediana complejidad aprovechando eficientemente máquinas con múltiples núcleos.
\item Que el alumno sea capaz de comparar aplicaciones secuenciales y paralelas.
\item Que el alumno sea capaz de convertir, cuando la situación lo amerite, aplicaciones secuenciales a paralelas de forma eficiente.
\end{goals}

\begin{outcomes}
\ExpandOutcome{a}{3}
\ExpandOutcome{b}{4}
\ExpandOutcome{c}{4}
\ExpandOutcome{g}{3}
\ExpandOutcome{h}{3}
\ExpandOutcome{i}{3}
\ExpandOutcome{j}{4}
\end{outcomes}

\begin{unit}{\CNParallelComputationDef}{progpara}{5}{4}
      \CNParallelComputationAllTopics %% Esto esta cubriendo todos los topicos.
      \CNParallelComputationAllObjectives
\end{unit}

\begin{unit}{\ARMultiprocessingDef}{progpara}{5}{3}
      \ARMultiprocessingAllTopics
      \ARMultiprocessingAllObjectives
\end{unit}

\begin{unit}{\ALParallelAlgorithmsDef}{progpara}{3}{4}
      \ALParallelAlgorithmsAllTopics %% Esta parte en realidad esta embebida en la primera.
      \ALParallelAlgorithmsAllObjectives
\end{unit}

\begin{unit}{Modelos de Threads con PTHREADs}{pthread,progpara}{0}{3}
\begin{topics}
         \item ?`Qué es una hebra?
         \item ?`Qué es  pthread?
         \item Diseñando programas con pthreads.
         \item Creacuón y manejo de hebras.
         \item Sincronización de hebras con mutex.
\end{topics}

\begin{unitgoals}
	\item Entender los distintos modelos de programación paralela.
	\item Conocer ventajas y desventajas de los distintos modelos de programación paralela.
\end{unitgoals}
\end{unit}

\begin{unit}{Modelos de Threads con OpenMP}{openmp,progpara}{0}{3}
\begin{topics}
         \item ?`Qué es OpenMP?
         \item El modelo de programación OpenMP.
         \item Directivas de OpenMP.
         \item Constructores de trabajo compartido.
         \item Constructores de Tareas.
         \item Constructores de sincronización.
	 \item Manejo de datos privados y compartidos.
\end{topics}

\begin{unitgoals}
	\item Implementar programas multihebras por medio de OpenMP.
	\item Entender y aplicar conceptos de sincronización y trabajo compartido.
\end{unitgoals}
\end{unit}

\begin{unit}{Modelo de programación mediante paso de Mensajes con MPI}{mpi,progpara}{0}{3}
\begin{topics}
         \item ?`Qué es MPI?
         \item Rutinas de administración de ambiente.
         \item Rutinas de comunicación punto a punto.
         \item Rutinas de comunicación colectiva.
         \item Tipos de datos derivados.
         \item Rutinas de administración del comunicador y de grupo.
	 \item Topología virtual.
\end{topics}

\begin{unitgoals}
	\item Implementar programas multihebras por medio de OpenMP.
	\item Entender y aplicar conceptos de sincronización y trabajo compartido.
\end{unitgoals}
\end{unit}

\begin{unit}{{\it Threading Building Blocks (TBB)}}{tbb,progpara}{0}{3}
\begin{topics}
      \item Bucles Simples Paralelos.
      \item Bucles Complejos Paralelos.
      \item Cancelación y Exepciones.
      \item Contenedores paralelos. 
\end{topics}

\begin{unitgoals}
	\item Entender y aplicar el modelo de datos paralelos utilizando la herramienta TBB.
\end{unitgoals}
\end{unit}



\begin{coursebibliography}
\bibfile{Computing/CS/CS314}
\end{coursebibliography}

\end{syllabus}
