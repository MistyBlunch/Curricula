\begin{syllabus}

\course{FG106. Arte y Tecnología}{Obligatorio}{FG106} % Common.pm

\begin{justification}
Favorece al estudiante a identificarse a la ``Comunidad Académica'' de la Universidad, en la medida en que le brinda canales naturales de integración a su grupo y  a su Centro de Estudios y le permite,  desde una visión alternativa, visualizar la valTecnologíaa interior de las personas a su alrededor, a la vez que puede conocer mejor la suya propia.
Relaciona al universitario, a través de la experimentación, con un nuevo lenguaje, un medio de comunicación y expresión que va más allá de la expresión verbal conceptualizada.
Coadyuva al estudiante en su formación integral, desarrollando en él  capacidades corporales. Estimula en él, actitudes anTecnologíamicas positivas,  aptitudes cognitivas y afectivas. Enriquece su sensibilidad y despierta su solidaridad.
Desinhibe y socializa, relaja y alegra,  abriendo un camino de apertura de conocimiento del propio ser y el ser de los demás.

\end{justification}

\begin{goals}
\item Contribuir a la formación personal y profesional del estudiante, reconociendo, valorando y desarrollando su lenguaje corporal, integrándolo a su grupo, afianzando su seguridad personal, enriqueciendo su intuición, su imaginación y creatividad, motivándolo  a abrir caminos de búsqueda  de conocimiento de sTecnología mismo y de comunicación con los demás a través de su sensibilidad, de ejercicios de introspección y de nuevas vTecnologíaas de expresión.
\end{goals}

\begin{outcomes}{V1}
    \item \ShowOutcome{f}{2}
    \item \ShowOutcome{ñ}{2}
\end{outcomes}

\begin{competences}{V1}
    \item \ShowCompetence{C17}{f}
    \item \ShowCompetence{C18}{ñ}
    \item \ShowCompetence{C24}{ñ}
\end{competences}

\begin{unit}{}{El Arte, la Creatividad y el Teatro}{Majorana,PAVIS}{6}{C18,C24}
\begin{topics}
	\item ?`Qué es el Arte? Una experiencia vivencial y personal.
	\item La llave maestra: la creatividad.
	\item La importancia del teatro en la formación personal y profesional.
	\item Utilidad y enfoque del arte teatral.
\end{topics}
\begin{learningoutcomes}
	\item Reconocer la vigencia del Arte y la creatividad en el desarrollo personal y social [\Usage].
	\item Relacionar al estudiante con su grupo valorando la importancia de la comunicación humana y del colectivo  social [\Usage].
	\item Reconocer nociones  básicas del teatro [\Usage].
\end{learningoutcomes}
\end{unit}

\begin{unit}{}{El Juego: el  quehacer del actor}{Majorana,PAVIS}{6}{C17,C24}
\begin{topics}
	\item Juego, luego existo.
	\item El juego del niño y el juego dramático.
	\item Juegos de integración grupal y juegos de creatividad.
	\item La secuencia teatral.
\end{topics}
\begin{learningoutcomes}
	\item Reconocer el juego como herramienta fundamental del teatro [\Usage].
	\item Interiorizar y revalorar el juego como aprendizaje creativo [\Usage].
	\item Acercar al estudiante de manera espontánea y natural, a la vivencia teatral [\Usage].
\end{learningoutcomes}
\end{unit}

\begin{unit}{}{La expresión corporal y el uso dramático del Objeto}{Majorana,PAVIS}{9}{C17, C18, C24}
\begin{topics}
	\item Toma de conciencia del cuerpo.
	\item Toma de conciencia del espacio
	\item Toma de conciencia del tiempo
	\item Creación de secuencias individuales y colectivas: Cuerpo, espacio y tiempo.
	\item El uso dramático del elemento: El juego teatral.
	\item Presentaciones teatrales con el uso del elemento.

\end{topics}
\begin{learningoutcomes}
	\item Experimentar con nuevas formas de expresión y comunicación [\Usage].
	\item Conocer algunos mecanismos de control y manejo corporal [\Usage].
	\item Brindar caminos para que el alumno pueda desarrollar creativamente su imaginación, su capacidad de relación  y captación de estTecnologíamulos auditivos, rTecnologíatmicos y visuales [\Usage].
	\item Conocer y desarrollar el manejo de su espacio propio  y de sus  relaciones  espaciales  [\Usage].
	\item Experimentar  estados emocionales diferentes y climas  colectivos nuevos [\Usage].
\end{learningoutcomes}
\end{unit}

\begin{unit}{}{Comunicación no verbal en el Teatro}{Majorana,PAVIS}{12}{C18, C24}
\begin{topics}
	\item Relajación, concentración y respiración.
	\item Desinhibición e interacción con el grupo.
	\item La improvisación.
	\item Equilibrio, peso, tiempo y ritmo.
	\item Análisis del movimiento. Tipos de movimiento.
	\item La presencia teatral.
	\item La danza, la coreografTecnologíaa teatral.

\end{topics}
\begin{learningoutcomes}
	\item Ejercitarse en el manejo de destrezas comunicativas no verbales [\Usage].
	\item Practicar juegos y ejercicios de lenguaje  corporal, individual y grupalmente [\Usage].
	\item Expresar libre y creativamente sus emociones y sentimientos y su visión de la sociedad  a través de representaciones originales con diversos lenguajes [\Usage].
	\item Conocer los tipos de actuación [\Usage].
\end{learningoutcomes}
\end{unit}

\begin{unit}{}{Huellas del teatro en el tiempo  (El Teatro en la historia)}{Majorana,PAVIS}{3}{C24}
\begin{topics}
	\item El orTecnologíagen del teatro, el teatro griego y el teatro romano.
	\item El teatro medieval , la comedia del arte.
	\item De la pasión a la razón: Romanticismo e Ilustración.
	\item El teatro realista, teatro épico. Brech  y  Stanislavski.
	\item El teatro del absurdo, teatro contemporáneo y teatro total.
	\item Teatro en el Perú: Yuyashkani, La Tarumba, pataclaun, otros.
\end{topics}
\begin{learningoutcomes}
	\item Conocer la influencia que la sociedad ha ejercido en el teatro y la respuesta de este arte ante los diferentes momentos de la historia [\Usage].
	\item Apreciar el valor y aporte de las obras de dramaturgos importantes [\Usage].
	\item Analizar el contexto social del arte teatral [\Usage].
	\item Reflexionar sobre el Teatro Peruano y arequipeño [\Usage].
\end{learningoutcomes}
\end{unit}

\begin{unit}{}{El  Montaje Teatral}{Majorana,PAVIS}{12}{C17,C18, C24}
\begin{topics}
	\item Apreciación teatral. Expectación de una o más obras teatrales.
	\item El espacio escénico.
	\item Construcción del personaje
	\item Creación y montaje de una obra teatral .
	\item Presentación en público de pequeñas obras haciendo uso de vestuario, maquillaje, escenografTecnologíaa, utilerTecnologíaa y del empleo dramático del objeto.
\end{topics}
\begin{learningoutcomes}
	\item Emplear  la creación teatral, como manifestación de ideas y sentimientos propios ante la sociedad [\Usage].
	\item Aplicar las técnicas practicadas y los conocimientos aprendidos en una apreciación y/o expresión teatral concreta que vincule el rol de la educación [\Usage].
	\item Intercambiar experiencias y realizar presentaciones breves de ejercicios teatrales en grupo, frente a público [\Usage].
\end{learningoutcomes}
\end{unit}

\begin{coursebibliography}
\bibfile{GeneralEducation/FG101}
\end{coursebibliography}

\end{syllabus}
