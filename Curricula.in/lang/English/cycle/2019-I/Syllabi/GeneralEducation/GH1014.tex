\begin{syllabus}

\course{GH0014. Culturas de gobernanza y distribución de poder}{Obligatorio}{GH0014} % Common.pm

\begin{justification}
The objective of the course is for the student to understand the interrelation that exists between political and economic systems of a country or region. The thread of this course will be the book “Why Nations Fail: The Origins of Power, Prosperity, and Poverty” by Acemoglu-Robinson.
At the end of the course, students must have learned an informed interpretation of different social dynamics through which power is organized and distributed, be it of a symbolic, economic and/or political nature. This course aims to train the student’s ability to use more complex concepts and to develop more elaborate interpretations of reality. 
\end{justification}

\begin{goals}
\item Ability to interpret information
\item Ability to formulate solution alternatives
\item Ability to understand texts 
\end{goals}

\begin{outcomes}{V1}
    \item \ShowOutcome{d}{2}
    \item \ShowOutcome{e}{2}
    \item \ShowOutcome{n}{2}
    
\end{outcomes}

\begin{competences}{V1}
    \item \ShowCompetence{C10}{d,n}
    \item \ShowCompetence{C17}{d}
    \item \ShowCompetence{C18}{n}
    \item \ShowCompetence{C21}{e}
\end{competences}

\begin{unit}{Culturas de Gobernanza y Distribución de Poder}{}{Lessig15}{12}{4}
   \begin{topics}
      \item How is the economy related to politics?
      \item The role of Institutions.
      \item Case analysis.
   \end{topics}
   \begin{learningoutcomes}
      \item Interest to know about current issues in Peruvian society and the world.
   \end{learningoutcomes}
\end{unit}

\begin{coursebibliography}
\bibfile{GeneralH/GH1014}
\end{coursebibliography}

\end{syllabus}
