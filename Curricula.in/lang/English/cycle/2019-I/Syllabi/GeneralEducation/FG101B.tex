\begin{syllabus}

\course{FG101B Oral and written communication .II}{Obligatorio}{FG101B}

\begin{justification}
In order to achieve an effective communication in the personal and professional field, the proper handling of the Language in oral and written form is a priority. It is therefore justified that the students know, understand and apply the conceptual and operative aspects of their language, for the development of their fundamental communicative skills: Listening, speaking, reading and writing.
Consequently the permanent exercise and the contribution of the fundamentals contribute greatly in the academic formation and, in the future, in the performance of its profession
\end{justification}

\begin{goals}
\item Develop communicative skills through the theory and practice of language that help the student to overcome the academic requirements of the undergraduate and contribute to his humanistic training and as a human person.
\end{goals}

\begin{outcomes}
   \item \ShowOutcome{f}{2}
   \item \ShowOutcome{h}{2}
   \item \ShowOutcome{n}{2}
\end{outcomes}

\begin{competences}
    \item \ShowCompetence{C17}{f,h,n}
    \item \ShowCompetence{C20}{f,n}
    \item \ShowCompetence{C24}{f,h}
\end{competences}

\begin{unit}{Course Presentation}{}{Real}{16}{C17,C20}
  \begin{topics}
      \item Oral and written communication II.
      \item Characteristics of argumentative text.
  \end{topics}

  \begin{learningoutcomes}
   \item Let the student reaffirm the differences between an Argumentative text from examples.
  \end{learningoutcomes}
\end{unit}

\begin{unit}{What is a laboratory report?}{}{Real}{16}{C17,C20}
  \begin{topics}
      \item Characteristics and parts.
      \item Presentation of a model and analysis.
      \item Presentation of the methodology
  \end{topics}

  \begin{learningoutcomes}
   \item .%ToDo
  \end{learningoutcomes}
\end{unit}

\begin{unit}{Laboratory Report: Laboratory Results and Applications}{}{Real}{16}{C17,C20}
  \begin{topics}
      \item Presentation of the characteristics of these parts.
      \item Writing exercise.
  \end{topics}

  \begin{learningoutcomes}
   \item .%ToDo
  \end{learningoutcomes}
\end{unit}

\begin{unit}{Laboratory report: Introduction and conclusions}{}{Real}{16}{C17,C20}
  \begin{topics}
      \item Presentation of the characteristics of these parts.
      \item Writing exercise.
  \end{topics}

  \begin{learningoutcomes}
   \item .%ToDo
  \end{learningoutcomes}
\end{unit}

\begin{unit}{Cited, parenthetical references and bibliography construction}{}{Real}{16}{C17,C20}
  \begin{topics}
      \item APA format.
  \end{topics}

  \begin{learningoutcomes}
   \item .%ToDo
  \end{learningoutcomes}
\end{unit}

\begin{unit}{Orality characteristics}{}{Real}{16}{C17,C20}
  \begin{topics}
      \item Analysis of a ted talk.
      \item Oral exposure characteristics.
  \end{topics}

  \begin{learningoutcomes}
   \item .%ToDo
  \end{learningoutcomes}
\end{unit}

\begin{unit}{Preparation for oral presentation}{}{Real}{16}{C17,C20}
  \begin{topics}
      \item Peer evaluation rubric.
  \end{topics}

  \begin{learningoutcomes}
   \item .%ToDo
  \end{learningoutcomes}
\end{unit}


\begin{unit}{Presentation of an argumentative text and characteristics of the argumentation}{}{Real}{16}{C17,C20}
  \begin{topics}
      \item Guidelines to delimit theme.
      \item Scheme and Summary.
      \item Elaboration of argumentative schemes.
      \item Writing brief arguments.
  \end{topics}

  \begin{learningoutcomes}
   \item .%ToDo
  \end{learningoutcomes}
\end{unit}

\begin{unit}{How to build an argument?}{}{Real}{16}{C17,C20}
  \begin{topics}
      \item The pragmatic argument.
      \item Identification exercises
      \item Pragmatic argument writing exercise.
  \end{topics}

  \begin{learningoutcomes}
   \item .%ToDo
  \end{learningoutcomes}
\end{unit}

\begin{unit}{Counterargumentation}{}{Real}{16}{C17,C20}
  \begin{topics}
      \item What is counterargument?.
      \item Counterargument texts models.
      \item Elaboration of a counterargumentative scheme.
      \item Drafting of counterargumentation.
  \end{topics}

  \begin{learningoutcomes}
   \item .%ToDo
  \end{learningoutcomes}
\end{unit}

\begin{unit}{Argumentative robustness of counterargumentation}{}{Real}{16}{C17,C20}
  \begin{topics}
      \item Presentation of counterargument
  \end{topics}

  \begin{learningoutcomes}
   \item .%ToDo
  \end{learningoutcomes}
\end{unit}


\begin{unit}{Debates}{}{Real}{16}{C17,C20}
  \begin{topics}
      \item Debates.
  \end{topics}

  \begin{learningoutcomes}
   \item .%ToDo
  \end{learningoutcomes}
\end{unit}




\begin{coursebibliography}
\bibfile{GeneralEducation/FG101B}
\end{coursebibliography}

\end{syllabus}
