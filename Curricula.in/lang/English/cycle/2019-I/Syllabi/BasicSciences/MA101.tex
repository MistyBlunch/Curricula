\begin{syllabus}

\course{EG0005. Matemática II}{Obligatorio}{EG0005} % Common.pm

\begin{justification}

The course develops in students the skills to deal with models of science and engineering skills. In the first part
of the course a study of the functions of several variables, partial derivatives, multiple integrals and an
introduction to vector fields is performed. Then the student will use the basic concepts of calculus to model
and solve ordinary differential equations using techniques such as Laplace transforms and Fourier series.
\end{justification}

\begin{goals}
  \item Apply derivation rules and partial differentation in functions of several variables.
  \item Apply techniques for calculating multiple integrals.
  \item Understand and use the concepts of vector calculus.
  \item Understand the importance of series.
  \item Identify and solve differential equations of the first order and their applications in chemical and physical problems.
\end{goals}

\begin{outcomes}{V1}
    \item \ShowOutcome{a}{3}  
    \item \ShowOutcome{j}{3}
\end{outcomes}

\begin{competences}{V1}
    \item \ShowCompetence{C1}{a}
    \item \ShowCompetence{C20}{j}
\end{competences}

\begin{unit}{Multi-Variable Function Differential}{}{Stewart,DennisZ}{24}{C1,C20}
   \begin{topics}      
    \item Concept of multi-variable functions.
    \item Directional Derivates
    \item Tangent line, normal plane to curve line and tangent plane, normal line to a curve plan. Know to calculate their equations.
    \item Concept of extreme value and conditional extreme value of multi-variable functions
    \item Applications problems such as modeling total production of an economic system, speed of sound through the ocean, thickener optimization, etc.
      \end{topics}

   \begin{learningoutcomes}
    \item Understand the concept of multi-variable functions.
    \item Master the concept and calculation method of the direction derivative and gradient of the guide.
    \item Master the calculation method of the first order and second order partial derivative of composite functions.
    \item Master the calculation method of the partial derivatives for implicit functions.
    \item Understand tangent line, normal plane to curve line and tangent plane, normal line to a curve plan. Know to calculate their equations.
    \item Learn the concept of extreme value and conditional extreme value of multi-variable functions; know to find out the binary function extreme value.
    \item Be able to solve simple applications problems.
    \end{learningoutcomes}
\end{unit}

\begin{unit}{Multi-Variable function Integral}{}{Stewart,DennisZ}{12}{C1,C20}
  \begin{topics}
    \item Double integral, triple integral and nature of the multiple integral.
    \item Method of double integral
    \item Line Integral
    \item The Divergence, Rotation and Laplacian
   \end{topics}
  
  \begin{learningoutcomes}
    \item Understand the double integral, triple integral, and understand the nature of the multiple integral.
    \item Master the calculation method of double integral (Cartesian coordinates, polar coordinates) the triple integral (Cartesian coordinates, cylindrical coordinates, spherical coordinates).
    \item Understand the concept of line Integral, their properties and relationships.
    \item Know to calculate the line integral.
    \item Master the calculation the rotational, divergence and Laplacian.   
  
    \end{learningoutcomes}

\end{unit}

\begin{unit}{Series}{}{Stewart,DennisZ}{24}{C1,C20}
   \begin{topics}
    \item Convergent series
    \item Taylor and McLaurin series
    \item Orthogonal functions 
 \end{topics}

   \begin{learningoutcomes}
    \item Master to calculation if series is convergent, and if convergent, find the sum of the series trying to find the radius of convergence and the interval of convergence of a power series.
    \item Represent a function as a power series and find the Taylor and McLaurin Series to estimate function values to a desired accuracy.
    \item Understand the concepts of orthogonal functions and the expansion of a given function f to find its Fourier series.
     \end{learningoutcomes}
\end{unit}

\begin{unit}{Ordinary Differential Equations}{}{Stewart,DennisZ}{30}{C1,C20}
   \begin{topics}
    \item Concept of differential equations
    \item Methods to resolve differential equations
    \item Methods to resolve the secod order linear differential equations 
    \item Higher order linear ordinary differential equations
    \item Applications problems using Laplace transforms
      \end{topics}

   \begin{learningoutcomes}
    \item Understand differential equations, solutions, order, general solution, initial conditions and special solutions etc.
    \item Master the calculation method for variables separable equation and first order linear equations. Known to solve homogeneous equation and Bernoulli (Bernoulli) equations; understand variable substitution to solve the equation.
    \item Master to solve total differential equations.
    \item Be able to use reduced order method to solve equations.
    \item Understand the structure of the second order linear differential equation.
    \item Master calculation method for the constant coefficient homogeneous linear differential equations; and understand calculation method for the higher order homogeneous linear differential equations.
    \item Know to apply the differential equation calculation method to solve simple geometric and physic application problems.
    \item Solve properly certain types of differential equations using Laplace transforms.

   \end{learningoutcomes}
\end{unit}

\begin{coursebibliography}
\bibfile{BasicSciences/MA101}
\end{coursebibliography}

\end{syllabus}
