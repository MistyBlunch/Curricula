\begin{syllabus}

\course{CS221. Arquitectura de Computadores}{Obligatorio}{CS221}

\begin{justification}
It is necessary that the professional in Computer Science has a solid knowledge of the organization and operation of the various computer systems in which the programming environment is installed. This will also know how to establish the scope and limits of the applications that are developed according to the platform being used.

The following topics will be addressed: basic digital logic components in a computer system, design of instruction sets, microarchitecture of the processor and execution in pipelining, organization of memory: cache and virtual memory, protection and sharing, system I / O and interrupts, super-scalar architectures and out-of-order execution, vector computers, multithreading architectures, symmetric multiprocessors, memory and synchronization models, integrated systems and parallel computers.

\end{justification}

\begin{goals}
\item This course is intended to provide the student with a solid foundation in the evolution of computer architectures and the factors that influenced the design of hardware and software in today's computer systems.
\item Ensure understanding of what hardware is itself and how it interacts with hardware and software in a current computing system.
\item To deal with the following topics: basic digital logic components in a computer system, design of instruction sets, microarchitecture of the processor and execution in pipelining, organization of memory: cache and virtual memory, protection and sharing, system I / O and interrupts, super-scalar architectures and out-of-order execution, vector computers, multithreading architectures, symmetric multiprocessors, memory and synchronization models, integrated systems and parallel computers.
\end{goals}

\begin{outcomes}
    \item \ShowOutcome{b}{2}
    \item \ShowOutcome{i}{3}
\end{outcomes}

\begin{competences}
    \item \ShowCompetence{C4}{i} 
    \item \ShowCompetence{C8}{b,i}
    \item \ShowCompetence{C9}{b}
    \item \ShowCompetence{CS3}{i}
\end{competences}

\begin{unit}{\ARDigitallogicanddigitalsystems}{}{Parhami2005,Patterson2004}{18}{C8}
\begin{topics}%
\item \ARDigitallogicanddigitalsystemsTopicOverview
\item \ARDigitallogicanddigitalsystemsTopicCombinational
\item \ARDigitallogicanddigitalsystemsTopicMultiple
\item \ARDigitallogicanddigitalsystemsTopicComputer
\item \ARDigitallogicanddigitalsystemsTopicRegister
\item \ARDigitallogicanddigitalsystemsTopicPhysical
\end{topics}
\begin{learningoutcomes}
\item \ARDigitallogicanddigitalsystemsLODescribeTheComputer [\Familiarity]
\item \ARDigitallogicanddigitalsystemsLOComprehend [\Usage]
\item \ARDigitallogicanddigitalsystemsLOExplainTheThe [\Usage]
\item \ARDigitallogicanddigitalsystemsLOArticulate [\Familiarity]
\item \ARDigitallogicanddigitalsystemsLODesignThe [\Usage]
\item \ARDigitallogicanddigitalsystemsLOUseCad [\Familiarity]
\item \ARDigitallogicanddigitalsystemsLOEvaluate [\Assessment]
\end{learningoutcomes}
\end{unit}

\begin{unit}{\ARMachinelevelrepresentationofdata}{}{Parhami2005,Stallings2010}{8}{C9}
\begin{topics}
\item \ARMachinelevelrepresentationofdataTopicBits
\item \ARMachinelevelrepresentationofdataTopicNumeric
\item \ARMachinelevelrepresentationofdataTopicFixed
\item \ARMachinelevelrepresentationofdataTopicSigned
\item \ARMachinelevelrepresentationofdataTopicRepresentation
\item \ARMachinelevelrepresentationofdataTopicRepresentationOf
\end{topics}

\begin{learningoutcomes}
\item \ARMachinelevelrepresentationofdataLOExplainWhyData [\Assessment]
\item \ARMachinelevelrepresentationofdataLOExplainTheUsing [\Familiarity]
\item \ARMachinelevelrepresentationofdataLODescribeHowAre [\Usage]
\item \ARMachinelevelrepresentationofdataLOExplainHowNumber [\Usage]
\item \ARMachinelevelrepresentationofdataLODescribeTheOf [\Usage]
\item \ARMachinelevelrepresentationofdataLOConvertNumerical [\Usage]
\end{learningoutcomes}
\end{unit}

\begin{unit}{\ARAssemblylevelmachineorganization}{}{Parhami2005,Patterson2004,Hennessy2006}{8}{C4,CS3}
\begin{topics}
  \item \ARAssemblylevelmachineorganizationTopicBasic
  \item \ARAssemblylevelmachineorganizationTopicControl
  \item \ARAssemblylevelmachineorganizationTopicInstruction
  \item \ARAssemblylevelmachineorganizationTopicAssembly
  \item \ARAssemblylevelmachineorganizationTopicInstructionFormats
  \item \ARAssemblylevelmachineorganizationTopicAddressing
  \item \ARAssemblylevelmachineorganizationTopicSubroutine
  \item \ARAssemblylevelmachineorganizationTopicI
  \item \ARAssemblylevelmachineorganizationTopicHeap
\end{topics}

\begin{learningoutcomes}
  \item \ARAssemblylevelmachineorganizationLOExplainTheTheNeumann [\Familiarity]
  \item \ARAssemblylevelmachineorganizationLODescribeHowIs [\Familiarity]
  \item \ARAssemblylevelmachineorganizationLODescribeInstruction [\Familiarity]
  \item \ARAssemblylevelmachineorganizationLOSummarize [\Familiarity]
  \item \ARAssemblylevelmachineorganizationLODemonstrateHow [\Usage]
  \item \ARAssemblylevelmachineorganizationLOExplainDifferent [\Usage]
  \item \ARAssemblylevelmachineorganizationLOExplainHowAre [\Usage]
  \item \ARAssemblylevelmachineorganizationLOExplainTheOf [\Familiarity]
  \item \ARAssemblylevelmachineorganizationLOWriteSimple [\Usage]
  \item \ARAssemblylevelmachineorganizationLOShow  [\Usage]
\end{learningoutcomes}
\end{unit}

%% Unidad Organizaci�n Funcional para ejecuci�n de instrucciones  
\begin{unit}{\ARFunctionalorganization}{}{Parhami2005,Hennessy2006}{8}{C9}
\begin{topics}
      \item \ARFunctionalorganizationTopicImplementation
      \item \ARFunctionalorganizationTopicControl
      \item \ARFunctionalorganizationTopicInstruction
      \item \ARFunctionalorganizationTopicIntroductionTo
\end{topics}

\begin{learningoutcomes}
\item \ARFunctionalorganizationLOCompareAlternative [\Assessment]
\item \ARFunctionalorganizationLODiscussTheControl [\Familiarity]
\item \ARFunctionalorganizationLOExplainBasic [\Usage]
\item \ARFunctionalorganizationLODesignAnd [\Usage]
\item \ARFunctionalorganizationLODetermineFor [\Assessment]
\end{learningoutcomes}
\end{unit}

%% Organizaci�n y Arquitectura de la Memoria
\begin{unit}{\ARMemorysystemorganizationandarchitecture}{}{Parhami2005,Patterson2004,Denning2005}{8}{CS3}
\begin{topics}
  \item \ARMemorysystemorganizationandarchitectureTopicStorage
  \item \ARMemorysystemorganizationandarchitectureTopicMemory
  \item \ARMemorysystemorganizationandarchitectureTopicMain
  \item \ARMemorysystemorganizationandarchitectureTopicLatency
  \item \ARMemorysystemorganizationandarchitectureTopicCache
  \item \ARMemorysystemorganizationandarchitectureTopicMultiprocessor
  \item \ARMemorysystemorganizationandarchitectureTopicVirtual
  \item \ARMemorysystemorganizationandarchitectureTopicFault
  \item \ARMemorysystemorganizationandarchitectureTopicError
\end{topics}

\begin{learningoutcomes}
  \item \ARMemorysystemorganizationandarchitectureLOIdentify [\Familiarity]
  \item \ARMemorysystemorganizationandarchitectureLOExplainTheMemory [\Familiarity]
  \item \ARMemorysystemorganizationandarchitectureLODescribeHowOf [\Usage]
  \item \ARMemorysystemorganizationandarchitectureLODescribeTheMemory [\Usage] 
  \item \ARMemorysystemorganizationandarchitectureLOExplainTheA [\Usage] 
  \item \ARMemorysystemorganizationandarchitectureLOCompute [\Assessment]
\end{learningoutcomes}
\end{unit}

%% Unidad I/O: Interfacing and Communications
\begin{unit}{\ARInterfacingandcommunication}{}{Parhami2005,Stallings2010}{8}{C4,C9,CS3}
\begin{topics}
	\item \ARInterfacingandcommunicationTopicI
	\item \ARInterfacingandcommunicationTopicInterrupt
	\item \ARInterfacingandcommunicationTopicExternal
	\item \ARInterfacingandcommunicationTopicBuses
	\item \ARInterfacingandcommunicationTopicIntroduction
	\item \ARInterfacingandcommunicationTopicMultimedia
	\item \ARInterfacingandcommunicationTopicRaid
 \end{topics}
 
\begin{learningoutcomes}
	\item \ARInterfacingandcommunicationLOExplainHowUsed [\Familiarity]
	\item \ARInterfacingandcommunicationLOIdentifyVarious [\Familiarity]
	\item \ARInterfacingandcommunicationLODescribeData [\Usage]
	\item \ARInterfacingandcommunicationLOCompare [\Assessment]
	\item \ARInterfacingandcommunicationLOIdentifyThe [\Familiarity]
	\item \ARInterfacingandcommunicationLODescribeTheLimitations [\Familiarity]
\end{learningoutcomes}
\end{unit}

%% Unidad Arquitecturas Alternativas y en Paralelo
\begin{unit}{\ARMultiprocessingandalternativearchitectures}{}{Parhami2005,Parhami2002,ElRewini2005}{8}{C9}
\begin{topics}
	\item \ARMultiprocessingandalternativearchitecturesTopicPower
	\item \ARMultiprocessingandalternativearchitecturesTopicExample
	\item \ARMultiprocessingandalternativearchitecturesTopicInterconnection
	\item \ARMultiprocessingandalternativearchitecturesTopicShared
	\item \ARMultiprocessingandalternativearchitecturesTopicMultiprocessor
\end{topics}

\begin{learningoutcomes}
	\item \ARMultiprocessingandalternativearchitecturesLODiscussTheParallel [\Assessment]
	\item \ARMultiprocessingandalternativearchitecturesLODescribeAlternative [\Familiarity]
	\item \ARMultiprocessingandalternativearchitecturesLOExplainTheInterconnection [\Usage]
	\item \ARMultiprocessingandalternativearchitecturesLODiscussTheThat [\Familiarity]
	\item \ARMultiprocessingandalternativearchitecturesLODescribeTheMemoryMemory [\Assessment]
\end{learningoutcomes}
\end{unit}

%% Unidad Mejoras en Desempe�o
\begin{unit}{\ARPerformanceenhancements}{}{Parhami2005,Parhami2002,Patterson2004,Dongarra2006,Johnson1991}{8}{C8,C9}
\begin{topics}
	\item \ARPerformanceenhancementsTopicSuperscalar
	\item \ARPerformanceenhancementsTopicBranch
	\item \ARPerformanceenhancementsTopicPrefetching
	\item \ARPerformanceenhancementsTopicVector
	\item \ARPerformanceenhancementsTopicHardware
	\item \ARPerformanceenhancementsTopicScalability
	\item \ARPerformanceenhancementsTopicAlternative
\end{topics}

\begin{learningoutcomes}
  \item \ARPerformanceenhancementsLODescribeSuperscalar [\Familiarity]
  \item \ARPerformanceenhancementsLOExplainTheBranch [\Usage]
  \item \ARPerformanceenhancementsLOCharacterize [\Assessment]
  \item \ARPerformanceenhancementsLOExplainSpeculative [\Assessment]
  \item \ARPerformanceenhancementsLODiscussTheThatIn [\Assessment]
  \item \ARPerformanceenhancementsLODescribeTheScalability [\Assessment]
\end{learningoutcomes}
\end{unit}



\begin{coursebibliography}
\bibfile{Computing/CS/CS221}
\end{coursebibliography}

\end{syllabus}
