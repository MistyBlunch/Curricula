\begin{syllabus}

\course{CS331. Cloud Computing}{Obligatorio}{CS331}

\begin{justification}
La capacidad de procesamiento de una sola m�quina es limitada y la Ley de Moore se ha encontrado 
con barreras antes de lo previsto, a pesar de esto la necesidad de mayor poder computacional es cresciente. 

El uso de las computadoras como elementos conectados entre s� es cada vez m�s com�n y cada vez en mayor escala, 
la capacidad de comunicaci�n entre dispositivos (computadoras, celulares, pdas, etc.), abre las puertas 
a la existencia de una �nica plataforma donde la informaci�n de los usuarios
est� disponible siempre, sin importar el medio de acceso a esta (\textit{Cloud computing}).

La computaci�n en la nube de internet o un grupo de computadores 
permite conseguir ambos objetivos, traspasando la barrera de una sola m�quina para poder
integrar las capacidades de distintos dispositivos y permitirles interactuar en un entorno que
el usuario perciba como unificado; adem�s, al conectarlos, el tope de desempe�o
del sistema ya no es la capacidad de un s�lo elemento (e.g. CPU) sino la cantidad de participantes en este,
por lo cual existe una escalabilidad del poder computacional much�simo mayor.
\end{justification}

\begin{goals}
\item Comprender los conceptos b�sicos de la computaci�n en nube, incluyendo definiciones, historia, pros y cons de la misma, comparaciones con tecnolog�as relacionadas, tales como grid computing, o utility computing.
\item Conocer la tecnolog�a que soporta a la computaci�n en nube.
\item Comprender la relaci�n entre data-intensive applications y cloud computing, y
\item Evaluar el nuevo modelo de computaci�n para conocer las tendencias de esta �rea emergente.
\end{goals}

\begin{outcomes}
\ExpandOutcome{a}{3}
\ExpandOutcome{b}{4}
\ExpandOutcome{c}{4}
\ExpandOutcome{d}{3}
\ExpandOutcome{i}{3}
\ExpandOutcome{j}{4}
\ExpandOutcome{k}{4}
\end{outcomes}

\begin{unit}{Introducci�n a cloud computing}{Armbrust:EECS-2009-28,nistcc}{7}{2}
   \begin{topics}
        \item \ARDistributedArchitecturesTopicNetwork%
        \item \SESpecializedSystemsTopicClient%
        \item \SESpecializedSystemsTopicDistributed%
        \item \SESpecializedSystemsTopicParallel%
        \item \SESpecializedSystemsTopicWeb%
   \end{topics}

   \begin{unitgoals}
        \item \ARDistributedArchitecturesObjSIX%
        \item \SESpecializedSystemsObjONE%
        \item \SESpecializedSystemsObjFIVE%
        \item Comprender como aparecio el paradigma de computaci�n en nube.
   \end{unitgoals}
\end{unit}

\begin{unit}{Temas de investigaci�n en cloud computing}{vaquero09,lmei2008}{8}{2}
   \begin{topics}
        \item Data Center Network Architecture
        \item Network Management
        \item Resource and Performance Management
        \item Data management
   \end{topics}

   \begin{unitgoals}
        \item Entender la relaci�n entre los diferentes tipos de investigaci�n que procedieron a la computaci�n en nube.
        \item Conocer distintas l�neas de investigaci�n de computaci�n en nube.
   \end{unitgoals}
\end{unit}

\begin{unit}{Cloud data management}{stonebrakersharednothing86,stonebrakerendofera07,claremont09}{10}{3}
   \begin{topics}
      \item \IMInformationModelsTopicInformationStorage%
      \item \IMInformationModelsTopicSearch%
      \item \IMInformationModelsTopicScalability%
      \item \IMDatabaseSystemsTopicDatabase%
      \item \IMDistributedDatabasesAllTopics%
      \item Big Data.
      \item Large small data.
      \item Bases de datos {\it NoSQL}.
   \end{topics}

    \begin{unitgoals}
      \item \IMInformationModelsObjTWO%
      \item \IMInformationModelsObjSEVEN%
      \item \IMInformationModelsObjEIGHT%
      \item \IMInformationModelsObjNINE%
      \item \IMDistributedDatabasesObjTWO%
      \item Conocer diferentes casos de objetos distribuidas.
   \end{unitgoals}
\end{unit}


\begin{unit}{Data-intensive applications}{heystewart09,bryantdisc07,dean08}{10}{3}
    \begin{topics}
      \item Modelo de programaci�n MapReduce.
      \item Ejemplos de aplicaciones en la academia y en la industria.
      \item Aplicaciones usando MapReduce.
      \item Otros lenguajes de programaci�n para Cloud Computing.
   \end{topics}

   \begin{unitgoals}
      \item Entender el modelo de programaci�n MapReduce.
      \item Conocer diferentes modos de uso de MapReduce. 
      \item \IMInformationModelsObjEIGHT%
      \item \IMInformationModelsObjNINE%
      \item \IMDistributedDatabasesObjTWO%
   \end{unitgoals}
\end{unit}


\begin{unit}{Programando para Cloud Computing}{dean08,Nurmi:2009:EOC:1577849.1577895,aws}{10}{3}
   \begin{topics}
      \item Usando Amazon Web Services.
      \item MapReduce en Amazon Web Services.
      \item Proveedores de Cloud Computing.
      \item Frameworks para crear servicios de Cloud Computing.
   \end{topics}

   \begin{unitgoals}
      \item Conocer los diferentes services de Amazon Web Services.
      \item Aplicar conocimientos de Cloud Computing para crear aplicaciones que usen otros servicios de Cloud Computing.
      \item Conocer los diferentes proveedores de servicios de Cloud Computing.
      \item Entender las similitudes y diferencias, ventajas y desventajas de los diferentes frameworks para crear {\it private clouds}.
   \end{unitgoals}
\end{unit}



\begin{coursebibliography}
\bibfile{Computing/CS/CS331}
\end{coursebibliography}

\end{syllabus}
