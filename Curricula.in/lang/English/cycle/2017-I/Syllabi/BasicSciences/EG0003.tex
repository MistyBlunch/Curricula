\begin{syllabus}

\course{MA100. Matemática I}{Obligatorio}{MA100}

\begin{justification}
The course aims to develop in students the skills to deal with models in science and engineering related to single variable differential calculus skills. In the course it is studied and applied concepts related to calculation limits, derivatives and integrals of real and vector functions of single real variables to be used as base and support for the study of new contents and subjects. Also seeks to achieve reasoning capabilities and
applicability to interact with real-world problems by providing a mathematical basis for further professional development activities.
\end{justification}

\begin{goals}
\item Apply the concepts of complex numbers and functions to solve problems related to science.
\item Apply mathematical concepts and techniques of differential calculus of one variable to solve problematic situations of science.
\item Calculate mathematical expressions of indefinite integrals with accuracy, order and clarity in the treatment of the data.
\end{goals}

\begin{outcomes}
    \item \ShowOutcome{a}{3}
    \item \ShowOutcome{j}{3}
\end{outcomes}

\begin{competences}
    \item \ShowCompetence{C1}{a}
    \item \ShowCompetence{C20}{j}
    \item \ShowCompetence{C24}{j}
\end{competences}

\begin{unit}{Complex numbers}{}{Stewart,RonLarson}{20}{C1}
   \begin{topics}
    \item Operations with complex numbers
    \item Moivre's theorem
   \end{topics}

   \begin{learningoutcomes}
      \item Define and operates with complex numbers, calculating their the polar and exponential form [\Assessment].
      \item Use Moivre's theorem to simplify calculations of complex [\Assessment].
      \end{learningoutcomes}
\end{unit}

\begin{unit}{Functions of a single variable}{}{Stewart,RonLarson}{10}{C20}
  \begin{topics}
    \item Domain and range.
    \item Types of functions.
    \item Graph of exponentials and logarithmic functions.
    \item Trigonometric functions.
    \item Apply rules to transform functions.
    \item Applications problems using Excel,modelling bacterial growing, Logarithmic scale, etc.
  \end{topics}

   \begin{learningoutcomes}
      \item Define a function of single variable and understand and be able to determine its domain and range. [\Assessment].
      \item Recognize different specific types of functions and create scatter plots and select an appropriate model. [\Assessment].
      \item Understand how a change in base affects the graph of exponentials and logarithmic functions. [\Assessment].
      \item Recognizes and builds trigonometric functions.[\Assessment].
      \item Apply rules to transform functions[\Assessment].
      \item Be able to solve simple applications problems such as regression and curve fitting. [\Assessment].
      \end{learningoutcomes}
\end{unit}

\begin{unit}{Limits and derivatives}{}{Stewart,RonLarson}{20}{C1}
   \begin{topics}
      \item Limits
      \item Derivatives
      \item Derivate concepts and compute relative errors.
      \item L'Hospital theorem 
      \item Applications problems such as velocity, exponential growth and decay, pile increasing gravel, optimization of a can, etc
   \end{topics}

   \begin{learningoutcomes}
      \item Understand the concept of limits and guess limits from the graph of a function. [\Assessment].
      \item Find limits using the limit laws and algebraic simplification. [\Assessment].
      \item Find vertical and horizontal asymptotes. [\Assessment].
      \item Compute and estimate derivatives. [\Assessment].
      \item Interpret the derivative as a rate of change. [\Assessment].
      \item Find the derivatives of basic and composed function[\Assessment].
      \item Approximates functions using derivate concepts and compute relative errors. [\Assessment].
      \item Find critical numbers, and absolute and local maximum and minimum values for continuous function. [\Assessment].
      \item Apply L'Hospital theorem to calculate some limits. [\Assessment].
      \item Understand optimization problems, find the function to be optimized and solve.[\Assessment].
      \item Be able to solve simple applications problems. [\Assessment].
   \end{learningoutcomes}
\end{unit}

\begin{unit}{Integrals}{}{Stewart,RonLarson}{22}{C20}
   \begin{topics}
    \item Strategy for integration.
    \item Technic to integrate functions.
    \item Additional Tools to Find Integrals.
    \item Applications problems.
   \end{topics}

   \begin{learningoutcomes}
      \item Solve properly estimate area using left and right endpoint and midpoint rectangles. [\Assessment].
      \item Use the Fundamental theorem to find derivatives of functions of evaluate definite and indefinite integrals using substitution. [\Assessment].
      \item Use different technic to integrate functions. [\Assessment].
      \item Apply integrals to found areas. [\Assessment].
      \item Compute volumes of solids obtained by rotating a bounded region about either the x-axis or the y-axis. [\Assessment].
      \item Compute the volume of solids obtained by rotating a bounded region about either the x-axis or the y-axis by considering cylindrical shells. [\Assessment].
      \item Compute the average value of a function. [\Assessment].
      \item Compute work done by a force and compute center of mass for a flat plate in the plane. [\Assessment].
      \item Define parametric curves and vectorials functions finding relationships between them. [\Assessment].
      \item Apply integrals to calculate the length of curves described by a vectorial functions[\Assessment].
      \item Be able to solve simple applications problems such as traffic on an Internet service, fuel consumption, tomography: volume of the brain, pump the water, mass in thickener, superformula, volume in Wankel machine, length of DNA molecule helix, etc. [\Assessment].
    \end{learningoutcomes}
\end{unit}

\begin{coursebibliography}
\bibfile{BasicSciences/EG003}
\end{coursebibliography}

\end{syllabus}

%\end{document}
