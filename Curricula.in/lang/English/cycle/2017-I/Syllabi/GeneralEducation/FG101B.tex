\begin{syllabus}

\course{FG101B Oral and written communication .II}{Obligatorio}{FG101B}

\begin{justification}
In order to achieve an effective communication in the personal and professional field, the proper handling of the Language in oral and written form is a priority. It is therefore justified that the students know, understand and apply the conceptual and operative aspects of their language, for the development of their fundamental communicative skills: Listening, speaking, reading and writing.
Consequently the permanent exercise and the contribution of the fundamentals contribute greatly in the academic formation and, in the future, in the performance of its profession
\end{justification}

\begin{goals}
\item Develop communicative skills through the theory and practice of language that help the student to overcome the academic requirements of the undergraduate and contribute to his humanistic training and as a human person.
\end{goals}

\begin{outcomes}
   \item \ShowOutcome{f}{2}
   \item \ShowOutcome{h}{2}
   \item \ShowOutcome{n}{2}
\end{outcomes}

\begin{competences}
    \item \ShowCompetence{C17}{f,h,n}
    \item \ShowCompetence{C20}{f,n}
    \item \ShowCompetence{C24}{f,h}
\end{competences}

\begin{unit}{Oral and written communication II}{}{Real}{16}{C17,C20}
\begin{topics}
      \item Cited, parenthetical references and bibliography construction.
\end{topics}

\begin{learningoutcomes}
   \item Handle properly the cited system and bibliographic references, and recognize the importance of its use.
\end{learningoutcomes}
\end{unit}



\begin{coursebibliography}
\bibfile{GeneralEducation/FG101B}
\end{coursebibliography}

\end{syllabus}
