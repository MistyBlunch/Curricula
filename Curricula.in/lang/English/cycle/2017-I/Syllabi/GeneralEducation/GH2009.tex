\begin{syllabus}

\course{GH2009.Peru:Industrial Country?}{Obligatorio}{CB101}

\begin{justification}

\end{justification}

\begin{goals}
\item Capability to interpret information.
\item Capability to identify problems.
\item Capability to understand texts.
\item Interest to know about contemporary issues in Peruvian society and the World.
\end{goals}


\begin{outcomes}
    \item \ShowOutcome{d}{2} % Multidiscip teams
    \item \ShowOutcome{e}{2} % ethical, legal, security and social implications
    \item \ShowOutcome{f}{2} % communicate effectively
    \item \ShowOutcome{n}{2} % Apply knowledge of the humanities
    \item \ShowOutcome{o}{2} % TASDSH
\end{outcomes}

\begin{competences}
    \item \ShowCompetence{C10}{d,n,o}
    \item \ShowCompetence{C17}{f}
    \item \ShowCompetence{C18}{f}
    \item \ShowCompetence{C21}{e}
\end{competences}

\begin{unit}{Peru:Industrial Country?}{}{Mayer44}{12}{4}
   \begin{topics}
      \item Description of Latin America in the 60’s,with a view to contextualizing Peru in relation to the rest of the world: influence of the Cold War, the Cuban Revolution (1959), the industrial vision of the ECLAC that implied the substitution of imports, etc..
      \item The economic and social picture of Peru in 1960: Population, poverty, distribution of wealth,economic structure.
      \item The Agrarian Reform: logic and economic and social effects. Processes and results. How much of Peru is current reality is explained by the agrarian reform?
      \item Industrial Policy: logic and outcome of statist/protectionist policies in industry.
      \item Terrorism and society. What do we know of its causes and consequences?
      \item The crisis of the late 80’s. Hyperinflation and economic interventionism.
      \item The principles of the Peruvian model. The Constitution of 1993. Why was private ownership of businesses preferred? What was it that really changed and what for?
      \item Industrial policy, development framework, and results. Current industry structure and potential.
      \item General changes in the relations of power: political parties, unions, development of local leadership, and finally, the decentralization of 2004.
      \item Taxation of extractive activities, the canon and local development.
      \item The relationship between companies and local communities.
   \end{topics}
   \begin{learningoutcomes}
      \item Ability to analyze information and Interests to know about Peruvian society and the world.
   \end{learningoutcomes}
\end{unit}





\begin{coursebibliography}
\bibfile{GeneralH/GH2009}
\end{coursebibliography}

\end{syllabus}
