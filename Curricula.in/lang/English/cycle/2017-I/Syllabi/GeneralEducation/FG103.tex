\begin{syllabus}

\course{FG103. Introducción a la Vida Universitaria}{Obligatorio}{FG103}

\begin{justification}
El ingreso a la universidad es un momento de nuevos desafíos y decisiones en la vida de una persona. En ese sentido, la Universidad Católica San Pablo busca, mediante el presente espacio, escuchar y acoger al joven ingresante con sus inquietudes y anhelos personales, presentar la identidad y misión de la universidad como su ``alma mater'', señalando los principales desafíos que el futuro profesional enfrentará en el mundo actual  y orientando a nuestros jóvenes estudiantes, a través de diversos principios, medios y otros recursos, con el fin de que puedan formarse integralmente y desplegarse plenamente en la fascinante aventura de la vida universitaria.  Su realización como buen profesional depende de una buena formación personal y cultural que le brinde horizontes amplios, que sustenten y proyecten su conocimiento y quehacer técnicos e intelectuales y que le permitan contribuir siendo agentes de cambio cultural y social.
\end{justification}

\begin{goals}
\item Que el alumno canalice sus inquietudes y anhelos a través del encuentro y descubrimiento de sí mismo, que le brinden espacios de análisis y reflexión personales para asumir posturas bien fundamentadas hacia los valores e ideales de su entorno. Mediante su inserción en la vida universitaria, logrará una disposición de apertura a su propio mundo interior y a su misión en el mundo, cuestionando su cosmovisión y a sí mismo para obtener un conocimiento y crecimiento personales que permitan su despliegue integral y profesional.
\end{goals}

\begin{outcomes}
    \item \ShowOutcome{e}{1}
    \item \ShowOutcome{n}{1}
    \item \ShowOutcome{ñ}{1}
    \item \ShowOutcome{o}{1}
\end{outcomes}

\begin{competences}
    \item \ShowCompetence{C20}{e,n,ñ}
    \item \ShowCompetence{C21}{e,n,ñ}
\end{competences}

\begin{unit}{}{Primera Unidad: La experiencia existencial y el descubrimiento del sentido de la vida}{Sanz,Rilke,Marias,Frankl}{24}{C20}
\begin{topics}
	\item Introducción al curso: presentación y dinámicas.
	\item Sentido de la Vida, búsqueda de propósito y vocación profesional.
	\item Obstáculos para el autoconocimiento: el ruido, la falta de comunicación, la mentira existencial, máscaras.
	\item Ofertas Intramundanas: Hedonismo, Relativismo, Consumismo, Individualismo, Inmanentismo
	\item Las consecuencias: la falta de interioridad, masificación y el desarraigo, soledad
	\item Los vicios capitales como plasmación en lo personal
\end{topics}
\begin{learningoutcomes}
	\item Identificar y caracterizar la propia cosmovisión y los criterios personales predominantes en sí mismos acerca del propósito y sentido de la vida y la felicidad. [\Usage]
	\item Crear un vínculo de confianza con el docente del curso para lograr apertura a nuevas perspectivas[\Usage].
\end{learningoutcomes}
\end{unit}

\begin{unit}{}{Segunda Unidad: Visión sobre la persona humana}{Guardini,Fromm,Figari,Pieper}{15}{C20, C21}
\begin{topics}
	\item Quién soy yo, las preguntas fundamentales
	\item El hombre como unidad.
	\item El hombre: nostalgia de infinito.
	\item La libertad como elemento fundamental en las elecciones personales: la experiencia del mal.
	\item Análisis del Amor y la Amistad.
	\item Aceptación y Reconciliación personal.
    \item Llamados a ser personas: la vivencia de la virtud según un modelo concreto.
\end{topics}
\begin{learningoutcomes}
	\item Reconocer la importancia de iniciar un proceso de autoconocimiento [\Usage].
	\item Identificar las manifestaciones que evidencian la unidad de la persona humana y su anhelo de trascendencia [\Usage].
    \item Contrastar los modelos de amor y libertad ofertados por la cultura actual con los propuestos en el curso [\Usage].
    \item Distinguir los criterios que conducen a una recta valoración personal [\Usage].
\end{learningoutcomes}
\end{unit}

\begin{unit}{}{Tercera Unidad: Vida Universitaria y Horizontes de Misión}{JuanPablo,Guardini2,Identidad}{12}{C20, C21}
\begin{topics}
	\item Origen y propósito de la Universidad: breve reseña histórica.
	\item La identidad católica de la UCSP: comunidad académica, búsqueda de la verdad, la formación integral y la  evangelización de la cultura.
	\item Proyecto final.
\end{topics}
\begin{learningoutcomes}
	\item Conocer e identificar  a la UCSP dentro del contexto histórico de las universidades [\Usage].
	\item Reconocer a su universidad como un ámbito de despliegue y espacio para crear cultura [\Usage].
    \item Afirmar, desde su vocación profesional, la necesidad de transformar el mundo que le toca vivir [\Usage].
\end{learningoutcomes}
\end{unit}



\begin{coursebibliography}
\bibfile{GeneralEducation/FG101}
\end{coursebibliography}

\end{syllabus}
