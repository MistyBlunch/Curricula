\begin{syllabus}

\course{GH2012.Economies and Development}{Obligatorio}{CB101}

\begin{justification}
This course seeks to introduce the student to the general concepts of microeconomics and macroeconomics. The goal is for students to be able to explain processes of reality from the logic of economics. After having completed the micro and macroeconomics modules, students must choose one of the two proposed elective tracks.
The elective tracks are i) cases of fast-growing economies and dramatic recessions and ii) public policies for poverty reduction in Latin America. The idea is that students can choose between two options that allow them to analyze concepts of economics applied to reality. In terms of contents, the student must have a clear understanding of the general concepts and topics that make up macro and microeconomics. Concerning the competencies to be worked on in this course, the student is expected to be able to apply theoretical concepts to the analysis of cases
\end{justification}

\begin{goals}
\item Ability to interpret information.
\item Ability to formulate solution alternatives.
\item Ability to understand texts .
\end{goals}

\begin{outcomes}
    \item \ShowOutcome{d}{2} % Multidiscip teams
    \item \ShowOutcome{e}{2} % ethical, legal, security and social implications
    \item \ShowOutcome{f}{2} % communicate effectively
    \item \ShowOutcome{n}{2} % Apply knowledge of the humanities

\end{outcomes}

\begin{competences}
    \item \ShowCompetence{C10}{d,n}
    \item \ShowCompetence{C17}{f}
    \item \ShowCompetence{C18}{f}
    \item \ShowCompetence{C21}{e}
\end{competences}

\begin{unit}{Economies and Development.}{}{Gregory02}{12}{4}
   \begin{topics}
      \item Microeconomics.
      \item Macroeconomics.
      \item Cases of Rapidly Growing Economies of Dramatic Recessions.
      \item Public Policies for Poverty Reduction in Latin America.
   \end{topics}
   \begin{learningoutcomes}
      \item Interest to know about current issues in Peruvian society and the world .
   \end{learningoutcomes}
\end{unit}


\begin{coursebibliography}
\bibfile{GeneralH/GH2012}
\end{coursebibliography}

\end{syllabus}
