\begin{syllabus}

\course{CB101. �lgebra y Geometr�a}{Obligatorio}{CB101}

\begin{justification}
This course aims to provide students with a real-life hands-on experience in the first steps within a business life cycle, through which an idea becomes a formal business model.
It is the first of a set of three courses designed to accompany students as they transform an idea into a prospective business or business, from idea to review of current business strategy.
\end{justification}

\begin{goals}
   \item Ability to analyze information.
   \item Interpretation of information and results.
�� \item Teamwork Ability.
�� \item Ethics.
�� \item Oral communication.
�� \item Written communication.
�� \item Graphic communication.
�� \item Understand the need to learn continuously
\end{goals}

\begin{outcomes}
\ExpandOutcome{f}{2}
\ExpandOutcome{g}{2}
\ExpandOutcome{h}{1}
\ExpandOutcome{i}{2}
\ExpandOutcome{j}{3}
\end{outcomes}

\begin{unit}{El ciclo de vida empresarial: desde la idea hasta la revisi�n de su estrategia.}{Fitzpatrick13}{12}{4}
   \begin{topics}
      \item .
   \end{topics}
   \begin{unitgoals}
      \item .
   \end{unitgoals}
\end{unit}

\begin{unit}{El proceso de ideaci�n y la visi�n del cliente}{Osterwalder10}{24}{3}
   \begin{topics}
      \item . 
   \end{topics}

   \begin{unitgoals}
      \item .
      \end{unitgoals}
\end{unit}

\begin{unit}{�C�mo construir y mantener equipos eficaces?}{Osterwalder10}{24}{3}
   \begin{topics}
      \item . 
      \end{topics}

   \begin{unitgoals}
      \item .
   \end{unitgoals}
\end{unit}

\begin{unit}{Ejecuci�n de LEAN: lo b�sico}{Osterwalder10}{30}{3}
   \begin{topics}
      \item .
   \end{topics}

   \begin{unitgoals}
      \item . 
   \end{unitgoals}
\end{unit}

\begin{unit}{Dise�o de un modelo de negocio: herramientas de dise�o y lienzo.}{Fitzpatrick13}{30}{3}
   \begin{topics}
      \item .
   \end{topics}

   \begin{unitgoals}
      \item .
   \end{unitgoals}
\end{unit}

\begin{unit}{Generaci�n de Modelos de Negocio: la Lona Modelo de Negocio (Osterwalder).}{Osterwalder10}{30}{3}
   \begin{topics}
      \item . 
   \end{topics}

   \begin{unitgoals}
      \item . 
   \end{unitgoals}
\end{unit}

\begin{unit}{Ingenier�a de Venture: usar las habilidades de la inform�tica para construir un modelo de negocio efectivo.}{Fitzpatrick13}{30}{3}
   \begin{topics}
      \item .
   \end{topics}

   \begin{unitgoals}
      \item .
   \end{unitgoals}
\end{unit}

\begin{unit}{Herramientas de investigaci�n de mercado primario y nichos de mercado.}{Grossman96}{30}{3}
   \begin{topics}
      \item .
   \end{topics}

   \begin{unitgoals}
      \item . 
   \end{unitgoals}
\end{unit}

\begin{unit}{La Importancia del Capital: Humano, Financiero e Intelectual.}{Grossman96}{30}{3}
   \begin{topics}
      \item .
   \end{topics}

   \begin{unitgoals}
      \item . 
   \end{unitgoals}
\end{unit}

\begin{unit}{T�cnicas de monetizaci�n y financiamiento.}{Fitzpatrick13}{30}{3}
   \begin{topics}
      \item . 
   \end{topics}

   \begin{unitgoals}
      \item. 
   \end{unitgoals}
\end{unit}


\begin{unit}{Comunicaci�n eficaz: crear una presentaci�n de un modelo de negocio de impacto.}{Fitzpatrick13}{30}{3}
   \begin{topics}
      \item . 
   \end{topics}

   \begin{unitgoals}
      \item .
   \end{unitgoals}
\end{unit}
\begin{coursebibliography}
\bibfile{GeneralEducation/GH0007}
\end{coursebibliography}

\end{syllabus}
