\begin{syllabus}

\course{CB101. �lgebra y Geometr�a}{Obligatorio}{CB101}

\begin{justification}
El curso de Quechua comunicativo permite acercar a los estudiantes al uso pr�ctico de la lengua andina en su variedad chanca. Esta es una de las variedades de mayor difusi�n y modelo para abordar otras variedades del denominado quechua sure�o o quechua II. Adem�s, se presenta sencilla en su aprendizaje por compartir sonidos con el castellano. Asimismo, el curso busca familiarizar al alumno con las estructuras b�sicas de esta lengua, as� como con la traducci�n y producci�n de textos. El objetivo �ltimo es proporcionar las herramientas b�sicas de aprendizaje de modo que el estudiante pueda expresarse en ella a un nivel b�sico y funcional, as� como conducir y desarrollar su propio aprendizaje de la lengua.
Consideramos que hablar quechua en ciertas situaciones donde los ingenieros UTEC tienen que desarrollarse es una ventaja important�sima: los hablantes nativos de quechua practican un trato diferenciado con las personas que lo hablan por sentir que se est� respetando su tradici�n y, a la vez, se est� haciendo un esfuerzo por entablar un di�logo en su propia lengua. Esto representa ventajas operativas muy puntuales en el trato y el acuerdo de intereses. 
\end{justification}

\begin{goals}
\item Otorgar herramientas b�sicas para presentarse y conversar en la lengua quechua, en la variedad chanca.
\item Acercar al estudiante a las estructuras b�sicas de la lengua con el fin de dirigir su estudio y auto aprendizaje.
\item Entrenar al alumno en la traducci�n y producci�n de textos en la lengua nativa.
\item Proporcionar herramientas para que el alumno desarrolle el conocimiento de esta lengua de manera individual.
\item Dar herramientas para reconocer la procedencia del quechua al cual se enfrentan a trav�s de elementos de an�lisis ling��stico
\end{goals}

\begin{outcomes}
\ExpandOutcome{a}{3}
\ExpandOutcome{i}{2}
\ExpandOutcome{j}{4}
\end{outcomes}

\begin{unit}{Sistemas de coordenadas. La recta.}{Lehmann05}{12}{4}
   \begin{topics}
      \item 
      \item 
   \end{topics}
   \begin{unitgoals}
      \item 
   \end{unitgoals}
\end{unit}

\begin{unit}{C�nicas y Coordenadas polares}{Lehmann05}{24}{3}
   \begin{topics}
      \item 
      \item 
   \end{topics}

   \begin{unitgoals}
      \item 
      \item
      \item 
      \end{unitgoals}
\end{unit}

\begin{unit}{Sistemas de ecuaciones. Matrices y determinantes}{Strang03,Grossman96}{24}{3}
   \begin{topics}
      \item 
      \item 
      \item 
      \end{topics}

   \begin{unitgoals}
      \item 
      \item 
      \item 
     
   \end{unitgoals}
\end{unit}

\begin{unit}{Vectores en $R^2$ y vectores en $R^3$}{Grossman96}{30}{3}
   \begin{topics}
      \item 
      \item 
   \end{topics}

   \begin{unitgoals}
      \item 
      \item 
      \item 
   \end{unitgoals}
\end{unit}



\begin{coursebibliography}
\bibfile{GeneralH/GH0017}
\end{coursebibliography}

\end{syllabus}
