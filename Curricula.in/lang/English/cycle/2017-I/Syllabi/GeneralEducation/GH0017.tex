\begin{syllabus}

\course{GH0017. Introducción al Quechua}{Electivos}{GH0017}

\begin{justification}
The Quechua communicative course allows students to approach the practical use of the Andean language in their Chanca variety. This is one of the varieties of greater diffusion and model to approach other varieties of denominated Quechua southern or Quechua II. In addition, it is simple in its learning to share sounds with Castilian. Also, the course seeks to familiarize the student with the basic structures of this language, as well as with the translation and production of texts. The ultimate goal is to provide the basic learning tools so that the student can express himself at a basic and functional level, as well as lead and develop his own language learning.
We believe that speaking Quechua in certain situations where UTEC engineers have to develop is a very important advantage: native Quechua speakers practice a differentiated treatment with people who speak it because they feel that their tradition is being respected and, at the same time, Making an effort to engage in dialogue in their own language. This represents very specific operational advantages in the treatment and the agreement of interests. 
\end{justification}

\begin{goals}
\item Grant basic tools to introduce and converse in the Quechua language, in the Chanca variety.
\item Approach the student to the basic structures of the language in order to direct his study and self-learning.
\item Train the student in the translation and production of texts in the native language.
\item Provide tools for the student to develop the knowledge of this language individually.
\item Give tools to recognize the origin of Quechua that they face through elements of linguistic analysis.
\end{goals}

\begin{outcomes}
    \item \ShowOutcome{n}{2}
    \item \ShowOutcome{ñ}{2}
\end{outcomes}

\begin{competences}
    \item \ShowCompetence{C24}{n,ñ}
\end{competences}

\begin{unit}{}{Semana 1.}{Zariquiey08}{12}{4}
   \begin{topics}
      \item History of Quechua: a brief overview.
      \item Dialectology.
      \item Greetings and basic questions.
   \end{topics}

   \begin{learningoutcomes}
      \item .
   \end{learningoutcomes}
\end{unit}

\begin{unit}{}{Semana 2.}{Zariquiey08}{24}{3}
   \begin{topics}
      \item Phonological system.
      \item Review of sound materials.
   \end{topics}

   \begin{learningoutcomes}
      \item . 
<<<<<<< HEAD
      \end{learningoutcomes}
=======
   \end{learningoutcomes}
>>>>>>> 7330f789e7e986f683bfb3bd497dd3b1c0f4aa6b
\end{unit}

\begin{unit}{}{Semana 3}{Zariquiey08}{24}{3}
   \begin{topics}
      \item Morphology.
      \item Formulation of questions. Noun phrase.
      \item Basic Quechua language requests.
   \end{topics}

   \begin{learningoutcomes}
      \item .
     
   \end{learningoutcomes}
\end{unit}

\begin{unit}{}{Semana 4.}{Adelaar77}{30}{3}
   \begin{topics}
      \item Review of nominal phrase and grammatical topics.
   \end{topics}

   \begin{learningoutcomes}
      \item .
   \end{learningoutcomes}
\end{unit}

\begin{unit}{}{Semana 5.}{Adelaar77}{30}{3}
   \begin{topics}
      \item Verbal sentence
      \item Use of multimedia materials
      \item Basic requests
   \end{topics}

   \begin{learningoutcomes}
      \item .
   \end{learningoutcomes}
\end{unit}

\begin{unit}{}{Semana 6.}{Zariquiey08}{30}{3}
   \begin{topics}
      \item Verbal sentence.
      \item Use of multimedia materials.
   \end{topics}

   \begin{learningoutcomes}
      \item .
   \end{learningoutcomes}
\end{unit}

\begin{unit}{}{Semana 7.}{Adelaar77}{30}{3}
   \begin{topics}
      \item Review of what has been seen up to this point.
   \end{topics}

   \begin{learningoutcomes}
      \item .
   \end{learningoutcomes}
\end{unit}

\begin{unit}{}{Semana 8.}{Adelaar77}{30}{3}
   \begin{topics}
      \item WEEK OF PARTIAL EXAMS.
   \end{topics}

   \begin{learningoutcomes}
      \item .
   \end{learningoutcomes}
\end{unit}

\begin{unit}{}{Semana 9.}{Zariquiey08}{30}{3}
   \begin{topics}
      \item Interview preparation.
   \end{topics}

   \begin{learningoutcomes}
      \item .
   \end{learningoutcomes}
\end{unit}

\begin{unit}{}{Semana 10.}{Adelaar77}{30}{3}
   \begin{topics}
      \item Morfología deverbativa. XYZ
      \item Speech morphology
   \end{topics}

   \begin{learningoutcomes}
      \item .
   \end{learningoutcomes}
\end{unit}

\begin{unit}{}{Semana 11.}{Adelaar77}{30}{3}
   \begin{topics}
      \item Translation exercises.
      \item Conversation exercises.
   \end{topics}

   \begin{learningoutcomes}
      \item . 
   \end{learningoutcomes}
\end{unit}

\begin{unit}{}{Semana 12.}{Zariquiey08}{30}{3}
   \begin{topics}
      \item Interview preparation.
   \end{topics}

   \begin{learningoutcomes}
      \item .
   \end{learningoutcomes}
\end{unit}

\begin{unit}{}{Semana 13.}{Adelaar77}{30}{3}
   \begin{topics}
      \item Group work: drafting a script.
   \end{topics}

   \begin{learningoutcomes}
      \item .
   \end{learningoutcomes}
\end{unit}

\begin{unit}{}{Semana 14.}{Zariquiey08}{30}{3}
   \begin{topics}
      \item Preparation and presentation of exhibitions.
   \end{topics}

   \begin{learningoutcomes}
      \item .
   \end{learningoutcomes}
\end{unit}

\begin{unit}{}{Semana 15.}{Adelaar77}{30}{3}
   \begin{topics}
      \item Preparation and presentation of exhibitions.
   \end{topics}

   \begin{learningoutcomes}
      \item .
   \end{learningoutcomes}
\end{unit}

\begin{unit}{}{Semana 16.}{Adelaar77}{30}{3}
   \begin{topics}
      \item WEEK OF FINAL EXAMS.
   \end{topics}

   \begin{learningoutcomes}
      \item .
   \end{learningoutcomes}
\end{unit}



\begin{coursebibliography}
\bibfile{GeneralEducation/GH0017}
\end{coursebibliography}

\end{syllabus}
