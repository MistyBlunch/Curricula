\begin{syllabus}

\course{ET301. Formación de Empresas de Base Tecnológica II}{Obligatorio}{ET301} % Common.pm

\begin{justification}
Este curso tiene como objetivo dotar al futuro profesional de conocimientos, actitudes y aptitudes que le permitan formar su propia empresa de desarrollo de software y/o consultorTecnologíaa en informática. El curso está dividido en tres unidades: Valorización de Proyectos, Marketing de Servicios y Negociaciones. En la primera unidad se busca que el alumno pueda analizar y tomar decisiones en relación a la viabilidad de un proyecto y/o negocio.

En la segunda unidad se busca preparar al alumno para que este pueda llevar a cabo un plan de marketing satisfactorio del bien o servicio que su empresa pueda ofrecer al mercado. La tercera unidad busca desarrollar la capacidad negociadora de los participantes a través del entrenamiento vivencial y práctico y de los conocimientos teóricos que le permitan cerrar contrataciones donde tanto el cliente como el proveedor resulten ganadores. Consideramos estos temas sumamente crTecnologíaticos en las etapas de lanzamiento, consolidación y eventual relanzamiento de una empresa de base tecnológica.
\end{justification}

\begin{goals}
\item Que el alumno comprenda y aplique la terminologTecnologíaa y conceptos fundamentales de ingeniería económica que le permitan valorizar un proyecto para tomar la mejor decisión económica.
\item Que el alumno adquiera las bases para formar su propia empresa de base tecnológica.
\end{goals}

%% (1) familiar  (2)usar (3)evaluar
\begin{outcomes}{V1}
    \item \ShowOutcome{d}{2}
    \item \ShowOutcome{f}{2}
    \item \ShowOutcome{m}{3}
\end{outcomes}

\begin{competences}{V1}
    \item \ShowCompetence{C17}{f} 
    \item \ShowCompetence{C18}{d}
    \item \ShowCompetence{C19}{m}
    \item \ShowCompetence{C20}{m}
    \item \ShowCompetence{C21}{m}
    \item \ShowCompetence{C22}{m}
    \item \ShowCompetence{C23}{m}
    \item \ShowCompetence{C24}{m}
\end{competences}

%% Nivel = 1(Familiarity),  2(Usage),  3(Assessment) 
\begin{unit}{}{Valorización de Proyectos}{blank06}{20}{C19}
\begin{topics}
      \item Introducción
       \item Proceso de toma de decisiones
       \item El valor del dinero en el tiempo
       \item Tasa de interés y tasa de rendimiento
       \item Interés simple e interés compuesto
       \item Identificación de costos
       \item Flujo de Caja Neto
       \item Tasa de Retorno de Inversión (TIR)
      \item Valor Presente Neto (VPN)
       \item Valorización de Proyectos
   \end{topics}
   \begin{learningoutcomes}
      \item Permitir al alumno tomar decisiones sobre como invertir mejor los fondos disponibles, fundamentadas en el análisis de los factores tanto económicos como no económicos que determinen la viabilidad de un emprendimiento. [\Assessment]
   \end{learningoutcomes}
\end{unit}

\begin{unit}{}{Marketing de Servicios}{kotler06,love09}{30}{C20}
\begin{topics}
      \item Introducción
      \item Importancia del marketing en las empresas de servicios
      \item El Proceso estratégico.
      \item El Plan de Marketing
      \item Marketing estratégico y marketing operativo
      \item Segmentación, targeting y posicionamiento de servicios en mercados competitivos
      \item Ciclo de vida del producto
       \item Aspectos a considerar en la fijación de precios en servicios
       \item El rol de la publicidad, las ventas y otras formas de comunicación
      \item El comportamiento del consumidor en servicios
      \item Fundamentos de marketing de servicios
      \item Creación del modelo de servicio
      \item Gestión de la calidad de servicio
   \end{topics}
   \begin{learningoutcomes}
      \item Brindar las herramientas al alumno para que pueda identificar, analizar y aprovechar las oportunidades de marketing que generan valor en un emprendimiento. [\Usage]
      \item Lograr que el alumno conozca, entienda e identifique criterios, habilidades, métodos y procedimientos que permitan una adecuada formulación de estrategias de marketing en sectores y medios especTecnologíaficos como lo es una empresa de base tecnológica. [\Usage]
   \end{learningoutcomes}
\end{unit}

\begin{unit}{}{Negociaciones}{fish96,dasi06}{10}{C18}
\begin{topics}
      \item Introducción. ?`Qué es una negociación?
      \item TeorTecnologíaa de las necesidades de la negociación
      \item La proceso de la negociación
      \item Estilos de negociación
      \item TeorTecnologíaa de juegos
      \item El método Harvard de negociación
   \end{topics}
   \begin{learningoutcomes}
      \item Conocer los puntos clave en el proceso de negociación. [\Usage]
      \item Establecer una metodologTecnologíaa de negociación eficaz. [\Usage]
      \item Desarrollar destrezas y habilidades que permitan llevar a cabo una negociación exitosa. [\Usage]
   \end{learningoutcomes}
\end{unit}





\begin{coursebibliography}
\bibfile{Enterpreneurship/ET301}
\end{coursebibliography}

\end{syllabus}

%\end{document}
