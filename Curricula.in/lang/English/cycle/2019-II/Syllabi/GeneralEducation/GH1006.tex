
\begin{syllabus}

\course{GH0006. Communication Laboratory II}{Obligatorio}{GH0006}
% Source file: ../Curricula.in/lang/English/cycle/2019-II/Syllabi/GeneralEducation/GH1006.tex

\begin{justification}
This laboratory is oriented to consolidate the student's communicative skills, both oral and written in the framework of the discipline under study. In particular, the student will strengthen his / her expositive abilities by exercising throughout the first part of the course in writing a type of text that
will develop throughout his career as an engineer: laboratory reports. He will reflect on the rhetorical situation he faces when writing this type of text: who will be his reader, what is the communicative intention of that text and the subject on which he is writing.
In a second part, the course is presented as a space for discussion about argumentative discourse and critical reading of argumentative texts, so that the student reflects, knows and uses the communicative tools to produce formal argumentative texts. In this sense, the course is oriented towards the production
Permanent written and oral texts, so that the student will participate not only in discussion forums but is expected to be able to discuss with his colleagues on a topic proposed by the teacher. In short, the course seeks to consolidate the skills of reading, analysis and preparation of written and oral texts, both expository and argumentative.
\end{justification}

\begin{goals}
\item Develop skills that enable students to improve their communication skills, both oral and written.
\item Understand and produce expository texts in which they report on the application of theoretical knowledge in a different experiment or context.
\item Understand and produce oral and written argumentative texts.
\item Be able to discuss using solid arguments.
\item Use appropriately and reflexively the information obtained from different sources.
\item Show openness and respect to listen to the diversity of opinions or points of view of classmates
\end{goals}

\begin{outcomes}{V1}
    \item \ShowOutcome{i}{2}
    \item \ShowOutcome{f}{2}
\end{outcomes}

\begin{competences}{V1}
    \item \ShowCompetence{C24}{n,ñ}
\end{competences}

\begin{unit}{Communication Laboratory II}{}{Cassany08}{12}{C24}
   \begin{topics}
      \item What is lab report ?
      \item Laboratory development or methodology.
      \item Laboratory results and apploications.
      \item Introduction and conclusions.
      \item Quotation, parenthetical references and bibliography construction.
      \item Review characteristics of orality.
      \item Presentation of an argumentative text:formal text and informal texts.
      \item How do you build an argument?
      \item Pragmatic argument.
      \item Quotation,references and APA format.
      \item Counter argumentation 
      
   \end{topics}
   \begin{learningoutcomes}
      \item Appropriately handle the system of citation and bibliographical references,and recognize the importance of it is use.
   \end{learningoutcomes}
\end{unit}

\begin{coursebibliography}
\bibfile{GeneralEducation/GH1006}
\end{coursebibliography}

\end{syllabus}
