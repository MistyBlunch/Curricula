\section{Impacto sobre la carrera de Ingeniería de Sistemas}\label{sec:impacto-en-sistemas}
Como ya fue explicado en secciones anteriores, el foco es formar
profesionales orientados a generar nueva tecnología computacional.
Este profesional está muy ligado a la investigación y tiene la
computación \underline{como un fin} en si misma.

Por otro lado, en las definiciones en internet de la carrera de Ingeniería de Sistemas en nuestro país está definida como \footnote{La
dirección es http://www.unsa.edu.pe/ y luego se entra a los enlaces:
Área académica->Estudios Pre-Grado->Ingenierías->Ing. de Sistemas}:

{\it El Ingeniero de Sistemas es el profesional capaz de analizar, diseñar, investigar, desarrollar y administrar todo tipo de sistemas, aplicando las ciencias básicas, las tecnologías de la información y comunicaciones, y la Teoría General de Sistemas.

Fundamentalmente, es capaz de integrar y optimizar los recursos organizacionales para la adecuada toma de decisiones, además de especificar y desarrollar software base y de aplicación generando tecnología nacional.}

El campo ocupacional que se observa en la misma página menciona que:

{\it El Ingeniero de Sistemas está capacitado para desempeñarse en el sector publico ó privado en las áreas de producción y/o servicios, desarrollando las siguientes actividades :

\begin{itemize}
    \item Planear, analizar, diseñar, implementar y administrar sistemas de información y organizacionales.
    \item Analizar, diseñar, desarrollar e implementar software específico.
    \item Administración y gestión de Centros de Computo e Informática.
    \item Administración Total de Proyectos y Sistemas complejos.
    \item Docencia e Investigación. o Asesoría y Consultoría.
\end{itemize}
}

En el texto anterior, se puede observar que la orientación de la carrera de Ingeniería de Sistemas desea utilizar a la computación \underline{como un medio} para mejorar el desempeño de las organizaciones. El objetivo del Ingeniero de Sistemas peruano es el de \underline{aplicar} y llevar la tecnología computacional a las empresas.

La orientación actual encaja mucho mejor con el estandar
internacional denominado Sistemas de Información que también tiene
un documento propuesto por IEEE-CS, ACM y la ac{AIS} en
\cite{InformationSystems2002Journal}. Consideramos que este perfil debe ser
fortalecido ya que es muy necesario en las empresas de la localidad.

Si consideramos que la carrera de Ingeniería de Sistemas se
orientaría al área de Sistemas de Información, ambas carreras serán
complementarias y podrían dar mayor atención a sus respectivos
campos.

Por un lado, los profesionales de Ciencia de la Computación están orientados a generar empresas
de base tecnológica, generar tecnología de punta y expandir las
fronteras del conocimiento en el área de computación. Tan importante
como el anterior, está el profesional en Sistemas de Información
orientado a llevar y aplicar esa tecnología al entorno organizacional. Este profesional está orientado al uso de
herramientas tecnológicas para la mejora de la productividad de la empresa.

Es necesario también tomar en cuenta que la carrera de Sistemas de Información es tradicionalmente formada en facultades de administración y de negocios en todo el mundo. Esto demuestra su fuerte tendencia al ámbito empresarial. Ambos perfiles han crecido tanto en las últimas décadas que ya se han desprendido en carreras separadas con sus propios cuerpos de conocimiento bien definidos.

Por las razones antes expuestas, consideramos que ambas carreras son necesarias en la misma proporción. La falta de la carrera de Ciencia de la Computación dificulta enormemente el desarrollo de la industria de software nacional. Por otro lado, si no tuvieramos la carrera de Sistemas de Información, tendríamos serias dificultades para satisfacer los requerimientos informáticos de las empresas. Hoy en dia, el campo de acción de ambas carreras ha crecido tanto que ya no es posible pensar en un solo profesional que las abarque bien, los necesitamos a ambos.

El efecto sobre la carrera de Ingeniería de Sistemas debe ser positivo desde el punto de vista de ayudar a fortalecer un perfil internacional que permita apuntar a una acreditación. Consideramos que ese perfil es claramente el de Sistemas de Información y, bajo esa perspectiva, la nueva carrera ayudará en su fortalecimiento.