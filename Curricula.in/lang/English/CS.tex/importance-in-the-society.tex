\section{Importancia de la carrera en la sociedad}\label{sec:importance-in-the-society}
Uno de los caminos que se espera que siga un profesional del área de computación es 
que el se dedique a producir software o que se integre a las empresas productoras 
de software. En el ámbito de la computación, es común observar que los países 
cuentas con Asociaciones de Productores de Software cuyas políticas están 
orientadas a la exportación. Siendo así, no tendría sentido preparar a nuestros 
alumnos sólo para el mercado local o nacional. Nuestros egresados deben estar 
preparados para desenvolverse en el mundo globalizado que nos ha tocado vivir.

Nuestros futuros profesionales deben estar orientados a crear nuevas empresas 
de base tecnológica que puedan incrementar las exportaciones de software peruano. 
Este nuevo perfil está orientado a generar industria innovadora. Si nosotros somos 
capaces de exportar software competitivo también estaremos en condiciones de 
atraer nuevas inversiones. Las nuevas inversiones generarían más puestos de 
empleo bien remunerados y con un costo bajo en relación a otros tipos de 
industria. Bajo esta perspectiva, podemos afirmar que esta carrera será un 
motor que impulsará al desarrollo del país de forma decisiva con una inversión 
muy baja en relación a otros campos.

Es necesario recordar que la mayor innovación de productos comerciales de versiones 
recientes utiliza tecnología que se conocía en el mundo académico hace 20 años o más. 
Un ejemplo claro son las bases de datos que soportan datos y consultas espaciales 
desde hace muy pocos años. Sin embargo, utilizan estructuras de datos que ya 
existían hace algunas décadas. Es lógico pensar que la gente del área académica 
no se dedique a estudiar en profundidad la última versión de un determinado 
software cuando esa tecnología ya la conocían hace mucho tiempo. Por esa misma 
razón es raro en el mundo observar que una universidad tenga convenios con una 
transnacional de software para dictar solamente esa tecnología pues, nuestra 
función es generar esa tecnología y no sólo saber usarla.

Tampoco debemos olvidar que los alumnos que ingresan hoy saldrán al mercado 
dentro de 5 años aproximadamente y, en un mundo que cambia tan rápido, no podemos 
ni debemos enseñarles tomando en cuenta solamente el mercado actual. 
Nuestros profesionales deben estar preparados para resolver los problemas 
que habrá dentro de 10 o 15 años y eso sólo es posible a través de la 
investigación.
