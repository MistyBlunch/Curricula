\section{Tendencias del Conocimiento}\label{sec:cs-tendencias-del-conocimiento}
En un mundo globalizado, como el que vivimos, es indispensable ajustarse a los est�ndares internacionales. En esta carrera espec�fica, el est�ndar internacional est� claramente liderado por la \textit{Computing Curricula} que fue propuesta por IEEE-CS/ACM antes mencionada.

El tema de ajuste a est�ndares internacionales toma especial importancia debido al proceso de acreditaci�n que est� entrando cada vez con mayor fuerza en nuestro pa�s. La acreditaci�n exige niveles de calidad que incluyen el ajuste a normas y nomenclatura internacional.

Recientemente, el Colegio de Ingenieros del Per� emiti� un informe titulado: ``Denominaciones y perfiles de las carreras de Ingenier�a de Sistemas, Computaci�n e Inform�tica" \cite{CIPInforme2006}. En este informe se observa una clara tendencia del CIP hacia los est�ndares internacionales ya mencionados en este documento. Este informe es, sin duda alguna, un gran paso para nuestro pa�s y nuestra carrera.

En nuestro pa�s, la llegada inminente del Tratado de Libre Comercio nos da cada vez menos tiempo para ajustarnos a un nivel de competitividad internacional. Si no hacemos esto estamos en grave peligro de dejar escapar oportunidades muy valiosas. 

Por otro lado, debemos considerar que el �rea de la computaci�n ha crecido de forma exponencial especialmente debido a la expansi�n de Internet. Siendo as�, estamos en un �rea que no conoce fronteras internacionales en ning�n aspecto.

En cuanto a nuestro pa�s, la entidad que agrupa a las empresas dedicadas a la producci�n de software es la \ac{APESOFT}. Esta asociaci�n ha tomado como pol�tica principal dedicarse a la producci�n de software para \underline{exportaci�n}. Siendo as�, no tendr�a sentido preparar a nuestros alumnos s�lo para el mercado local o nacional. Nuestros egresados deben estar preparados para desenvolverse en el mundo globalizado que nos ha tocado vivir.

Debido a estas consideraciones, es claro que la tendencia de esta carrera es tener como referencia al mundo y no solamente a nuestro pa�s. En la medida que esto sea posible estaremos creando las condiciones para atraer inversiones extranjeras relacionadas al desarrollo de software a nivel internacional.
