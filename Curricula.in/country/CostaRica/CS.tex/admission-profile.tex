\section{Requisitos de ingreso y graduaci�n}
\subsection{Requisitos de ingreso}
La carrera de Ciencia de la Computaci�n est� dirigida a todas aquellas personas que deseen obtener un grado universitario y hayan logrado alcanzar el Bachillerato en Educaci�n Media.

Para ingresar a la carrera se requiere:
\begin{itemize}
\item Presentar el original y dos copias del T�tulo de Bachillerato en Educaci�n Media.
Cuando se trate de un t�tulo fuera del pa�s, se deben presentar los documentos debidamente autenticados y cumplir los requisitos de legalidad que solicita Costa Rica.

\item Aportar dos fotograf�as recientes, tama�o pasaporte.
\item Llenar la f�rmula de \underline{Solicitud de matr�cula} y aportar los dem�s documentos requeridos para abrir su \underline{Expediente de estudiante}.

\item Cancelar la suma correspondiente a los derechos de matr�cula, colegiatura y dem�s pagos que deba hacer de acuerdo con los tr�mites que haya tenido que efectuar.

\item Aceptar las normas universitarias existentes, en particular el Estatuto Org�nico y los Reglamentos de la Instituci�n.
\end{itemize}

\subsection{Requisitos de graduaci�n}
Para optar por el grado acad�mico de \underline{Licenciatura en Ciencia de la Computaci�n}, el estudiante debe cumplir con los siguientes requisitos:

\begin{itemize}
\item Aprobar el plan de estudio que exige el programa de licenciatura, con una nota igual o mayor a 70 puntos en una escala de 1 a 100.

\item Alcanzar el nivel B2 del Test TOEIC en los �reas de {\it Listening} y {\it Reading} (ToeicListening: 400 y ToeicReading: 385) o dominio instrumental de la Lengua Portuguesa.

\item Realizar el trabajo final de graduaci�n para lo cual existen dos alternativas: la tesis� el proyecto de graduaci�n el cual se debe aprobar con una nota igual o mayor a 80 puntos en una escala de 1 a 100.

\item Aprobar el programa de Trabajo Comunal Universitario.

\item Presentar la solicitud de graduaci�n escrita ante el Decano del �rea de Docencia, por medio del Registro de la Universidad.

\item Tener cumplidas las obligaciones financieras con la Universidad, incluyendo los aranceles de graduaci�n.
\end{itemize}
