\section{Antecedentes}
El estado académico que viven las Universidades en el Ecuador, es cada día  una de las mayores preocupaciones de control, 
administración y servicio, por su condición poco objetiva de sus programas, que estacionan carreras profesionales 
sin cambios o innovaciones  en su estructura educativa, promulgando un sistema de formación profesional sin 
mayor sorpresa o  respuesta para los  sectores productivos y sociales.

En esta época de cambio, las Universidades y en especial la \University, admite 
qué  su estado  académico requiere estudios y evaluaciones en todos sus niveles de actuación,  
que permita proponer   innovaciones  o cambios  en sus programas y que al mismo tiempo, 
faciliten el encuentro con las necesidades del sector empresarial y social, 
para que su actividad académica se fundamente en una lógica de conciencia y 
expresión ciudadana, con el propósito de solventar un desarrollo armónico y 
sustentable en cada escenario de servicio que  oferta al sector productivo. 

La población del Ecuador, según el censo de noviembre del 2001 fue de 12'156,608, 
según proyecciones en el 2005 será de 13'798,000 habitantes con un crecimiento 
poblacional de  del 1.74\% (CEPAR: Estimación  y proyecciones de población).

El Ecuador ha dejado desde hace mucho tiempo atrás de ser un país rural, 
en 1950 la población rural representaba el 72\%; en el 2001 fue del 38\%. 
En cambio la población urbana en la misma fecha fue de 7'372,528 y la rural de 
4'718,276 (INEC, VI Censo  de  Población  y vivienda 2001. En el 2005 dos de 
cada tres Ecuatoriano vivirán en la ciudad)

La población proyectada para el 2010 será de 14'899,000, repartida en el 
sector rural 4'649,000 y el la urbana 10'250,000. (CEPAR  proyecciones de población)

La población se concentra principalmente en la costa y en la sierra En la
primera existe un total de 6'056,223  y en la segunda 5'460,738 habitantes.

La Provincia de Manabí de acuerdo al censo 2001, su población es de 1'186,025
habitantes y Portoviejo en esa fecha tenía 171,847 habitantes.

Manabí en el 2001 tenía cinco universidades con estructuras funcionales y 
marco jurídico, que asumían la responsabilidad en un 95\% en su matriz, en 
5\% con extensiones, paralelos o cursos en varios cantones de la provincia.

Somos una Universidad con principios  filosóficos  en crecimiento. 
Durante ocho años nos hemos preocupado por lograr estabilidad y desarrollo 
académico, para lo cual, nos hemos  relacionado con los de safios de la 
ciencia y la tecnología, con la dinámica del proceso que requiere la 
formación profesional, con los espacios de la investigación y vinculación social, 
solo en esas circunstancias, hemos resaltado nuestra identidad de proyección y 
contribución al desarrollo de la sociedad y sus componentes de actuación.

La \University, actualmente tiene en funcionamiento 16 carreras 
profesionales con exitosa aceptación, tanto en su  parte académica como en su 
derivación profesional;  de tal manera, que nuestras ofertas  cubren las 
principales expectativas de la comunidad educativa y del sector empresarial.



Estamos empeñados en seguir contribuyendo al desarrollo social y empresarial, 
para ello, hemos crecido en posturas académicas, en investigación y vinculación 
con la comunidad, mantenemos innovadas propuestas académicas, acuerdos y convenios 
con instituciones de servicio social y productivo, prestamos servicios 
profesionales a la comunidad, a través de las áreas técnica, empresarial, 
turismo, cultura, social y otras  que nos ha permitido el reconocimiento de la comunidad.

Nuestra actividad académica mantiene espacios importante de análisis, 
reflexión y producción de conocimientos a través de investigación, nos 
hemos preocupado por mejorar los espacios físicos que hacen el desarrollo 
educativo tanto en el eje teórico, práctico y de investigación.  Mantenemos 
un plan de capacitación y actualización pedagógica para los docentes y un en 
los actuales momentos nos preparamos para desarrollar el proyecto de 
auto-evaluación  institucional con fines a la evaluación y acreditación.

Estos referentes  académicos e institucional, nos  acentúa para
elaborar nuevos propuestas  de formación profesional, que aporten al 
desarrollo social, productivo a través avances  técnicos  y tecnológicos, 
que permitan ser más competitivas y competentes las instituciones, para ello, 
hemos diseñado una propuesta técnica académica que forme profesionales en 
Ciencia de la Computación, como respuestas a los 
grandes desafíos de globalización, organización y comunicación.
