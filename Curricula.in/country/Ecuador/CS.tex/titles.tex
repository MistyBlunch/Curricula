\section{Título que otorga}\label{sec:cs-grados-y-titulos}
Cumplida la aprobación de los eventos académicos propuestos en el \textit{pensum} 
de estudios y luego de haber realizado y aprobado su trabajo de investigación 
operativo (tesis de grado) y habiendo cumplido con las obligaciones 
administrativas, la respectiva solvencia  institucional, 
recibirá el título profesional universitario de:

\GradosyTitulos

El trabajo de investigación lo iniciará en el séptimo semestre y podrá 
denunciar su tema una vez cumplido el 80\% del programa académico, 
para  lo cual asistirá a un seminario taller de diseño de proyecto, 
gestión y elaboración de informe de investigación, con una duración de 
10 créditos (160 horas). El proyecto o propuesta de investigación 
será sustentado y aprobado por un tribual, dando la validez para su 
trabajo de campo y demás actividades y procesos académicos que exige 
la carrera profesional. 