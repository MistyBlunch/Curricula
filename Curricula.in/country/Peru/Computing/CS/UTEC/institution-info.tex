\newcommand{\DocumentVersion}{2010}
\newcommand{\fecha}{20 de Abril de 2016}
\newcommand{\city}{Lima\xspace}
\newcommand{\country}{Perú\xspace}
\newcommand{\dictionary}{Español\xspace}
\newcommand{\SyllabusLangs}{Español,English}
\newcommand{\GraphVersion}{2\xspace}

% newcommand{\CurriculaVersion}{1\xspace} % Malla 2006: 1, Malla 2010: 2
% newcommand{\YYYY}{2006\xspace}          % Plan 2006
% newcommand{\Range}{8-10}                % Plan 2010 1-8, Plan 2006 7-10
% newcommand{\SchoolShortName}{Ingeniería Informática\xspace}
% newcommand{\SchoolFullName}{Escuela Profesional de \SchoolShortName}
% newcommand{\SchoolFullNameBreak}{\FacultadName\\ Escuela Profesional de \\\SchoolShortName\xspace}

\newcommand{\CurriculaVersion}{2016\xspace} % Malla 2006: 1, Malla 2010: 2
\newcommand{\YYYY}{2017\xspace}          % Plan 2006
\newcommand{\Range}{1-10}                % Plan 2010 1-8, Plan 2006 7-10
\newcommand{\SchoolShortName}{Ciencia de la Computación\xspace}
\newcommand{\SchoolFullName}{Escuela Profesional de \SchoolShortName}
\newcommand{\SchoolFullNameBreak}{\FacultadName\\ Escuela Profesional de \\\SchoolShortName\xspace}

\newcommand{\Semester}{2017-I\xspace}      

% convert ./fig/UTEC.jpg ./html/img3.png
% cp ./fig/big-graph-curricula.png ./html/img18.png
% convert ../Curricula2.0.out/Peru/CS-UTEC/cycle/2014-1/Plan2010/fig/UTEC.jpg ../Curricula2.0.out/Peru/CS-UTEC/cycle/2014-1/Plan2010/html/img3.png
% cp ../Curricula2.0.out/Peru/CS-UTEC/cycle/2014-1/Plan2010/fig/big-graph-curricula.png ../Curricula2.0.out/Peru/CS-UTEC/cycle/2014-1/Plan2010/html/img18.png

\newcommand{\OutcomesList}{a,b,c,d,e,f,g,h,i,j,k,l,m,HU,FH,TASDSH}
\newcommand{\logowidth}{20cm}

\newcommand{\University}{Universidad de Ingeniería e Tecnología\xspace}
\newcommand{\InstitutionURL}{\htmladdnormallink{http://www.utec.edu.pe}{http://www.utec.edu.pe}\xspace}
\newcommand{\underlogotext}{}
\newcommand{\FacultadName}{Facultad de Ingeniería y Computación\xspace}
\newcommand{\DepartmentName}{Ciencia de la Computación\xspace}
\newcommand{\SchoolAcro}{EPCC\xspace}
\newcommand{\SchoolURL}{\htmladdnormallink{http://cs.utec.edu.pe}{http://cs.utec.edu.pe}\xspace}

\newcommand{\GradoAcademico}{Bachiller en Ciencia de la Computación\xspace}
\newcommand{\TituloProfesional}{Licenciado en Ciencia de la Computación\xspace}
\newcommand{\GradosyTitulos}%
{\begin{description}%
\item [Grado Académico: ] \GradoAcademico\xspace y% 
\item [Titulo Profesional: ] \TituloProfesional%
\end{description}%
}

\newcommand{\doctitle}{Plan Curricular \YYYY\xspace del \SchoolFullName\\ \SchoolURL}

\newcommand{\AbstractIntro}{Este documento representa el informe final de la nueva 
malla curricular \YYYY del \SchoolFullName de la \University (\textit{\InstitutionURL}) 
en la ciudad de \city-\country.}

\newcommand{\OtherKeyStones}%
{}

\newcommand{\profile}{%
El perfil profesional de este programa profesional puede ser mejor entendido a partir de
\OnlyMainDoc{la Fig. \ref{fig.cs} (Pág. \pageref{fig.cs})}\OnlyPoster{las figuras del lado derecho}. 
Este profesional tiene como centro de su estudio a la computación. Es decir, tiene a la computación 
como fin y no como medio. De acuerdo a la definición de esta área, este profesional está llamado 
directamente a ser un impulsor del desarrollo de nuevas técnicas computacionales que 
puedan ser útiles a nivel local, nacional e internacional.

Nuestro perfil profesional está orientado a ser generador de puestos de empleo a través de la innovación permanente. 
Nuestra formación profesional tiene 3 pilares fundamentales: 
Formación Humana, un contenido de acuerdo a normas internacionales y una orientación marcada a la innovación.
}


\newcommand{\mission}{La \University ... .\xspace}

\newcommand{\HTMLFootnote}{{Generado por <A HREF='http://socios.spc.org.pe/ecuadros/'>Ernesto Cuadros-Vargas</A> <ecuadros AT spc.org.pe>, <A HREF='http://www.spc.org.pe/'>Sociedad Peruana de Computación-Perú</A> <BR>basado en el modelo de la <I>Computing Curricula</I> de <A HREF='http://www.computer.org/'>IEEE-CS</A>/<A HREF='http://www.acm.org/'>ACM</A>}}

\newcommand{\Copyrights}{Generado por Ernesto Cuadros-Vargas (ecuadros AT spc.org.pe), Sociedad Peruana de Computación (http://www.spc.org.pe/) basado en la {\it Computing Curricula} de IEEE-CS (http://www.computer.org) y ACM (http://www.acm.org/)}
