\newcommand{\DocumentVersion}{V2.0}
\newcommand{\fecha}{\today}
\newcommand{\YYYY}{2010\xspace}
\newcommand{\Semester}{2010-1\xspace}
\newcommand{\city}{Arequipa\xspace}
\newcommand{\country}{Perú\xspace}
\newcommand{\dictionary}{Español\xspace}
\newcommand{\GraphVersion}{2\xspace}
\newcommand{\CurriculaVersion}{2\xspace}
\newcommand{\OutcomesList}{a,b,c,d,e,f,g,h,i,j,k,l,m,HU}
\newcommand{\logowidth}{7cm}

% convert ../Curricula2.0.out/Peru/CS-SPC/cycle/2010-1/Plan2010/fig/SPC.eps                 ../Curricula2.0.out/Peru/CS-SPC/cycle/2010-1/Plan2010/html/img3.png 
% cp      ../Curricula2.0.out/Peru/CS-SPC/cycle/2010-1/Plan2010/fig/big-graph-curricula.png ../Curricula2.0.out/Peru/CS-SPC/cycle/2010-1/Plan2010/html/img33.png

\newcommand{\University}{Sociedad Peruana de Computación\xspace}
\newcommand{\InstitutionURL}{http://www.spc.org.pe\xspace}
\newcommand{\underlogotext}{}

\newcommand{\FacultadName}{Computación\xspace}
\newcommand{\DepartmentName}{Ciencia de la Computación\xspace}
\newcommand{\SchoolFullName}{Programa Profesional de Ciencia de la Computación (\DocumentVersion)}
\newcommand{\SchoolFullNameBreak}{Facultad de Computación\\ Programa Profesional de\\ ({\bf Ciencia de la Computación}) (\DocumentVersion)\xspace}
\newcommand{\SchoolShortName}{({\bf Ciencia de la Computación})\xspace}
\newcommand{\SchoolAcro}{PPII\xspace}
\newcommand{\SchoolURL}{http://www.spc.org.pe/education/PCC/Peru/CS-SPC/}

\newcommand{\GradoAcademico}{Bachiller en Ciencia de la Computación\xspace}
\newcommand{\TituloProfesional}{Licenciado en Ciencia de la Computación\xspace}
\newcommand{\GradosyTitulos}%
{\begin{description}%
\item [Grado Académico: ] \GradoAcademico y%
\item [Título Profesional: ] \TituloProfesional%
\end{description}%
}

\newcommand{\doctitle}{Plan Curricular \YYYY\xspace del \SchoolFullName\\ \SchoolURL}

\newcommand{\AbstractIntro}{Este documento representa el informe final de la nueva malla curricular \YYYY del 
\SchoolFullName de la \University (\textit{\InstitutionURL}) en la ciudad de \city-Perú. 
En la actualidad esta carrera está siendo orientada a Ciencia de la Computación, 
el Bachillerato ya está con esta denominación y el título profesional está en proceso de cambio.}

\newcommand{\OtherKeyStones}%
{Un pilar que merece especial consideración en el caso de la \University 
es el aspecto de valores humanos, básicos y cristianos debido a que forman 
parte fundamental de los lineamientos básicos de la existencia de la institución.\xspace}

\newcommand{\profile}{%
El perfil profesional de este programa profesional puede ser mejor entendido a partir de
\OnlyMainDoc{la Fig. \ref{fig.cs} (Pág. \pageref{fig.cs})}\OnlyPoster{las figuras del lado derecho}. 
Este profesional tiene como centro de su estudio a la computación. Es decir, tiene a la computación 
como fin y no como medio. De acuerdo a la definición de esta área, este profesional está llamado 
directamente a ser un impulsor del desarrollo de nuevas técnicas computacionales que 
puedan ser útiles a nivel local, nacional e internacional.

Nuestro perfil profesional está orientado a ser generador de puestos de empleo a través de la innovación permanente. 
Nuestra formación profesional tiene 2 pilares fundamentales: 
Un contenido de acuerdo a recomendaciones internacionales y una orientación marcada a la innovación.
}

\newcommand{\HTMLFootnote}{{Generado por <A HREF='http://socios.spc.org.pe/ecuadros/'>Ernesto Cuadros-Vargas</A> <ecuadros AT spc.org.pe>, 
                           <A HREF='http://www.spc.org.pe/'>Sociedad Peruana de Computaci&oacute;n-Peru</A>, 
                           <A HREF='http://www.utec.edu.pe/'>Universidad de Ingenier&iacute;a y Tecnolog&iacute;a, Lima-Per&uacute;</A><BR>
                           basado en el modelo de la Computing Curricula de 
                           <A HREF='http://www.computer.org/'>IEEE-CS</A>/<A HREF='http://www.acm.org/'>ACM</A>}}

\newcommand{\Copyrights}{Generado por Ernesto Cuadros-Vargas (ecuadros AT spc.org.pe), 
                           Sociedad Peruana de Computación (http://www.spc.org.pe/), 
                           Universidad de Ingenier\'{i}a y Tecnolog\'{i}a (UTEC) (http://www.utec.edu.pe) \country 
                           basado en la {\it Computing Curricula} de 
                           IEEE-CS (http://www.computer.org) y ACM (http://www.acm.org/)}