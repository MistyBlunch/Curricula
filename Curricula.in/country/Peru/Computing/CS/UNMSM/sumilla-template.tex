\documentclass[final]{article}
\input{current-institution}
\usepackage{\InStyAllDir/syllabus-final}

\newcommand{\INST}{}
\newcommand{\AREA}{}

\begin{document}

\begin{tabularx}{\textwidth}{p{3cm}cX}
\includegraphics[width=3cm]{\InLogosDir/\INST} 
&
\begin{minipage}{0.75\textwidth}
\begin{center}
{\noindent\Large\bf\University\\ \SchoolFullNameBreak \\SILABO}
\section*{--COURSE_CODE--. --COURSE_NAME-- (--COURSE_TYPE--)}\label{sec:--COURSE_CODE--}%
\addcontentsline{toc}{subsection}{--COURSE_CODE--. --COURSE_NAME-- (--COURSE_TYPE--)}%
\end{center}
\end{minipage}
\end{tabularx}

\addtocounter{SyllabiSectionCount}{1}

\noindent \Semester
\begin{center}
\begin{tabularx}{\textwidth}{|p{5cm}cX|}\hline
\multicolumn{3}{|l|}{{\bf \arabic{SyllabiSectionCount}. DATOS GENERALES}}                \\ \hline
\arabic{SyllabiSectionCount}.1 CARRERA PROFESIONAL     & : & \SchoolShortName           \\
\arabic{SyllabiSectionCount}.2 ASIGNATURA              & : & --COURSE_CODE--. --COURSE_NAME-- \\
\arabic{SyllabiSectionCount}.3 SEMESTRE ACADÉMICO      & : & --SEMESTER--          \\
\arabic{SyllabiSectionCount}.4 PREREQUISITO(S)         & : & --PREREQUISITES--     \\
\arabic{SyllabiSectionCount}.5 CARÁCTER                & : & --COURSE_TYPE--              \\
\arabic{SyllabiSectionCount}.6 HORAS                   & : & --HOURS--          \\ 
\arabic{SyllabiSectionCount}.7 CRÉDITOS                & : & --CREDITS--          \\ \hline
\end{tabularx}
\end{center}

\addtocounter{SyllabiSectionCount}{1}
\begin{center}
\begin{tabularx}{\textwidth}{|X|}      \hline
{\bf \arabic{SyllabiSectionCount}. DOCENTE}\\ \hline
--PROFESSOR_SHORT_CVS--
\\ \hline
\end{tabularx}
\end{center}

\addtocounter{SyllabiSectionCount}{1}
\begin{center}
\begin{tabularx}{\textwidth}{|X|}      \hline
{\bf \arabic{SyllabiSectionCount}. FUNDAMENTACIÓN DEL CURSO}        \\ \hline
--JUSTIFICATION--                  \\ \hline
\end{tabularx}
\end{center}

\addtocounter{SyllabiSectionCount}{1}
\begin{center}
\begin{tabularx}{\textwidth}{|X|}      \hline
{\bf \arabic{SyllabiSectionCount}. SUMILLA}                         \\ \hline
--SHORT_DESCRIPTION--                         \\ \hline
\end{tabularx}
\end{center}

\addtocounter{SyllabiSectionCount}{1}
\begin{center}
\begin{tabularx}{\textwidth}{|X|}      \hline
{\bf \arabic{SyllabiSectionCount}. OBJETIVO GENERAL}                \\ \hline
--GOAL--                 \\ \hline
\end{tabularx}
\end{center}

\addtocounter{SyllabiSectionCount}{1}
\begin{center}
\begin{tabularx}{\textwidth}{|X|}      \hline
{\bf \arabic{SyllabiSectionCount}. CONTRIBUCIÓN A LA FORMACIÓN PROFESIONAL Y FORMACIÓN GENERAL} \\ \hline
\ContribInitMsg:
--OUTCOMES--\\ \hline
\end{tabularx}
\end{center}


\addtocounter{SyllabiSectionCount}{1}
\begin{center}
\begin{tabularx}{\textwidth}{|X|}      \hline
{\bf \arabic{SyllabiSectionCount}. CONTENIDOS}                      \\ \hline
\end{tabularx}
\end{center}

%%%%%%%%%%%%%%%%%%%%%%%%%%%%%%%%%%%%%%%%%%%%%%%%%%%%%%%%%%%%%%%%%%%%%%%%%%%%%%
\setcounter{SyllabiUnitCount}{0}
--BEGINUNIT--
% A new unit begins here
\addtocounter{SyllabiUnitCount}{1}
\begin{center}
\begin{tabularx}{\textwidth}{|X|X|}                 \hline
\multicolumn{2}{|l|}{{\bf UNIDAD \arabic{SyllabiUnitCount}: --UNIT_TITLE-- (--HOURS--)}} \\ \hline
\multicolumn{2}{|l|}{{\bf Nivel Bloom: --BLOOM_LEVEL--}} \\ \hline
{\bf OBJETIVO GENERAL}  & {\bf CONTENIDO}                    \\ \hline
--UNIT_GOAL--
& 
--UNIT_CONTENT--
\\ \hline
\multicolumn{2}{|l|}
{\begin{minipage}{0.95\textwidth}
{\bf Lecturas:} --CITATIONS--
\end{minipage}
}
\\ \hline
\end{tabularx}
\end{center}

--ENDUNIT--
%%%%%%%%%%%%%%%%%%%%%%%%%%%%%%%%%%%%%%%%%%%%%%%%%%%%%%%%%%%%%%%%%%%%%%%%%%%%%%

\addtocounter{SyllabiSectionCount}{1}
\begin{center}
\begin{tabularx}{\textwidth}{|X|}      \hline
\arabic{SyllabiSectionCount}. METODOLOGÍA  \\ \hline
\begin{evaluation}
	\item El profesor del curso presentará clases teóricas de los temas señalados en el programa propiciando la intervención de los alumnos. 
	\item El profesor del curso presentará demostraciones para fundamentar clases teóricas.
	\item El profesor y los alumnos realizarán prácticas
	\item Los alumnos deberán asistir a clase habiendo leído lo que el profesor va a presentar. 
	De esta manera se facilitará la comprensión y los estudiantes estarán en mejores condiciones de hacer consultas en clase.
\end{evaluation}
\\ \hline
\end{tabularx}
\end{center}

\addtocounter{SyllabiSectionCount}{1}
\begin{center}
\begin{tabularx}{\textwidth}{|X|}      \hline
\arabic{SyllabiSectionCount}. EVALUACIONES  \\ \hline
\begin{evaluation}
	\item[Evaluación Permanente 1] : 20 \%
	\item[Examen Parcial] : 30 \%
	\item[Evaluación Permanente 2] : 20 \%
	\item[Examen Final] : 30 \%
\end{evaluation}
\\ \hline
\end{tabularx}
\end{center}

\bibliographystyle{apalike}
\bibliography{--BIBFILE--}
\end{document}
