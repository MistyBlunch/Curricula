\newcommand{\DocumentVersion}{2014}
\newcommand{\fecha}{\today}
\newcommand{\Semester}{2014-1\xspace}
\newcommand{\city}{Lima\xspace}
\newcommand{\country}{Perú\xspace}
\newcommand{\dictionary}{Español\xspace}
\newcommand{\GraphVersion}{2\xspace}
\newcommand{\OutcomesList}{a,b,c,d,e,f,g,h,i,j,k,l,m,HU,FH,TASDSH}
\newcommand{\logowidth}{7cm}

\newcommand{\CurriculaVersion}{2\xspace}
\newcommand{\YYYY}{2014\xspace}
\newcommand{\Range}{1-10}

\newcommand{\University}{Universidad Nacional Mayor de San Marcos\xspace}
\newcommand{\InstitutionURL}{http://www.unmsm.edu.pe\xspace}
\newcommand{\underlogotext}{}
\newcommand{\FacultadName}{Facultad de Ciencias Matemáticas\xspace}
\newcommand{\DepartmentName}{Ciencia de la Computación\xspace}
\newcommand{\SchoolFullName}{Programa Profesional de Ciencia de la Computación}
\newcommand{\SchoolFullNameBreak}{\FacultadName\\ Escuela Académico Profesional de \\Ciencia de la Computación \\ Malla Curricular \DocumentVersion\xspace}
\newcommand{\SchoolShortName}{Ciencia de la Computación\xspace}
\newcommand{\SchoolAcro}{EAPCC\xspace}
\newcommand{\SchoolURL}{http://cs.unmsm.edu.pe}

\newcommand{\GradoAcademico}{Bachiller en Ciencia de la Computación\xspace}
\newcommand{\TituloProfesional}{Licenciado en Ciencia de la Computación\xspace}
\newcommand{\GradosyTitulos}%
{\begin{description}%
\item [Grado Académico: ] \GradoAcademico\xspace y% 
\item [Titulo Profesional: ] \TituloProfesional%
\end{description}%
}

\newcommand{\doctitle}{Plan Curricular \YYYY\xspace de la \SchoolFullName\\ \SchoolURL}

\newcommand{\AbstractIntro}{Este documento representa el informe final de la nueva 
malla curricular \YYYY del \SchoolFullName de la \University (\textit{\InstitutionURL}) 
en la ciudad de \city-\country.}

\newcommand{\OtherKeyStones}{}


\newcommand{\profile}{%
El perfil profesional de este programa profesional puede ser mejor entendido a partir de
\OnlyMainDoc{la Fig. \ref{fig.cs} (Pág. \pageref{fig.cs})}\OnlyPoster{las figuras del lado derecho}. 
Este profesional tiene como centro de su estudio a la computación. Es decir, tiene a la computación 
como fin y no como medio. De acuerdo a la definición de esta área, este profesional está llamado 
directamente a ser un impulsor del desarrollo de nuevas técnicas computacionales que 
puedan ser útiles a nivel local, nacional e internacional.

Nuestro perfil profesional está orientado a ser generador de puestos de empleo a través de la innovación permanente. 
Nuestra formación profesional tiene 3 pilares fundamentales: 
Formación Humana, un contenido de acuerdo a normas internacionales y una orientación marcada a la innovación.
}

\newcommand{\mission}{Contribuir al desarrollo científico, tecnológico y técnico del país,  
formando profesionales competentes, orientados a la creación de nueva 
ciencia y tecnología computacional, como motor que impulse y consolide la industria 
del software en base a la investigación científica y tecnológica en 
áreas innovadoras formando, EN NUESTROS profesionales, un conjunto de habilidades y 
destrezas para la solución de problemas computacionales con un compromiso social.\xspace}
