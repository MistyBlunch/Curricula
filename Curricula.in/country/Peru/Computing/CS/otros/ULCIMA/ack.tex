\chapter*{Agradecimientos}\label{chap:cs-ack}
\addcontentsline{toc}{chapter}{Agradecimientos}%

Deseamos también dejar nuestro agradecimiento a la Sociedad Peruana de Computación (SPC) por 
haber publicado el documento matriz de esta malla curriculas y a la Universidad Católica San Pablo (UCSP) 
por que su malla curricular también ha sido una referencia sólida para la construcción y adaptación de este modelo.

Además de los autores directos de este documento, también deseamos dejar manifiesto de nuestro 
agradecimiento a otros colegas de diversas universidades del paÍs y del mundo que gentilmente 
han aportado parte de su tiempo a darnos sus sugerencias. Entre ellos debemos mencionar a:

\begin{itemize}
\item Ernesto Cuadros-Vargas (UCSP-Perú), 
\item Luis Fernando DÍaz Basurco (UCSP-Perú), 
\item Nelly Condori-Fernández (UPV-España), 
\item AgustÍn Tornés (ITESM-CCM, México), 
\item Alex Cuadros-Vargas (UCSP, Perú),
\item Alvaro Cuno-Parari (SPC),
\item Alfredo Paz (UNSA, Perú), 
\item Rodrigo Lazo Paz (UCSP, Perú),
\item Juan Manuel Gutiérrez Cárdenas (\textit{University of the Witwatersrand}-Sud-Africa),
\item Lenin Henry Cari Mogrovejo por su valiosa colaboración en los aspectos de redacción y corrección ortográfica del documento.
\end{itemize}

Todo este equipo de trabajo asumió como premisa que el centro de nuestro esfuerzo, 
es la formación académica y humana \underline{de los estudiantes}.

A todos ellos deseamos agradecerles por su aporte que ha permitido generar 
este documento, úinico en su género en nuestro paÍs, que servirá para sentar las 
bases de una carrera más sólida en esta fantástica área que nos ha tocado estudiar y 
de la cual nos sentimos orgullosos de formar parte: \textbf{Computación}.
