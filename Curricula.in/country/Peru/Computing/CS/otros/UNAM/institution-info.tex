\newcommand{\DocumentVersion}{2010}
\newcommand{\fecha}{\today}
\newcommand{\city}{Ilo\xspace}
\newcommand{\country}{Perú\xspace}
\newcommand{\dictionary}{Español\xspace}
\newcommand{\GraphVersion}{2\xspace}

\newcommand{\CurriculaVersion}{2\xspace} % Malla 2006: 1, Malla 2010: 2
\newcommand{\YYYY}{2014\xspace}          % Plan 2006
\newcommand{\Range}{1-10}                % Plan 2010 1-7, Plan 2006 6-10

% newcommand{\CurriculaVersion}{2\xspace} % Malla 2006: 1, Malla 2010: 2
% newcommand{\YYYY}{2010\xspace}          % Plan 2006
% newcommand{\Range}{1-7}                 % Plan 2010 1-7, Plan 2006 6-10

\newcommand{\Semester}{2014-1\xspace}

% convert ./fig/UNAM.jpg ./html/img3.png
% cp ./fig/big-graph-curricula.png /html/img17.png
% cp ./fig/small-graph-curricula.png ./html/img18.png 

% convert ../Curricula2.0.out/Peru/CS-UNAM/cycle/2014-1/Plan2014/fig/UNAM.jpg ../Curricula2.0.out/Peru/CS-UNAM/cycle/2014-1/Plan2014/html/img3.png
% cp ../Curricula2.0.out/Peru/CS-UNAM/cycle/2014-1/Plan2014/fig/big-graph-curricula.png ../Curricula2.0.out/Peru/CS-UNAM/cycle/2014-1/Plan2014/html/img17.png
% cp ../Curricula2.0.out/Peru/CS-UNAM/cycle/2014-1/Plan2014/fig/small-graph-curricula.png ../Curricula2.0.out/Peru/CS-UNAM/cycle/2014-1/Plan2014/html/img18.png
% convert ../Curricula2.0.out/Peru/CS-UNAM/cycle/2014-1/Plan2014/fig/Bloom.eps ../Curricula2.0.out/Peru/CS-UNAM/cycle/2014-1/Plan2014/html/img6.png

\newcommand{\OutcomesList}{a,b,c,d,e,f,g,h,i,j,k,l,m,HU,FH,TASDSH}
\newcommand{\logowidth}{8cm}

\newcommand{\University}{Universidad Nacional de Moquegua\xspace}
\newcommand{\InstitutionURL}{http://www.unam.edu.pe\xspace}
\newcommand{\underlogotext}{}
\newcommand{\FacultadName}{Facultad de Computación\xspace}
\newcommand{\DepartmentName}{Ciencia de la Computación\xspace}
\newcommand{\SchoolFullName}{Carrera Profesional de Ciencia de la Computación}
\newcommand{\SchoolFullNameBreak}{\FacultadName\\ Carrera Profesional de \\Ciencia de la Computación\xspace}
\newcommand{\SchoolShortName}{Ciencia de la Computación\xspace}
\newcommand{\SchoolAcro}{CPCC\xspace}
\newcommand{\SchoolURL}{http://cs.unam.edu.pe}

\newcommand{\GradoAcademico}{Bachiller en Ciencia de la Computación\xspace}
\newcommand{\TituloProfesional}{Licenciado en Ciencia de la Computación\xspace}
\newcommand{\GradosyTitulos}%
{\begin{description}%
\item [Grado Académico: ] \GradoAcademico\xspace y%
\item [Titulo Profesional: ] \TituloProfesional%
\end{description}%
}

\newcommand{\doctitle}{Plan Curricular \YYYY\xspace del \SchoolFullName\\ \SchoolURL}

\newcommand{\AbstractIntro}{Este documento representa el informe final de la nueva malla curricular \YYYY del \SchoolFullName de la 
\University (\textit{\InstitutionURL}) en la ciudad de \city-\country.}

\newcommand{\OtherKeyStones}%
{Un pilar que merece especial consideración en el caso de la \University es el aspecto de 
valores humanos, básicos y cristianos debido a que forman parte fundamental 
de los lineamientos básicos de la existencia de la institución.\xspace}

\newcommand{\profile}{%
El perfil profesional de este programa profesional puede ser mejor entendido a partir de
\OnlyMainDoc{la Fig. \ref{fig.cs} (Pág. \pageref{fig.cs})}\OnlyPoster{las figuras del lado derecho}. 
Este profesional tiene como centro de su estudio a la computación. Es decir, tiene a la computación 
como fin y no como medio. De acuerdo a la definición de esta área, este profesional está llamado 
directamente a ser un impulsor del desarrollo de nuevas técnicas computacionales que 
puedan ser útiles a nivel local, nacional e internacional.

Nuestro perfil profesional está orientado a ser generador de puestos de empleo a través de la innovación permanente. 
Nuestra formación profesional tiene 3 pilares fundamentales: 
Formación Humana, un contenido de acuerdo a normas internacionales y una orientación marcada a la innovación.
}

\newcommand{\HTMLFootnote}{{Generado por <A HREF='http://socios.spc.org.pe/ecuadros/'>Ernesto Cuadros-Vargas</A> <ecuadros AT spc.org.pe>, <A HREF='http://www.spc.org.pe/'>Sociedad Peruana de Computación-Perú</A>, <A HREF='http://www.ucsp.edu.pe/'>Universidad Católica San Pablo, Arequipa-Perú</A><BR>basado en el modelo de la {\it Computing Curricula} de <A HREF='http://www.computer.org/'>IEEE-CS</A>/<A HREF='http://www.acm.org/'>ACM</A>}}

\newcommand{\Copyrights}{Generado por Ernesto Cuadros-Vargas (ecuadros AT spc.org.pe), Sociedad Peruana de Computación (http://www.spc.org.pe/), Universidad Católica San Pablo (http://www.ucsp.edu.pe) \country basado en la {\it Computing Curricula} de IEEE-CS (http://www.computer.org) y ACM (http://www.acm.org/)}
