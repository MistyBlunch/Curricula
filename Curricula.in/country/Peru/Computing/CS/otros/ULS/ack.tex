\chapter*{Agradecimientos}\label{chap:cs-ack}
\addcontentsline{toc}{chapter}{Agradecimientos}%

Además de los autores directos de este documento, también deseamos dejar manifiesto de nuestro 
agradecimiento a otros colegas de diversas universidades del país y del mundo que gentilmente 
han aportado parte de su tiempo a darnos sus sugerencias. Entre ellos debemos mencionar a:

\begin{itemize}
\NotUNI{
\item Luis Fernando Díaz Basurco (UCSP-Perú), 
\item Nelly Condori-Fernández (UPV-España), 
\item Agustín Tornés (ITESM-CCM, México), 
\item Alex Cuadros-Vargas (UCSP, Perú),
\item Alvaro Cuno-Parari (SPC),
\item Alfredo Paz (UNSA, Perú), 
\item Rodrigo Lazo Paz (UCSP, Perú),
\item Juan Manuel Gutiérrez Cárdenas (\textit{University of the Witwatersrand}-Sud-Africa),
\item Lenin Henry Cari Mogrovejo por su valiosa colaboración en los aspectos de redacción y corrección ortográfica del documento.
}%\NotUNI

\OnlyUNI{%
\item Sociedad Peruana de Computación por apoyarnos y facilitarnos su documento de Curricula de Computación.

\item Dr. Ernesto Cuadros-Vargas (Universidad Catóica de San Pablo, Arequipa) por su invaluable ayuda en la elaboración de la malla propuesta. 

\item Dr. César Beltrán Casta\~nón (IME - Universidad Catóica de San Pablo, Arequipa) por su invaluable ayuda en la elaboración de la malla propuesta. 

\item Julieta Flores y Johan Chicana Díaz por facilitarnos la base de sus estudios de mercado, recolección de datos y tabulación de resultados.

\item Dr. Glen Rodriguez (Universidad Nacional de Ingeniería) por su ayuda en la elaboración de la línea de Computación Distribuída.

\item Dr. José Luis Segovia (CONCYTEC y Universidad Peruana Cayetano Heredia) por su ayuda en la elaboración de la línea de Interacción Humano-Computador.

\item Dr. Jes\'us Castagnetto (Universidad Peruana Cayetano Heredia) por ayudarnos en la elaboración de la línea de Computación Distribuida y Bio-informática.

\item Dr. Mirko Zimic (Universidad Peruana Cayetano Heredia) por ayudarnos en la elaboración de la línea de Bio-informática.

\item Dr. Bruno Schulze (Laboratorio Nacional de Computa\c{c}\~ao Científica, Brasil) por su apoyo en la línea de investigación de aplicación de computación distribuida a la bio-informática.

\item Mg. Robinson Oliva (Universidad Nacional de Ingeniería) por su ayuda en la elaboración de los cursos relacionados con Arquitectura y Organización

\item Dr. A. M. Coronado (Universidad Nacional de Ingeniería) por su apoyo en la línea de investigación de Ciencia Computacional y Métodos Numéricos.
}
\end{itemize}

\OnlyUCSP{También deseamos agradecer a la Universidad Catóica San Pablo (UCSP) de Arequipa-Perú por su colaboración decidida de 
forma institucional y de forma individual a través de sus autoridades: Dr. Alonso Quintanilla Pérez-Witch (Rector), 
Dr. José Corrales Nieves-Lazarte (Vicerrector Académico) y al 
Mag. Germán Chávez (Director de Investigación). 
No hay duda de que cuando las autoridades están decididas a hacer las cosas bien todo el trabajo 
se realiza de forma rápida. 
Realmente es un ejemplo a seguir y llevar a todas las universidades de nuestro país.
}

\OnlyUNI{También deseamos agradecer a la Facultad de Ciencias de la 
Universidad Nacional de Ingeniería (UNI) de Lima-Perú por su colaboración 
decidida de forma institucional y de forma individual a través de sus autoridades: 
Dr. Pedro Canales (Decano). No hay duda de que cuando las autoridades están 
decididas a hacer las cosas bien todo el trabajo se realiza de forma rápida. 
Realmente es un ejemplo a seguir y llevar a todas las universidades de nuestro país.
}

Todo este equipo de trabajo asumió como premisa que el centro de nuestro esfuerzo, 
es la formación académica y humana \underline{de los estudiantes}.

A todos ellos deseamos agradecerles por su aporte que ha permitido generar 
este documento, único en su género en nuestro país, que servirá para sentar las 
bases de una carrera más sólida en esta fantástica área que nos ha tocado estudiar y 
de la cual nos sentimos orgullosos de formar parte: \textbf{Computación}.
