\newcommand{\DocumentVersion}{2012}
\newcommand{\fecha}{5 de Octubre de 2012}
\newcommand{\city}{Andahuaylas, Apurímac\xspace}
\newcommand{\country}{Perú\xspace}
\newcommand{\dictionary}{Español\xspace}
\newcommand{\GraphVersion}{2\xspace}

\newcommand{\CurriculaVersion}{2\xspace} % Malla 2006: 1
\newcommand{\YYYY}{2012\xspace}          % Plan 2006
\newcommand{\Range}{1-10}                 %Plan 2010 1-6, Plan 2006 5-10
\newcommand{\Semester}{2012-2\xspace}

% convert ./in/country.Peru/logos/UNAJMA.jpg ./out/html/Peru/CS-UNAJMA/Plan2012/img3.png 
% cp ./out/Peru/CS-UNAJMA/cycle/2012-2/Plan2012/fig/big-graph-curricula.png ./out/html/Peru/CS-UNAJMA/Plan2012/img17.png
% cp ./out/Peru/CS-UNAJMA/cycle/2012-2/Plan2012/fig/big-graph-curricula.png ./out/html/Peru/CS-UNAJMA/Plan2012/img18.png

\newcommand{\OutcomesList}{a,b,c,d,e,f,g,h,i,j,k,l,m,HU,FH,TASDSH}
\newcommand{\logowidth}{20cm}

\newcommand{\University}{Asamblea Nacional José María Arguedas\xspace}
\newcommand{\InstitutionURL}{http://www.unajma.edu.pe\xspace}
\newcommand{\underlogotext}{}
\newcommand{\FacultadName}{Facultad de Computación\xspace}
\newcommand{\DepartmentName}{Ciencia de la Computación\xspace}
\newcommand{\SchoolFullName}{Escuela Profesional de Ciencia de la Computación}
\newcommand{\SchoolFullNameBreak}{\FacultadName\\ Escuela Profesional de \\Ciencia de la Computación\xspace}
\newcommand{\SchoolShortName}{Ciencia de la Computación\xspace}
\newcommand{\SchoolAcro}{PPII\xspace}
\newcommand{\SchoolURL}{http://www.unajma.edu.pe}

\newcommand{\GradoAcademico}{Bachiller en Ciencia de la Computación\xspace}
\newcommand{\TituloProfesional}{Licenciado en Ciencia de la Computación\xspace}
\newcommand{\GradosyTitulos}%
{\begin{description}%
\item [Grado Académico: ] \GradoAcademico\xspace y%
\item [Titulo Profesional: ] \TituloProfesional%
\end{description}%
}

\newcommand{\doctitle}{Plan Curricular \YYYY\xspace del \SchoolFullName\\ \SchoolURL}

\newcommand{\AbstractIntro}{Este documento representa el informe final para la \SchoolFullName 
en base a las propuestas mundiales para carreras de Computación.}

\newcommand{\OtherKeyStones}%
{.\xspace}

\newcommand{\profile}{%
El perfil profesional de este programa profesional puede ser mejor entendido a partir de
\OnlyMainDoc{la Fig. \ref{fig.cs} (Pág. \pageref{fig.cs})}\OnlyPoster{las figuras del lado derecho}. 
Este profesional tiene como centro de su estudio a la computación. Es decir, tiene a la computación 
como fin y no como medio. De acuerdo a la definición de esta área, este profesional está llamado 
directamente a ser un impulsor del desarrollo de nuevas técnicas computacionales que 
puedan ser útiles a nivel local, nacional e internacional.

Nuestro perfil profesional está orientado a ser generador de puestos de empleo a través de la innovación permanente. 
Nuestra formación profesional tiene 3 pilares fundamentales: 
Formación Humana, un contenido de acuerdo a normas internacionales y una orientación marcada a la innovación.
}

\newcommand{\HTMLFootnote}{Generado por <A HREF='http://socios.spc.org.pe/ecuadros/'>Ernesto Cuadros-Vargas</A> <ecuadros AT spc.org.pe>, <A HREF='http://www.spc.org.pe/'>Sociedad Peruana de Computación-Perú</A>, <A HREF='http://www.ucsp.edu.pe/'>Universidad Católica San Pablo, Arequipa-Perú</A><BR>basado en el modelo de la {\it Computing Curricula} de <A HREF='http://www.computer.org/'>IEEE-CS</A>/<A HREF='http://www.acm.org/'>ACM</A>}

\newcommand{\Copyrights}{Generado por Ernesto Cuadros-Vargas (ecuadros AT spc.org.pe), Sociedad Peruana de Computación (http://www.spc.org.pe/), Universidad Católica San Pablo (http://www.ucsp.edu.pe) \country basado en la {\it Computing Curricula} de IEEE-CS (http://www.computer.org) y ACM (http://www.acm.org/)}
