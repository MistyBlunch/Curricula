\newcommand{\DocumentVersion}{V2.0}
\newcommand{\fecha}{6 de Mayo de 2010}
\newcommand{\YYYY}{2010\xspace}
\newcommand{\Semester}{2010-1\xspace}
\newcommand{\city}{Abancay\xspace}
\newcommand{\country}{Perú\xspace}
\newcommand{\dictionary}{Español\xspace}
\newcommand{\GraphVersion}{2\xspace}
\newcommand{\CurriculaVersion}{2\xspace}
\newcommand{\OutcomesList}{a,b,c,d,e,f,g,h,i,j,k,l,m,HU,FH,TASDSH}

\newcommand{\University}{Universidad Nacional Micaela Bastidas\xspace}
\newcommand{\InstitutionURL}{http://www.unamba.edu.pe\xspace}
\newcommand{\underlogotext}{}
\newcommand{\FacultadName}{Computación\xspace}
\newcommand{\DepartmentName}{Ciencia de la Computación\xspace}
\newcommand{\SchoolFullName}{Escuela Académico Profesional Ciencia de la Computación}
\newcommand{\SchoolFullNameBreak}{Facultad de Computación\\ Escuela Académico  Profesional de\\ Ciencia de la Computación\xspace}
\newcommand{\SchoolShortName}{Ciencia de la Computación\xspace}
\newcommand{\SchoolAcro}{EAPCC\xspace}
\newcommand{\SchoolURL}{http://cs.unamba.edu.pe}

\newcommand{\GradoAcademico}{Bachiller en Ciencia de la Computación\xspace}
\newcommand{\TituloProfesional}{Ciencia de la Computación\xspace}
\newcommand{\GradosyTitulos}%
{\begin{description}%
\item [Grado Académico: ] \GradoAcademico\xspace y%
\item [Titulo Profesional: ] \TituloProfesional%
\end{description}%
}

\newcommand{\doctitle}{Plan Curricular \YYYY\xspace del \SchoolFullName\\ \SchoolURL}

\newcommand{\AbstractIntro}{Este documento representa el informe final de la nueva malla curricular \YYYY del \SchoolFullName de la \University (\textit{\InstitutionURL}) en la ciudad de \city-\country. En la actualidad esta carrera está siendo orientada a Ciencia de la Computación, el Bachillerato ya está con esta denominación y el título profesional está en proceso de cambio.}

\newcommand{\OtherKeyStones}{}
% {Un pilar que merece especial consideración en el caso de la \University es el aspecto de 
% valores humanos, básicos y cristianos debido a que forman parte fundamental 
% de los lineamientos básicos de la existencia de la institución.\xspace}

\newcommand{\profile}{%
El perfil profesional de este programa profesional puede ser mejor entendido a partir de
\OnlyMainDoc{la Fig. \ref{fig.cs} (Pág. \pageref{fig.cs})}\OnlyPoster{las figuras del lado derecho}. 
Este profesional tiene como centro de su estudio a la computación. Es decir, tiene a la computación 
como fin y no como medio. De acuerdo a la definición de esta área, este profesional está llamado 
directamente a ser un impulsor del desarrollo de nuevas técnicas computacionales que 
puedan ser útiles a nivel local, nacional e internacional.

Nuestro perfil profesional está orientado a ser generador de puestos de empleo a través de la innovación permanente. 
Nuestra formación profesional tiene 3 pilares fundamentales: 
Formación Humana, un contenido de acuerdo a normas internacionales y una orientación marcada a la innovación.
}

\newcommand{\HTMLFootnote}{Generado por <A HREF='http://socios.spc.org.pe/ecuadros/'>Ernesto Cuadros-Vargas</A> <ecuadros AT spc.org.pe>, basado en el modelo de la <A HREF='http://www.spc.org.pe/'>Sociedad Peruana de Computación, Perú</A><BR> y en la Computing Curricula de <A HREF='http://www.computer.org/'>IEEE-CS</A>/<A HREF='http://www.acm.org/'>ACM</A>}


