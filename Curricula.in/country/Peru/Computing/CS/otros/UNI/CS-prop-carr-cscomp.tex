\section{Propuesta para la creación de la Carrera de \SchoolShortName}\label{sec:cs-prop-carr-cscomp}

\subsection{Objetivos de la Carrera de \SchoolShortName}
Los objetivos del Departamento de Computación se agrupan alrededor de las grandes tareas de nuestra Universidad:
\begin{inparaenum}[ a) ]
\item la docencia
\item la investigación y 
\item la divulgación y vinculación.
\end{inparaenum}

\begin{enumerate}
\item El personal académico de la carrera de \SchoolShortName realizará labores de docencia en esta Licenciatura. Además, impartirá cursos de Computación en otras carreras y posgrados en los cuales participa la Facultad. Sin embargo, quedará por definir entre todos los involucrados en la licenciatura el mecanismo por medio del cual se dirigirá la licenciatura en \SchoolShortName. Como parte de su labor de formación de recursos humanos, el personal de esta carrera dirigirá tesis de licenciatura.

\item La labor de investigación abarcará las áreas de la Computación en las que trabaja su personal académico, ya sea en proyectos internos o en colaboración con investigadores de otros grupos afines o de otras áreas que así lo requieran.

\item El trabajo de divulgación y vinculación incluirá, entre otras, las siguientes actividades: organización de eventos, actualización de profesores y profesionales, asesoría y colaboración con la industria de la Computación y con entidades gubernamentales y no gubernamentales, la divulgación de la labor de la carrera y la transferencia de conocimiento.
\end{enumerate}

\subsection{Naturaleza y Funciones de la Carrera de \SchoolShortName}
En un principio, la Escuela de Computación agrupará a los profesores de las áreas de conocimiento de la Computación:
\begin{enumerate}[ a) ]
\item Cómputo Visual,
\item Procesamiento Paralelo y Sistemas Distribuidos,
\item Computación Científica
\end{enumerate}
y de otras áreas relacionadas con la computación que se puedan desarrollar dentro del Departamento. Asimismo, integrará a los técnicos académicos y personal de administración y servicios adscritos al mismo.

Son funciones específicas de la carrera de \SchoolShortName la organización y coordinación de la docencia, la investigación, la divulgación y la vinculación correspondientes a las áreas de conocimiento de su competencia, conforme a las atribuciones siguientes: 

\begin{itemize}
\item Organizar, desarrollar, evaluar y co ordinar el programa de Licenciatura en \SchoolShortName. Además, impartir cursos de Computación en otras carreras y posgrados en los cuales participa la Facultad de Ciencias. Esta labor se desarrollará respetando la especialización de su profesorado y la libertad de cátedra. 

\item Realizar investigación en la disciplina.

\item Organizar, desarrollar y evaluar cursos de especialización, actualización y formación permanente, en las áreas de su competencia.

\item Proponer para su aprobación el establecimiento, modificación o supresión de aspectos de los planes de estudio relativos a sus áreas de conocimiento.

\item Impulsar la actualización científica, técnica y pedagógica de sus miembros.

\item Cooperar con las demás áreas de la Facultad, con dependencias de la propia Universidad y con otras instituciones y organismos públicos y privados en la realización de programas de investigación o docentes y en proyectos de desarrollo.

\item Proponer sus plantillas académicas y presupuestos.

\item Proponer la contratación de personas físicas y entidades u organismos públicos o privados para llevar a cabo proyectos científicos y técnicos, según lo establecido en la Ley Orgánica y el Estatuto General de la Universidad. Administrar sus propios recursos.

\item Organizar y distribuir entre sus miembros las tareas inherentes a su funcionamiento.
\end{itemize}
