\newcommand{\fecha}{\today}
\newcommand{\YYYY}{2008\xspace}
\newcommand{\country}{Perú\xspace}
\newcommand{\dictionary}{Español\xspace}
\newcommand{\Semester}{\YYYY-I\xspace}

\newcommand{\University}{Universidad Nacional de Ingeniería\xspace}
\newcommand{\InstitutionURL}{http://www.uni.edu.pe\xspace}
\newcommand{\underlogotext}{}

\newcommand{\FacultadName}{Ciencias\xspace}
\newcommand{\DepartmentShortName}{Ciencia\xspace}
\newcommand{\SchoolFullName}{Escuela Profesional de Ciencia de la Computación\xspace}
\newcommand{\SchoolFullNameBreak}{\SchoolFullName}
\newcommand{\SchoolShortName}{Ciencia de la Computación\xspace}
\newcommand{\SchoolAcro}{CC\xspace}

\newcommand{\GradoAcademico}{Bachiller en Ciencia de la Computación\xspace}
\newcommand{\TituloProfesional}{Licenciado en Ciencia de la Computación\xspace}
\newcommand{\GradosyTitulos}%
{\begin{description}%
\item [Grado Académico: ] \GradoAcademico y%
\item [Titulo Profesional: ] \TituloProfesional%
\end{description}%
}
\newcommand{\SchoolURL}{http://www.uni.edu.pe\xspace}
\newcommand{\doctitle}{Proyecto de creación de la \SchoolFullName{\Large\footnote{\SchoolURL}}\xspace}
\newcommand{\city}{Lima\xspace}
\newcommand{\AbstractIntro}{Este documento es el proyecto de creación de la \SchoolFullName de la \ac{\currentinstitution}.\xspace}

% renews go here
\newcommand{\OtherKeyStones}{}
%\newcommand{\OnlyUNSA}[1]{#1}

