\newcommand{\DocumentVersion}{V1.95}
\newcommand{\fecha}{\today}
\newcommand{\YYYY}{2010\xspace}
\newcommand{\Semester}{2010-1\xspace}
\newcommand{\city}{Pucallpa\xspace}
\newcommand{\country}{Perú\xspace}
\newcommand{\dictionary}{Español\xspace}
\newcommand{\GraphVersion}{2\xspace}
\newcommand{\CurriculaVersion}{2\xspace}
\newcommand{\OutcomesList}{a,b,c,d,e,f,g,h,i,j,k,l,m,HU}

\newcommand{\University}{Universidad Nacional de Ucayali\xspace}
\newcommand{\InstitutionURL}{http://www.unu.edu.pe\xspace}
\newcommand{\underlogotext}{}

\newcommand{\FacultadName}{Facultad de Educación y Ciencias Sociales\xspace}
\newcommand{\DepartmentName}{Ciencia de la Computación\xspace}
\newcommand{\SchoolFullName}{Escuela Académico Profesional de Ciencia de la Computación (\DocumentVersion)}
\newcommand{\SchoolFullNameBreak}{\FacultadName \\Escuela Académico Profesional de\\Ciencia de la Computación (\DocumentVersion)\xspace}
\newcommand{\SchoolShortName}{({\bf Ciencia de la Computación})\xspace}
\newcommand{\SchoolAcro}{EAPCC\xspace}
\newcommand{\SchoolURL}{http://www.unu.edu.pe/}

\newcommand{\GradoAcademico}{Bachiller en Ciencia de la Computación\xspace}
\newcommand{\TituloProfesional}{Licenciado en Ciencia de la Computación\xspace}
\newcommand{\GradosyTitulos}%
{\begin{description}%
\item [Grado Académico: ] \GradoAcademico y%
\item [Título Profesional: ] \TituloProfesional%
\end{description}%
}

\newcommand{\doctitle}{Plan Curricular \YYYY\xspace de la \SchoolFullName\\ \SchoolURL}

\newcommand{\AbstractIntro}{Este documento representa el informe final de la nueva malla curricular \YYYY del 
\SchoolFullName de la \University (\textit{\InstitutionURL}) en la ciudad de \city-\country}

\newcommand{\OtherKeyStones}{}

\newcommand{\profile}{%
El perfil profesional de este programa profesional puede ser mejor entendido a partir de
\OnlyMainDoc{la Fig.~\ref{fig.cs} (Pág.~\pageref{fig.cs})}\OnlyPoster{las figuras del lado derecho}.
Este profesional tiene como centro de su estudio a la computación. Es decir, tiene a la computación 
como fin y no como medio. De acuerdo a la definición de esta área, este profesional está llamado 
directamente a ser un impulsor del desarrollo de nuevas técnicas computacionales que 
puedan ser útiles a nivel local, nacional e internacional.

Nuestro perfil profesional está orientado a ser generador de puestos de empleo a través de la innovación permanente. 
Nuestra formación profesional tiene 2 pilares fundamentales:
Un contenido de acuerdo a recomendaciones internacionales y una orientación marcada a la innovación.
}


\newcommand{\HTMLFootnote}{Generado por <A HREF='http://socios.spc.org.pe/ecuadros/'>Ernesto Cuadros-Vargas</A> <ecuadros AT spc.org.pe>, <A HREF='http://www.unu.edu.pe/'>Universidad Nacional de Ucayali, Pucallpa-Per</A><BR>basado en el modelo de la {\it Computing Curricula} de <A HREF='http://www.computer.org/'>IEEE-CS</A>/<A HREF='http://www.acm.org/'>ACM</A>}

