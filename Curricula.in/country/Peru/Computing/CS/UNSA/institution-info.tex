\newcommand{\DocumentVersion}{V2.0}
\newcommand{\fecha}{5 de Enero de 2017}
\newcommand{\city}{Arequipa\xspace}
\newcommand{\country}{Perú\xspace}
\newcommand{\dictionary}{Español\xspace}
\newcommand{\SyllabusLangs}{Español,English}
\newcommand{\GraphVersion}{2\xspace}

\newcommand{\OutcomesVersion}{V1}
\OutcomesList{V1}{a,b,c,d,e,f,g,h,i,j,k,l,m,n,ñ,o}

\newcommand{\logowidth}{7cm}

\newcommand{\CurriculaVersion}{2016\xspace} % Malla 2006: 1
\newcommand{\YYYY}{2017\xspace}          % Plan 2006
\newcommand{\Range}{1-10}                % Plan 2010 1-6, Plan 2006 5-10
\newcommand{\Semester}{2017-I\xspace}
\newcommand{\equivalences}{2010} %  {2006,2010}

% convert ./in/country.Peru/logos/UNSA.jpg ./out/html/Peru/CS-UNSA/Plan2010/img3.png 
% cp ./out/Peru/CS-UNSA/cycle/2013-II/Plan2010/fig/big-graph-curricula.png ./out/html/Peru/CS-UNSA/Plan2010/img32.png
% cp ./out/Peru/CS-UNSA/cycle/2013-II/Plan2010/fig/big-graph-curricula.png ./out/html/Peru/CS-UNSA/Plan2010/img33.png  

\newcommand{\University}{Universidad Nacional de San Agustí­n\xspace}
\newcommand{\InstitutionURL}{http://www.unsa.edu.pe\xspace}
\newcommand{\underlogotext}{}

\newcommand{\FacultadName}{Facultad de Ingenierí­a de Producción y Servicios\xspace}
\newcommand{\DepartmentShortName}{Ingenierí­a de Sistemas e Informática\xspace}
\newcommand{\SchoolFullName}{Escuela Profesional de Ciencia de la Computación\xspace}
\newcommand{\SchoolFullNameBreak}{\FacultadName \\ Escuela Profesional de \\ \SchoolShortName \\ Malla Curricular \YYYY}
\newcommand{\SchoolShortName}{Ciencia de la Computación\xspace}
\newcommand{\SchoolAcro}{CC\xspace}

\newcommand{\GradoAcademico}{Bachiller en Ciencia de la Computación\xspace}
\newcommand{\TituloProfesional}{Licenciado en Ciencia de la Computación\xspace}
\newcommand{\GradosyTitulos}%
{\begin{description}%
\item [Grado Académico: ] \GradoAcademico y%
\item [Titulo Profesional: ] \TituloProfesional%
\end{description}%
}
\newcommand{\SchoolURL}{http://www.unsa.edu.pe\xspace}
\newcommand{\doctitle}{Curriculo de la \SchoolFullName{\Large\footnote{\SchoolURL}}\xspace}
\newcommand{\AbstractIntro}{Este documento es el proyecto de creación de la \SchoolFullName de la \University.\xspace}

\newcommand{\OtherKeyStones}{}

\newcommand{\profile}{%
El perfil profesional de este programa profesional puede ser mejor entendido a partir de
\OnlyMainDoc{la Fig. \ref{fig.cs} (Pág. \pageref{fig.cs})}\OnlyPoster{las figuras del lado derecho}.
Este profesional tiene como centro de su estudio a la computación. Es decir, tiene a la computación
como fin y no como medio. De acuerdo a la definición de esta área, este profesional está llamado
directamente a ser un impulsor del desarrollo de nuevas técnicas computacionales que
puedan ser útiles a nivel local, nacional e internacional.

Nuestro perfil profesional está orientado a ser generador de puestos de empleo a través de la innovación permanente.
Nuestra formación profesional tiene 3 pilares fundamentales:
Formación Humana, un contenido de acuerdo a normas internacionales y una orientación marcada a la innovación.
}

\newcommand{\mission}{La \University ... .\xspace}

\newcommand{\HTMLFootnote}{{Generado por <A HREF='http://socios.spc.org.pe/ecuadros/'>Ernesto Cuadros-Vargas</A> <ecuadros AT spc.org.pe>, <A HREF='http://www.spc.org.pe/'>Sociedad Peruana de Computación-Perú</A><BR>basado en el modelo de la {\it Computing Curricula} de <A HREF='http://www.computer.org/'>IEEE-CS</A>/<A HREF='http://www.acm.org/'>ACM</A>}}
\newcommand{\Copyrights}{Generado por Ernesto Cuadros-Vargas (ecuadros AT spc.org.pe), Sociedad Peruana de Computación (http://www.spc.org.pe/) \country basado en la {\it Computing Curricula} de IEEE-CS (http://www.computer.org) y ACM (http://www.acm.org/)}
