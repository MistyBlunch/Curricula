\chapter*{Agradecimientos}\label{chap:cs-ack}
\addcontentsline{toc}{chapter}{Agradecimientos}%

Adem�s de los autores directos de este documento, tambi�n deseamos dejar manifiesto de nuestro 
agradecimiento a otros colegas de diversas universidades del pa�s y del mundo que gentilmente 
han aportado parte de su tiempo a darnos sus sugerencias. Entre ellos debemos mencionar a:

\begin{itemize}
\item Luis Fernando D�az Basurco (UCSP-Per�), 
\item Nelly Condori-Fern�ndez (UPV-Espa�a), 
\item Agust�n Torn�s (ITESM-CCM, M�xico), 
\item Alex Cuadros-Vargas (UCSP, Per�),
\item Alvaro Cuno-Parari (SPC),
\item Alfredo Paz (UNSA, Per�), 
\item Rodrigo Lazo Paz (UCSP, Per�),
\item Juan Manuel Guti�rrez C�rdenas (\textit{University of the Witwatersrand}-Sud-Africa),
\item Lenin Henry Cari Mogrovejo por su valiosa colaboraci�n en los aspectos de redacci�n y correcci�n ortogr�fica del documento.

\item Sociedad Peruana de Computaci�n por apoyarnos y facilitarnos su documento de Curricula de Computaci�n.
\item Dr. Ernesto Cuadros-Vargas (Universidad Cat�lica de San Pablo, Arequipa) por su invaluable ayuda en la elaboraci�n de la malla propuesta. 
\end{itemize}

Todo este equipo de trabajo asumi� como premisa que el centro de nuestro esfuerzo, 
es la formaci�n acad�mica y humana \underline{de los estudiantes}.

A todos ellos deseamos agradecerles por su aporte que ha permitido generar 
este documento, �nico en su g�nero en nuestro pa�s, que servir� para sentar las 
bases de una carrera m�s s�lida en esta fant�stica �rea que nos ha tocado estudiar y 
de la cual nos sentimos orgullosos de formar parte: \textbf{Computaci�n}.
