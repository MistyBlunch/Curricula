\newcommand{\DocumentVersion}{2010}
\newcommand{\fecha}{\today}
\newcommand{\city}{Lima\xspace}
\newcommand{\country}{Perú\xspace}
\newcommand{\dictionary}{Español\xspace}
\newcommand{\SyllabusLangs}{Español} %,English
\newcommand{\GraphVersion}{2\xspace}

%Para UTEC 2018
\newcommand{\CurriculaVersion}{2018\xspace} % Malla 2006: 1, Malla 2010: 2
\newcommand{\YYYY}{2020\xspace}             % Plan 2006
\newcommand{\Range}{1-10}                   % Plan 2010 1-8, Plan 2006 7-10
\newcommand{\Semester}{2020-I\xspace}      

% newcommand{\OutcomesList}{a,b,c,d,e,f,g,h,i,j,k,l,m,HU,FH,TASDSH}
\newcommand{\OutcomesVersion}{V2}
\OutcomesList{V1}{a,b,c,d,e,f,g,h,i,j,k,l,m,n,ñ,o}
\OutcomesList{V2}{1,2,3,4,5,6,7,8,9,10}

\newcommand{\logowidth}{6.3cm}
\newcommand{\InstitutionURL}{\htmladdnormallink{http://www.utec.edu.pe}{http://www.utec.edu.pe}\xspace}

\newcommand{\UniversityES}{Universidad de Ingeniería y Tecnología (UTEC)\xspace}
\newcommand{\UniversityEN}{Universidad de Ingeniería y Tecnología (UTEC)\xspace}
%newcommand{\University}{\UniversityES}
\newcommand{\University}{Universidad de Ingeniería y Tecnología (UTEC)\xspace}

\newcommand{\FacultadNameES}{Facultad de Computación\\ \xspace}
\newcommand{\FacultadNameEN}{}
\newcommand{\FacultadName}{\FacultadNameES}

\newcommand{\DepartmentNameES}{Departamento de Ciencia de Datos\xspace}
\newcommand{\DepartmentNameEN}{Department of Data Science\xspace}
\newcommand{\DepartmentName}{\DepartmentNameES}

\newcommand{\SchoolShortNameES}{Ciencia de Datos\xspace}
\newcommand{\SchoolShortNameEN}{Computer Science\xspace}
\newcommand{\SchoolShortName}{\SchoolShortNameES}

\newcommand{\SchoolFullNameES}{Escuela Profesional de \SchoolShortNameES}
\newcommand{\SchoolFullNameEN}{School of \SchoolShortNameEN}
\newcommand{\SchoolFullName}{\SchoolFullNameES}

\newcommand{\SchoolFullNameBreakES}{Escuela Profesional de \\ \SchoolShortNameES\xspace}
\newcommand{\SchoolFullNameBreakEN}{School of \SchoolShortNameEN\xspace}
\newcommand{\SchoolFullNameBreak}{\SchoolFullNameBreakES}

\newcommand{\PosterTitle}{}

\newcommand{\SchoolAcro}{EPCC\xspace}
\newcommand{\SchoolURL}{\htmladdnormallink{http://ds.utec.edu.pe}{http://ds.utec.edu.pe}\xspace}
\newcommand{\underlogotext}{}

\newcommand{\GradoAcademico}{Bachiller en Ciencia de Datos\xspace}
\newcommand{\TituloProfesional}{Licenciado en Ciencia de Datos\xspace}
\newcommand{\GradosyTitulos}%
{\begin{description}%
\item [Grado Académico: ] \GradoAcademico\xspace y% 
\item [Titulo Profesional: ] \TituloProfesional%
\end{description}%
}

\newcommand{\doctitle}{Plan Curricular \YYYY\xspace de la \SchoolFullName\\ \SchoolURL}

\newcommand{\AbstractIntro}{Este documento representa el informe final de la nueva 
malla curricular \YYYY del \SchoolFullName de la \University (\textit{\InstitutionURL}) 
en la ciudad de \city-\country.}

\newcommand{\OtherKeyStones}{}


\newcommand{\profileES}{%
El perfil profesional de este programa profesional puede ser mejor entendido a partir de ...

Este profesional tiene como centro de sus estudios al tratamiento y estudio de los datos.
De acuerdo a la definición de esta área, este profesional está llamado 
directamente a ser un impulsor del desarrollo de nuevas técnicas computacionales derivadas del tratamiento de datos que
puedan ser útiles a nivel local, nacional e internacional.

Nuestro perfil profesional está orientado a ser líder en el tratamiento y estudios de datos. 
Nuestra formación profesional tiene 3 pilares fundamentales: 
un contenido de acuerdo a ACM/IEEE-CS Computing Curricula CC2020
un contenido de acuerdo a normas internacionales, una orientación marcada a la innovación y formación humana.
}

\newcommand{\profileEN}{The professional profile of this professional program can be better understood from 
\OnlyMainDoc{la Fig. \ref{fig.cs} (P. \pageref{fig.cs})} \OnlyPoster{figures on the right side}. 
This professional has Computing as the center of his studies. That is, it has computating as an end and not as a means. 
According to the definition of this area, this professional is called directly to be a promoter of 
the development of new computational techniques that can be useful at local, national and international level.

Our professional profile is aimed at generating jobs through permanent innovation. Our professional training has three fundamental pillars: 
a content according to ACM/IEEE-CS Computing Curricula CC2020, a marked orientation to innovation and human/soft skills.
}

\newcommand{\missionES}{Contribuir al desarrollo científico, tecnológico y técnico del país  
formando profesionales competentes, orientados al tratamiento y transformación de datos como motor que impulse y consolide el desarrollo de diversas industrias 
incluyendo la investigación científica y tecnológica alrededor de los datos.
Además buscamos formar profesionales  un conjunto de habilidades y 
destrezas para la solución de problemas alrededor del tratamiento y transformación de grandes volumens de datos con un compromiso social.\xspace}

\newcommand{\missionEN}{To contribute to the scientific, technological and technical development of the country 
forming competent professionals oriented to process and transformation of data as the engine to push and consolidate creation of new industries 
including research and technology arounf data. 
Besides, we also aim to have students with a groups of skills to solve problems around treatment of big data with a social commitment.\xspace}
