\begin{syllabus}

\course{CB210. Investigaci�n Operativa I}{Obligatorio}{CB210}

\begin{justification}
Este curso es importante en la medida que proporciona modelos de optimizaci�n �tiles para la toma de decisiones en negocios.
\end{justification}

\begin{goals}
\item Reconocer, modelar, resolver, implementar e interpretar modelos de optimizaci�n linear en problemas reales.
\end{goals}

\begin{outcomes}
\ExpandOutcome{a}{1}
\ExpandOutcome{b}{1}
\ExpandOutcome{c}{1}
\ExpandOutcome{g}{1}
\ExpandOutcome{j}{1}
\end{outcomes}

\begin{unit}{Introducci�n a la Programaci�n Lineal}{Taha2004,Hillier2006}{14}{1}
   \begin{topics}
      \item Soluciones gr�ficas.
      \item Ejemplos de problemas de Programaci�n Lineal.
      \item Casos especiales de Programaci�n Lineal.
   \end{topics}

   \begin{unitgoals}
      \item Que el alumno conozca las t�cnicas y/o conceptos b�sicos de la Programaci�n asi como su interpretaci�n.
   \end{unitgoals}
\end{unit}

\begin{unit}{El Algoritmo Simplex}{Taha2004,Hillier2006}{12}{1}
   \begin{topics}
      \item Preliminares del Algoritmo Simplex.
      \item Soluci�n de problemas de minimizaci�n con mediante el Algoritmo Simplex.
      \item Soluciones alternativas.
      \item Problemas Lineales no acotados.
      \item Uso de paquetes.
      \item Degeneraci�n y convergencia.
   \end{topics}

   \begin{unitgoals}
      \item Que el alumno sea capaz de comprender, dise�ar y aplicar el algoritmo simplex en problemas reales de una empresa.
   \end{unitgoals}
\end{unit}

\begin{unit}{An�lisis de sensibilidad}{Taha2004,Hillier2006}{12}{1}
   \begin{topics}
      \item Introducci�n gr�fica del an�lisis de sensibilidad.
      \item An�lisis de sensibilidad cuando cambian los par�metros.
      \item Determinaci�n del dual.
      \item Interpretaci�n econ�mica del problema Dual.
      \item Precios sombra.
      \item Holgura complemantaria.
   \end{topics}

   \begin{unitgoals}
      \item Que el alumno pueda medir la sensibilidad de la soluci�n de un problema frente a la variaci�n de par�metros.
   \end{unitgoals}
\end{unit}

\begin{unit}{Problemas de Transporte y Asignaci�n}{Eppen1999,Izar2010}{12}{1}
   \begin{topics}
      \item ?`C�mo formular problemas de transporte?
      \item Soluciones b�sicas factibles para problemas de transporte.
      \item El m�todo simplex en el transporte.
      \item An�lisis de sensibilidad.
      \item Problemas de asignaci�n.
      \item Problemas de transbordo.
   \end{topics}

   \begin{unitgoals}
      \item Que el alumno sea capaza de resolver problemas de optimizaci�n en el �rea de transporte y optimizaci�n.
   \end{unitgoals}
\end{unit}

\begin{unit}{Programaci�n entera}{Eppen1999,Izar2010}{10}{1}
   \begin{topics}
      \item Planteamiento de problemas de programaci�n entera.
      \item M�todos de ramificaci�n y acotaci�n.
      \item Algoritmos de plano cortante.
   \end{topics}

   \begin{unitgoals}
      \item Resolver problemas de optimizaci�n lineal para la toma de decisiones booleanas.
   \end{unitgoals}
\end{unit}



\begin{coursebibliography}
\bibfile{BasicSciences/CB210}
\end{coursebibliography}

\end{syllabus}
