\begin{syllabus}

\course{FG210. Moral}{Obligatorio}{FG210}

\begin{justification}
La �tica-moral comienza cuando se trata de elegir un sentido correcto de realizaci�n humana en su l�nea propia, un sentido capaz de desarrollar en plenitud sus posibilidades.
El problema de dar sentido a la vida es fundamental en el ser humano, ya que lo acompa�a durante toda su existencia, y la �tica-moral interpela a la persona a vivir seg�n su fin �ltimo. En este sentido, la �tica-moral busca la realizaci�n del hombre en la elecci�n correcta de dicho fin.
\end{justification}

\begin{goals}
\item Formar la conciencia del estudiante para que pueda conducirse con criterio moralmente correcto en los �mbitos personal y profesional.

\end{goals}

\begin{outcomes}
    \item \ShowOutcome{e}{2}
    \item \ShowOutcome{�}{2}
\end{outcomes}

\begin{competences}
    \item \ShowCompetence{C10}{e}
    \item \ShowCompetence{C20}{e}
    \item \ShowCompetence{C21}{e, �}
    \item \ShowCompetence{C22}{�}
\end{competences}

\begin{unit}{}{Primera Unidad: La �tica Filos�fica}{Lewis, Bourmaud, RodriguezL, AristotelesE}{9}{C10,C20}
\begin{topics}
    \item Presentaci�n del curso.
    \item Lo �tico y moral. La �tica como rama de la filosof�a.
    \item La necesidad de la metaf�sica.
    \item La experiencia moral.
    \item El problema del relativismo y su soluci�n.
	
\end{topics}
\begin{learningoutcomes}
	\item Incorporar una primera noci�n de la �tica y la moral, junto con los problemas que buscan resolver.[\Familiarity]
\end{learningoutcomes}
\end{unit}

\begin{unit}{}{Segunda Unidad: La acci�n moral}{SanchezM,Genta}{15}{C10,C20}
\begin{topics}
    \item Caracterizaci�n del actuar humano.
    \item Libertad, conciencia y voluntariedad. Distintos niveles de libertad. Factores que afectan la voluntariedad.
    \item El papel de la afectividad en la moralidad.
    \item La felicidad como fin �ltimo del ser humano.

\end{topics}
\begin{learningoutcomes}
	\item Analizar el acto humano, presentando sus condiciones y especificando su moralidad.[\Familiarity]
\end{learningoutcomes}
\end{unit}

\begin{unit}{}{Tercera Unidad: La vida virtuosa}{Piper,Droste,Lego}{12}{C21,C22}
\begin{topics}
    \item �Qu� se entiende por virtud?
    \item La virtud moral: caracterizaci�n y modo de adquisici�n; el car�cter din�mico de la virtud.
    \item Relaci�n entre las distintas virtudes �ticas. Las virtudes cardinales. Los vicios.
\end{topics}
\begin{learningoutcomes}
	\item Reflexionar respecto al ideal filos�fico y moral de la vida virtuosa desde la pr�ctica estable de bien y el rechazo constante de lo da�ino.[\Familiarity]
\end{learningoutcomes}
\end{unit}

\begin{unit}{}{Cuarta Unidad: Lo �ticamente correcto y su conocimiento}{ReydeCastro2010,SanchezM,Genta}{9}{C10,C20,C21}
\begin{topics}
    \item La correcci�n en lo �tico.
    \item El conocimiento de lo �ticamente correcto.
    \item La llamada ``recta raz�n'' y la ``verdad pr�ctica''.
    \item Las leyes morales: ley natural y ley positiva.
    \item La conciencia moral: definici�n, tipos, deformaciones.
    \item La valoraci�n moral de las acciones concretas.
\end{topics}

\begin{learningoutcomes}
	\item Discernir las nociones de recta raz�n, conciencia moral, y moral natural, remarcando la necesidad de la ley moral natural como el par�metro de conducta.[\Familiarity]
\end{learningoutcomes}
\end{unit}



\begin{coursebibliography}
\bibfile{GeneralEducation/FG101}
\end{coursebibliography}

\end{syllabus}
