\begin{syllabus}

\course{CS232W. Programaci�n de Dispositivos M�viles}{Obligatorio}{CS232W}

\begin{justification}
El siempre creciente desarrollo de las tecnolog�as de comunicaci�n y la
informaci�n hace que exista una marcada tendencia a  establecer medios de 
comunicaci�n m�s simples y eficientes. De esta forma es que las soluciones
m�biles aparecen como respuesta a esta nueva tendencia.

En este curso se brindar� a los participantes una introducci�n a los
problemas que conlleva la comunicaci�n usando dispositivos m�viles, a trav�s del
estudio e implementaci�n de aplicativos; tomando como referencia otros aplicativos
m�biles creados por diferentes grupos de investigaci�n, y tambi�n de la industria.

\end{justification}

\begin{goals}
      \item Explorar problemas de investigaci�n en computaci�n m�vil.
      \item Conocer tecnolog�as usadas para computaci�n m�vil.
      \item Entender y construir sistemas que soporten la computaci�n m�vil.
      \item Comprender las razones por las que dispositivos m�viles sean convertido ubicuos, y
      \item Evaluar y proponer aplicaciones cuya soluci�n es apropiada a la computaci�n m�vil.

\end{goals}

\begin{outcomes}
\ExpandOutcome{b}{3}
\ExpandOutcome{c}{4}
\ExpandOutcome{e}{3}
\ExpandOutcome{g}{3}
\ExpandOutcome{i}{3}
\ExpandOutcome{j}{4}
\end{outcomes}

\begin{unit}{Mobilidad y Manejo de Localidad}{AGGL:2005}{8}{4}
   \begin{topics}
     \item Definiciones y visiones sobre mobilidad.
     \item Historia de la computaci�n ubicua.
     \item Sistemas ubicuos.
     \item Localidad.
     \item Context aware computing.
   \end{topics}

  \begin{unitgoals}
     \item Conocer los conceptos relaciones con la computaci�n m�vil.
     \item Comprender nuevas tendencias en la computaci�n ubicua.
  \end{unitgoals}

\end{unit}

\begin{unit}{Manejo de datos en ambientes m�viles}{Pitoura:1997:DMM:550358}{10}{2}
   \begin{topics}
     \item Privacidad en Ubiquitous Computing.
     \item Manejo de datos en ambientes m�viles.
     \item Manejo de recursos.
   \end{topics}

   \begin{unitgoals}
     \item Comparar el manejo de datos en sistemas convencionales con el manejo de datos de sistemas m�viles y/o ubicuos.
     \item Evaluar las ventajas y desventajas del manejo de recursos en dispositivos m�viles.
  \end{unitgoals}
\end{unit}

\begin{unit}{Mobile Ad Hoc y Sensor Networks}{AGGL:2005}{8}{2}
        \NCMobileComputingAllTopics
        \NCMobileComputingAllObjectives
\end{unit}

\begin{unit}{Aplicaciones de computaci�n m�vil y ubicua}{Krumm:2009:UCF:1803789}{20}{6}
   \begin{topics}
     \item �reas de aplicaci�n.
     \item Procesamiento de sensores y datasets.
     \item Mobile social networking.
   \end{topics}
   \begin{unitgoals}
     \item Conocer los tipos de aplicaciones que pueden usarse en diferentes �reas de la industria.
     \item Evaluar formas de procesamiento de se�ales de dispositivos m�viles para generar datasets, y posteriomente poder analizarlos.
  \end{unitgoals}

\end{unit}



\begin{coursebibliography}
\bibfile{Computing/CS/CS232W}
\end{coursebibliography}

\end{syllabus}

%\end{document}
