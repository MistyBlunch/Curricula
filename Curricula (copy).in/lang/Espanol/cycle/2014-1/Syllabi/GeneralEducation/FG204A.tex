\begin{syllabus}

\course{FG204A. Teolog�a II}{Obligatorio}{FG204A}

\begin{justification}
Para que la formaci�n de un buen profesional no se desligue ni se oponga sino mas bien contribuya al autentico crecimiento personal requiere de la asimilaci�n de valores s�lidos, horizontes espirituales amplios y una visi�n profunda del entorno cultural. La fe cristiana es uno de los elementos fundamentales de la configuraci�n de la vida y el que hacer cultural de nuestro pa�s, propone los m�s altos valores humanos y el horizonte espiritual mas rico posible.
Por esta raz�n dar una clara y explicita formaci�n cristiana es indispensable.  En este curso tratamos de brindar al alumno el conocimiento b�sico de las razones.
\end{justification}

\begin{goals}
\item \OutcomeFH
\end{goals}

\begin{outcomes}
\ExpandOutcome{FH}{2}
\end{outcomes}

\begin{unit}{Teolog�a fundamental}{Catecismo,Ratzinger,Biblia}{15}{2}
\begin{topics}
	\item La demanda. 
	      \begin{inparaenum}
		    \item Presentaci�n del curso.
		    \item An�lisis de la Felicidad.
		    \item An�lisis del Amor.
		    \item Esquema de antropolog�a.
	      \end{inparaenum}
	\item Ofertas intramundanas. 
	      \begin{inparaenum}
		    \item Hedonismo.
		    \item Materialismo.
		    \item Individualismo.
		    \item Ideolog�as de dominio.
	      \end{inparaenum}
	\item La respuesta cristiana (Parte I). 
	      \begin{inparaenum}
		    \item Conversi�n
		    \item Fe y Raz�n
	      \end{inparaenum}
\end{topics}
\begin{unitgoals}
	\item Que el alumno interprete las manifestaciones de su entorno cultural concreto a la luz de los elementos fundamentales de la persona humana.
	\item Que el alumno descrubra existencialmente la insuficiencia de las respuestas del mero poder, tener o placer y al mismo tiempo se proponga que es lo que realmente anhela en la vida.
	\item Que el alumno descubra la fe cristiana como una respuesta a los anhelos mas profundos del ser humano que ha creado toda una cultura y que est� en la base de nuestra vida nacional.
\end{unitgoals}
\end{unit}

\begin{unit}{Teolog�a dogm�tica}{Catecismo,Ratzinger,Biblia}{15}{2}
\begin{topics}
	\item La respuesta cristiana (Parte II). 
	      \begin{inparaenum}
		  \item Dios Unitrino.
		  \item Doctrina de la creaci�n.
		  \item El pecado original.
	      \end{inparaenum}
	 \item La respuesta cristiana (Parte III). 
	      \begin{inparaenum}
		  \item Identidad y misi�n de Jes�s de Nazareth.
		  \item Vida y obra de Jes�s de Nazareth.
		  \item La gracia
		  \item El Esp�ritu Santo.
		  \item La Iglesia Una, Santa, Cat�lica y Apost�lica.
		  \item La Virgen Mar�a y la Vida Cristiana.
	      \end{inparaenum}
\end{topics}
\begin{unitgoals}
	\item Que el alumno tenga pleno conocimiento de lo que significa la palabra de Dios y de las formas en las que �l se relaciona con nosotros con el fin de que reafirme y fortalezca su f�, enfrente dificultades y asuma el compromiso al que Dios lo invita.
\end{unitgoals}
\end{unit}

\begin{unit}{Teolog�a moral y espiritual}{Catecismo,Ratzinger,Biblia}{15}{2}
\begin{topics}
      \item Fundamentos de teolog�a moral.
      \item Fundamentos de teolog�a espiritual.
      \item La vida espiritual.
      \item La vida de oraci�n.
      \item El combate espiritual.
\end{topics}

\begin{unitgoals}
      \item Que el alumno comprenda la necesidad de una vida espiritual que le permita encontrar el sentido de su vida.
\end{unitgoals}
\end{unit}



\begin{coursebibliography}
\bibfile{GeneralEducation/FG204}
\end{coursebibliography}

\end{syllabus}
