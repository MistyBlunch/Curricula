\begin{syllabus}

\course{CS361. T�picos en Inteligencia Artificial}{Electivos}{CS361}

\begin{justification}
Provee una serie de herramientas para resolver problemas que son dif�ciles de solucionar con los m�todos algor�tmicos tradicionales. Incluyendo heur�sticas, planeamiento, formalismos en la representaci�n del conocimiento y del razonamiento, t�cnicas de aprendizaje en m��quinas, t�cnicas aplicables a los problemas de acci�n y reacci�n: asi como el aprendizaje de lenguaje natural, visi�n artificial y rob�tica entre otros. 
\end{justification}

\begin{goals}
\item Realizar alg�n curso avanzado de Inteligencia Artificial sugerido por el curriculo de la ACM/IEEE.
\end{goals}

%% (1) familiar  (2)usar (3)evaluar
\begin{outcomes}
\item \ShowOutcome{a}{2}
\item \ShowOutcome{h}{2}
\item \ShowOutcome{i}{2}
\item \ShowOutcome{j}{3}
\end{outcomes}

\begin{competences}
\item \ShowCompetence{C1}{a} 
\item \ShowCompetence{C8}{h,i} 
\item \ShowCompetence{CS5}{i,j}
\end{competences}

\begin{unit}{}{Levantamiento del estado del arte}{Russell03,Haykin99,Goldberg89}{60}{C1,C20,CS2}
\begin{topics}
      \item CS360. Sistemas Inteligentes
      \item CS361. Razonamiento automatizado
      \item CS362. Sistemas Basados en Conocimiento
      \item CS363. Aprendizaje de Maquina \cite{Russell03},\cite{Haykin99}
      \item CS364. Sistemas de Planeamiento
      \item CS365. Procesamiento de Lenguaje Natural
      \item CS366. Agentes
      \item CS367. Rob�tica
      \item CS368. Computaci�n Simb�lica
      \item CS369. Algoritmos Gen�ticos \cite{Goldberg89}
\end{topics}
\begin{learningoutcomes}
  \item Profundizar en diversas t�cnicas relacionadas a la Inteligencia Artificial [\Usage]
\end{learningoutcomes}
\end{unit}



\begin{coursebibliography}
\bibfile{Computing/CS/CS361}
\end{coursebibliography}

\end{syllabus}
