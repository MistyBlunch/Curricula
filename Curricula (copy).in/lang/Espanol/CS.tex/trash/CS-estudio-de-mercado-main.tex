\documentclass[a4paper,10pt]{article}
\usepackage[spanish]{babel}

\usepackage{acronym}
%\usepackage{algorithm}
%\usepackage{algorithmic}
\usepackage{amsmath}
\usepackage{amssymb}
\usepackage{graphicx}
\usepackage{hlundef} %If error just comment it, it is to highlight unreferenced citations (?)
\usepackage{ae}
%\usepackage{psfig}
\usepackage{subfigure}
\usepackage{tabularx}
\usepackage{xspace}
\usepackage{pstricks}
\usepackage{pst-tree}
\usepackage{pst-node}
\usepackage{xspace}
%\usepackage{silabo}
\usepackage[a4paper]{geometry}%top=25mm,bottom=5mm,left=15mm,right=15mm
\usepackage{lscape}
\usepackage{longtable}
\usepackage{fancyhdr}


\title{Estudio de mercado para la creaci�n de la carrera de Ciencia de la Computaci�n}
\author{Julieta Flores Luna\footnote{Julieta Flores Luna <jfloresenator@gmail.com>}
\and Johan Chicana D�az\footnote{Johan Chicana D�az <johanchicana@gmail.com>}}
%\institute{Sociedad Peruana de Computaci�n\texttt{ecuadros@spc.org.pe,johanchicana@gmail.com
%\and
%Universidad Nacional de San Agust�n
%}

\begin{document}

\maketitle

\begin{abstract}
El estudio de mercado realizado tuvo como base una consulta a empresarios del medio local y nacional para saber su opini�n en relaci�n a que tan deseable es tener en nuestro medio una carrera de Ciencia de la Computaci�n. Considerando que el nombre de la carrera no es difundido en nuestro medio, lo que se hizo fue preguntar por las caracter�sticas de este perfil profesional y no mucho en funci�n del nombre en si.
\end{abstract}

\input{cs-ficha-tecnica-encuesta}
\input{cs-lista-de-encuestados}
\input{cs-market-survey}

\section{Conclusiones}
Como conclusi�n de toda esta encuesta podemos afirmar que existe un nicho muy interesante de productos de software que no se cubre actualmente y corresponde a profesionales con formaci�n de investigador e innovaci�n permanente de la tecnolog�a existente.

Este breve estudio tambi�n nos permite afirmar que a pesar de ser una carrera poco difundida en nuestro medio hay una clara necesidad de este tipo de profesionales de parte del empresariado.

Es tambi�n importante resaltar que el empresariado peruano no confunde este tipo de desarrollo de software con la carrera de Ingenier�a de Sistemas o Ingenier�a Inform�tica por lo que su creaci�n no afectar�a el campo laboral de los profesionales formados con estos perfiles.
\end{document}
