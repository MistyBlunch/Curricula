\section{Campo y mercado ocupacional}\label{sec:job-positions}
Nuestro egresado podr� prestar sus servicios profesionales en empresas e instituciones p�blicas y privadas 
que requieran sus capacidades en funci�n del desarrollo que oferta, entre ellas:

\begin{itemize}
\item Empresas dedicadas a la producci�n de software con calidad internacional orientado a asuntos organizacionales.
\item Empresas, instituciones y organizaciones que requieran software de calidad para mejorar sus actividades y/o 
      servicios ofertados.
\end{itemize}

Nuestro profesional puede desempe�arse en el mercado laboral sin ning�n problema ya que, en general, la 
exigencia del mercado y campo ocupacional est� mucho m�s orientada al uso de herramientas. 
Nuestro profesional se diferencia debido a que el considera la tecnolog�a como una herramienta fundamental
para el eficiente funcionamiento de una organizaci�n en un ambiente globalizado como el que tenemos en estos momentos.

El profesional en Sistemas de Informaci�n cuenta con rasgos de formaci�n en Ciencia de la Computaci�n por 
lo que est� en condiciones de complementarse con dicha profesi�n en beneficio de la organizaci�n donde se desempe�a.

A medida que la informatizaci�n de las empresas del pa�s avanza, la necesidad de personas 
capacitadas para resolver los problemas de mayor complejidad aumenta. Las organizaciones, sin ninguna duda, 
tambi�n son fuertemente impactadas por un aumento de informaci�n que ya no es posible tratar de forma manual.
En este escenario, el profesional en Sistemas de Informaci�n se perfila como alguien fundamental en la toma 
de decisiones basada en informaci�n que el mismo es capaz de producir en la organizaci�n. 
Esta caracter�stica representa una enorme ventaja en relaci�n a profesionales de otras �reas como 
Administraci�n de Empresas o Ingenier�a Industrial quienes, en muchos casos, toman decisiones empresariales 
con informaci�n procesada en baja escala o en forma manual debido a que no carecen del componente 
tecnol�gico en la misma profundidad que este profesional si tiene a su disposici�n. 

Debido al correcto entendimiento de la computaci�n como una herramienta que puede ser utilizada para 
generar y procesar gran volumen de informaci�n, el profesional de Sistemas de Informaci�n presenta 
la ventaja de poder tomar decisiones apoyadas en informaci�n concreta y a gran escala que el mismo es 
capaz de producir y que otras profesiones no relacionadas a la computaci�n no podr�an hacer con 
la misma velocidad y eficiencia.
