\begin{syllabus}

\course{CB101. �lgebra y Geometr�a}{Obligatorio}{CB101}

\begin{justification}
This course seeks to provide students with certain frameworks with which to analyze the dilemmas that can be presented in their professional practice. The course puts in practice the critical and responsible reasoning of the students, being this a fundamental competence for the decision-making processes that we will assume as professionals and citizens.
\end{justification}

\begin{goals}
\item Introduce students to critical and ethical thinking applied to their professional field.
\item Strengthen the student's ability to think interdisciplinarily.
\item Developing the competence to look at a phenomenon from various disciplines and perspectives generates in the person empathy and respect for diversity of opinion.
\item Ability to work in a team.
\item Ability to identify problems
\item Understand professional and ethical responsibilities.
\item Oral communication skills
\item Understands the impact of engineering solutions in a global, economic, environmental and societal context-
\item He is interested in learning about current issues of Peruvian society and the world.
\item Written communication skills..
\end{goals}

\begin{outcomes}
\ShowOutcome{a}{3}
\ShowOutcome{i}{2}
\ShowOutcome{j}{4}
\end{outcomes}

\begin{unit}{}{�tica, ciencia y tecnolog�a.}{Garcia06}{12}{4}
   \begin{topics}
      \item Definition and scope of ethics Critical thinking / ethical argumentation.
������\item Science and Technology, are engineering and technology issues objective?
������\item Technology: concept and limits.
������\item Importance of ethics in science and engineering.
   \end{topics}
   \begin{learningoutcomes}
      \item .
   \end{learningoutcomes}
\end{unit}

\begin{unit}{}{Traditional ethical values and norms vs. new technological context}{Garcias06}{24}{3}
   \begin{topics}
      \item Ethical considerations in issues of authorship and rights of use in the digital age.Copyright / digital piracy Case: Free software movement.
   \end{topics}

   \begin{learningoutcomes}
      \item .
      \end{learningoutcomes}
\end{unit}

\begin{unit}{}{Responsabilidad en la ciencia e ingenier�a}{Alvarado05}{24}{3}
   \begin{topics}
      \item Scope of the concept Responsibility in science (Imperative of Responsability)
������\item Introduction to the subject Responsibility / freedom
   \end{topics}

   \begin{learningoutcomes}
      \item .
   \end{learningoutcomes}
\end{unit}

\begin{unit}{}{Bio�tica}{Alvarado05}{30}{3}
   \begin{topics}
      \item Central concept. Origins. Significant cases (first case of genome, Bioethics Committee in the USA)
������\item Case: Transhumanism
������\item Case: Right to die. Discussion about the removal of artificial ventilation
   \end{topics}

   \begin{learningoutcomes}
      \item . 
   \end{learningoutcomes}
\end{unit}

\begin{unit}{}{Robot Uprising}{Alvarado05}{30}{3}
   \begin{topics}
      \item  Case:An�lisis: 2021 '?Humans without work?
      \item  Case:Legalization of human marriage / robots
   \end{topics}

   \begin{learningoutcomes}
      \item .
   \end{learningoutcomes}
\end{unit}

\begin{unit}{}{ Tecnolog�a e Ingenier�a Sustentable}{Alvarado05}{30}{3}
   \begin{topics}
      \item Reflection from the Peruvian context about responsibility in science and engineering.
      \item Peruvian cases: Engineering works vs populations
      \item Case:Renewable and sustainable energy in Peru
   \end{topics}

   \begin{learningoutcomes}
      \item . 
   \end{learningoutcomes}
\end{unit}

\begin{unit}{}{ Ciudadan�a y ejercicio de la justicia en la era digital}{Alvarado05}{30}{3}
   \begin{topics}
      \item Introduction to the issue of citizenship in the digital age
      \item Technology, new activism and citizenship
   \end{topics}

   \begin{learningoutcomes}
      \item .
   \end{learningoutcomes}
\end{unit}

\begin{unit}{}{Hackctivismo}{Alvarado05}{30}{3}
   \begin{topics}
      \item Case: Anonnymus
   \end{topics}

   \begin{learningoutcomes}
      \item .
   \end{learningoutcomes}
\end{unit}

\begin{unit}{}{Ciberfeminismo}{Alvarado05}{30}{3}
   \begin{topics}
      \item Citizenship and gender in the digital age Cyberfeminism and techno-feminism: Case: '90: Women in network and E-leusis.
   \end{topics}

   \begin{learningoutcomes}
      \item .
   \end{learningoutcomes}
\end{unit}


\begin{coursebibliography}
\bibfile{GeneralEducation/GH0010}
\end{coursebibliography}

\end{syllabus}
