\begin{syllabus}

\course{ID201. Ingl�s III}{Obligatorio}{ID103}

\begin{justification}
A fundamental part of the integral formation of a professional is the ability to communicate in a foreign language in addition to the native language itself.
It not only broadens its cultural horizon but also allows a more humane and comprehensive view of life. In the case of foreign languages, undoubtedly English is the most practical because it is spoken around the world. There is no country where it is not spoken. In careers related to tourist services, English is perhaps the most important practical tool that the student must master from the outset as part of his / her integral education

\end{justification}

\begin{goals}
\item Train the student to understand and hold a conversation.
\item Provide techniques of llation of ideas .
\end{goals}

\begin{competences}
    \item \ShowCompetence{C25}{f,i}
\end{competences}

\begin{outcomes}
\item \ShowOutcome{f}{2}
\item \ShowOutcome{i}{2}
\end{outcomes}

\begin{unit}{}{Getting to know you!}{Soars022S,Soars022W,Soars022T, Cambridge06, MacGrew99}{0}{3}
   \begin{topics}
      \item Present, Past, and Future Times.
      \item Interrogative sentences with Wh-.
      \item Words with more than one meaning.
      \item Parts of the sentence
      \item Expressions for free time
   \end{topics}

   \begin{learningoutcomes}
      \item At the end of the first unit, each of the students, understanding the grammar of present, past and future times, is able to express a greater number of actions in the form of sentences. He is also able to express ideas in the form of questions.
            Assume the idea of words with more than one meaning. Use social expressions in entertainment situations. 
   \end{learningoutcomes}
\end{unit}

\begin{unit}{}{The way we live!}{Soars022S,Soars022W,Soars022T, Cambridge06, MacGrew99}{0}{3}
   \begin{topics}
      \item Simple present tense.
      \item Present Continuous Time.
      \item Collocations.
      \item Vocabulary of the countries of the world.
      \item Expressions of anger.
      \item Connectors.
   \end{topics}

   \begin{learningoutcomes}
      \item Ath the end of the second unit, students having identified the present form of expression recognize the difference between the forms of the same and apply it properly. They describe the countries accurately. They take expressions to show interest. Use connectors to join various ideas.  
   \end{learningoutcomes}
\end{unit}

\begin{unit}{}{It all went wrong!}{Soars022S,Soars022W,Soars022T, Cambridge06, MacGrew99}{0}{3}
   \begin{topics}
      \item Past simple tense.
      \item Continuous past tense.
      \item Irregular Verbs.
      \item Time expressions.
      \item Connectors of time.
   \end{topics}

   \begin{learningoutcomes}
      \item  At the end of the third unit, students having recognized the characteristics of past times use them properly. They use prefixes and suffixes to create and recognize new words. They describe time in a broad way. They will use conjunctions to unite type ideas.
   \end{learningoutcomes}
\end{unit}

\begin{unit}{}{Let's go shopping!}{Soars022S,Soars022W,Soars022T, Cambridge06, MacGrew99}{0}{3}
   \begin{topics}
      \item Expressions of Indefinite Quantity.
      \item Affirmative sentences, Negatives and Questions.
      \item Use of Articles.
      \item Product prices.
      \item Filling of formats and surveys
      \item Expressions for shopping
   \end{topics}

   \begin{learningoutcomes}
      \item At the end  of the fourth unit, students having identified the idea of quantity express different situations that involve it. Recognize and apply articles to nouns. They assume the idea of shopping with the help of expressions. They express money prices and ideas. They fill several formats. They express attitudes.
   \end{learningoutcomes}

\end{unit}

\begin{unit}{}{What do you want to do?}{Soars022S,Soars022W,Soars022T, Cambridge06, MacGrew99}{0}{3}
   \begin{topics}
      \item Verbal Patterns I.
      \item Future Intentions.
      \item Verbs of Perception.
      \item Vocabulary of feelings.
      \item Expressions of Plans and Ambitions.
   \end{topics}

   \begin{learningoutcomes}
      \item At the end of the fifth unit, students, from the understanding of the idea of verbal patterns, will elaborate sentences using the necessary elements. They will also assimilate the need to express future intentions. They will acquire vocabulary to describe feelings. Expressions will be presented to describe plans and ambitions.
   \end{learningoutcomes}
\end{unit}

\begin{unit}{}{The best in the world!}{Soars022S,Soars022W,Soars022T, Cambridge06, MacGrew99}{0}{3}
   \begin{topics}
      \item What�s it like?.
      \item Adjectives 
      \item Comparative and superlative.
      \item Synonyms and antonyms. 
      \item Indications of direction .
      \item Readings.
   \end{topics}

   \begin{learningoutcomes}
      \item At the end of the sixth unit, students having known the fundamentals of using adjectives, structure sentences with different forms of adjectives in appropriate contexts. They emphasize the difference between types of cities and towns and lifestyles. They use expressions indicating directions.
   \end{learningoutcomes}

\end{unit}

\begin{unit}{}{Fame!}{Soars022S,Soars022W,Soars022T, Cambridge06, MacGrew99}{0}{3}
   \begin{topics}
      \item Present Perfect and Simple Past
      \item Expressions for, ever, since
      \item Adverbs
      \item Expressions that come in pairs
      \item Short answers
      \item Celebrities
   \end{topics}

   \begin{learningoutcomes}
      \item At the end of the seventh unit, students have learned the fundamentals of structuring the present perfect time and differentiate it from the simple past. They emphasize the difference between forms of adjectives. Describe ideas of music. They use expressions to give short answers. They assume the idea of giving extra explanations of the elements of a sentence.
   \end{learningoutcomes}
\end{unit}



\begin{coursebibliography}
\bibfile{ForeignLanguages/ID101}
\end{coursebibliography}

\end{syllabus}
%\end{document}
