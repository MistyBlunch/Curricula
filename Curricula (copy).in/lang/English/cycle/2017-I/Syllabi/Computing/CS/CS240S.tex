\begin{syllabus}

\course{CS240S. Compiladores}{Obligatorio}{CS240S}

\begin{justification}
Que el alumno conozca y comprenda los conceptos y principios
fundamentales de la teor�a de compilaci�n para realizar la
construcci�n de un compilador
\end{justification}

\begin{goals}
\item Conocer las t�cnicas b�sicas empleadas durante el proceso de generaci�n intermedio, optimizaci�n y generaci�n de c�digo.
\item Aprender a implementar peque�os compiladores.
\end{goals}

\begin{outcomes}
\ExpandOutcome{a}{3}
\ExpandOutcome{b}{4}
\ExpandOutcome{j}{4}
\end{outcomes}

\begin{unit}{\PLOverviewDef}{Lou004LP,Pratt98}{8}{4}
   \PLOverviewAllTopics
   \PLOverviewAllObjectives
\end{unit}

\begin{unit}{\PLBasicLanguageTranslationDef}{Aho2008,Aho90,Teu98,Lou004CO,Appe002}{12}{3}
   \PLBasicLanguageTranslationAllTopics
   \PLBasicLanguageTranslationAllObjectives
\end{unit}

\begin{unit}{\PLLanguageTranslatioSystemsDef}{Aho2008,Aho90,Lou004CO,Teu98,Lem96,Appe002}{24}{2}
   \PLLanguageTranslatioSystemsAllTopics
   \PLLanguageTranslatioSystemsAllObjectives
\end{unit}

\begin{unit}{Paralelismo a nivel de instrucci�n}{Aho2008}{4}{2}
  \begin{topics}
     \item Arquitectura de procesadores.
     \item Restricciones de programaci�n de c�digo.
     \item Programaci�n de bloques b�sicos.
     \item Programaci�n de c�digo global.
     \item Canalizaci�n por software.
  \end{topics}

  \begin{unitgoals}
     \item Describir la importancia y poder de la extracci�n de paralelismo de las secuencias de instrucciones.
     \item Explicar los conceptos de bloques b�sicos y c�digo global.
     \item Distinguir los conceptos entre canalizaci�n de instrucciones por software.
  \end{unitgoals}
\end{unit}

\begin{unit}{Optimizaci�n para el paralelismo y la localidad}{Aho2008}{4}{2}
  \begin{topics}
     \item Conceptos b�sicos.
     \item Multiplicaci�n de matrices.
     \item Espacios de iteraciones.
     \item Indices de arreglos afines.
     \item An�lisis de dependencias de datos de arreglos.
     \item B�squeda del paralelismo sin sincronizaci�n.
     \item Sincronizaci�n entre ciclos paralelos.
  \end{topics}

  \begin{unitgoals}
     \item Dise�ar, codificar programas para c�lculos paralelos.
     \item Identificar las propiedades b�sicas del paralelismo.
     \item Aplicar los fundamentos del paralelismo en la programaci�n.
  \end{unitgoals}
\end{unit}



\begin{coursebibliography}
\bibfile{Computing/CS/CS240S}
\end{coursebibliography}

\end{syllabus}
