\begin{sumilla}

\curso{CF141. F\'isica General I}{Obligatorio}{CF141}

\begin{fundamentacion}
Este curso es \'util en esta carrera para que el alumno aprenda a mostrar un alto grado de dominio de las leyes del movimiento de la F\'isica General.
\end{fundamentacion}

\begin{objetivosdelcurso}
\item Capacitar y presentar al estudiante los principios b�sicos de la F�sica como ciencia natural abarcando sus t�picos m�s importantes y su relaci�n con los problemas cotidianos.
\end{objetivosdelcurso}

\begin{outcomes}
\ExpandOutcome{a}
\ExpandOutcome{i}
\ExpandOutcome{j}
\end{outcomes}

\begin{unit}{FI1. Introducci\'on}{Serway2002,Alonso95}{4}
\begin{topicos}
      \item La investigaci\'on cient\'ifica. El m\'etodo cient\'ifico.
      \item Concepto de Qu\'imica. La Qu\'imica en la actualidad.
      \item Materia. Clasificaci\'on y propiedades f\'isicas, qu\'imicas, intensivas y extensivas.
      \item Modelo idealizado.
      \item Magnitudes f\'isicas.
      \item Propiedades de los vectores.
      \item Componentes de un vector y vectores unitarios.
      \item Producto de vectores.
      \item Ejercicios y problemas.
   \end{topicos}

   \begin{objetivos}
      \item Entender y trabajar con las magnitudes f\'isicas del SI.
      \item Abstraer de la naturaleza los conceptos f\'isicos rigurosos y
      representarlos en modelos vectoriales.
      \item Entender y aplicar los conceptos vectoriales a problemas f\'isicos reales.
   \end{objetivos}
\end{unit}

\begin{unit}{FI2. Movimiento de part\'iculas en una dimensi\'on}{Serway2002,Alonso95}{2}
\begin{topicos}
      \item Desplazamiento, velocidad y rapidez.
      \item Velocidad instant\'anea.
      \item Aceleraci\'on media e instant\'anea.
      \item Movimiento con aceleraci\'on constante.
      \item Ca\'ida libre de los cuerpos.
      \item Ejercicios y problemas.
    \end{topicos}
   \begin{objetivos}
      \item Describir matem\'aticamente el movimiento mec\'anico de una part\'icula unidimensional como un cuerpo de dimensiones despreciables.
      \item Conocer y aplicar conceptos de magnitudes cinem\'aticas.
      \item Describir el comportamiento de movimiento de part'iculas, te\'orica y gr\'aficamente.
      \item Conocer representaciones vectoriales de estos movimientos unidimensionales.
      \item Resolver problemas.
   \end{objetivos}
\end{unit}

\begin{unit}{FI3. Movimiento de part\'iculas en dos y tres dimensiones}{Serway2002,Alonso95}{4}
\begin{topicos}
      \item Desplazamiento y velocidad.
      \item El vector aceleraci\'on.
      \item Movimiento parab\'olico.
      \item Movimiento circular.
      \item Componentes tangencial y radial de la aceleraci\'on.
      \item Ejercicios y problemas
\end{topicos}

   \begin{objetivos}
      \item Describir matematicamente el movimiento mec\'anico de una part\'icula en dos y tres dimensiones como un cuerpo de dimensiones despreciables.
      \item Conocer y aplicar conceptos de magnitudes cinem\'aticas vectoriales en dos y tres dimensiones.
      \item Describir el comportamiento de movimiento de part\'iculas te\'orica y gr\'aficamente en dos y tres dimensiones.
      \item Conocer y aplicar conceptos del movimiento circular.
      \item Resolver problemas.
   \end{objetivos}
\end{unit}

\begin{unit}{FI4. Leyes del movimiento}{Serway2002,Alonso95}{6}
\begin{topicos}
      \item Fuerza e interacciones.
      \item Primera ley de Newton.
      \item Masa inercial.
      \item Segunda ley de Newton.
      \item Peso.
      \item Diagramas de cuerpo libre.
      \item Tercera Ley de newton.
      \item Fuerzas de fricci\'on.
      \item Din\'amica del movimiento circular.
      \item Ejercicios y problemas.
   \end{topicos}

   \begin{objetivos}
      \item Conocer los conceptos de fuerza.
      \item Conocer las interacciones mas importantes de la naturaleza y representarlos en un diagrama de cuerpo libre.
      \item Conocer los conceptos de equilibrio est\'atico.
      \item Conocer y aplicar las leyes del movimiento y caracterizarlos vectorialmente.
      \item Conocer y aplicar las leyes de Newton.
      \item Resolver problemas.
   \end{objetivos}
\end{unit}

\begin{unit}{FI5. Trabajo y Energ\'ia}{Serway2002,Alonso95}{4}
\begin{topicos}
	\item Trabajo realizado por una fuerza constante.
	\item Trabajo realizado por fuerzas variables.
	\item Trabajo y energ\'ia cin\'etica.
	\item Potencia.
	\item Energ\'ia potencial gravitatoria.
	\item Energ\'ia potencial el\'astica.
	\item Fuerzas conservativas y no conservativas.
	\item Principios de conservaci\'on de la energ\'ia.
	\item Ejercicios y problemas.
\end{topicos}

   \begin{objetivos}
      \item Establecer los conceptos de energ\'ia f\'isica. (F\'isica cl\'asica)
      \item Conocer algunas formas de energ\'ia.
      \item Establecer la relaci\'on entre trabajo y energ\'ia.
      \item Conocer y aplicar los conceptos de conservaci\'on de energ\'ia.
      \item Resolver problemas.
   \end{objetivos}
\end{unit}

\begin{unit}{FI6. Momento lineal}{Serway2002,Alonso95}{3}
\begin{topicos}
      \item Momento lineal.
      \item Conservaci\'on del momento lineal.
      \item Centro de masa y de gravedad.
      \item Movimiento de un sistema de part\'iculas.
      \item Ejercicios y problemas.
  \end{topicos}

   \begin{objetivos}
      \item Establecer los conceptos de momento lineal.
      \item Conocer los conceptos de conservaci\'on del momento lineal.
      \item Conocer el movimiento de un sistema de part\'iculas.
      \item Resolver problemas.
   \end{objetivos}
\end{unit}

\begin{unit}{FI7. Rotaci\'on de cuerpos r\'igidos}{Serway2002,Alonso95}{4}
\begin{topicos}
      \item Velocidad y aceleraciones angulares.
      \item Rotaci\'on con aceleraci\'on angular constante.
      \item Relaci\'on entre cinem\'atica lineal y angular.
      \item Energ\'ia en el movimiento de rotaci\'on.
      \item Momento de torsi\'on.
      \item Relaci\'on entre momento de torsi\'on y aceleraci\'on angular.
      \item Ejercicios y problemas.
   \end{topicos}

   \begin{objetivos}
      \item Conocer los conceptos b\'asicos de cuerpo r\'igido.
      \item Conocer y aplicar conceptos de rotaci\'on de cuerpo r\'igido.
      \item Conocer conceptos de torsi\'on.
      \item Aplicar conceptos de energ\'ia al movimiento de rotaci\'on.
      \item Resolver problemas.
   \end{objetivos}
\end{unit}

\begin{unit}{FI8. Din\'amica del movimiento de rotaci\'on}{Serway2002,Alonso95}{3}
\begin{topicos}
      \item Momento de torsi\'on y aceleraci\'on angular de un cuerpo r\'igido.
      \item Rotaci\'on de un cuerpo r\'igido sobre un eje m\'ovil.
      \item Trabajo y potencia en el movimiento de rotaci\'on.
      \item Momento angular.
      \item Conservaci\'on del momento angular.
      \item Ejercicios y problemas.
    \end{topicos}

   \begin{objetivos}
      \item Conocer conceptos b\'asicos de din\'amica de rotaci\'on.
      \item Conocer y aplicar conceptos de torsi\'on.
      \item Entender el momento angular y su conservaci\'on.
      \item Resolver problemas.
   \end{objetivos}
\end{unit}

\begin{bibliografia}
\bibfile{CF141}
\end{bibliografia}
\end{sumilla}


