\begin{sumilla}

\curso{CM251. \'Algebra Lineal}{Obligatorio}{CM251}

\begin{fundamentacion}
En este curso se estudiar\'an los espacios vectoriales, determinantes, transformaciones lineales, \'Algebra multilineal.
Autovalores y formas can\'onicas, operadores sobre espacios con producto interno formas bilineales y cuadr\'aticas. Geometr\'ia Af\'in y Transformaciones afines.
\end{fundamentacion}

\begin{objetivosdelcurso}
\item  Lograr que el alumno asimile los conceptos b\'asicos sobre espacios vectoriales, transformaciones lineales, matrices, as\'i como determinantes y sus aplicaciones
\item  Dotar al estudiante de los conocimientos b\'asicos de temas de \'Algebra Lineal que son de utilidad para el estudio de otros cursos y sus aplicaciones
\end{objetivosdelcurso}

\begin{outcomes}
\ExpandOutcome{a}
\ExpandOutcome{i}
\ExpandOutcome{j}
\end{outcomes}

\begin{unit}{Espacios Vectoriales en general}{Halmos58}{6}
   \begin{topicos}
         \item  Definici\'on y ejemplos
	 \item  Subespacios, sus propiedades. Suma y suma directa
         \item  Independencia lineal, base y dimensi\'on
	 \item  Producto interno. Bases ortogonales; ortogonalizaci\'on de Gram-Schmidt
         \item  Distancia de un punto a una variedad lineal. (Aplicaci�n a la Geometr�a)
         \item  El espacio cociente
   \end{topicos}

   \begin{objetivos}
         \item  Entender los conceptos y caracter\'isticas de los espacios vectoriales
         \item  Resolver problemas
   \end{objetivos}
\end{unit}

\begin{unit}{Transformaciones lineales}{Halmos58,Hoffman71}{8}
   \begin{topicos}
         \item  Definici\'on y ejemplos.
	 \item  Teorema fundamental de las transformaciones lineales y sus consecuencias.
         \item  \'Algebra de las transformaciones lineales. Espacio de las transformaciones lineales. Espacio dual.
	 \item  Matrices. Sus operaciones. Rango e inversa. Matriz asociada a una transformaci\'on lineal. Matrices equivalentes y semejantes.
         \item  Autovalores y autovectores. Forma triangular. Teorema de Cayley-Hamilton. Forma racional y de  Jordan. Transformaciones lineales diagonalizables, criterios.
	 \item  Tipos especiales de matrices: Sim\'etricas, antisim\'etricas, unitaria y ortogonal. Su diagonalizaci\'on.
   \end{topicos}

   \begin{objetivos}
         \item  Entender los conceptos y caracter\'isticas de las Transformaciones lineales
         \item  Resolver problemas
   \end{objetivos}
\end{unit}

\begin{unit}{Determinantes}{Lages95}{6}
   \begin{topicos}
         \item  Funci\'on determinante.
	 \item  Propiedades.
         \item  Existencia y Unicidad del determinante
	 \item  C\'alculo del determinante y determinante de una transformaci\'on lineal.
         \item  Cofactores, menores y adjuntos.
	\item Determinante y rango de una matriz. Aplicaciones.
   \end{topicos}

   \begin{objetivos}
         \item  Entender los conceptos y caracter\'isticas de los determinantes
         \item  Resolver problemas
   \end{objetivos}
\end{unit}

\begin{unit}{\'Algebra Multilineal}{Lang90}{8}
   \begin{topicos}
         \item  Aplicaciones bilineales.
	 \item  Productos tensoriales.
         \item  Isomorfismos can\'onicos.
	 \item  Producto tensoriales de aplicaciones lineales.
         \item  Cambio de coordenadas de un tensor.
	 \item  Producto tensorial de espacios vectoriales.
         \item  \'Algebra tensorial de un espacio vectorial.
   \end{topicos}

   \begin{objetivos}
         \item  Entender y aplicar los conceptos del \'Algebra Multilineal
         \item  Resolver problemas
   \end{objetivos}
\end{unit}

\begin{unit}{Autovalores y formas can\'onicas}{Chavez05}{8}
   \begin{topicos}
	\item  Valores y vectores propios.
	\item  Triangulaci\'on de matrices. El Teorema Cayley-Hamilton
	\item  Criterios de diagonalizaci\'on.
	\item  Matrices nilpotentes.
	\item Forma can�nica de Jordan.
	\item La exponencial de una matriz.
   \end{topicos}

   \begin{objetivos}
         \item  Entender y aplicar los conceptos de Autovalores y formas can\'onicas.
         \item  Resolver problemas.
   \end{objetivos}
\end{unit}

\begin{unit}{Operadores sobre espacios con producto interno}{Nomizu66}{6}
   \begin{topicos}
	\item  La adjunta de un operador.
	\item  Matrices positivas.
	\item  Isometr�as.
	\item  Proyecci�n perpendicular.
	\item  Operadores autoadjuntos. El Teorema Espectral.
	\item  Operadores normales.
	\item Funciones definidas sobre transformaciones lineales.
   \end{topicos}

   \begin{objetivos}
         \item  Entender y aplicar los conceptos de Operadores sobre espacios con producto interno
         \item  Resolver problemas
   \end{objetivos}
\end{unit}

\begin{unit}{Formas bilineales y cuadr\'aticas}{Kaplansky74}{8}
   \begin{topicos}
	\item Formas bilineales.
	\item Suma directa y diagonalizaci\'on.
	\item El teorema de inercia.
	\item Teorema de cancelaci\'on de Witt.
	\item Planos hiperb\'olicos, formas alternadas.
	\item Witt equivalencia.
	\item Formas hermitianas.
   \end{topicos}

   \begin{objetivos}
         \item Entender y aplicar los conceptos de Formas bilineales y cuadr\'aticas.
         \item Resolver problemas.
   \end{objetivos}
\end{unit}

\begin{unit}{Geometr\'ia afin}{Chavez05}{6}
   \begin{topicos}
         \item  Planos afines.
	 \item  Planos proyectivos.
         \item  Transformaciones proyectivas.
	 \item  Raz\'on doble.
         \item  C\'onicas.
         \item  Espacios de dimensi\'on superior.
   \end{topicos}

   \begin{objetivos}
         \item  Entender y aplicar los conceptos de Geometr\'ia afin
         \item  Resolver problemas
   \end{objetivos}
\end{unit}

\begin{bibliografia}
\bibfile{CM251}
\end{bibliografia}

\end{sumilla}


