
\newcommand{\DSUNODef}{DS1. Funciones, Relaciones y Conjuntos. }
\newcommand{\DSUNOHours}{6}
\newcommand{\DSUNOTopicFunciones}{Funciones (surjeción, injección, inversos, composición). }
\newcommand{\DSUNOTopicRelaciones}{Relaciones (reflexibilidad, simetría, transitividad, relaciones de equivalencia). }
\newcommand{\DSUNOTopicConjuntos}{Conjuntos (Diagramas de Venn, complementos, Producto cartesiano, conjunto potencia). }
\newcommand{\DSUNOTopicPrincipio}{Principio de las casillas (\textit{pigeonhole}). }
\newcommand{\DSUNOTopicCardinalidad}{Cardinalidad y Conteo.}
\newcommand{\DSUNOAllTopics}%
{%
\begin{topicos}%
\item \DSUNOTopicFunciones%
\item \DSUNOTopicRelaciones%
\item \DSUNOTopicConjuntos%
\item \DSUNOTopicPrincipio%
\item \DSUNOTopicCardinalidad%
\end{topicos}%
}
\newcommand{\DSUNOObjUNO}{Explicar con ejemplos la terminología básica de funciones, relaciones y conjuntos. }
\newcommand{\DSUNOObjDOS}{Desarrollar las operaciones asociadas con conjuntos, funciones y relaciones. }
\newcommand{\DSUNOObjTRES}{Relacionar ejemplos prácticos al modelo apropiado de conjunto, función o relación e interpretar la operación asociada y terminología en el contexto. }
\newcommand{\DSUNOObjCUATRO}{Demostrar los principios básicos del conteo, incluyendo el uso de la diagonalización y el principio de las casillas (\textit{pigeonhole}). }
\newcommand{\DSUNOAllObjectives}%
{%
\begin{objetivos}%
\item \DSUNOObjUNO%
\item \DSUNOObjDOS%
\item \DSUNOObjTRES%
\item \DSUNOObjCUATRO%
\end{objetivos}%
}

\newcommand{\DSDOSDef}{DS2. Lógica Básica. }
\newcommand{\DSDOSHours}{10}
\newcommand{\DSDOSTopicLogica}{Lógica proposicional. }
\newcommand{\DSDOSTopicConectivos}{Conectivos lógicos. }
\newcommand{\DSDOSTopicTablas}{Tablas de verdad. }
\newcommand{\DSDOSTopicFormas}{Formas normales (conjuntiva y disyuntiva). }
\newcommand{\DSDOSTopicValidacion}{Validación. }
\newcommand{\DSDOSTopicLogicade}{Lógica de predicados. }
\newcommand{\DSDOSTopicCuantificacion}{Cuantificación universal y existencial. }
\newcommand{\DSDOSTopicModus}{Modus ponens y modus tollens. }
\newcommand{\DSDOSTopicLimitaciones}{Limitaciones de la lógica de predicados. }
\newcommand{\DSDOSAllTopics}%
{%
\begin{topicos}%
\item \DSDOSTopicLogica%
\item \DSDOSTopicConectivos%
\item \DSDOSTopicTablas%
\item \DSDOSTopicFormas%
\item \DSDOSTopicValidacion%
\item \DSDOSTopicLogicade%
\item \DSDOSTopicCuantificacion%
\item \DSDOSTopicModus%
\item \DSDOSTopicLimitaciones%
\end{topicos}%
}
\newcommand{\DSDOSObjUNO}{Aplicar métodos formales de lógica simbólica proposicional y de predicados. }
\newcommand{\DSDOSObjDOS}{Describir como las herramientas formales de lógica simbólica son utilizadas para modelar algoritmos en situaciones reales. }
\newcommand{\DSDOSObjTRES}{Usar demostraciones lógico-formales y razonamiento lógico para solucionar problemas tales como rompecabezas (\textit{puzzles}). }
\newcommand{\DSDOSObjCUATRO}{Describir la importancia y limitaciones de la lógica de predicados. }
\newcommand{\DSDOSAllObjectives}%
{%
\begin{objetivos}%
\item \DSDOSObjUNO%
\item \DSDOSObjDOS%
\item \DSDOSObjTRES%
\item \DSDOSObjCUATRO%
\end{objetivos}%
}

\newcommand{\DSTRESDef}{DS3. Técnicas de Validación. }
\newcommand{\DSTRESHours}{12}
\newcommand{\DSTRESTopicNociones}{Nociones de implicación, converso, inversa, contrapositiva, negación y contradicción. }
\newcommand{\DSTRESTopicEstructura}{Estructura de pruebas formales. }
\newcommand{\DSTRESTopicPruebas}{Pruebas Directas. }
\newcommand{\DSTRESTopicPruebasy}{Pruebas y contra-ejemplos. }
\newcommand{\DSTRESTopicPruebaspor}{Pruebas por contraposición. }
\newcommand{\DSTRESTopicPruebasporcontradiccion}{Pruebas por contradicción. }
\newcommand{\DSTRESTopicInduccion}{Inducción Matemática. }
\newcommand{\DSTRESTopicInduccionfuerte}{Inducción fuerte. }
\newcommand{\DSTRESTopicDefiniciones}{Definiciones matemáticas recursivas. }
\newcommand{\DSTRESTopicEl}{El principio del buen orden. }
\newcommand{\DSTRESAllTopics}%
{%
\begin{topicos}%
\item \DSTRESTopicNociones%
\item \DSTRESTopicEstructura%
\item \DSTRESTopicPruebas%
\item \DSTRESTopicPruebasy%
\item \DSTRESTopicPruebaspor%
\item \DSTRESTopicPruebasporcontradiccion%
\item \DSTRESTopicInduccion%
\item \DSTRESTopicInduccionfuerte%
\item \DSTRESTopicDefiniciones%
\item \DSTRESTopicEl%
\end{topicos}%
}
\newcommand{\DSTRESObjUNO}{Bosquejar la estructura básica y dar ejemplos de cada tipo de prueba descrita en esta unidad. }
\newcommand{\DSTRESObjDOS}{Discutir que tipo de prueba es mejor para un problema dado. }
\newcommand{\DSTRESObjTRES}{Relacionar las ideas de inducción matemática con la recursividad y con estructuras definidas recursivamente. }
\newcommand{\DSTRESObjCUATRO}{Identificar las diferencias entre inducción matemática e inducción fuerte dando ejemplos de su apropiado uso en cada caso. }
\newcommand{\DSTRESAllObjectives}%
{%
\begin{objetivos}%
\item \DSTRESObjUNO%
\item \DSTRESObjDOS%
\item \DSTRESObjTRES%
\item \DSTRESObjCUATRO%
\end{objetivos}%
}

\newcommand{\DSCUATRODef}{DS4. Conceptos Básicos de Conteo. }
\newcommand{\DSCUATROHours}{5}
\newcommand{\DSCUATROTopicArgumentos}{Argumentos de conteo. 
	\begin{inparaenum}[ a)]%
		\item Reglas de suma y producto. %
		\item Principios de inclusión y exclusión. %
		\item Progresiones aritméticas y geométricas. %
		\item Números de Fibonacci. %
	\end{inparaenum}%
}
\newcommand{\DSCUATROTopicPrincipio}{Principio de las casillas (\textit{pigeonhole}). }
\newcommand{\DSCUATROTopicPermutaciones}{Permutaciones y combinaciones. 
	\begin{inparaenum}[ a)]%
		\item Definiciones básicas. %
		\item Identidad de Pascal. %
		\item El teorema del binomio. %
	\end{inparaenum}%
}
\newcommand{\DSCUATROTopicSolucion}{Solución de relaciones de recurrencia. 
	\begin{inparaenum}[ a)]%
		\item Ejemplos comunes. %
		\item El teorema maestro. %
	\end{inparaenum}%
}
\newcommand{\DSCUATROAllTopics}%
{%
\begin{topicos}%
\item \DSCUATROTopicArgumentos%
\item \DSCUATROTopicPrincipio%
\item \DSCUATROTopicPermutaciones%
\item \DSCUATROTopicSolucion%
\end{topicos}%
}
\newcommand{\DSCUATROObjUNO}{Calcular permutaciones y combinaciones de un conjunto e interpretar el significado en el contexto de la aplicación particular. }
\newcommand{\DSCUATROObjDOS}{Establecer la definición del Teorema Maestro. }
\newcommand{\DSCUATROObjTRES}{Solucionar una clase de ecuaciones recurrentes básicas. }
\newcommand{\DSCUATROObjCUATRO}{Analizar un problema para crear ecuaciones de recurrencia relevantes o identificar preguntas importantes de conteo. }
\newcommand{\DSCUATROAllObjectives}%
{%
\begin{objetivos}%
\item \DSCUATROObjUNO%
\item \DSCUATROObjDOS%
\item \DSCUATROObjTRES%
\item \DSCUATROObjCUATRO%
\end{objetivos}%
}

\newcommand{\DSCINCODef}{DS5. Gráfos y Árboles. }
\newcommand{\DSCINCOHours}{4}
\newcommand{\DSCINCOTopicArboles}{Árboles. }
\newcommand{\DSCINCOTopicGrafos}{Grafos no dirigidos. }
\newcommand{\DSCINCOTopicGrafosdirigidos}{Grafos dirigidos. }
\newcommand{\DSCINCOTopicArbolesde}{Árboles de expansión. }
\newcommand{\DSCINCOTopicEstrategias}{Estrategias de recorrido (\textit{traversal strategies}). }
\newcommand{\DSCINCOAllTopics}%
{%
\begin{topicos}%
\item \DSCINCOTopicArboles%
\item \DSCINCOTopicGrafos%
\item \DSCINCOTopicGrafosdirigidos%
\item \DSCINCOTopicArbolesde%
\item \DSCINCOTopicEstrategias%
\end{topicos}%
}
\newcommand{\DSCINCOObjUNO}{Ilustrar con ejemplos la terminología básica de teoría de grafos, y algunas de las propiedades y casos especiales de cada una. }
\newcommand{\DSCINCOObjDOS}{Mostrar diferentes métodos de recorrido en árboles y grafos. }
\newcommand{\DSCINCOObjTRES}{Modelar problemas en Ciencias de la Computación usando grafos y árboles. }
\newcommand{\DSCINCOObjCUATRO}{Relacionar grafos y árboles con estructura de datos, algoritmos y conteo. }
\newcommand{\DSCINCOAllObjectives}%
{%
\begin{objetivos}%
\item \DSCINCOObjUNO%
\item \DSCINCOObjDOS%
\item \DSCINCOObjTRES%
\item \DSCINCOObjCUATRO%
\end{objetivos}%
}

\newcommand{\DSSEISDef}{DS6. Probabilidad Discreta. }
\newcommand{\DSSEISHours}{6}
\newcommand{\DSSEISTopicEspacios}{Espacios de probabilidad finita, medidas de probabilidad, eventos. }
\newcommand{\DSSEISTopicProbabilidad}{Probabilidad condicional, independencia, Teorema de Bayes. }
\newcommand{\DSSEISTopicVariables}{Variables aleatorias enteras, esperanza. }
\newcommand{\DSSEISAllTopics}%
{%
\begin{topicos}%
\item \DSSEISTopicEspacios%
\item \DSSEISTopicProbabilidad%
\item \DSSEISTopicVariables%
\end{topicos}%
}
\newcommand{\DSSEISObjUNO}{Calcular las probabilidades de eventos y la esperanza de variables aleatorias para problemas elementales como juegos de azar. }
\newcommand{\DSSEISObjDOS}{Diferenciar entre eventos dependientes e indepedientes. }
\newcommand{\DSSEISObjTRES}{Aplicar el teorema del binomio a eventos independientes y el teorema Bayes a eventos dependientes. }
\newcommand{\DSSEISObjCUATRO}{Aplicar las herramientas de probabilidad para resolver problemas tales como el método de Monte Carlo y el análisis de caso promedio de algoritmos, y \textit{hashing}. }
\newcommand{\DSSEISAllObjectives}%
{%
\begin{objetivos}%
\item \DSSEISObjUNO%
\item \DSSEISObjDOS%
\item \DSSEISObjTRES%
\item \DSSEISObjCUATRO%
\end{objetivos}%
}

\newcommand{\PFUNODef}{PF1. Fundamentos de  Programación. }
\newcommand{\PFUNOHours}{9}
\newcommand{\PFUNOTopicSintaxis}{Sintaxis básica y semántica de un lenguaje de más alto nivel. }
\newcommand{\PFUNOTopicVariables}{Variables, tipos, expresiones, y asignaciones. }
\newcommand{\PFUNOTopicEntrada}{Entrada y salida simple. }
\newcommand{\PFUNOTopicEstructuras}{Estructuras de control condicionales e iterativas. }
\newcommand{\PFUNOTopicFunciones}{Funciones y paso de parámetros. }
\newcommand{\PFUNOTopicDescomposicion}{Descomposición estructurada. }
\newcommand{\PFUNOAllTopics}%
{%
\begin{topicos}%
\item \PFUNOTopicSintaxis%
\item \PFUNOTopicVariables%
\item \PFUNOTopicEntrada%
\item \PFUNOTopicEstructuras%
\item \PFUNOTopicFunciones%
\item \PFUNOTopicDescomposicion%
\end{topicos}%
}
\newcommand{\PFUNOObjUNO}{Analizar y explicar el comportamiento de programas simples involucrando las estructuras de programación fundamental cubiertas por esta unidad. }
\newcommand{\PFUNOObjDOS}{Modificar y extender programas cortos que usan condicionales estándar, estructuras de control iterativo y funciones. }
\newcommand{\PFUNOObjTRES}{Diseñar, implementar, probar y depurar un programa que use cada una de  las siguientes estructuras fundamentales de programación: cálculos básicos, entrada y salida simple, estructuras estándar condicionales e iterativas y definición de funciones. }
\newcommand{\PFUNOObjCUATRO}{Escoger la estructura apropiada condicional y/o iterativa para una estructura de programación dada. }
\newcommand{\PFUNOObjCINCO}{Aplicar técnicas de descomposición  estructurada o funcional para dividir un programa en pequeñas partes. }
\newcommand{\PFUNOObjSEIS}{Describir los mecanismos de paso de parámetros. }
\newcommand{\PFUNOAllObjectives}%
{%
\begin{objetivos}%
\item \PFUNOObjUNO%
\item \PFUNOObjDOS%
\item \PFUNOObjTRES%
\item \PFUNOObjCUATRO%
\item \PFUNOObjCINCO%
\item \PFUNOObjSEIS%
\end{objetivos}%
}

\newcommand{\PFDOSDef}{PF2. Algoritmos y Resolución de Problemas. }
\newcommand{\PFDOSHours}{6}
\newcommand{\PFDOSTopicEstrategias}{Estrategias para la solución de problemas. }
\newcommand{\PFDOSTopicEl}{El rol de los algoritmos en el proceso de solución de problemas. }
\newcommand{\PFDOSTopicEstrategiasde}{Estrategias de implementación para algoritmos. }
\newcommand{\PFDOSTopicEstrategiasdedepuracion}{Estrategias de depuración. }
\newcommand{\PFDOSTopicElConcepto}{El Concepto y propiedades de algoritmos. }
\newcommand{\PFDOSAllTopics}%
{%
\begin{topicos}%
\item \PFDOSTopicEstrategias%
\item \PFDOSTopicEl%
\item \PFDOSTopicEstrategiasde%
\item \PFDOSTopicEstrategiasdedepuracion%
\item \PFDOSTopicElConcepto%
\end{topicos}%
}
\newcommand{\PFDOSObjUNO}{Discutir la importancia de los algoritmos en el proceso de solución de problemas. }
\newcommand{\PFDOSObjDOS}{Identificar las propiedades necesarias de un buen algoritmo. }
\newcommand{\PFDOSObjTRES}{Crear algoritmos para resolver  problemas simples. }
\newcommand{\PFDOSObjCUATRO}{Usar pseudocódigo o un lenguaje de programación para implementar, probar y depurar algoritmos para resolver problemas simples. }
\newcommand{\PFDOSObjCINCO}{Describir estrategias útiles para depuración. }
\newcommand{\PFDOSAllObjectives}%
{%
\begin{objetivos}%
\item \PFDOSObjUNO%
\item \PFDOSObjDOS%
\item \PFDOSObjTRES%
\item \PFDOSObjCUATRO%
\item \PFDOSObjCINCO%
\end{objetivos}%
}

\newcommand{\PFTRESDef}{PF3. Estructuras de Datos Fundamentales. }
\newcommand{\PFTRESHours}{14}
\newcommand{\PFTRESTopicTipos}{Tipos primitivos. }
\newcommand{\PFTRESTopicArreglos}{Arreglos. }
\newcommand{\PFTRESTopicRegistros}{Registros. }
\newcommand{\PFTRESTopicCadenas}{Cadenas y procesamiento de cadenas. }
\newcommand{\PFTRESTopicRepresentacion}{Representación de datos en memoria. }
\newcommand{\PFTRESTopicAsignacion}{Asignación de memoria estática, en la pila y en el montículo (\textit{heap}). }
\newcommand{\PFTRESTopicAdministracion}{Administración del almacenamiento en tiempo de ejecución. }
\newcommand{\PFTRESTopicPunteros}{Punteros y referencias. }
\newcommand{\PFTRESTopicEstructuras}{Estructuras enlazadas. }
\newcommand{\PFTRESTopicEstrategias}{Estrategias de implementación para pilas, colas y tablas \textit{hash}. }
\newcommand{\PFTRESTopicEstrategiasde}{Estrategias de implementación para grafos y árboles. }
\newcommand{\PFTRESTopicEstrategiaspara}{Estrategias para escoger la estructura de datos correcta. }
\newcommand{\PFTRESAllTopics}%
{%
\begin{topicos}%
\item \PFTRESTopicTipos%
\item \PFTRESTopicArreglos%
\item \PFTRESTopicRegistros%
\item \PFTRESTopicCadenas%
\item \PFTRESTopicRepresentacion%
\item \PFTRESTopicAsignacion%
\item \PFTRESTopicAdministracion%
\item \PFTRESTopicPunteros%
\item \PFTRESTopicEstructuras%
\item \PFTRESTopicEstrategias%
\item \PFTRESTopicEstrategiasde%
\item \PFTRESTopicEstrategiaspara%
\end{topicos}%
}
\newcommand{\PFTRESObjUNO}{Discutir la representación y uso de tipos de datos primitivos y estructuras de datos incorporadas en el lenguaje. }
\newcommand{\PFTRESObjDOS}{Describir como la estructuras de datos en la lista de temas son asignadas y usadas en memoria. }
\newcommand{\PFTRESObjTRES}{Describir aplicaciones comunes para cada estructura de datos en la lista de temas. }
\newcommand{\PFTRESObjCUATRO}{Implementar estructuras de datos  definidas por el usuario en un lenguaje de alto nivel. }
\newcommand{\PFTRESObjCINCO}{Comparar implementaciones alternativas de estructuras de datos considerando su desempeño. }
\newcommand{\PFTRESObjSEIS}{Escribir programas que usan cada una de las siguientes estructuras de datos: arreglos, registros, cadenas, listas enlazadas, pilas, colas y tablas \textit{hash}. }
\newcommand{\PFTRESObjSIETE}{Comparar y contrastar los costos y beneficios de las implementaciones dinámicas y estáticas de las estructuras de datos. }
\newcommand{\PFTRESObjOCHO}{Escoger la estructura de datos apropiada para modelar un problema dado. }
\newcommand{\PFTRESAllObjectives}%
{%
\begin{objetivos}%
\item \PFTRESObjUNO%
\item \PFTRESObjDOS%
\item \PFTRESObjTRES%
\item \PFTRESObjCUATRO%
\item \PFTRESObjCINCO%
\item \PFTRESObjSEIS%
\item \PFTRESObjSIETE%
\item \PFTRESObjOCHO%
\end{objetivos}%
}

\newcommand{\PFCUATRODef}{PF4. Recursividad. }
\newcommand{\PFCUATROHours}{5}
\newcommand{\PFCUATROTopicEl}{El concepto de recursividad. }
\newcommand{\PFCUATROTopicFunciones}{Funciones matemáticas recursivas. }
\newcommand{\PFCUATROTopicProcedimientos}{Procedimientos recursivos simples. }
\newcommand{\PFCUATROTopicEstrategias}{Estrategias de dividir y conquistar. }
\newcommand{\PFCUATROTopicBacktracking}{Backtracking recursivo. }
\newcommand{\PFCUATROTopicImplementacion}{Implementación de recursividad. }
\newcommand{\PFCUATROAllTopics}%
{%
\begin{topicos}%
\item \PFCUATROTopicEl%
\item \PFCUATROTopicFunciones%
\item \PFCUATROTopicProcedimientos%
\item \PFCUATROTopicEstrategias%
\item \PFCUATROTopicBacktracking%
\item \PFCUATROTopicImplementacion%
\end{topicos}%
}
\newcommand{\PFCUATROObjUNO}{Describir el concepto de recursividad y dar ejemplos de su uso. }
\newcommand{\PFCUATROObjDOS}{Identificar el caso base y el caso general de un problema definido recursivamente. }
\newcommand{\PFCUATROObjTRES}{Comparar soluciones iterativas y recursivas para problemas elementales tal como factorial. }
\newcommand{\PFCUATROObjCUATRO}{Describir la técnica dividir y conquistar. }
\newcommand{\PFCUATROObjCINCO}{Implementar, probar y depurar funciones y procedimientos recursivos simples. }
\newcommand{\PFCUATROObjSEIS}{Describir como la recursividad puede ser implementada usando una pila. }
\newcommand{\PFCUATROObjSIETE}{Discutir problemas para los cuales el backtracking es una solución apropiada. }
\newcommand{\PFCUATROObjOCHO}{Determinar cuando una solución recursiva es apropiada para un problema. }
\newcommand{\PFCUATROAllObjectives}%
{%
\begin{objetivos}%
\item \PFCUATROObjUNO%
\item \PFCUATROObjDOS%
\item \PFCUATROObjTRES%
\item \PFCUATROObjCUATRO%
\item \PFCUATROObjCINCO%
\item \PFCUATROObjSEIS%
\item \PFCUATROObjSIETE%
\item \PFCUATROObjOCHO%
\end{objetivos}%
}

\newcommand{\PFCINCODef}{PF5. Programación Orientada a Eventos. }
\newcommand{\PFCINCOHours}{4}
\newcommand{\PFCINCOTopicMetodos}{Métodos para la manipulación de eventos. }
\newcommand{\PFCINCOTopicPropagacion}{Propagación de eventos. }
\newcommand{\PFCINCOTopicManejo}{Manejo de excepciones. }
\newcommand{\PFCINCOAllTopics}%
{%
\begin{topicos}%
\item \PFCINCOTopicMetodos%
\item \PFCINCOTopicPropagacion%
\item \PFCINCOTopicManejo%
\end{topicos}%
}
\newcommand{\PFCINCOObjUNO}{Explicar la diferencia entre programación orientada a eventos y programación por línea de comandos. }
\newcommand{\PFCINCOObjDOS}{Diseñar, codificar, probar y depurar programas de manejo de eventos simples que respondan a eventos del usuario. }
\newcommand{\PFCINCOObjTRES}{Desarrollar código que responda a las condiciones de excepción lanzadas durante la ejecución. }
\newcommand{\PFCINCOAllObjectives}%
{%
\begin{objetivos}%
\item \PFCINCOObjUNO%
\item \PFCINCOObjDOS%
\item \PFCINCOObjTRES%
\end{objetivos}%
}

\newcommand{\ALUNODef}{AL1. Análisis de Algoritmos Básicos. }
\newcommand{\ALUNOHours}{4}
\newcommand{\ALUNOTopicAnalisis}{Análisis asintótico de límites en los casos promedio y superior. }
\newcommand{\ALUNOTopicIdentificar}{Identificar la diferencias entre casos de mejor, mediano y peor comportamiento. }
\newcommand{\ALUNOTopicNotacion}{Notación grande O, pequeña o, omega y theta. }
\newcommand{\ALUNOTopicClases}{Clases de complejidad estándar. }
\newcommand{\ALUNOTopicMedidas}{Medidas empíricas de desempeño. }
\newcommand{\ALUNOTopicCambios}{Cambios de tiempo y espacio en algoritmos. }
\newcommand{\ALUNOTopicUsar}{Usar relación de recurrencia para analizar algoritmos recursivos. }
\newcommand{\ALUNOAllTopics}%
{%
\begin{topicos}%
\item \ALUNOTopicAnalisis%
\item \ALUNOTopicIdentificar%
\item \ALUNOTopicNotacion%
\item \ALUNOTopicClases%
\item \ALUNOTopicMedidas%
\item \ALUNOTopicCambios%
\item \ALUNOTopicUsar%
\end{topicos}%
}
\newcommand{\ALUNOObjUNO}{Explicar el uso de anotaciones big O, omega y theta para describir la cantidad de trabajo hecha por un algoritmo. }
\newcommand{\ALUNOObjDOS}{Usar anotaciones big O, omega y theta y dar límites superior, bajos y estrechos en complejidad de algoritmos en tiempo y espacio. }
\newcommand{\ALUNOObjTRES}{Determinar la complejidad de tiempo y espacios de algoritmos simples. }
\newcommand{\ALUNOObjCUATRO}{Deducir relación de recurrencia que describe la complejidad de tiempo de algoritmos definidos recursivamente. }
\newcommand{\ALUNOObjCINCO}{Solucionar relaciones de recurrencia elemental. }
\newcommand{\ALUNOAllObjectives}%
{%
\begin{objetivos}%
\item \ALUNOObjUNO%
\item \ALUNOObjDOS%
\item \ALUNOObjTRES%
\item \ALUNOObjCUATRO%
\item \ALUNOObjCINCO%
\end{objetivos}%
}

\newcommand{\ALDOSDef}{AL2. Estrategias Algorítmicas. }
\newcommand{\ALDOSHours}{6}
\newcommand{\ALDOSTopicAlgoritmos}{Algoritmos de fuerza bruta (\textit{brute-force}). }
\newcommand{\ALDOSTopicAlgoritmosgolosos}{Algoritmos golosos (\textit{greedy}). }
\newcommand{\ALDOSTopicDividir}{Dividir y conquistar. }
\newcommand{\ALDOSTopicRegresion}{Regresión (\textit{Backtracking}). }
\newcommand{\ALDOSTopicBifurcacion}{Bifurcación y límites. }
\newcommand{\ALDOSTopicHeuristicas}{Heurísticas. }
\newcommand{\ALDOSTopicCasamiento}{Casamiento de patrones y algoritmos de cadenas/texto. }
\newcommand{\ALDOSTopicAlgoritmosde}{Algoritmos de aproximación numérica. }
\newcommand{\ALDOSAllTopics}%
{%
\begin{topicos}%
\item \ALDOSTopicAlgoritmos%
\item \ALDOSTopicAlgoritmosgolosos%
\item \ALDOSTopicDividir%
\item \ALDOSTopicRegresion%
\item \ALDOSTopicBifurcacion%
\item \ALDOSTopicHeuristicas%
\item \ALDOSTopicCasamiento%
\item \ALDOSTopicAlgoritmosde%
\end{topicos}%
}
\newcommand{\ALDOSObjUNO}{Describir el defecto de los algoritmos de fuerza bruta. }
\newcommand{\ALDOSObjDOS}{Para cada una de las diferentes clases de algoritmos (fuerza bruta, golosos, dividir y conquistar, regresión, bifurcación y límites, y heurístico), identificar un ejemplo del comportamiento humano cotidiano que ejemplifique el concepto básico. }
\newcommand{\ALDOSObjTRES}{Implementar un algoritmo goloso para solucionar un problema dado. }
\newcommand{\ALDOSObjCUATRO}{Implementar un algoritmo de dividir y conquistar para solucionar un problema apropiado. }
\newcommand{\ALDOSObjCINCO}{Utilizar regresión para solucionar problemas tal como el de navegación en un laberinto. }
\newcommand{\ALDOSObjSEIS}{Describir varios métodos de solución de problemas heurísticos. }
\newcommand{\ALDOSObjSIETE}{Utilizar casamiento de patrones para analizar subcadenas. }
\newcommand{\ALDOSObjOCHO}{Utilizar aproximación numérica para resolver problemas matemáticos, tal como el de encontrar las raíces de un polinomio. }
\newcommand{\ALDOSAllObjectives}%
{%
\begin{objetivos}%
\item \ALDOSObjUNO%
\item \ALDOSObjDOS%
\item \ALDOSObjTRES%
\item \ALDOSObjCUATRO%
\item \ALDOSObjCINCO%
\item \ALDOSObjSEIS%
\item \ALDOSObjSIETE%
\item \ALDOSObjOCHO%
\end{objetivos}%
}

\newcommand{\ALTRESDef}{AL3. Algoritmos de Computación Fundamental. }
\newcommand{\ALTRESHours}{12}
\newcommand{\ALTRESTopicAlgoritmos}{Algoritmos numéricos simples. }
\newcommand{\ALTRESTopicBusqueda}{Búsqueda secuencial y binaria. }
\newcommand{\ALTRESTopicAlgoritmoscuadraticos}{Algoritmos cuadráticos de ordenamiento (selección, inserción). }
\newcommand{\ALTRESTopicAlgoritmosde}{Algoritmos de tipo O(N log N) (Quicksort, heapsort, mergesort). }
\newcommand{\ALTRESTopicTablas}{Tablas (\textit{hash}) incluyendo estrategias de solución para las colisiones. }
\newcommand{\ALTRESTopicArboles}{Árboles de búsqueda binaria. }
\newcommand{\ALTRESTopicRepresentacion}{Representación de grafos (Listas y Matrices de adyacencia). }
\newcommand{\ALTRESTopicRecorridos}{Recorridos por amplitud y profundidad. }
\newcommand{\ALTRESTopicEl}{El algoritmo del camino más corto (algoritmos de Dijkstra y Floyd). }
\newcommand{\ALTRESTopicCerradura}{Cerradura transitiva (algoritmo de Floyd). }
\newcommand{\ALTRESTopicArbol}{Árbol de expansión mínima (algoritmos de Kruskal y Prim). }
\newcommand{\ALTRESTopicOrdenamiento}{Ordenamiento Topológico. }
\newcommand{\ALTRESAllTopics}%
{%
\begin{topicos}%
\item \ALTRESTopicAlgoritmos%
\item \ALTRESTopicBusqueda%
\item \ALTRESTopicAlgoritmoscuadraticos%
\item \ALTRESTopicAlgoritmosde%
\item \ALTRESTopicTablas%
\item \ALTRESTopicArboles%
\item \ALTRESTopicRepresentacion%
\item \ALTRESTopicRecorridos%
\item \ALTRESTopicEl%
\item \ALTRESTopicCerradura%
\item \ALTRESTopicArbol%
\item \ALTRESTopicOrdenamiento%
\end{topicos}%
}
\newcommand{\ALTRESObjUNO}{Implementar los algoritmos cuadráticos más comunes y los algoritmos de ordenamiento O(N log N). }
\newcommand{\ALTRESObjDOS}{Diseñar e implementar una función (\textit{hash}) apropiada para una aplicación. }
\newcommand{\ALTRESObjTRES}{Diseñar e implementar un algoritmo de resolución de colisiones para tablas hash. }
\newcommand{\ALTRESObjCUATRO}{Discutir la eficiencia computacional de los principales algoritmos de ordenamiento, búsqueda y (\textit{hashing}). }
\newcommand{\ALTRESObjCINCO}{Discutir factores distintos a la eficiencia computacional que influyen en la elección de los algoritmos, tales como tiempo de programación, mantenibilidad y el uso de patrones específicos de aplicación en los datos de entrada. }
\newcommand{\ALTRESObjSEIS}{Resolver problemas usando los algoritmos de grafos fundamentales, incluyendo búsqueda por amplitud y profundidad; caminos más cortos con una fuente y con múltiples fuentes, cerradura transitiva, ordenamiento topológico, y al menos un algoritmo de árbol de expansión mínima. }
\newcommand{\ALTRESObjSIETE}{Demostrar las siguientes capacidades: evaluar algoritmos, seleccionar una opción de un rango posible, proveer una justificación para tal elección e implementar el algoritmo. }
\newcommand{\ALTRESAllObjectives}%
{%
\begin{objetivos}%
\item \ALTRESObjUNO%
\item \ALTRESObjDOS%
\item \ALTRESObjTRES%
\item \ALTRESObjCUATRO%
\item \ALTRESObjCINCO%
\item \ALTRESObjSEIS%
\item \ALTRESObjSIETE%
\end{objetivos}%
}

\newcommand{\ALCUATRODef}{AL4. Algoritmos Distribuidos. }
\newcommand{\ALCUATROHours}{3}
\newcommand{\ALCUATROTopicConsenso}{Consenso y elección. }
\newcommand{\ALCUATROTopicDeteccion}{Detección de terminación. }
\newcommand{\ALCUATROTopicTolerancia}{Tolerancia a fallas. }
\newcommand{\ALCUATROTopicEstabilizacion}{Estabilización. }
\newcommand{\ALCUATROAllTopics}%
{%
\begin{topicos}%
\item \ALCUATROTopicConsenso%
\item \ALCUATROTopicDeteccion%
\item \ALCUATROTopicTolerancia%
\item \ALCUATROTopicEstabilizacion%
\end{topicos}%
}
\newcommand{\ALCUATROObjUNO}{Explicar el paradigma distribuido. }
\newcommand{\ALCUATROObjDOS}{Explicar un algoritmo distribuido simple. }
\newcommand{\ALCUATROObjTRES}{Determinar cuando usar los algoritmos de consenso o elección. }
\newcommand{\ALCUATROObjCUATRO}{Distinguir entre relojes físicos y lógicos. }
\newcommand{\ALCUATROObjCINCO}{Describir el ordenamiento relativo de eventos en un algoritmo distribuido. }
\newcommand{\ALCUATROAllObjectives}%
{%
\begin{objetivos}%
\item \ALCUATROObjUNO%
\item \ALCUATROObjDOS%
\item \ALCUATROObjTRES%