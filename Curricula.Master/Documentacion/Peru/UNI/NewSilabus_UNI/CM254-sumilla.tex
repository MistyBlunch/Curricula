\begin{sumilla}

\curso{CM254. Estructuras Discretas}{Obligatorio}{CM254}

\begin{fundamentacion}
Las estructuras discretas son fundamentales para la ciencia de la computaci\'on. Es evidente que las estructuras discretas son usadas en las \'areas de estructura de datos y algoritmos, sin embargo son tambi\'en importantes en otras, como por ejemplo en la verificaci\'on, en criptograf\'ia y m\'etodos formales.

Para entender las t\'ecnicas computacionales avanzadas, los estudiantes deber\'an tener un fuerte conocimiento de las diversas estructuras discretas, estructuras que ser\'an implementadas y usadas en laboratorio en el lenguaje de programaci\'on.

El \'algebra abstracta tiene un lado pr\'actico que explotaremos para comprender en profundidad temas de computaci\'on como criptograf\'ia y \'algebra relacional.
\end{fundamentacion}

\begin{objetivosdelcurso}
\item Desarrollar Operaciones asociadas con conjuntos, funciones y relaciones.
\item Relacionar ejemplos pr\'acticos al modelo apropiado de conjunto, funci\'on o relaci\'on.
\item Conocer las diferentes t\'ecnicas de conteo m\'as utilizadas.
\item Describir como las herramientas formales de l\'ogica simb\'olica son utilizadas.
\item Describir la importancia y limitaciones de la l\'ogica de predicados.
\item Bosquejar la estructura b\'asica y dar ejemplos de cada tipo de prueba descrita en esta unidad.
\item Relacionar las ideas de inducci\'on matem\'atica con la recursividad y con estructuras definidas recursivamente.
\item Enunciar, identificar y habituarse a los conceptos m\'as importantes de Conjuntos Parcialmente Ordenados y L\'atices
\item Analizar, comentar y aceptar las nociones b\'asicas de \'algebras Booleanas.
\item Que el alumno sea capaz de modelar problemas de ciencia de la computaci\'on usando grafos y \'arboles relacionados con estructuras de datos
\item Que el alumno aplicar eficientemente estrategias de recorrido para poder buscar datos de una manera \'optima
\item Conocer las t\'ecnicas y m\'etodos de encriptaci\'on de datos.
\end{objetivosdelcurso}

\begin{outcomes}
\ExpandOutcome{a}
\ExpandOutcome{b}
\ExpandOutcome{j}
\end{outcomes}

\begin{unit}{\DSUNODef}{Kolman97,Grassmann97,Johnsonbaugh99}{13}
\begin{topicos}
	\item  \DSUNOTopicFunciones
	\item  Conjuntos producto, especificaci\'on de relaciones.
	\item  \DSUNOTopicRelaciones
	\item  Clases de equivalencia, operaciones entre relaciones.
	\item  \DSUNOTopicConjuntos
	\item  \DSUNOTopicPrincipio
	\item  \DSUNOTopicCardinalidad
\end{topicos}

\begin{objetivos}
	\item \DSUNOObjUNO
	\item \DSUNOObjDOS
	\item \DSUNOObjTRES
	\item \DSUNOObjCUATRO
\end{objetivos}
\end{unit}

\begin{unit}{\DSDOSDef}{Grassmann97,Iranzo05,Paniagua03,Johnsonbaugh99}{14}
\begin{topicos}
         \item \DSDOSTopicLogica
         \item \DSDOSTopicConectivos
         \item \DSDOSTopicTablas
         \item \DSDOSTopicFormas
         \item \DSDOSTopicValidacion
         \item \DSDOSTopicLogicade
     \item \DSDOSTopicCuantificacion
         \item \DSDOSTopicModus
         \item \DSDOSTopicLimitaciones
   \end{topicos}

   \begin{objetivos}
      \item \DSDOSObjUNO
         \item \DSDOSObjDOS
         \item \DSDOSObjTRES
         \item \DSDOSObjCUATRO
   \end{objetivos}
\end{unit}

\begin{unit}{\DSTRESDef}{Scheinerman01,Brassard97,Kolman97,Johnsonbaugh99}{14}
\begin{topicos}
      \item \DSTRESTopicNociones
      \item \DSTRESTopicEstructura
      \item \DSTRESTopicPruebas
      \item \DSTRESTopicPruebasy
      \item \DSTRESTopicPruebaspor
      \item \DSTRESTopicPruebasporcontradiccion
      \item \DSTRESTopicInduccion
      \item \DSTRESTopicInduccionfuerte
      \item \DSTRESTopicDefiniciones
      \item \DSTRESTopicEl
   \end{topicos}

   \begin{objetivos}
      \item \DSTRESObjUNO
      \item \DSTRESObjDOS
      \item \DSTRESObjTRES
      \item \DSTRESObjCUATRO
   \end{objetivos}
\end{unit}

\begin{unit}{\ARUNODef}{Kolman97, Grimaldi97, Gersting87}{19}
\begin{topicos}
      \item Conjuntos Parcialmente Ordenados.
      \item Elementos extremos de un conjunto parcialmente ordenado.
      \item L\'atices.
      \item \'Algebras Booleanas.
      \item Funciones Booleanas.
      \item \ARUNOTopicExpresiones
      \item \ARUNOTopicBloques
   \end{topicos}
   \begin{objetivos}
      \item \DSTRESObjUNO
      \item \DSTRESObjDOS
      \item \DSTRESObjTRES
   \end{objetivos}
\end{unit}

\begin{unit}{\DSCUATRODef}{Grimaldi97}{25}
   \begin{topicos}
	 \item 	\DSCUATROTopicArgumentos
	 \item 	\DSCUATROTopicPermutaciones
	 \item 	\DSCUATROTopicPrincipio
	 \item 	\DSCUATROTopicSolucion 
   \end{topicos}

   \begin{objetivos}
	\item \DSCUATROObjUNO
	\item \DSCUATROObjDOS
	\item \DSCUATROObjTRES 
	\item \DSUNOObjCUATRO 
   \end{objetivos}
\end{unit}

\begin{unit}{\DSCINCODef}{Johnsonbaugh99}{25}
   \begin{topicos}
	 \item \DSCINCOTopicArboles
	 \item \DSCINCOTopicGrafos 
	 \item \DSCINCOTopicGrafosdirigidos
	 \item \DSCINCOTopicArbolesde 
	 \item \DSCINCOTopicEstrategias 
   \end{topicos}

   \begin{objetivos}
	 \item \DSCINCOObjUNO
	 \item \DSCINCOObjDOS
	 \item \DSCINCOObjTRES
	 \item \DSCINCOObjCUATRO
   \end{objetivos}
\end{unit}

\begin{unit}{\DSSEISDef}{Micha98,Rosen2004}{10}
   \begin{topicos}
      \item \DSSEISTopicEspacios
      \item \DSSEISTopicProbabilidad 
      \item \DSSEISTopicVariables 
   \end{topicos}

   \begin{objetivos}
      \item \DSSEISObjUNO
      \item \DSSEISObjDOS
      \item \DSSEISObjTRES
      \item \DSSEISObjCUATRO
   \end{objetivos}
\end{unit}

\begin{unit}{\ALNUEVEDef}{Grimaldi97, Scheinerman01}{20}
   \begin{topicos}
         \item  \ALNUEVEDef
         \item  \ALNUEVETopicRevision
         \item  \ALNUEVETopicCriptografia
        \item   \ALNUEVETopicCriptografiade
        \item   \ALNUEVETopicFirmas
        \item   \ALNUEVETopicProtocolos
        \item   \ALNUEVETopicAplicaciones
   \end{topicos}

   \begin{objetivos}
         \item \ALNUEVEObjUNO
         \item \ALNUEVEObjDOS
          \item \ALNUEVEObjTRES
   \end{objetivos}
\end{unit}

\begin{unit}{\IMCUATRODef}{Grassmann97}{20}
   \begin{topicos}
         \item \IMCUATRODef
         \item \IMCUATROTopicMapeo
         \item \IMCUATROTopicEntidad
         \item \IMCUATROTopicAlgebra
   \end{topicos}

   \begin{objetivos}
         \item \IMCUATROObjUNO
         \item \IMCUATROObjDOS
         \item \IMCUATROObjTRES
         \item \IMCUATROObjCUATRO
         \item \IMCUATROObjCINCO
   \end{objetivos}
\end{unit}

\begin{unit}{Teor\'ia de N\'umeros}{Grimaldi97, Scheinerman01}{20}
   \begin{topicos}
      \item Teor\'ia de los n\'umeros
     \item Aritm\'etica  Modular
      \item Teorema del Residuo Chino
       \item Factorizaci\'on
      \item Grupos, teor\'ia de la codificaci\'on y m\'etodo de enumeraci\'on de Polya
      \item Cuerpos finitos y dise\~nos combinatorios
   \end{topicos}

   \begin{objetivos}
      \item Establecer la importancia de la teor\'ia de n\'umeros en la criptograf\'ia
      \item Utilizar las propiedades de las estructuras algebraicas en el estudio de la teor\'ia algebraica de c\'odigos
   \end{objetivos}
\end{unit}


\begin{bibliografia}
\bibfile{CM254}
\end{bibliografia}

\end{sumilla}

%\end{document}

