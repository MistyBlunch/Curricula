\begin{sumilla}

\curso{CM142. C\'alculo Vectorial II}{Obligatorio}{CM142}

\begin{fundamentacion}
Se estudia los fundamentos de matrices para luego tratar determinantes, rango y sistemas lineales. Asimismo, se considera el estudio de valores y vectores propios con la finalidad de diagonalizar una matriz. Por otra parte, se trata espacios vectoriales, transformaciones lineales y formas cuadr\Åœaticas para tener aplicaci\Åœon a las secciones c\Åœonicas.
\end{fundamentacion}

\begin{objetivosdelcurso}
\item Al finalizar el curso el alumno comprender\Åœa los fundamentos de Matrices y Determinantes, el c\Åœalculo de valores y vectores propios, espacios vectoriales en Rn y las cu\Åœadricas.
\end{objetivosdelcurso}

\begin{outcomes}
\ExpandOutcome{a}
\ExpandOutcome{i}
\ExpandOutcome{j}
\end{outcomes}

\begin{unit}{1. Matrices y Determinantes}{Hasser97,Burgos08}{12}
\begin{topicos}
      \item Definici\Åœon de Matrices. Notaci\Åœon. Igualdad de matrices. Transpuesta de una matriz.
      \item Matrices Especiales: Matriz cuadrada, nula. Matriz diagonal. Matriz  Escalar. Matriz Identidad. Matriz sim\Åœetrica. Matriz antisim\Åœetrica. Matriz  triangular superior. Matriz triangular inferior.
      \item Operaciones con Matrices; Suma de matrices. Multiplicaci\Åœon de una matriz para escalar. Multiplicaci\Åœon de matrices.
	\item Potencia de una matriz. Inversa de una Matriz
	\item Determinantes: Definici\Åœon para una matriz de orden 2. Definci\Åœon para una matriz de orden 3
	\item Determinante de una matriz de orden nxn por menores y cofactores
	\item Propiedades de las determinantes
	\item Obtenci\Åœon de la Inversa de una Matriz
	\item Rango de una matriz. Definici\Åœon
	\item Operaciones elementales por filas y por columnas. Matrices equivalentes
	\item Matriz Escalonada
	\item Determinaci\Åœon del rango de una matriz por operaciones elementales
	\item Determinaci\Åœon de la Inversa de una matriz por operaciones elementales
	\item Sistemas de Ecuaciones Lineales. Definici\Åœon. Sistemas equivalentes
	\item Rango asociado a un sistema de Ecuaciones lineales. Condici\Åœon necesaria y suficiente para su consistencia.
	\item Soluciones Independientes de un sistema de Ecuaciones lineales
	\item Regla de Cramer.
	\item Sistemas de Ecuaciones Lineales, homogéneas. Propiedades.
   \end{topicos}

   \begin{objetivos}
      \item Conocer las caracter\'isticas de las matrices y el c\'alculo de determinantes
	\item Resolver problemas
   \end{objetivos}
\end{unit}

\begin{unit}{2. Espacios Vectoriales}{Hasser97,Burgos08}{8}
\begin{topicos}
	\item Espacio euclideano n-dimensional. n-ada ordenada. Operaciones: Suma, multiplicaci\Åœon por un escalar. Producto. Interior. Longitud de un vector.
	\item Espacio Vectorial tridimensional. Definici\Åœon. Subespacio. Combinaci\Åœon Lineal. Generaci\Åœon de sub espacios.
      \item Independencia Lineal. Base y dimensi\Åœon. Espacio fila y espacio columna de una matriz. Rango. M\Åœetodo para hallar bases
      \item Espacios con producto Interno. Desigualdad de Cauchy-Schwarz. Longitud de un vector. \ÅœAngulo entre vectores
      \item Bases ortonormales. Proceso de Gram-Schmidt. Coordenadas. Cambio de base. Rotaci\Åœon de los ejes coordenados en el plano, en el espacio. Matriz ortogonal.
    \end{topicos}
   \begin{objetivos}
      \item Describir matem\'aticamente el concepto de Espacio Vectorial
      \item Conocer y aplicar conceptos de espacios y bases
	\item Resolver problemas
   \end{objetivos}
\end{unit}

\begin{unit}{3. Transformaciones Lineales}{Anton,Burgos08}{6}
\begin{topicos}
      \item Definici\Åœon. Algunos tipos de transformaci\Åœon lineal: Identidad, dilataci\Åœon, contracci\Åœon, rotaci\Åœon. Proyecci\Åœon. N\Åœucleo e imagen.
      \item Transformaciones lineales de $R^n$ a $R^n$, de $R^n$ a $R^m$; geometr\Åœia de la transformaci\Åœon lineal de $R^2$ a $R^2$.
      \item Matrices asociadas a las transformaciones lineales. Equivalencia. Semejanza.
\end{topicos}

   \begin{objetivos}
      \item Describir matem\'aticamente los transformaciones lineales
	\item Resolver problemas
   \end{objetivos}
\end{unit}

\begin{unit}{4. Valores Propios y Vectores Propios}{Anton,Burgos08}{12}
\begin{topicos}
      \item Valor propio y Vector propio. Definiciones. Polinomio caracter\Åœistico. Propiedades.
      \item Diagonalizaci\Åœon de una matriz. Definici\Åœon. Condici\Åœon, necesaria y suficiente para la diagonalizaci\Åœon de una matriz.
      \item Diagonalizaci\Åœon ortogonal. Matrices Sim\Åœetricas.
	\item Formas Cuadr\Åœaticas. Definici\Åœon. Aplicaci\Åœon a las secciones c\Åœonicas en el plano y en el espacio.
	\end{topicos}

   \begin{objetivos}
      \item Conocer y aplicar conceptos de valores y vectores propios
	\item Resolver problemas
   \end{objetivos}
\end{unit}

\begin{unit}{5 Cu\Åœadricas}{Granero85,Saal84}{12}
\begin{topicos}
      \item Coordenadas Homog\Åœeneas en el Espacio.
	\item Ecuaci\Åœon param\Åœetrica de la recta que pasa por dos puntos.
	\item Cu\ÃÃÃ. Transformaciones Lineales}{Anton,Burgos08}{6}
\begin{topicos}
      \item Definici\Åœon. Algunos tipos de transformaci\Åœon lineal: Identidad, dilataci\Åœon, contracci\Åœon, rotaci\Åœon. Proyecci\Åœon. N\Åœucleo e imagen.
      \item Transformaciones lineales de $R^n$ a $R^n$, de $R^n$ a $R^m$; geometr\Åœia de la transformaci\Åœon lineal de $R^2$ a $R^2$.
      \item Matrices 