\begin{sumilla}

\curso{CM274. Introducci\'on a la Estad\'istica y Probabilidades}{Obligatorio }{CM274}

\begin{objetivosdelcurso}
\item  Estudiar algunos conceptos básicos de Estadística, tanto Descriptiva como Inferencial, para el mejor desenvolvimiento en el Área de Matemática con respecto a la recopilación de datos y la toma de decisiones a partir de la interpretación de dichos datos.
\item  Usar las diferentes distribuciones de probabilidad junto con las pruebas de hipótesis como entes comparativos de datos muestrales, y como ayuda para pronosticar diferentes situaciones reales a posteriori.
\end{objetivosdelcurso}

\begin{outcomes}
\ExpandOutcome{a}
\ExpandOutcome{i}
\ExpandOutcome{j}
\end{outcomes}

\begin{unit}{Nociones de estad\'istica descriptiva}{Garcia98}{4}
   \begin{topicos}
         \item  Estad\'istica
	 \item  Poblaci\'on y muestra.
	 \item  Variables estad\'isticas.
         \item  Organizaci\'on de los datos: Distribuciones de frecuencias.
   \end{topicos}

   \begin{objetivos}
         \item  Entender los conceptos b\'asicos de la Estad\'istica descriptiva
         \item  Resolver problemas
   \end{objetivos}
\end{unit}

\begin{unit}{Medidas de posici\'on}{Garcia99}{6}
   \begin{topicos}
	\item Introducci\'on.
	\item Media Aritm\'etica.
	\item Mediana.
	\item Moda.
	\item Relaci\'on entre media, mediana y moda.
	\item Uso de los promedios.
	\item Cuantiles
	\item Otras medias: Media geom\'etrica, media arm\'onica.
   \end{topicos}

   \begin{objetivos}
         \item  Entender y aplicar los conceptos y caracter\'isticas de las medidas de posici\'on
         \item  Resolver problemas
   \end{objetivos}
\end{unit}

\begin{unit}{Medidas de dispersi\'on}{Cordova97}{6}
   \begin{topicos}
         \item  Introducci\'on.
	 \item  Medidas de Dispersi\'on.
         \item  \'Indices de Asimetr\'ia
	 \item  \'Indice de curtosis o apuntamiento
   \end{topicos}

   \begin{objetivos}
         \item  Entender los conceptos y caracter\'isticas de las Medidas de Dispersi\'on.
         \item  Resolver problemas
   \end{objetivos}
\end{unit}

\begin{unit}{Regresi\'on lineal simple}{Garcia99}{4}
   \begin{topicos}
         \item  Introducci\'on.
	 \item  Regresi\'on lineal simple
         \item  Nociones de regresi\'on no lineal.
   \end{topicos}

   \begin{objetivos}
         \item  Entender y aplicar los conceptos de Regresi\'on lineal y no lineal
         \item  Resolver problemas
   \end{objetivos}
\end{unit}

\begin{unit}{Probabilidad}{Moya92}{8}
   \begin{topicos}
         \item  Experimento aleatorio, espacio muestral, eventos.
	 \item  Conteo de puntos muestrales (n\'umero de puntos muestrales, variaciones, permutaciones, combinaciones).
         \item  Probabilidad de un evento.
	 \item  C\'alculo de Probabilidades.
	\item Probabilidad Condicional.
	\item Eventos independientes.
	\item Reglas de la multiplicaci\'on, probabilidad total y de Bayes.
   \end{topicos}

   \begin{objetivos}
         \item  Entender y aplicar los conceptos de probabilidad aleatoria y de Bayes
         \item  Resolver problemas
   \end{objetivos}
\end{unit}

\begin{unit}{Variables aleatorias y distribución de probabilidad}{Mitacc99}{6}
   \begin{topicos}
         \item  Variable aleatoria.
	 \item  Variable aleatoria discreta: Funci\'on de probabilidad y funci\'on de distribuci\'on acumulada.
         \item  Variable aleatoria continua: Funci\'on de densidad y funci\'on de distribuci\'on acumulada.
	 \item  Propiedades de la funci\'on de distribuci\'on.
         \item  Valor esperado o Esperanza Matem\'atica.
   \end{topicos}

   \begin{objetivos}
         \item  Entender y aplicar los conceptos de variables aleatorias y  distribuci\'on de probabilidad
         \item  Resolver problemas
   \end{objetivos}
\end{unit}

\begin{unit}{Algunas distribuciones importantes}{Larson90}{6}
   \begin{topicos}
         \item  Algunas distribuciones importantes de variables aleatorias discretas: Bernoulli, Binomial,  Geométrica, Pascal o Binomial Negativa, Hipergeométrica, Poisson.
	 \item  Algunas distribuciones importantes de variables aleatorias continuas: Uniforme, Normal, Erlang, Gamma, exponencial, Chi-cuadrado, $t$ de Student, F de Fisher.
   \end{topicos}

   \begin{objetivos}
         \item  Entender y aplicar algunas distribuci\'ones importantes de variables aleatorias discretas y continuas.
         \item  Resolver problemas
   \end{objetivos}
\end{unit}

\begin{unit}{Distribuciones muestrales}{Moya92}{4}
   \begin{topicos}
         \item  Muestreo aleatorio.
	 \item  Distribuciones muestrales: de la media, de una proporción, varianza.
   \end{topicos}

   \begin{objetivos}
         \item  Entender y aplicar algunas distribuci\'ones muestrales
         \item  Resolver problemas
   \end{objetivos}
\end{unit}

\begin{unit}{Estimaci\'on de par\'ametros}{Mitacc99}{4}
   \begin{topicos}
         \item  Introducci\'on.
	 \item  Estimaci\'on puntual de par\'ametros.
	\item Estimaci\'on de par\'ametros por intervalos.
	\item Intervalos de confianza.
   \end{topicos}

   \begin{objetivos}
         \item  Aprender como aplicar la estimaci\'on de par\'ametros
         \item  Resolver problemas
   \end{objetivos}
\end{unit}

\begin{unit}{Pruebas de hipótesis}{Larson90}{4}
   \begin{topicos}
         \item  Hip\'otesis estad\'isticas: Hip\'otesis nula y alternativa, errores tipo I y tipo II, nivel de significancia.
	 \item  Pruebas de hip\'otesis.
   \end{topicos}

   \begin{objetivos}
         \item  Aprender como aplicar las pruebas de hip\'otesis.
         \item  Resolver problemas
   \end{objetivos}
\end{unit}

\begin{bibliografia}
\bibfile{CM274}
\end{bibliografia}

\end{sumilla}


