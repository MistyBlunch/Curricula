\begin{sumilla}

\curso{CQ121. Qu\'imica General I}{Obligatorio}{CQ121}

\begin{fundamentacion}
Este curso es \'util en esta carrera para que el alumno aprenda a mostrar un alto grado de dominio de las leyes de la Qu\'imica General.
\end{fundamentacion}

\begin{objetivosdelcurso}
\item Capacitar y presentar al estudiante los principios básicos de la química como ciencia natural abarcando sus tópicos más importantes y su relación con los problemas cotidianos.
\end{objetivosdelcurso}

\begin{outcomes}
\ExpandOutcome{a}
\ExpandOutcome{i}
\ExpandOutcome{j}
\end{outcomes}

\begin{unit}{QU1. Termodinámica}{Raymond99,Kennet92}{2}
\begin{topicos}
      \item Sistemas termodinámicos y su clasificación. Variables termodinámicas y funciones de estado.
      \item Estados de un sistema. Estados de equilibrio. Variables extensivas e intensivas.
      \item Equilibrios térmicos. Principio cero de la termodinámica.
      \item Primer principio de la termodinámica. Capacidad calorífica. Procesos reversibles y trabajo máximo.
      \item Energía interna de los gases ideales. Transformaciones adiabáticas. Termoquímica. Ley de Lavoisier y La Place, Ley de Hess. Ley de Kirchhoff.
      \item Segunda Ley de la termodinámica. Entropía. Eficiencia de un ciclo reversible.
	\item Energía libre. Tercera ley de la termodinámica.
   \end{topicos}

   \begin{objetivos}
      \item Entender y trabajar con los principios de la Termodin\'amica.
      \item Abstraer de la naturaleza los conceptos de las transformaciones de los gases.
   \end{objetivos}
\end{unit}

\begin{unit}{QU2. Equilibrio Químico}{Raymond99,Kennet92}{4}
\begin{topicos}
      \item Concepto. Constante de equilibrio.
      \item Ley de acción de las masas.
      \item Equilibrios homogéneos. Equilibrios heterogéneos. Equilibrios m\Ŝultiples.
      \item Factores que afectan el equilibrio químico. Principio de Le Chatelier.
    \end{topicos}
   \begin{objetivos}
      \item Describir, conocer y aplicar los conceptos del equilibrio qu\'imico.
      \item Resolver problemas.
   \end{objetivos}
\end{unit}

\begin{unit}{QU3. Estudios que Contribuyeron al Desarrollo de la Teoría del Átomo}{Raymond99}{4}
\begin{topicos}
      \item Propiedades de las ondas.
      \item Radiación electromagnética. Característica. Espectros.
      \item Teoría Cuántica de Max Planck.
      \item Efecto fotoeléctrico.
      \item Relación entre la materia y energía.
      \item Rayos X, Rayos catódicos y rayos canales.
      \item Ejercicios y problemas
\end{topicos}

   \begin{objetivos}
      \item Describir el comportamiento y caracter\'isticas de las ondas.
      \item Entender qualitativa y quantitativamente el comportamiento corpuscular de las ondas electromagn\'eticas.
      \item Resolver problemas.
   \end{objetivos}
\end{unit}

\begin{unit}{QU4. Teorías del Átomo}{Babor83,Kennet92}{6}
\begin{topicos}
      \item Postulados de Dalton. Modelo atómico de Thompson.
      \item Experimento de Rutherford, Modelo atómico de Rutherford. Inconsistencia.
      \item Modelo atómico de Bohr. Espectro de emisión del átomo de hidrógeno.
      \item Teoría atómica moderna. Dualidad de la materia.
      \item Principio de incertidumbre de Heisenberg.
      \item Orbitales atómicos. Ecuación de Schrodinguer.
      \item Descripción mecánico cuántica del átomo de hidrogeno N\Ŝumeros cuánticos.
      \item Configuración electrÃÂÂÂÂÂ