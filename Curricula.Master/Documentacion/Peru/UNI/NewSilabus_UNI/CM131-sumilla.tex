\begin{sumilla}

\curso{CM131. C\'alculo Diferencial}{Obligatorio}{CM131}

\begin{fundamentacion}
Conceptos básicos de lógica matemática, métodos de demostración. Definición axiomática de los números reales, El axioma del supremo, densidad de los racionales. Funciones reales: funciones biyectivas, funciones monótonas, composición e inversa de una función. Sucesiones reales: limite de sucesiones, sucesiones monótonas.  Limites de funciones, limites trigonométricos, limites y el infinito, asíntotas  oblicuas. Continuidad, el teorema del valor intermedio, el teorema de los valores extremos, Derivadas: regla de la cadena, derivación implícita, la diferencial., el teorema del valor medio, criterios de la primera y segunda derivada, concavidad y puntos de inflexión, el teorema del valor medio generalizado, las reglas de L'hospital, el método de Newton para hallar raíces y el  teorema de Taylor  con resto diferencial.
\end{fundamentacion}

\begin{objetivosdelcurso}
\item Al finalizar el curso el alumno comprender\'a los fundamentos del C\'alculo Diferencial, y habr\'a adquirido habilidades que le permitan usar los conceptos estudiados, en el desarrollo de otras asignaturas, as\'i como tambi\'en en la soluci\'on de problemas vinculados a su especialidad 
\end{objetivosdelcurso}

\begin{outcomes}
\ExpandOutcome{a}
\ExpandOutcome{i}
\ExpandOutcome{j}
\end{outcomes}

\begin{unit}{Introducci\'on a la L\'ogica Proposicional}{Hasser97,Spivak96}{4}
\begin{topicos}
      \item Disyunci\'on y conjunci\'on de proposiciones.
      \item Negaci\'on.
      \item Implicaci\'on, equivalencia.
      \item Reglas de Inferencia y demostraciones.
   \end{topicos}

   \begin{objetivos}
      \item Conocer y aplicar conceptos de L\'ogica Proposicional.
	\item Resolver problemas.
   \end{objetivos}
\end{unit}

\begin{unit}{Propiedades B\'asicas de los N\'umeros}{Helfgott89}{8}
\begin{topicos}
	\item Axiomas de Cuerpo
	\item Axioma de orden
	\item Conjuntos Acotados
	\item Axioma del supremo  (completitud)
	\item Propiedades del supremo y del \'infimo
	\item Representaci\'on decimal de los n\'umeros reales
	\item Fracciones continuas
\end{topicos}
\begin{objetivos}
	\item Describir matem\'aticamente las propiedades b\'asicas de los N\'umeros
	\item Resolver problemas
\end{objetivos}
\end{unit}

\begin{unit}{Funciones}{Bartle90}{12}
\begin{topicos}
      \item Definici\'on. Dominio. Rango. Imagen. Pre imagen
      \item Operaciones de funciones: Suma, resta, multiplicaci\'on, divisi\'on y composici\'on
      \item Funciones mon\'otonas. Funciones inyectivas,  suryectivas y biyectivas
      \item Funci\'on inversa. Gr\'afica de funciones
      \item Modelaci\'on con funciones
\end{topicos}

   \begin{objetivos}
      \item Describir matem\'aticamente las funciones
      \item Conocer y aplicar las operaciones de funciones
	\item Resolver problemas
   \end{objetivos}
\end{unit}

\begin{unit}{L\'imites y Continuidad}{Leithold82}{12}
\begin{topicos}
      \item El \'limite de una funci\'on. L\'imites laterales. L\'imites  infinito. Teoremas sobre límites y aplicaciones
      \item As\'intotas horizontales, verticales y oblicuas a las gr\'aficas de una funci\'on
      \item Continuidad. El Teorema del Valor Intermedio. Teorema del Cero. Tipos de discontinuidad
      \item Funciones acotadas Teorema fundamental de las funciones continuas
	\end{topicos}

   \begin{objetivos}
      \item Describir matem\'aticamente los l\'imites y continuidad de funciones
	\item Resolver problemas
   \end{objetivos}
\end{unit}

\begin{unit}{La Derivada}{Edwards96}{12}
\begin{topicos}
	\item La derivada de una funci\'on en un punto. Interpretaciones geom\'etrica y f\'isica de la derivada
	\item Regla de derivaci\'on. La regla de la cadena
	\item Derivaci\'on impl\'icita. Derivada de la funci\'on inversa
	\item Funciones derivables en un intervalo
	\item Representaci\'on param\'etrica de una curva. Curvas diferenciables
\end{topicos}

\begin{objetivos}
	\item Describir matem\'aticamente la derivada
	\item Conocer y aplicar conceptos de derivada en la soluci\'on de problemas
\end{objetivos}
\end{unit}

\begin{unit}{Aplicaciones de La Derivada}{Stewart99}{8}
\begin{topicos}
	\item Extremos locales y globales de una funci\'on. Funciones crecientes y decrecientes
	\item Puntos cr\'iticos. Los Teoremas de Rolle y del Valor Medio
	\item La regla de L'Hospital para el c\'alculo de l\'imites
	\item Criterio de la primera derivada. Concavidad. Puntos de inflexi\'on
	\item Criterio de la segunda derivada. As\'intotas oblicuas. Trazo del gr\'afico de una funci\'on
	\item Problemas de Optimizaci\'on. La derivada como raz\'on de cambio instant\'aneo
	\item Velocidades relacionadas. La derivada en otras ciencias
	\item Diferenciales. Aproximaci\'on, usando diferenciales
\end{topicos}

\begin{objetivos}
	\item Conocer y aplicar conceptos de derivada
	\item Resolver problemas
\end{objetivos}
\end{unit}

\begin{bibliografia}
\bibfile{CM131}
\end{bibliografia}
\end{sumilla}


