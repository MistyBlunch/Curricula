\begin{sumilla}

\curso{CQ122. Qu\'imica General II}{Obligatorio}{CQ122}

\begin{fundamentacion}
Este curso es \'util en esta carrera para que el alumno aprenda a mostrar un alto grado de dominio de las leyes de la Qu\'imica General.
\end{fundamentacion}

\begin{objetivosdelcurso}
\item Capacitar y presentar al estudiante los principios básicos de la química como ciencia natural abarcando sus tópicos más importantes y su relación con los problemas cotidianos.
\end{objetivosdelcurso}

\begin{outcomes}
\ExpandOutcome{a}
\ExpandOutcome{i}
\ExpandOutcome{j}
\end{outcomes}

\begin{unit}{Termodinámica}{Chang99,Garritz94}{2}
\begin{topicos}
	\item Sistemas termodinámicos y su clasificación. Variables termodinámicas y funciones de estado.
	\item Estados de un sistema. Estados de equilibrio. Variables extensivas e intensivas.
	\item Equilibrios térmicos. Principio cero de la termodinámica.
	\item Primer principio de la termodinámica. Capacidad calorífica. Procesos reversibles y trabajo máximo.
	\item Energía interna de los gases ideales. Transformaciones adiabáticas. Termoquímica. Ley de Lavoisier y La Place, Ley de Hess. Ley de Kirchhoff.
	\item Segunda Ley de la termodinámica. Entropía. Eficiencia de un ciclo reversible.
	\item Energía libre. Tercera ley de la termodinámica.
\end{topicos}

\begin{objetivos}
	\item Entender y trabajar con los principios de la Termodin\'amica.
	\item Abstraer de la naturaleza los conceptos de las transformaciones de los gases.
\end{objetivos}
\end{unit}

\begin{unit}{Equilibrio químico}{Chang99,Garritz94}{4}
\begin{topicos}
      \item Concepto. Constante de equilibrio.
      \item Ley de acción de las masas.
      \item Equilibrios homogéneos. Equilibrios heterogéneos. Equilibrios múltiples.
      \item Factores que afectan el equilibrio químico. Principio de Le Chatelier.
    \end{topicos}
   \begin{objetivos}
      \item Describir, conocer y aplicar los conceptos del equilibrio qu\'imico.
      \item Resolver problemas.
   \end{objetivos}
\end{unit}

\begin{unit}{Ácidos y bases}{Chang99,Garritz94}{4}
\begin{topicos}
	\item Ácidos y bases de Bronsted. Propiedades ácido-base del agua. El pH.
	\item Fuerza de los ácidos y bases. Ácidos débiles y su constante de ionización ácida. Bases débiles y su constante de ionización básica. 
	\item Relación entre la constante de acidez de los ácidos y sus bases conjugadas.
	\item Ácidos dipróticos y polipróticos. Propiedades ácido-base de las sales.
	\item Hidrólisis.  Ácidos y bases de Lewis
\end{topicos}

\begin{objetivos}
	\item Describir el comportamiento y caracter\'isticas de los \'acidos y las bases.
	\item Resolver problemas.
\end{objetivos}
\end{unit}

\begin{unit}{Equilibrio ácido-base y equilibrio de solubilidad}{Whitten98,Brady98}{6}
\begin{topicos}
	\item El efecto del ion común. Disoluciones reguladoras.
	\item Titulaciones ácido-base. Tipos.  Indicadores ácido-base.
	\item El producto de solubilidad. SeparaciÃÃÂÃÂ�