\begin{syllabus}

\section{Yes!}{Obligatorio}{GH2009}

\begin{justification}
El objetivo de este curso es situar ala estudiante dentro de la historia del desarrollo de la industria peruana tomando como punto de partida la Reforma Agraria .Durante el curso , se espera que el estudiante logre articular los distintos 
procesos que han dado forma a  la industria peruana hoy enfocándose  especialmente en las industrias extractivas .Se Busca que el estudiante  comprenda,como partes de un todo cómo las condiciones sociales y políticas,
económicas y geográficas del Perú  han configurado nuestro desarrollo industrial en los últimos 50 años.En términos de aprendizaje, el curso debeb ser capaz de desarrollar en el estudiante  una narrativa más crítica y compleja de escenarios como ,por ejemplo,Conga,el  Baguazo 
o la  miniría informal .En términos de competencias,este curso debe centrarse en  trabajar las capacidades de análisis e interpretación del estudiante.

\end{justification}

\begin{goals}
\item Capacidad de interpretar información.
\item Capacidad para identificar problemas.
\item Capacidad de comprender textos.
\item Interés por conocer sobre temas actuales en la sociedad peruana y el mundo. 
\end{goals}

\begin{outcomes}
    \item \ShowOutcome{d}{2} % Multidiscip teams
    \item \ShowOutcome{e}{2} % ethical, legal, security and social implications
    \item \ShowOutcome{f}{2} % communicate effectively
    \item \ShowOutcome{n}{2} % Apply knowledge of the humanities
    \item \ShowOutcome{o}{2} % TASDSH
\end{outcomes}

\begin{competences}
    \item \ShowCompetence{C10}{d,n,o}
    \item \ShowCompetence{C17}{f}
    \item \ShowCompetence{C18}{f}
    \item \ShowCompetence{C21}{e}
\end{competences}

\begin{unit}{Perú país industrial?}{}{Mayer44}{12}{4}
   \begin{topics}
      \item Descripción de América Latina y el mundo en la década de 1960 con miras a contextualizar el Perú en relación al resto del mundo: influencia de la guerra fría,la revolución cubana (1959), la visión industrial de la Cepal que implicaba la sustitución de importaciones, etc.
      \item La foto económica y social del Perú en 1960: población, pobreza, distribución de la riqueza, estructura económica.
      \item La reforma agraria: lógica y efectos económicos y sociales. Procesos y resultados. ?`Cuánto de la realidad actual del Perú se explica por la reforma agraria?
      \item Política industrial: lógica y desenlace de las políticas estatistas/proteccionistas en la industria.
      \item El terrorismo y la sociedad: ?`Qué sabemos sobre sus causas y sus consecuencias?
      \item La crisis de fines de los 80. La hiperinflación y el intervencionismo económico.
      \item Los principios del modelo peruano. La constitución del 93.?` Por qué se prefirió la propiedad privada de las empresas? ?`Qué fue lo que realmente cambió y para qué?
      \item Política industrial, marco promotor y resultados. Estructura actual de la industria y potencial.
      \item Cambios generales en las relaciones de poder: partidos políticos, sindicatos, desarrollo de los liderazgos locales, y finalmente, la descentralización del 2004.
      \item La tributación de actividades extractivas, el canon y el desarrollo local.
      \item Las políticas ambientales y los conflictos sociales.
      \item La relación de las empresas con las comunidades locales.
   \end{topics}
   \begin{learningoutcomes}
      \item Capacidad de analizar información e Interés por conocer sobre la sociedad peruana y el mundo.
   \end{learningoutcomes}
\end{unit}



\begin{coursebibliography}
\bibfile{GeneralEducation/GH2009}
\end{coursebibliography}

\end{syllabus}
